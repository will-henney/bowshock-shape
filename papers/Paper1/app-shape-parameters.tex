
\section{Planitude and alatude of thin-shell bow shapes}
\label{sec:thin-shell-shapes}

In this appendix, we provide analytic calculations of the planitude
(exact) and alatude (approximate) for the wilkinoid, cantoids, and
ancantoids.  We first consider the most general case of the
ancantoids, and then show how results for cantoids and the wilkinoid
follow as special cases.

From equations~\eqref{eq:radius-curvature} and~\eqref{eq:planitude},
the planitude depends on the apex second derivative,
\(R_{\theta\theta,0}\), as
\begin{equation}
  \label{eq:planitude-from-2nd-derivative}
  \Pi = \left(  1 - R_{\theta\theta,0} / R_0\right)^{-1} \ .
\end{equation}
From equation~\eqref{eq:taylor-R-theta}, the second derivative can be
found from the coefficient of \(\theta^2\) in the Taylor expansion
of \(R(\theta)\).  Since we do not have \(R(\theta)\) in explicit analytic form,
we proceed via a Taylor expansion of the implicit
equations~\eqref{eq:crw-angles}
and~\eqref{eq:ancantoid-theta-theta1-implicit}, retaining terms up to
\(\theta^4\) to obtain from
equation~\eqref{eq:ancantoid-theta-theta1-implicit}:
\begin{equation}
  \label{eq:taylor-expansion-implicit}
  \theta_1^2 = \beta \theta^2 \left( 1 + C_{k\beta} \theta^2\right) + \mathcal{O}(\theta^6)\ , 
\end{equation}
with the coefficient \(C_{k\beta}\) given by
\begin{equation}
  \label{eq:C-k-beta}
  C_{k\beta} = \frac{1}{15} - \frac{3k}{20} - \frac{\beta}{15}  \ .
\end{equation}
Then, from equation~\eqref{eq:crw-angles} we find
\begin{align}
  \label{eq:taylor-R-over-D}
  \frac{R}{D} & = \frac{\sin \theta_1} {\sin (\theta + \theta_1)} \nonumber \\
              & = \frac{\beta^{1/2}}{1+\beta^{1/2}}
                \left\lbrace 1 + \theta^2
                \left[ \frac{C_{k\beta}} {2 \left(1+\beta^{1/2}\right)}
                + \frac{1}{6} \left(1+2\beta^{1/2} \right)
                \right]
                \right\rbrace + \mathcal{O}(\theta^4) \ ,
\end{align}
where in the second line we have carried out a Taylor expansion of the
two \(\sin\) terms and substituted
\eqref{eq:taylor-expansion-implicit}.  Comparing coefficients of unity
and \(\theta^2\) between equations~\eqref{eq:taylor-R-theta} and
\eqref{eq:taylor-R-over-D} we find
\begin{align}
  \label{eq:again-R0-over-D}
  \frac{R_0} {D} &= \frac{\beta^{1/2}}{1+\beta^{1/2}} \\
  \label{eq:final-second-derivative}
  \frac{R_{\theta\theta,0}} {R_0} &= \frac{C_{k\beta}}{1+\beta^{1/2}}+\frac{1}{3}\left(1+2\beta^{1/2}\right) \ ,
\end{align}
so that the final result for the planitude, from~\eqref{eq:planitude-from-2nd-derivative}, is
\begin{equation}
  \label{eq:final-planitude}
  \Pi = \left[ {1 - \frac{C_{k\beta}}{1+\beta^{1/2}} - \frac{1}{3}\left(1+2\beta^{1/2}\right)}
  \right]^{-1} \ .
\end{equation}




%%% Local Variables:
%%% mode: latex
%%% TeX-master: "quadrics-bowshock"
%%% End:
