%\RequirePackage{amsmath}
\documentclass[useAMS, usenatbib, a4paper]{mnras}
\pdfsuppresswarningpagegroup=1

% The following is needed to fix the margins if using Letter-size paper
% REMOVE if your LaTeX uses A4 paper by default
%\addtolength\topmargin{-1.8cm}

% Standard LaTeX packages
% \usepackage[varg]{txfonts}
\usepackage[varg]{newtxmath}
\usepackage{newtxtext}
\usepackage{graphicx}
\usepackage{microtype}
\usepackage{xcolor}
\usepackage{fixltx2e}
\usepackage{booktabs}
\usepackage{siunitx}
\usepackage{color}
\usepackage{enumerate}
\usepackage{pdflscape}
\usepackage{rotating}
\usepackage{hyperref}

% Define a fermata symbol for use inline with text
% 04 Jul 2017 WJH
% Based on info in Secs 18.4.1 and 24.1 of musixdoc.pdf
\usepackage{musixtex}
\makeatletter
\newcommand\textfermata{%
  {\let\extractline\relax
    \setlines10\smallmusicsize \nobarnumbers \nostartrule
    \staffbotmarg0pt \setclefsymbol1\empty \global\clef@skip0pt
    \raisebox{0ex}[0ex][0ex]{
      \startextract\addspace{-\afterruleskip}%
      \notes\fermataup{-2}\en
      \zendextract}}}
\makeatother

% % The \eye command is in the dingbat package, but we need to take care
% % of a naming conflict with the AMS package
% \usepackage{savesym}
% \savesymbol{checkmark}
% \usepackage{dingbat}

% The "eye" symbol is drawn with the \faEye command of the fontawesome
% package
\usepackage{fontawesome}
% The ``irregular'' symbol is drawn from the Icelandic Sorcery and Witchcraft 
\usepackage{staves}

\hypersetup{colorlinks=True, linkcolor=blue!50!black, citecolor=black,
  urlcolor=blue!50!black}

\usepackage{etoolbox}
\robustify\bfseries
\robustify\itshape


%% Bold italic
\newcommand\hmmax{0}            % we don't need heavy fonts
\newcommand\bmmax{1}            % reduce use of math alphabets for bold
\usepackage{bm}

%% Bundled custom packages
\usepackage{aastex-compat}

% Definitions needed in abstract
\newcommand\hii{\ion{H}{ii}}



\title[Bowshock shapes]{True versus apparent shapes of astrophysical bow shocks}

\newcommand\AddressCRyA{Instituto de Radioastronom\'{\i}a y Astrof\'{\i}sica,
  Universidad Nacional Aut\'onoma de M\'exico, Apartado Postal 3-72,
  58090 Morelia, Michoac\'an, M\'exico}
\author[Tarango Yong \& Henney]{
  Jorge A. Tarango Yong \& William J. Henney\\
  \AddressCRyA
}
\begin{document}
\maketitle
\begin{abstract}
  Stellar bow shocks are the result of the supersonic interaction
  between a stellar wind and its environment.  Some of these are
  "runaways": high-velocity stars that have been ejected from a star
  cluster.  Others are "weather vanes", where it is the local
  interstellar medium itself that is moving, perhaps as the result of
  a champagne flow of ionized gas from a nearby \hii{} region.  We
  present a new two-dimensional classification scheme for the shape of
  such bow shocks, which accounts for both the apex (pointy versus
  flat) and the wings (open versus closed).  For different theoretical
  bow shock models, we show how variations in the true shape can be
  disentangled from projection effects due to variations in the
  inclination angle.  We apply our classification scheme to three
  different observational datasets: mid-infrared arcs around hot
  main-sequence stars (\(N = 227\)), far-infrared arcs around luminous
  cool stars (\(N = 7\)), and emission-line arcs around proplyds and other
  young stars in the Orion Nebula (\(N = 18\)).  We find significant
  differences between the three datasets: cool star bow shocks have
  markedly more closed wings than hot star bow shocks, while
  differences in the shape of the apex region are only marginally
  significant.  The Orion Nebula arcs, on the other hand, have both
  significantly more open wings and significantly flatter apices than
  the OhotB star bow shocks.  We discuss the implications of these
  differences for understanding the physics of the bow shock
  interaction in the three classes of sources.
\end{abstract}

% Force paper II to be before paper III in the reference list
%\nocite{Henney:2018a}

\section{Introduction}
\label{sec:intro}


%%
%% Circumstances when bowshocks arise
%%

The archetypal bow shock is formed when a solid body moves
supersonically through a compressible fluid.  Terrestrial examples
include the atmospheric re-entry of a space capsule, or the sonic boom
produced by a supersonic jet \citep{van-Dyke:1982a}.  In astrophysics
the term bow shock is employed more widely, to refer to many different
types of curved shocks that have approximate cylindrical symmetry.
Instead of a solid body, astrophysical examples usually involve the
interaction of \emph{two} supersonic flows, such as the situation of a
stellar wind emitted by a star that moves supersonically through the
interstellar medium \citep{van-Buren:1988a, Kobulnicky:2010a,
  van-Marle:2011a, Mackey:2012b, Mackey:2015a}.  In such cases, two
shocks are generally produced, one in each flow.  Sometimes,
especially in heliospheric studies \citep{Zank:1999a, Scherer:2014a},
the term ``bow shock'' is reserved for the shock in the ambient
medium, with the other being called the ``wind shock'' or
``termination shock''.  However, in other contexts such as colliding
wind binaries \citep{Stevens:1992a, Gayley:2009a} such a distinction
is not so useful.  

%% 
%% Examples of astrophysical bow shocks
%%

\begin{figure}
  \centering
  \bigskip
  \includegraphics[width=0.9\linewidth]{figs/bow-terminology}
  \caption{Descriptive terminology for a stellar bow shock.  The apex
    is the closest approach of the bow to the star, while the wings
    are the parts of the bow that curve back past the star.}
  \label{fig:bow-terminology}
\end{figure}
A further class of astrophysical bow shock is driven by highly
collimated, supersonic jets of material, such as the Herbig Haro
objects \citep{Schwartz:1978b, Hartigan:1987a} that are powered by
jets from young stars or protostars.  Additional examples are seen in
planetary nebulae \citep{Phillips:2010a, Meaburn:2013a}, active
galaxies \citep{Wilson:1987a}, and in galaxy clusters
\citep{Markevitch:2002a}.  In the jet-driven case, the term ``working
surface'' is often applied to the entire structure comprising the two
shocks plus the shocked gas in between them, separated by a
\textit{contact discontinuity}.  The working surface may be due to the
interaction of the jet with a relatively quiescent medium, or may be an
``internal working surface'' within the jet that is due to
supersonic temporal variations in the flow velocity
\citep{Raga:1990a}.

In empirical studies the relationship between these theoretical
constructs and the observed emission structures is not always clear.
In such cases the term ``bow shock'' is often used in a more general
sense to refer to the entire arc of emission.  In this paper, we will
concentrate on \textit{stellar bow shocks}, in which the position of
the star can serve as a useful reference point for describing the bow
shape.  The empirical terminology that we will employ is illustrated
in Figure~\ref{fig:bow-terminology}.  The \textit{apex} is the point
of closest approach of the bow to the star, which lies on the
approximate symmetry axis, and the region around the apex is sometimes
referred to as the \textit{head} of the bow.  The \textit{wings} are
the swept-back sides of the bow, which lie in a direction from the
star that is orthogonal to the axis, with the \textit{far wings} being
the wing region farthest from the apex. Finally, the \textit{tail} is
the region near the axis but in the opposite direction from the apex.

\begin{figure}
  % \includegraphics[width=\linewidth]{2winds-scheme}
  \includegraphics[width=\linewidth]{figs/generic-bowshock}
  \caption{Quasi-stationary bow shock structure formed by the
    interaction of two supersonic winds.  Lower-left inset box shows
    the case where the inner wind is anisotropic. The streamlines
    (thin lines) are drawn to be qualitatively realistic: they are
    straight in regions of hypersonic flow, but curved in subsonic
    regions, responding to pressure gradients in the shocked
    shells. Streamline slopes are discontinuous across oblique
    shocks.}
\label{fig:2-winds}
\end{figure}
\newcommand\Mach{\ensuremath{\mathcal{M}}} Figure~\ref{fig:2-winds}
shows an idealized schematic of how a double bow-shock shell is formed
from the interaction of two supersonic streams: an \textit{inner wind}
and an \textit{outer wind}, with the inner wind being the weaker of
the two (in terms of momentum), so that the shell curves back around
the inner source.  The outer wind may be from another star, or may be
a larger scale flow of the interstellar medium, such as the
\textit{champagne flow} produced by the expansion of an \hii{} region
away from a molecular cloud \citep{Tenorio-Tagle:1979a, Shu:2002a,
  Medina:2014a}.  Alternatively, it may be due to the supersonic
motion of the inner source through a relatively static medium, in
which case the outer wind will not be divergent as shown in the figure
but rather plane-parallel.  The thickness of the shocked shells at the
apex depends on the Mach number, \Mach{}, of the flows and the
efficiency of the post-shock cooling.  For sufficiently strong
cooling, the post-shock cooling zone thickness is negligible and the
shock can be considered isothermal.  In this case, the shell thickness
is of order \(\Mach^{-2}\) times the source-apex separation
\citep{Henney:2002a}, which can become very small for high Mach
numbers.  The shell thickness will tend to increase towards the wings,
due to the increasing shock obliqueness, which reduces the
perpendicular Mach number.

\begin{figure}
  \centering
  \includegraphics[width=\linewidth]{figs/bowshock-crw-variables}
  \caption[]{Schematic diagram of cylindrically symmetric two-wind
    interaction problem in the thin-shell limit, following
    \citet{Canto:1996}.}
  \label{fig:crw-schema}
\end{figure}
In the extreme thin-shell limit, the entire bow structure can be
treated as a surface.  The bow radius measured from the inner source
(star) is \(R(\theta, \phi)\), where \(\theta\) is the polar angle, measured from
the star-apex axis, and \(\phi\) is the azimuthal angle, measured around
that axis.  Assuming cylindrical symmetry about the axis, this reduces
to \(R(\theta)\), which is illustrated in Figure~\ref{fig:crw-schema},
following \citet{Canto:1996}.  The separation between the two sources
is \(D\) and the complementary angle, as measured at the position of
the outer source, is \(\theta_1\).  The minimum value of \(R(\theta)\) is the
stagnation radius, \(R_0\), which occurs at the apex (\(\theta = 0\)).  In
a steady state, ram-pressure balance on the axis implies that
\begin{equation}
  \label{eq:stagnation-radius}
  \frac{R_0} {D} = \frac{\beta^{1/2}} {1 + \beta^{1/2}} ,
\end{equation}
where \(\beta\) is the momentum ratio between the two winds.  If the winds
are isotropic, with inner wind mass-loss rate \(\dot{M}_{\w}\) and
terminal velocity \(V_{\w}\), while the outer wind has corresponding
values \(\dot{M}_{\w1}\) and \(V_{\w1}\), then the momentum ratio is
\begin{equation}
  \label{eq:beta-definition}
  \beta = \frac{\dot{M}_{\w} V_{\w}} {\dot{M}_{\w1} V_{\w1}} .
\end{equation}
The case where the outer wind is a parallel stream
\citep{Wilkin:1996a} corresponds to the limit \(\beta \to 0\), in which case
\(D\) is no longer a meaningful parameter.

% \TODO{Bow shocks from pulsars and neutron stars
%   \citep{Cordes:1993a, Brownsberger:2014a}.}


The paper is organized as follows.
%
In \S~\ref{sec:plan-alat-bow} we outline the geometric parameters that
are necessary for describing bow shapes and introduce two
dimensionless ratios: planitude and alatude.
%
In \S~\ref{sec:projection} we derive general results for the
projection of bow shapes on to the plane of the sky.
%
In \S~\ref{sec:conic} we apply the results to the simplest possible
class of geometric bow models: the quadrics of revolution, which
comprise spheroids, paraboloids, and hyperboloids, each of which
occupies a distinct region of the planitude--alatude plane.
%
In \S~\ref{sec:crw-scenario} we consider thin-shell hydrodynamic
models for the parallel-stream case (wilkinoids) and wind-wind case
(cantoids), including extension to an anisotropic inner wind
(ancantoids).  We calculate the location of the models in the
planitude--alatude plane as a function of the inclination of the bow
shock axis to the plane of the sky.
%
In \S~\ref{sec:more-realistic-bow} we test our methods against the
results of more realistic numerical simulations of bow shocks,
including the derivation of the shape parameters from maps of infrared
dust emission.
%
In \S~\ref{sec:obs} we apply our methods to example observations of
proplyd bow shocks in the Orion Nebula, paying close attention to the
systematic uncertainties that arise when our algorithms are applied to
real data.
%
In \S~\ref{sec:conc} we summarise our results and outline how
following papers will apply these ideas to a more extensive set of
observations, models and numerical simulations.


% The bow shocks we consider are originated by a source located at the
% origin, emitting a wind with a mass loss rate of $\dot{M}_{\w}$ and a
% terminal (supersonic) velocity $v_{\w}$. This wind interacts with another
% wind originated by another source located at a distance $D$ from the
% first one.  The mass loss rate of the second source is $\dot{M}_{\w1}$
% and the terminal velocity is $v_{\w1}$. The momentum of the wind of the
% second source is higher than the momentum of the first one, and the
% resultant bow shock is stationary due to pressure balance.

\section{Planitude and alatude of bow shapes}
\label{sec:plan-alat-bow}

\begin{figure}
  \centering
  \includegraphics[width=\linewidth]{figs/characteristic-radii}
  \caption[]{Parameters for characterizing a bow shape.  Bow radius
    from the star, measured parallel (\(R_0\)) and perpendicular
    (\(R_{90}\)) to the symmetry axis, together with radius of
    curvature (\(R_{\C}\)) at apex and asymptotic opening angle
    (\(\theta_\infty\)) of the far wings. }
  \label{fig:characteristic-radii}
\end{figure}

The stagnation radius \(R_0\) describes the linear scale of the bow
shock, but in order to characterize its shape more parameters are
required.  To efficiently capture the diversity of bow shapes, we
propose the parameters shown in Figure~\ref{fig:characteristic-radii}.
The perpendicular radius \(R_{90}\) is the value of \(R(\theta)\) at
\(\theta = 90^\circ\), whereas \(R_{\C}\) is the radius of curvature of the bow
at the apex (\(\theta = 0\)).  For a cylindrically symmetric bow, we show
in Appendix~\ref{sec:radius-curvature} that this is given by
\begin{equation}
  \label{eq:radius-curvature}
  R_{\C} 
  = \frac{R_0^2}{R_0 - R_{\theta\theta,0}} \ , 
\end{equation}
where \(R_{\theta\theta,0}\) is \(d^2 \!R / d\theta^2\) evaluated at \(\theta = 0\).

A fourth parameter is the asymptotic opening angle of the far wings,
\(\theta_\infty\), which is useful in the case that the wings are asymptotically
conical.  However, in many bow shocks the wings tend towards the
asymptotic angle only slowly, making \(\theta_\infty\) difficult to measure,
especially since the emission from the far wings is often weak at
best.  In contrast, the three radii, \(R_0\), \(R_{90}\), and
\(R_{\C}\). are straightforward to determine from observations.  One
simple method to estimate the radius of curvature is to make use of
the Taylor expansion\footnote{%
  This method assumes both that \(R(\theta)\) is even (true for a
  cylindrically symmetric bow) and that the orientation of the axis is
  already known.  Generalization to cases where these assumptions do
  not hold is discussed in Appendix~\ref{app:rcurv-empirical}.} %
of \(R(\theta)\) about the apex (with \(\theta\) in radians):
\begin{equation}
  \label{eq:taylor-R-theta}
  R(\theta) = R_0 + \frac12 R_{\theta\theta,0} \,\theta^2 + \mathcal{O}(\theta^4) \ ,
\end{equation}
so that fitting a polynomial in \(\theta^2\) to \(R(\theta)\) for
\(|\theta| < \Delta\theta \) yields \(R_0\) and \(R_{\theta\theta,0}\) from the first two
coefficients, and hence \(R_{\C}\) from
equation~\eqref{eq:radius-curvature}.  Experience has shown that
\(\Delta\theta = 30^\circ\) and three terms in the polynomial are good choices,
where the third term is used only as a monitor (if the co-efficient of
\(\theta^4\) is not small compared with \(R_0\), then it may indicate a
problem with the fit).

Since we have three radii, we can construct two independent
dimensionless parameters:
\begin{align}
  \label{eq:planitude}
  \text{Planitude} \quad \Pi & \equiv  \frac{R_{\C}} {R_0} \\
  \label{eq:alatude}
  \text{Alatude} \quad \Lambda & \equiv  \frac{R_{90}} {R_0}
\end{align}
and these will be the principal shape parameters that we will use in
the remainder of the paper.  The \textit{planitude}, \(\Pi\), is a
measure of the flatness of the head of the bow around the apex, while
the \textit{alatude}, \(\Lambda\), is a measure of the openness of the bow
wings.  Although ``planitude'' can be found in English dictionaries,
``alatude'' is a new word that we introduce here, derived from the
latin \textit{ala} for ``wing''.

Several previous studies have discussed the relation between
\(R_{90}\) and \(R_0\) as a diagnostic of bow shape (for example
\citealp{Robberto:2005a, Cox:2012a, Meyer:2016a}), but as far as we
know, we are the first to include \(R_{\C}\).  \citet{Robberto:2005a}
\S~4.2 use the ratios \(R_0/D\) and \(R_{90}/D\) in analyzing proplyd
bow shapes in the Trapezium cluster in the center of the Orion Nebula
\citep{Hayward:1994a, Garcia-Arredondo:2001a, Smith:2005a}.  In that
case, the source of the outer wind is known, and so \(D\) is
well-determined (at least, in projection), but for many bow shocks
\(D\) is not known, and is not even defined for the moving-star or
parallel-stream case. \citet{Cox:2012a} \S~4.1 compare the observed
shapes of bow shocks around cool giant stars with an analytic model,
and use \(A\) and \(B\) for the projected values of \(R_0\) and
\(R_{90}\), respectively (see next section for discussion of
projection effects).  \citet{Meyer:2016a} \S~3.2 analyze the
distribution of \(R_0 / R_{90}\) (the reciprocal of our \(\Lambda\)) for
hydrodynamic simulations of bow shocks around runaway OB stars.

% In order to contrast different bow shock models, we derive a set of
% measurable radii. Each model used should predict them and these
% predictions can be compared with observations.
% \begin{itemize}
% \item Radius at axis of symmetry. Denoted as $R_0$.
% \item Radius of Curvature at the axis of symmetry. Denoted as $R_{\C}$
% \item Radius at the perpendicular direction to the symmetry
%   axis. Denoted as $R_{90}$
% \item For open bow shocks, the asymptotic angle. Denoted as
%   $\theta_\infty$
% \end{itemize}



%% 
%% Restriction to cylindrical symmetry
%% 

% For simplicity, the current paper is restricted to cylindrically
% symmetric bow shock shapes. 

%%
%% Effects of instabilities
%%


%%% Local Variables:
%%% mode: latex
%%% TeX-master: "quadrics-bowshock"
%%% End:

\section{Generic bow shock model}
\label{sec:generic-model}

\begin{figure}
\includegraphics[width=\linewidth]{2winds-scheme}
\caption{Schematic representation of the two winds problem. Any point in the shell is located with the coordinates
$(R,\theta)$, and $\theta_1$ is measured from the external wind position}
\label{fig:2-winds}
\end{figure}

The bow shocks we consider are originated by a source located at the origin, emitting a wind with a mass loss rate of $\dot{M}_w$ and a terminal
(supersonic) velocity $v_w$. This wind interacts with another wind originated by another source located at a distance $D$ from the first one. 
The mass loss rate of the second source is $\dot{M}_{w1}$ and the terminal velocity is $v_{w1}$. The momentum of the wind of the second source is
higher than the momentum of the first one, and the resultant bow shock is stationary due to pressure balance. 

\subsection{Characteristic Radii}

In order to contrast different bowshock models, we  derive a set of measurable radii. Each model used should predict them and these predictions can be
compared with observations.

\begin{itemize}
\item Radius at axis of symemtry. Denoted as $R_0$. 
\item Radius of Curvature at the axis of symmetry. Denoted as $R_c$
\item Radius at the  perpendicular direction to the symmetry axis. Denoted as $R_{90}$
\item For open bow shocks, the assymptotic angle. Denoted as $\theta_\infty$
\end{itemize} 

\begin{figure}
\includegraphics[width=\linewidth]{ch-radii_ed}
\caption{Schematic representation of the characteristic radii $R_0$, $R_{90}$ and the radius of curvature at the symmetry axis $R_c$}
\end{figure}



%%% Local Variables:
%%% mode: latex
%%% TeX-master: "proplyd-bowshocks"
%%% End:

\section{Projection onto the plane of the sky}
\label{sec:projection}

In this section we calculate the apparent shape on the plane of the
sky of the limb brightened border of a shock or shell that is
idealized as as arbitrary cylindrically symmetric surface.

%Note: I'm aware that some of this material should be moved to an appendix, but I think it will be a future edition.
\subsection{Frames of reference}


Consider body-frame cartesian coordinates $(x,y,z)$, where \(x\) is
the symmetry axis, and spherical polar coordinates
\((R, \theta, \phi)\), where \(\theta\) is the polar angle and
\(\phi\) the azimuthal angle.  Since the surface is cylindrically
symmetric, it is can be specified as $R = R(\theta)$, so that
cartesian coordinates on the surface are:
\begin{equation}
\left(\begin{array}{c}
x \\ y \\ z
\end{array}
\right) = R(\theta)\left(\begin{array}{c}
\cos\theta \\
\sin\theta\cos\phi \\
\sin\theta\sin\phi
\end{array}\right).
\end{equation} 
Suppose that the viewing direction makes an angle \(i\) with the \(z\)
axis, so that we can define observer-frame coordinates
\((x', y', z')\), which are given by the rotating the body-frame
coordinates:
\begin{equation}
\left(\begin{array}{c}
x' \\ y' \\ z'
\end{array}
\right) = \left(\begin{array}{c}
x\cos i - z\sin i\\
y \\
z\cos i + x\sin i
\end{array}\right).
\label{eq:Trans}
\end{equation} 
In what follows, all quantities in the observer's frame are denoted by
attaching a prime to the equivalent quantity in the body frame.  The
coordinates \(x'\) and \(y'\) are in the plane of the sky, with \(x'\)
being the projected symmetry axis of the surface, while the line of
sight lies along \(-z'\).  The inclination angle \(i\) is defined so
that \(i = 0^\circ\) when the surface is viewed perpendicular to its
axis (\textit{side on}) and \(i = 90^\circ\) when it is viewed along
its axis (\textit{end on}).


\subsection{Tangent line}

The normal and tangent vectors to the shell's border in the shell's frame are given by:
\begin{align}
\hat{t} = \left(\begin{array}{c}
-\cos\alpha \\
\sin\alpha\cos\phi\\
\sin\alpha\sin\phi
\end{array}\right)\\
\hat{n} = \left(\begin{array}{c}
\sin\alpha \\
\cos\alpha\cos\phi \\
\cos\alpha\sin\phi
\end{array}\right)
\end{align}
Where:
\begin{align}
\tan\alpha = -\left.\frac{dy}{dx}\right|_{R(\theta)} \\
\label{eq:tanalpha}
\end{align}
Then we apply the transformation (\ref{eq:Trans}) to the normal and tangent vectors to obtain:
\begin{align}
\hat{n}' &= \left(\begin{array}{c}
(\cos\theta+\omega\sin\theta)\cos i -(\sin\theta-\omega\cos\theta)\sin i \sin\phi\\
(\sin\theta-\omega\cos\theta)\cos\phi \\
(\cos\theta+\omega\sin\theta)\sin i + (\sin\theta-\omega\cos\theta)\sin\phi\cos i
\end{array}\right)\\
\hat{t}' &= \left(\begin{array}{c}
-(\sin\theta-\omega\cos\theta)\cos i - (\cos\theta+\omega\sin\theta)\sin\phi\sin i \\
(\cos\theta+\omega\sin\theta)\cos\phi \\
-(\cos\theta+\omega\sin\theta)\sin i + (\sin\theta-\omega\cos\theta)\sin\phi\cos i
\end{array}\right)
\end{align}
Where $\omega = \frac{1}{R}\frac{dR}{d\theta}$

In the thin shell case, the limb brightened border of the shell is such that $\hat{n}'\cdot \hat{z}'$. 
The values for $\phi$ that satisfy this relation for each inclination $i$ are given by:
\begin{equation}
\sin\phi_t = \tan i\tan\alpha = \tan i \frac{1+\omega\tan\theta}{\omega-\tan\theta}
\label{eq:tanphi}
\end{equation}
With this, the coordinates of the limb brightened shell are given by:
\begin{equation}
\left(\begin{array}{c}
x'_t \\ y'_t \\ z'_t
\end{array}\right)= R(\theta)\left(\begin{array}{c}
\cos\theta\cos i - \sin\theta\sin\phi_t \sin i \\
\sin\theta(1-\sin^2\phi_t)^{1/2} \\
\cos\theta\sin i +\sin\theta\sin\phi_t\cos i
\end{array}\right)
\label{eq:tangential}
\end{equation} 

It is important to note that the equation (\ref{eq:tanphi}) does not have a solution for arbitrary values for $\theta$ and $i$, since
it's required that $|\sin\phi_t|<1$. If $i\neq 0$, then, the  allowed values for $\theta$ are such that $\theta > \theta_\parallel$, where
$\theta_\parallel$ is given implicitly by:
\begin{align}
\tan\theta_\parallel = \frac{|\tan i| + \omega(\theta_\parallel)}{1-\omega(\theta_\parallel) |\tan i|}
\label{eq:thetapar}
\end{align}

\subsection{Parallel and perpendicular projected shell radii}

Considering further applications to bow shocks, we will consider open shells. In order to compare the shell shape given by $R(\theta)$ with observations,
it is convenient to define the following apparent radii in the observer frame: $R'_\parallel$ and $R'_\perp$. These are projected distances of the shell tangent line
from the origin. The first is measured in the direction of the symetry axis, and the second in a perpendicular direction. More concretely $R'_\parallel = x'_t(y'_t=0)$
and $R'_\perp = y'_t(x'_t=0)$. From equations (\ref{eq:tanphi}) and (\ref{eq:tangential}) we find that:
\begin{align}
R'_\parallel = R(\theta_\parallel)\cos(\theta + i) \label{eq:Rpar} 
\end{align}
Where $\theta_\parallel$ is the solution of equation (\ref{eq:thetapar}), and
\begin{align}
R'_\perp = R(\theta_\perp)\sin\theta_\perp\left(1-\sin^2(\phi_t(\theta_\perp))\right)^{1/2}
\end{align}
Where $\theta_\perp$ is the solution of the next implicit equation:
\begin{align}
\cot\theta_\perp = \frac{1-\left(1+\omega(\theta_\perp)^2\sin^22i\right)^{1/2}}{2\omega(\theta_\perp\cos^2 i)}
\end{align}


%%% Local Variables:
%%% mode: latex
%%% TeX-master: "proplyd-bowshocks"
%%% End:


\section{Conic section approximation to bow shock shapes}
\label{sec:conic}

%%% Local Variables:
%%% mode: latex
%%% TeX-master: "proplyd-bowshocks"
%%% End:

In this section we will analyze the case where the resultant shape is a conic curve (circle, ellipse, parabola or hyperbola).
These curves are mathematical simple to model and give us a good reference to understand the effects of the projection effects
described in the last section on other bowshocks. Instead of the excentricity, we utilize the parameter $\theta_c$ to characterize the different curves, where
$\tan\theta_c = \frac{b}{a}$,  $b$ and $a$ are the typical parameters of conics. A positive value for $\theta_c$ indicates thethe given curve is a closed one, i.e
an ellipse, while a negative value indicates that is an hyperbola. %Insert figures if neccesary  

Due to the similitudes between the parametrization of the ellipse and the hyperbola, we can do the following parametrization:

\begin{align}
x = au_c(t)-x_0 \\ 
y = bv_c(t)
\end{align}

where:
\begin{align}
u_c(t) = \left\lbrace \begin{array}{c}
\cos t ~\mathrm{if~ellipse} \\
\cosh t ~mathrm{if~hyperbola}
\end{array}\right
\end{align}\\
v_c(t) = \left\lbrace \begin{array}{c}
\sin t ~\mathrm{if~ellipse} \\
\sinh t ~mathrm{if~hyperbola}
\end{array}\right \\
-\pi < t < \pi \\
R_0 = a - x_0 
\end{align}

\subsection{Projection onto the plane of sky} 

Once the parametrization is done, we can find the apparent shape of the shell in the observer's frame, following the procedure explained in section \ref{sec:projection}.

First of all, the intrinsic 3D shape of the shell is given by:

\begin{align}
x = au_c(t)-x_0 \\ 
y = bv_c(t)\cos\phi \\
z =  bv_c(t)\sin\phi
\end{align}

The azimutal angle where the line of sight is tangent to the shell is given by equation (\ref{eq:tanphi}) and (\ref{eq:tanalpha}), then:

\begin{align}
\tan\phi &= \frac{b}{a}w_c(t) 
\end{align}
where:

\begin{align}
w_c(t) = \left\lbrace \begin{array}{c}
\cot t ~\mathrm{if~ellipse} \\
-\coth t ~mathrm{if~hyperbola}
\end{array}\right
\end{align}
\subsection{Characteristic radii}


\newcommand\thC{\(\theta^1\)\,Ori~C}
\defcitealias{Canto:1996}{CRW}
\newcommand\CRW{\citetalias{Canto:1996}}


\section{Thin-shell bow shock models}
\label{sec:crw-scenario}

More physically realistic examples of bow shapes are provided by
steady-state hydrodynamic models for the interaction of hypersonic
flows in the thin-shell limit.  The classic examples are the solutions
for the wind--parallel stream and wind--wind problems (see
\S~\ref{sec:intro}) of \citet[][hereafter \CRW{}]{Canto:1996}, where
it is assumed that the two shocks are highly radiative and that the
post-shock flows are perfectly mixed to form a single shell of
negligible thickness. In this approximation, the shape of the shell is
found algebraically by \CRW{} from conservation of linear and angular
momentum, following an approach first outlined in
\citet{Wilkin:1996a}.  For the wind--stream case, the resulting bow
shape was dubbed \textit{wilkinoid} by \citet{Cox:2012a} and has the
form:
\begin{equation}
  \label{eq:wilkinoid-R-theta}
  R(\theta) = R_0\csc\theta\left( 3(1-\theta\cot\theta) \right)^{1/2} \ .
\end{equation}

For the wind--wind case, a family of solutions are found that depend on
the value of \(\beta\), the wind momentum ratio,\footnote{%
  By always placing the weaker of the two winds at the origin, it is
  only necessary to consider \(\beta \le 1\).  } %
see Figure~\ref{fig:crw-schema},
equations~(\ref{eq:stagnation-radius}, \ref{eq:beta-definition}), and
surrounding discussion in \S~\ref{sec:intro}.  We propose that these
shapes be called \textit{cantoids}.  The exact solution for the
cantoid shapes (eqs.~[23, 24] of \CRW{}) is only obtainable in
implicit form, but an approximate explicit solution (eq.~[26] of
\CRW{}) is very accurate for \(\beta \le 0.1\).  The wilkinoid shape
corresponds to the \(\beta \to 0\) limit of the cantoids.  Note that \CRW{}
employ cylindrical polar coordinates, \(z\) and \(r\), see our
Figure~\ref{fig:crw-schema}, and we follow this usage for the
thin-shell models discussed in this section.  \CRW{}'s \(z\) axis
corresponds to the cartesian \(x\) axis used in
sections~\ref{sec:projection} and \ref{sec:conic} of the current
paper, while the \(r\) axis corresponds to \(y\) when \(\phi = 0\).

A generalization of the cantoids to the case of an
anisotropic\footnote{Note that the wind anisotropy axis must be
  aligned with the star--star axis to maintain cylindrical symmetry.}
inner wind is developed next, giving rise
to what we call \textit{ancantoids}, which depend on an anisotropy
index, \(k\), in addition to \(\beta\).  

\subsection{Bow shocks from anisotropic wind--wind interactions}
\label{sec:ancantoid}
\begin{figure}
  \centering
  \includegraphics[width=\linewidth]{figs/anisotropic-arrows}
  \caption[]{Schematic diagram of wind flow patterns in isotropic and
    non-isotropic cases for different values of the anisotropy index,
    \(k\).  Arrow length represents the wind momentum loss rate per
    solid angle.}
  \label{fig:anisotropic-arrows}
\end{figure}


We wish to generalize the results of \citet[\CRW{}]{Canto:1996} to the
case where the inner wind is no longer isotropic, but instead has a
density that falls off with angle, \(\theta\), away from the symmetry axis.
Specifically, at some fiducial spherical radius, \(R_0\), from the
origin, the wind mass density is given by
\begin{equation}
  \label{eq:ancantoid-density}
  \rho(R_0, \theta) =
  \begin{cases}
    \rho_0 \cos^k \theta & \text{for \(\theta \le 90^\circ\)} \\
    0 & \text{for \(\theta > 90^\circ\)}
  \end{cases}
  \ ,
\end{equation}
where \(\rho_0\) is the density on the symmetry axis and \(k \ge 0\) is an
\textit{anisotropy index}.  The wind velocity is still assumed to be
constant and the wind streamlines to be radial, so the radial
variation of density at each angle is
\(\rho(R, \theta) = \rho(R_0, \theta)\, (R/R_0)^{-2}\) and the wind mass loss rate and
momentum loss rate per solid angle both have the same \(\cos^k\theta\)
dependence as the density.  Examples are shown in
Figure~\ref{fig:anisotropic-arrows} for a variety of different values
of \(k\).  As \(k\) increases, the wind becomes increasingly jet-like.

Our primary motivation for considering such an anisotropic wind is the
case of the Orion Nebula proplyds and their interaction with the
stellar wind of the massive star \thC{}
\citep{Garcia-Arredondo:2001a}.  The inner ``wind'' in this case is
the transonic photoevaporation flow away from a roughly hemispherical
ionization front, where photoionization equilibrium, together with
monodirectional illumination of the front, implies that the ionized
hydrogen density, \(n\), satisfies \(n^2 \propto \cos \theta\), which is
equivalent to \(k = 0.5\) in equation~\eqref{eq:ancantoid-density}.
Since the primary source of ionizing photons is the same star that is
the source of the outer wind, it is natural that the inner wind's axis
should be aligned with the star--star axis in this case.  For other
potential causes of wind anisotropy (for instance, bipolar flow from
an accretion disk), there is no particular reason for the axes to be
aligned, so cylindrical symmetry would be broken.  Nevertheless, we
calculate results for general \(k\) with aligned axes, so as to
provide a richer variety of cylindrically symmetric bow shock shapes
than are seen in the cantoids.

\begin{figure}
\includegraphics[width=\linewidth]{figs/ancantoid-shape}
\caption{Bow shock shapes for interacting winds in the thin-shell
  approximation: cantoids and ancantoids. Coordinates are normalized
  by $D$, the distance between the two wind sources, which are
  indicated by black dots on the axis.  The weaker source is at
  \((0.0, 0.0)\) and the stronger source is at \((1.0, 0.0)\).
  Results are shown for different values of the wind momentum ratio,
  \(\beta\) (different line widths), and for the case where the weaker
  wind is isotropic (black lines) or anisotropic (colored lines).}
\label{fig:r-beta}
\end{figure}


The general solution for the bow shock shape, \(R(\theta)\), in the \CRW{}
formalism is
\begin{equation}
  R(\theta) = \frac {\dot{J}_{\w} + \dot{J}_{\w{}1}}
  {\left(\dot{\Pi}_{\w{}r}+\dot{\Pi}_{\w{}r1}\right)\cos\theta
    - \left(\dot{\Pi}_{\w{}z}+\dot{\Pi}_{\w{}z1}\right)\sin\theta}
  \label{eq:Rmom}
\end{equation}
where \(\dot{\Pi}_{\w{}r}\), \(\dot{\Pi}_{\w{}z}\), \(\dot{J}_{\w}\) are
the accumulated linear radial momentum, linear axial momentum, and
angular momentum, respectively, due to the inner wind emitted between
the axis and \(\theta\). The equivalent quantities for the outer wind have
subscripts appended with ``1''.  The inner wind momenta for our
anisotropic case (replacing \CRW{}'s eqs.~[9, 10]) are:
\begin{gather}
  \label{eq:ancantoid-momenta}
  \begin{aligned}
    \dot{\Pi}_{\w{}z} &= \frac{k + 1}{2(k+2)}\, \dot{M}_{\w}^0 V_{\w}
    \max\left[\bigl(1- \cos^{k+2} \theta\bigr), 1 \right] \\
    \dot{\Pi}_{\w{}r} &= (k + 1)\, \dot{M}^0_{\w} V_{\w}\, I_k (\theta) 
  \end{aligned}
\end{gather}
where
\begin{equation}
  \label{eq:ancantoid-mass-loss}
  \dot{M}^0_{\w} = \frac{2 \pi} {k + 1} r_0^2 \rho_0 V_{\w}
\end{equation}
is the total mass-loss rate of the inner wind. The integral
\begin{equation}
  \label{eq:ancantoid-I-integral}
  I_k(\theta) = \int^{\max(\theta, \pi/2)}_0 \cos^k \theta \sin^2\theta \,d\theta 
\end{equation}
has an analytic solution in terms of the hypergeometric function,
\({}_2 F_1(-\tfrac12; \tfrac{1+k}2; \tfrac{3+k}2; \cos^2 \theta)\), but is
more straightforwardly calculated by numerical quadrature.  The
angular momentum of the inner wind about the origin is
\(\dot{J}_{\w} = 0\) because it is purely radial.  The outer wind
momenta are unchanged from the \CRW{} case, but are given here for
completeness:
\begin{gather}
  \label{eq:ancantoid-momenta-outer}
  \begin{aligned}
    \dot{\Pi}_{\w{}z1} & =
    -\frac{\dot{M}^0_{\w{}1}V_{\w{}1}}{4}\sin^2\theta_1\\
    \dot{\Pi}_{\w{}r1} & =
    \frac{\dot{M}^0_{\w{}1}V_{\w{}1}}{4}\left(\theta_1-\sin\theta_1\cos\theta_1\right)\\
    \dot{J}_{\w{}1} & =
    \frac{\dot{M}^0_{\w{}1}V_{\w{}1}}{4}\left(\theta_1-\sin\theta_1\cos\theta_1\right)D \ .
  \end{aligned} 
\end{gather}

\begin{figure}
  \centering
  \includegraphics[width=\linewidth]{figs/ancantoid-Pi-lambda-true}
  \caption{True shapes of cantoids and ancantoids in the
    \(\Pi\)--\(\Lambda\) plane, calculated according to results of
    App.~\ref{sec:thin-shell-shapes}.  For each line, \(\beta\) varies
    over the range \([0, 1]\) from lower left to upper right (although
    the black and red lines are truncated on the upper right), and
    line colors correspond to different anisotropy indices, matching
    those used in Fig.~\ref{fig:r-beta}. Circle symbols mark
    particular \(\beta\) values: \(0, 0.01, 0.1\), from largest to
    smallest circle.  Square symbols mark \(\beta = 0.5\), but with
    \(\Lambda\) calculated exactly, instead of using the approximation of
    equation~\eqref{eq:Lambda-approx}.  The white plus symbol marks
    the result for the wilkinoid:
    \((\Pi, \Lambda) = (\frac53, \sqrt{3})\).  Background shading indicates
    the domains of different quadric classes: hyperboloids (white),
    prolate spheroids (dark gray), and oblate spheroids (light gray).}
  \label{fig:ancantoid-Pi-lambda-true}
\end{figure}

We define \(\beta\) in this case as the momentum ratio \emph{on the symmetry axis}, which means that 
\begin{equation}
  \label{eq:ancantoid-momentum-ratio}
  \dot{M}^0_{\w{}1}V_{\w{}1} = 2 (k + 1)\, \beta\, \dot{M}^0_{\w} V_{\w} \ . 
\end{equation}
Substituting
equations~(\ref{eq:ancantoid-momenta}--\ref{eq:ancantoid-momentum-ratio})
into equation~\eqref{eq:Rmom} and making use of the geometric relation
between the interior angles of the triangle shown in
Figure~\ref{fig:crw-schema}:
\begin{equation}
  \label{eq:crw-angles}
  R \sin(\theta + \theta_1) = D \sin \theta_1 \ , 
\end{equation}
yields
\begin{equation}
  \label{eq:ancantoid-theta-theta1-implicit}
  \theta_1 \cot \theta_1 = 1 +
  2 \beta \left(
    I_k(\theta) \cot \theta
    - \frac{1 - \cos^{k+2} \theta} {k + 2} \right)   \ , 
\end{equation}
which is the generalization of \CRW{}'s equation~(24) to the
anisotropic case.  Equation~\eqref{eq:ancantoid-theta-theta1-implicit}
is solved numerically to give \(\theta_1(\theta)\), which is then combined with
equations~(\ref{eq:crw-angles}) and (\ref{eq:stagnation-radius}) to
give the dimensionless bow shape, \(R(\theta; \beta, k)/R_0\), where we now
explicitly indicate the dependence of the solution on two parameters:
axial momentum ratio, \(\beta\), and anisotropy index, \(k\).  We refer to
the resultant bow shapes as \textit{ancantoids}.



\begin{figure}
  \centering
  \includegraphics[width=\linewidth]{figs/ancantoid-angles}
  \caption{Equivalent quadric angles, \(\theta_{\Q}\), for ancantoids and
    cantoids.  Solid lines show values of \(\theta_{\Q}\) calculated from
    \((\Pi, \Lambda)\), which is representative of the shape of the head,
    while dashed lines show \(\theta_{\Q}\) calculated from
    \(\theta_\infty\), which is representative of the tail.  Dot-dashed line
    shows the result for cantoids, which differ from the \(k=0\)
    ancantoids in \(\theta_\infty\), but not in \((\Pi, \Lambda)\). Gray shading and
    line colors have the same meaning as in
    Fig.~\ref{fig:ancantoid-Pi-lambda-true}. }
  \label{fig:ancantoid-angles}
\end{figure}



\begin{figure}
  \centering
  \includegraphics[width=\linewidth]{figs/test_xyprime}
  \caption{Apparent bow shapes as a function of inclination angle for
    isotropic thin shell models. (a)~Confocal paraboloid for
    comparison (shape independent of inclination).
    (b)~Wilkinoid. (c)~Cantoid, \(\beta = 0.001\). (d)~Cantoid,
    \(\beta = 0.01\). }
  \label{fig:xyprime}
\end{figure}

\begin{figure}
  \centering
  \includegraphics[width=\linewidth]{figs/test_xyprime_ancantoid}
  \caption{Further apparent bow shapes as a function of inclination
    angle for anisotropic thin shell models (ancantoids).
    (a)~\(\beta = 0.001\), \(k = 0.5\); (b) (a)~\(\beta = 0.1\),
    \(k = 0.5\); (c)~\(\beta = 0.001\) \(k = 3\); (d)
    \(\beta = 0.1\) \(k = 3\).}
  \label{fig:xyprime-ancantoid}
\end{figure}


\subsection{True shapes of cantoids and ancantoids}
\label{sec:true-cantoids-ancantoids}

The shapes of the ancantoid bow shocks are shown in
Figure~\ref{fig:r-beta} for three different values of \(\beta\), and are
compared with the \CRW{} results for cantoids (dashed curves).  The
location of these shapes in the planitude--alatude plane is shown in
Figure~\ref{fig:ancantoid-Pi-lambda-true}, where the gray background
shading indicates the zones of different quadric classes, as in
\S~\ref{sec:conic}, Figures~\ref{fig:quadric-projection-continued}
and~\ref{fig:projected-R90-Rc-snapshots}.  Values of \(\Pi\) and
\(\Lambda\) are calculated via the analytic expressions derived in
Appendix~\ref{sec:ancantoid-planitude} and
\ref{sec:ancantoid-alatude}, respectively, which are only approximate
in the case of \(\Lambda\).  However, the filled square symbols show the
exact results for \(\beta = 0.5\), which can be seen to lie extremely close
to the approximate results, even for the worst case of \(k = 0\). The
leading term in the relative error of
equation~\eqref{eq:Lambda-approx} scales as \((\beta / (k + 2))^2\), so
the approximation is even better for smaller \(\beta\) and larger \(k\).

It is apparent from Figure~\ref{fig:r-beta} that the \(k=0\) ancantoid
is identical to the cantoid for \(\theta \le 90^\circ\) (\(z > 0\), to the right
of vertical dotted line in Fig.~\ref{fig:r-beta}), but is slightly
more swept back in the far wings.\footnote{%
  \label{fn:discontinuity}
  Due to the discontinuity in the inner wind density at
  \(\theta = 90^\circ\) (see Fig.~\ref{fig:anisotropic-arrows}), there is a
  discontinuity in the second derivative of the bow shape.} %
Since the true planitude and alatude depend on \(R(\theta)\) only in the
range \(\theta = [0, 90^\circ]\), the cantoid and the \(k = 0\) ancantoid
behave identically in Figure~\ref{fig:ancantoid-Pi-lambda-true}.
There is a general tendency for the bows to be flatter and more open
with increasing \(\beta\) and decreasing \(k\), with the cantoid being
most open at a given \(\beta\).  All the models cluster close to the
diagonal \(\Lambda \simeq \Pi\) in the planitude--alatude plane, but with a tendency
for \(\Lambda > \Pi\) at higher anisotropy.  There is therefore a degeneracy
between \(\beta\) and \(k\) for higher values of \(\beta\).  The wilkinoid
shape, which corresponds to the \(\beta \to 0\) limit of the cantoids, is
marked by a white plus symbol in
Figure~\ref{fig:ancantoid-Pi-lambda-true}, and lies in the prolate
spheroid region of the plane.  Cantoids lie either in the prolate
spheroid or hyperboloid regions, according to whether \(\beta\) is less
than or greater than about \(0.01\).  For ancantoids of increasing
\(k\), this dividing point moves to higher values of \(\beta\), until
almost the entire range of models with \(k = 8\) are within the
prolate spheroid zone.

However, the true planitude and alatude, which are what would be
observed for a side-on viewing angle (\(i = 0\)), are not at all
sensitive to the behavior of the far wings of the bow shock, which has
a rather different implication as to which variety of quadric best
approximates each shape.  We illustrate this is
Figure~\ref{fig:ancantoid-angles}, which shows two different ways of
estimating the quadric angle, \(\theta_{\Q}\) (see \S~\ref{sec:conic}).
The first is from \((\Pi, \Lambda)\), as in
Figure~\ref{fig:ancantoid-Pi-lambda-true}:
\newcommand\head{^{\text{head}}}
\newcommand\tail{^{\text{tail}}}
\begin{equation}
  \label{eq:thetaQ-head}
  \theta_{\Q}\head =
  \sgn{\bigl(2 \Pi - \Lambda^2\bigr)} \;
  \tan^{-1} \Abs{2\Pi - \Lambda^2}^{1/2} \ ,
\end{equation}
which follows from equations~\eqref{eq:Tq}, \eqref{eq:thetaQ}, and
\eqref{eq:Tq-from-Pi-Lambda}.  The second is from the asymptotic
opening angle of the wings, \(\theta_\infty\) (Fig.~\ref{fig:characteristic-radii}):
\begin{equation}
  \label{eq:thetaQ-tail}
  \theta_{\Q}\tail = \theta_\infty - 180^\circ \ , 
\end{equation}
where \(\theta_\infty\) is calculated from
equation~\eqref{eq:ancantoid-theta-inf} for ancantoids, or
\eqref{eq:cantoid-theta-inf} for cantoids.  If the bow shock shape
were truly a quadric, then these two definitions would agree.
However, as seen in Figure~\ref{fig:ancantoid-angles}, this is not the
case for the cantoids and ancantoids.  While
\(\smash[b]{\theta_{\Q}\head}\) generally corresponds to a prolate spheroid
(except for the largest values of \(\beta\)),
\(\smash[b]{\theta_{\Q}\tail}\) always corresponds to a hyperbola.  This
tension between the shape of the head and the shape of the far wings
has important implications for the projected shapes (as we will see in
the next section), since the far wings influence the projected
planitude and alatude when the inclination is large.


% \subsection{Characteristic Radii}
% $R_0$ is obtained directly from equation (27) of \CRW{} as the distance from the inner source where the RAM pressure of the interacting winds is in equilibrium.
% %Here goes a little introduction
% For the rest of the radii we need a relation between $\theta$ and $\theta_1$ as follows:


% \begin{align}
% \theta_1\cot\theta_1 -1 = 2\beta I_k(\theta) \cot\theta - \frac{2\beta}{k+2}\left(1-\cos^{k+2}\theta\right)
% \label{eq:th1th}
% \end{align}

% Equation (\ref{eq:th1th}) is reduced to equation (24) of \CRW{} when $k=0$.
% We can obtain $R_{90}$ by following the process shown in appendix \ref{app:r90-analytic},
% which lead us to a solution for $B \equiv \frac{R_{90}}{R_0}$:

% \begin{align}
% \tilde{R}_{90} = \frac{\sqrt(3\xi)\left(1+\beta^{1/2}\right)}{(1-\xi\beta)\left(1+\frac{1}{5}\xi\beta\right)^{1/2}}
% \label{eq:B}
% \end{align}

% Now, the solution for $R_c$ is explained in appendix \ref{app:rc-analytic},
% which lead us to derive the radius of curvature at the symmetry axis:

% \begin{align}
% R_c &= R_0\left(1-2\gamma\right)^{-1} \label{eq:Rcurv} \\
% \mathrm{where:~} & \gamma = \frac{C_{k\beta}}{1+\beta^{1/2}}+\frac{1}{6}(1-2\beta^{1/2})
% \end{align}

% Finally, using equations (\ref{eq:Rcurv}) and (\ref{eq:B}) we can estimate the parameter of
% conic curves $\theta_c$ as a function of $(\beta,\xi)$ using equation (\ref{eq:Thc})

% \begin{align}
% \tan^2\theta_c &= \left| \frac{3\xi\left(1+\beta^{1/2}\right)^2}{\left(1-\xi\beta\right)^2\left(1+\frac{1}{5}\xi\beta\right)}-\frac{2}{\left(1-2\gamma\right)}\right| 
% \label{eq:thc-CRW}
% \end{align}

% \begin{figure}
% \begin{tabular}{c}
% \includegraphics[width=\linewidth]{figs/AB-beta-log} \\
% \includegraphics[width=\linewidth]{figs/thc-beta-log}
% \end{tabular}
% \caption{Top: Characteristic radii, $\tilde{R}_c = R_c/R_0$ (solid lines) and $\tilde{R}_{90}
%   = R_{90}/R_0$ (dashed lines)
%   vs $\beta$, calculated from quadric fits to the generalized CRW
%   solutions with varying degrees of isotropy $\xi$.  Bottom: Conic
%   angle $\theta_c$ vs $\beta$ for the bow 
%   shock head (solid lines) and the bow shock tail (dashed lines).}
% \label{fig:rad-beta}
% \end{figure}


% % Observationally, we can measure the projected radii. In order to estimate the model parameters is neccesary to measure at least two of the mentioned radii,
% %being $R_0$ the
% %easiest to measure. Therefore, we may compare both $R_c$ and $R_{90}$ against $R_0$ as shown in figure (\ref{fig:prop-shell-rad}). 

% With this we can use the results of section \ref{sec:conic} to estimate the projected conic shapes for bow shocks with different winds 
% momenta and different density distributions. Figure \ref{fig:rad-beta} shows equations ( \ref{eq:B}), (\ref{eq:Rcurv}) and (\ref{eq:thc-CRW}) for different anisotropy indexes. 

% \subsection{Special Case: Isotropic Wind/Parallel Flow Interaction Problem}

% This problem has already been solved in \citet{Wilkin:1996a} and \CRW{}, which have an explicit form for the shell shape, given by:

% \begin{align}
%   R(\theta) = R_0\csc\theta\left[3(1-\theta\cot\theta)\right]^{1/2} \label{eq:R-Wilkin}
% \end{align}
% Where:
% \begin{align}
%   R_0 = \left(\frac{\dot{M}_wv_w}{4\pi\rho_a v_a^2}\right)^{1/2}
% \end{align}
% And $\rho_a$ and $v_a$ being the density and the velocity of the Parallel wind, respectively.
% The characteristic radii are given by:
% \begin{align}
%   \tilde{R}_{90} &= \sqrt{3} \\
%   \tilde{R}_c &= \frac{5}{3}
% \end{align}
% The derivation of these values is developed in appendix \ref{app:ch-rad-Wilkin}

% \subsection{Fits to the tail}
% \label{sec:fits-tail}

% In most cases, the quadric fit to the head of the bowshock does a very poor job of representing the ``wings'' or ``tail''.  The bowshock tail in all cases is asymptotically hyperbolic, with 

% We therefore 
% We carry out three-level nested fits to determine the 

% For the hyperbola ``center''
% \begin{multline}
%   \label{eq:tail-analytic-x0}
%   x_{0,\mathrm{t}} = 0.7 \beta^{-0.55} \biggl[
%     C_3 \bigl(\log_{10}\beta\bigr)^3 + C_2 \bigl(\log_{10}\beta\bigr)^2 
%   \\ + C_1 \log_{10}\beta + C_0
%   \biggr]
% \end{multline}
% \begin{equation}
%   \label{eq:tail-analytic-x0-minus-a}
%   (x_{0,\mathrm{t}} - a_{\mathrm{t}}) = D_2 (\log_{10}\beta)^2 + D_1 \log_{10}\beta + D_0
% \end{equation}
% where
% \begin{alignat}{2}
%   \label{eq:tail-analytic-coeffs-c}
%   C_k &= c_{2,k} \xi^2 + c_{1,k} \xi + c_{0,k} &\quad \text{for\ } k &= \{0, 1, 2, 3\} \\
%   \label{eq:tail-analytic-coeffs-d}
%   D_k &= d_{2,k} \xi^2 + d_{1,k} \xi + d_{0,k} &\quad \text{for\ } k &= \{0, 1, 2\}
% \end{alignat}


% % \newcommand\iso{\ensuremath{^{\mathrm{iso}}}}

% % \begin{table}
% %   \caption{Coefficients for hyperbola fit to bowshock tails}
% %   \label{tab:tail-fit-coeffs}
% %   \renewcommand\arraystretch{1.2}
% %   \setlength\tabcolsep{0.5\tabcolsep}
% %   \begin{tabular}{@{}llll@{}}
% %     \toprule
% %     Equation~(\ref{eq:tail-analytic-x0}) & 
% %     \multicolumn{3}{l}{
% %     Equation~(\ref{eq:tail-analytic-coeffs-c}) \dotfill
% %     } \\ \midrule
% %     \( C\iso_0 = +1.3195     \)    
% %     & \( c_{0,0} = +2.0758   \)  
% %     & \( c_{1,0} = -0.2309   \)  
% %     & \( c_{2,0} = -0.2532   \)\\
% %       \( C\iso_1 = +0.4229     \)    
% %     & \( c_{0,1} = +0.9571   \)  
% %     & \( c_{1,1} = -0.1530   \)  
% %     & \( c_{2,1} = -0.2487   \)\\
% %       \( C\iso_2 = +0.1092     \)    
% %     & \( c_{0,2} = +0.2528   \)  
% %     & \( c_{1,2} = -0.0360   \)  
% %     & \( c_{2,2} = -0.0794   \)\\
% %       \( C\iso_3 = +0.0051     \)    
% %     & \( c_{0,3} = +0.0171   \)  
% %     & \( c_{1,3} = -0.0010   \)  
% %     & \( c_{2,3} = -0.0095   \)\\ \midrule
% %     Equation~(\ref{eq:tail-analytic-x0-minus-a}) &
% %     \multicolumn{3}{l}{
% %     Equation~(\ref{eq:tail-analytic-coeffs-d}) \dotfill
% %     } \\ \midrule
% %     \( D\iso_0 = +0.7962   \)    
% %     & \( d_{0,0} = +0.8516 \)  
% %     & \( d_{1,0} = -0.0907 \)  
% %     & \( d_{2,0} = -0.2002 \)\\
% %       \( D\iso_1 = -0.2363   \)    
% %     & \( d_{0,1} = -0.7620 \)  
% %     & \( d_{1,1} = +0.1411 \)  
% %     & \( d_{2,1} = -0.0295 \)\\
% %       \( D\iso_2 = -0.0126   \)    
% %     & \( d_{0,2} = -0.0683 \)  
% %     & \( d_{1,2} = +0.0390 \)  
% %     & \( d_{2,2} = -0.0236 \)\\
% %     \bottomrule
% %   \end{tabular}
% % \end{table}




% % \begin{figure*}
% %   \centering
% %   \includegraphics[width=0.8\textwidth]{figs/conic-head-tail-analytic}
% %   \caption{Double quadric fits to thin shell
% %     solutions.}
% %   \label{fig:head-tail}
% % \end{figure*}

% % \begin{figure}
% %   \centering
% %   \includegraphics[width=\linewidth]{figs/two-quadric-th90-vs-i}
% %   \caption{Variation with inclination of the body-frame polar angle
% %     \(\theta_{90}\) that corresponds to the projected perpendicular radius
% %     \(R’_{90}\).  Line colors and thicknesses represent different
% %     quadrics of revolution, as in
% %     Fig.~\ref{fig:projected-R90-Rc-snapshots}.  This quantity is
% %     important because the quadric fits to the bow shock shape must be
% %     accurate for \(\theta < \theta_{90}\) in order to give a reliable estimate
% %     for \(R’_{90}\).}
% %   \label{fig:projected-th90}
% % \end{figure}



% \begin{figure*}
%   \centering
%   \includegraphics[width=0.8\textwidth]{figs/conic-head-tail-residuals}
%   \caption[]{Residuals of double quadric fits to thin shell solutions.
%     Residuals are expressed as the fractional error in the
%     complementary angle \(\theta_1\) (see Fig.~\ref{fig:crw-schema}) for
%     the same parameters as are shown in Fig.~\ref{fig:head-tail}.}
%   \label{fig:head-tail-residuals}
% \end{figure*}

% \subsection{Projection effects in the thin shell model}
% \label{sec:proj-effects-thin}


\begin{figure}
  \centering
  \setkeys{Gin}{width=\linewidth}
  \begin{tabular}{@{}l@{}}
    (a) \\
    \includegraphics{figs/ancantoid-R90-vs-Rc-a} \\
    (b) \\
    \includegraphics{figs/ancantoid-R90-vs-Rc-b}
  \end{tabular}
  \caption{Apparent projected shapes of wilkinoid, cantoids and
    ancantoids in the \(\Pi'\)--\(\Lambda'\) plane.  Colored symbols indicate
    the \(\abs{i} = 0\) position for \(\beta = 0.001\), \(0.003\),
    \(0.01\), \(0.03\), \(0.1\), \(0.3\).  Thin lines show the
    inclination-dependent tracks of each model, with tick marks along
    each track for 20 equal-spaced values of \(\abs{\sin i}\).  Gray
    shaded regions are as in
    Fig.~\ref{fig:quadric-projection-continued}a.  The wilkinoid track
    is shown in white. (a)~Isotropic wind model (cantoid) and
    proplyd-like model (ancantoid, \(k = 0.5\)). (b)~Hemispheric wind
    model (ancantoid, \(k = 0\)) and jet-like model (ancantoid,
    \(k = 3\)).}
  \label{fig:thin-shell-R90-Rc}
\end{figure}

\subsection{Apparent shapes of projected cantoids and ancantoids}
\label{sec:proj-shap-cant}

Figures~\ref{fig:xyprime} and~\ref{fig:xyprime-ancantoid} show the
apparent bow shapes of various thin shell models (wilkinoid, cantoids,
ancantoids) for different inclination angles \(\abs{i}\).  For
comparison, Figure~\ref{fig:xyprime}a shows the confocal paraboloid,
whose apparent shape is independent of inclination (see
Appendix~\ref{app:parabola}).  The wilkinoid (Fig.~\ref{fig:xyprime}b)
shows only subtle changes, with the wings becoming slightly more swept
back as the inclination increases.  The cantoids
(Fig.~\ref{fig:xyprime}c and d) behave in the opposite way, with the
wings becoming markedly more open once \(\abs{i}\) exceeds
\(60^\circ\) (for \(\beta = 0.001\)), or \(45^\circ\) (for
\(\beta = 0.01\)).  The ancantoids (Fig.~\ref{fig:xyprime-ancantoid}) can
show more complex behavior.  For instance, in the \(k = 0.5\),
\(\beta = 0.001\) ancantoid (Fig.~\ref{fig:xyprime-ancantoid}a) the near
wings begin to become more closed with increasing inclination up to
\(\abs{i} = 60^\circ\), at which point they open up again, whereas the
opening angle of the far wings increases monotonically with
\(\abs{i}\).

The inclination-dependent tracks that are traced by the thin-shell
models in the projected planitude--alatude plane are shown in
Figure~\ref{fig:thin-shell-R90-Rc}.  The behavior is qualitatively
different from the quadric shapes shown in
Figure~\ref{fig:quadric-projection-continued}a in that the tracks are
no longer confined within the borders of the region of a single type
of quadric (hyperboloid or spheroid). At low inclinations, many of the
models behave like the prolate spheroids, but then transition to a
hyperboloid behavior at higher inclinations, which is due to the
tension between the shape of the head and the shape of the far wings,
as discussed in the previous section. This can be seen most clearly in
the \(\beta = 0.001\), \(k = 0.5\) ancantoid (lowest red line in
Fig.~\ref{fig:thin-shell-R90-Rc}a, see also zoomed version in
Fig.~\ref{fig:convergence-cantoid-wilkinoid}). The track begins
heading towards \((\Pi', \Gamma') = (1, 1)\), as expected for a spheroid, but
then turns around and crosses the paraboloid line to head out on a
hyperboloid-like track.

Ancantoids with different degrees of inner-wind anisotropy are shown
in Figure~\ref{fig:thin-shell-R90-Rc}b.  In all cases, the tracks
follow hyperboloid-like behavior at high inclinations, tending to
populate the region just above the diagonal \(\Lambda' = \Pi'\).  The
\(k = 0\) ancantoids show a kink in their tracks at the point where
the projected apex passes through \(\theta = 90^\circ\), due to the
discontinuity in the second derivative of \(R(\theta)\) there (see
footnote~\ref{fn:discontinuity}).  The wilkinoid has a much less
interesting track, most clearly seen in the zoomed
Figure~\ref{fig:convergence-cantoid-wilkinoid}, simply moving the
short distance from \((\nicefrac53, \sqrt3)\) to
\((\nicefrac32, \smash{\sqrt{\nicefrac83}})\).  Despite its location
in the ellipsoid region of the plane, the fact that it has
\(\theta_\infty = 180^\circ\) means that it behaves more like a parabola at high
inclination, but converges on
\((\nicefrac32, \smash{\sqrt{\nicefrac83}})\) instead of \((2, 2)\)
since the far wings are asymptotically cubic, rather than quadratic.

\begin{figure}
  \centering
  \includegraphics[width=\linewidth]{figs/ancantoid-R90-vs-Rc-lobeta-a}
  \caption{As Fig.~\ref{fig:thin-shell-R90-Rc}a but zoomed in to show
    the wilkinoid track (white) and the convergence of the cantoid
    tracks (purple) to the wilkinoid as \(\beta \to 0\).}
  \label{fig:convergence-cantoid-wilkinoid}
\end{figure}

The local density of tick marks gives an indication of how likely it
would be to observe each portion of the track, assuming an isotropic
distribution of viewing angles.  It can be seen that the ticks tend to
be concentrated towards the beginning of each track, near the
\(\abs{i} = 0\) point, so the hyperboloid-like portions of the tracks
would be observed for only a relatively narrow range of inclinations.
This concentration becomes more marked as \(\beta\) becomes smaller, which
helps to resolve the apparent paradox that the wilkinoid corresponds
to the \(\beta \to 0\) limit of the cantoids, and yet follows a
qualitatively different track.  The detailed behavior of the
small-\(\beta\) cantoid models is shown in
Figure~\ref{fig:convergence-cantoid-wilkinoid}, which zooms in on the
region around the wilkinoid track.  It can be seen that for
\(\beta < 0.001\) the cantoid tracks begin to develop a downward hook,
similar to the \(k = 0.5\) ancantoids discussed above.  For
\(\beta < 10^{-4}\) this begins to approach the wilkinoid track and the
high inclination, upward portion of the track becomes less and less
important as \(\beta\) decreases.




% \begin{figure*}
%   \centering
%   \includegraphics[width=0.8\textwidth]{figs/two-quadric-R90-Rc-snapshots}
%   \caption{Variation with inclination angle of radii diagnostics}
%   \label{fig:thin-shell-R90-Rc-snapshots}
% \end{figure*}


%%% Local Variables:
%%% mode: latex
%%% TeX-master: "quadrics-bowshock"
%%% End:

\clearpage
\newcommand\hii{\ion{H}{ii}}

\section{Comparison with observations}
\label{sec:comp-with-observ}


Placing various classes of objects on the \(R_{90}\)--\(R_c\) plane:
\begin{itemize}
\item LL arcs
\item runaway O stars
\item AGB stars
\end{itemize}

\subsection{Mid-infrared arcs around early-type stars}
\label{sec:mid-infrared-arcs}

\begin{figure*}
  \setlength\tabcolsep{0pt}
  \begin{tabular}{ll}
    (a)
    & (b) \\
    \includegraphics
    [width=0.5\linewidth, trim=20 15 30 10, clip]{figs/0510-3-star}
    & \includegraphics
      [width=0.5\linewidth, trim=20 15 30 10, clip]{figs/0506-4-star} \\
    (c)
    & (d) \\
    \includegraphics
    [width=0.5\linewidth, trim=20 15 30 10, clip]{figs/0517-5-star}
    & \multicolumn{1}{c}
      {\includegraphics[width=0.35\linewidth, trim=20 20 20 20]
      {figs/mipsgal-r0-r0-plus-dPA-edited}}
  \end{tabular}
  \caption[]{Examples of typical fits to the bow shock shapes of
    MIPSGAL sources with different star ratings: (a) K510, 3-star
    rating; (b) K506, 4-star rating; (c) K517, 5-star rating.  Right
    panels of parts (a)--(c) show a 4\('\) square \SI{24}{\um} image,
    centered on each source.  Contours are ten linearly spaced levels
    between the median brightness of the entire image and the maximum
    brightness of the bow shock arc. Grids of galactic coordinates
    (light blue lines, parallel to the box sides) and equatorial
    coordinates (tilted magenta lines) are shown.  The stellar source
    and the bow shock axis, as determined by \citet{Kobulnicky:2016a}
    are indicated by an orange star and an orange line, respectively,
    where the line extends from \(-2 R_0\) to \(+2 R_0\).  The
    automatically traced arc shapes using the ``mean'' and ``peak''
    methods (see text) are shown by blue and red dots, respectively.
    The magenta circle shows the fit to the arc points within
    \(\pm 45^\circ\) of the nominal bowshock axis, with the magenta dot
    showing the center of curvature and the magenta line showing the
    fitted bow shock axis, which is the line passing through the
    source and the center of curvature.  Left panels of parts (a)--(c)
    show the radius measured from the source (upper panel) and
    brightness (lower panel) of the arc points, plotted as a function
    of angle \(\theta\) from the nominal bow shock axis, and with the same
    color coding as used on the image. Angular ranges of
    \(\theta = \pm 45\degr\) and \(\pm 90\degr\) are shown by gray shaded
    boxes In the upper panel, the \(R_0\) value tabulated by
    \citet{Kobulnicky:2016a} is shown by a horizontal blue line. (d)
    Comparison of the bow shock sizes (scatter plot) and position
    angles (inset histograms) determined from our fits with those
    tabulated by \citet{Kobulnicky:2016a} for the MIPSGAL sources.
    The horizontal dotted line on the main plot shows the MIPS
    \SI{24}{\um} point spread function FWHM of \(5.5\arcsec\).  The
    standard deviation (s.d.) of the position angle differences is
    shown on each inset histogram.}
  \label{fig:mipsgal-examples}
\end{figure*}


The most extensive observational sample of stellar bow shock nebulae
to date is a catalog of 709 arcs \citep{Kobulnicky:2016a} detected in
mid-infrared surveys of the Galactic Plane by the \textit{Spitzer
  Space Telescope} (\textit{SST}, \citealp{Werner:2004a}) and
\textit{Wide-field Infrared Survey Explorer} (\textit{WISE},
\citealp{Wright:2010a}).  These sources are believed to be powered by
the winds of early-type stars, which are either moving supersonically
through the interstellar medium (runaway stars,
\citealp{Gvaramadze:2008a}), or are interacting with a local bulk
flow, such as the champagne flow from a nearby \hii{} region (weather
vanes, \citealp{Povich:2008a}).

In order to study the shapes of these bow shocks, we downloaded data
from the NASA/IPAC Infrared Science Archive archive\footnote{
  \url{http://irsa.ipac.caltech.edu/docs/program_interface/api_images.html}}
and extracted 4\arcmin{} square images in the \SI{24}{\um} bandpass of
the Multiband Imaging Photometer for \textit{Spitzer} (MIPS) centered
on each of the 471 \citet{Kobulnicky:2016a} sources that are covered
by the MIPSGAL \citep{Carey:2009a} survey, which includes most of the
sources with Galactic longitude within \(\pm 60\degr\) of the Galactic
center.

We developed a methodology for automatically tracing the arcs as follows:
\begin{enumerate}[1.]
\item Calculate arrays of celestial coordinates, \(C\), for each pixel
  of the image.
\item Using the central source coordinates, \(C_0\) and nominal
  bowshock radius, \(R_0\) from \citet{Kobulnicky:2016a}, construct a
  pixel mask that includes only those pixels with separations from the
  source that satisfy \(\frac12 R_0 \le |C - C_0| \le 3 R_0\).  This mask
  will be used for all subsequent operations, which serves to help
  avoid confusion from the star itself and other bright sources in the
  field of view.
\item Define a ``step-back'' point, \(C_1\), which is located at a
  separation \(2 R_0\) from the source, but in the opposite direction
  from the apex of the bow shock. That is, along a position angle
  180\degr{} from the nominal position angle, \(\text{PA}_0\), of the
  bow shock axis.  This point is at one end of the orange line shown
  superimposed on the bow shock images in
  Figure~\ref{fig:mipsgal-examples}.
\item Looping over a grid of 50 position angles, \(\text{PA}_k\),
  within \(\pm 60\degr\) of \(\text{PA}_0\), estimate the location of
  the arc along rays cast from the step-back point, using two
  different methods:
  \begin{enumerate}[(a)]
  \item The pixel with the peak brightness, with coordinates
    \(C_{k,\text{peak}}\) (red dots in
    Fig.~\ref{fig:mipsgal-examples}).
  \item The mean brightness-weighted separation from \(C_1\), with
    coordinates \(C_{k,\text{mean}}\) (light blue dots in
    Fig.~\ref{fig:mipsgal-examples}).
  \end{enumerate}
  For each \(\text{PA}_k\) in the grid, the calculation is performed
  over only those pixels that satisfy
  \(|\text{PA}(C, C_1) - \text{PA}_k| < \frac12 \delta\theta\), where
  \(\delta\theta = 120/50 = 2.4\degr\), which defines a thin radial wedge from
  \(C_1\).  The results are shown as red and blue dots superimposed
  on the images in Figure~\ref{fig:mipsgal-examples}. Each of the two
  methods, ``peak'' and ``mean'', works better in some objects and
  worse in others (according to the subjective judgment of
  ``correctly'' tracing the bow shock shape).  We therefore take the
  average by amalgamating all the \(C_{k,\text{peak}}\) and
  \(C_{k,\text{mean}}\) points into a single set, \(C_{k}\), for
  the following steps.
\item For each of the points \(C_{k}\), determine the radial
  separation from the central source, \(R_k = |C_k - C_0|\) and the
  angle from the bow shock axis about the central source
  \(\theta_k = \text{PA}(C_k, C_0) - \text{PA}_0\).  These are plotted in
  the upper left panels of Figure~\ref{fig:mipsgal-examples}.  Note
  that, even though the rays are cast from the step-back point \(C_1\)
  within \(\pm 60\degr\) of \(\text{PA}_0\), the angles \(\theta_k\) are
  measured from the source, \(C_0\), which is closer to the bow shock
  than \(C_1\) and therefore \(|\theta_k|\) can be much larger than
  \(60\degr\).
\item Make our own estimate of the axial size, \(R_0\), of the bow
  shock by calculating the mean and standard deviation of \(R_k\) over
  all points \(C_k\) with \(|\theta_k| \le 10\degr\).  Note that this is
  distinct from the nominal value of \(R_0\) given in the
  \citet{Kobulnicky:2016a} catalog, which was ``measured by eye''.
\item Estimate the radius of curvature, \(R_c\), by fitting a circle
  to all those points within \(\pm 45\degr\) of the nominal axis
  (\(|\theta_k| < 45\degr\)), but after excluding any point with
  \(R_k < \frac12 R_m\) or \(R_k > 2 R_m\), where \(R_m\) is the median
  \(R_k\) for \(|\theta_k| < 45\degr\).
\item Determine two separate estimates, \(R_{90+}\) and \(R_{90-}\),
  of the perpendicular radius, \(R_{90}\), by taking the mean and
  standard deviation of \(R_k\) over all points \(C_k\) with
  \(|\theta_k - 90\degr| \le 10\degr\) for \(R_{90+}\), and with
  \(|\theta_k + 90\degr| \le 10\degr\) for \(R_{90-}\).
\end{enumerate}

After these automatic steps, we subjectively evaluate the results by giving a star rating to each source:

\paragraph*{0 stars} The fitting algorithm failed for some reason. 

\paragraph*{1 star} The fit was formally successful, but the results
for \(R_c\) or \(R_{90}\) are far removed from what a human would
predict by looking at the image.  For example, in the smallest
bowshocks, which are only marginally resolved by Spitzer's 6\arcsec{}
beam, the dispersion in \(R_k\) can be a significant fraction of
\(R_0\), in which case our algorithm tends to erroneously favor
\(R_c < R_0\).

\paragraph*{2 stars} The fit results are not totally outlandish, but
nonetheless some problem is apparent that casts doubt on their
reliability.  For example, a double-shell structure to the bow shock
that leads to large differences between the ``peak'' and ``mean''
methods, or point sources near to the bow shock that interfere with
the tracing procedure.
  
\paragraph*{3 stars} A good fit, but where the dispersion in \(R_k\)
and/or the asymmetry in the bow shock reduces the precision in the
determination of \(R_c\) and \(R_{90}\), giving subjectively estimated
uncertainties around the 20\% level.  An example of a 3-star fit is
shown in Figure~\ref{fig:mipsgal-examples}a.

\paragraph*{4 stars} A high quality fit, with subjectively estimated
uncertainties in \(R_c\) and \(R_{90}\) around the 10\% level. An
example of a 4-star fit is shown in
Figure~\ref{fig:mipsgal-examples}b.

\paragraph*{5 stars} The highest-quality fit, usually corresponding to
large, sharply defined bow shocks, whose shape is determined with high
precision. An example of a 5-star fit is shown in
Figure~\ref{fig:mipsgal-examples}c.

\bigskip
%
Figure~\ref{fig:mipsgal-examples}d compares the bow shock size,
\(R_0\), determined by our fits (vertical axis) with the corresponding
value given in the \citet{Kobulnicky:2016a} catalog (horizontal axis).
For most sources with 3-star or higher rating, the two estimates agree
to within \(\pm 20\%\), but there are a small number of sources with a
discrepancy of more than a factor of two.  In all cases that we
checked, we believe that our estimates of \(R_0\) are more accurate
than those in the catalog.  It is apparent that the star ratings are
correlated with the bow shock size, with larger bow shocks tending to
receive higher ratings, although there is considerable overlap.  In
particular, most of the 1- and 2-star sources are close to the
resolution limit of the MIPSGAL \SI{24}{\um} images (\(6\arcsec\),
indicated by the dotted horizontal line in te figure).

\begin{figure*}
  \centering
  \begin{tabular}{ll}
    (a) & (b) \\
    \includegraphics[width=0.45\textwidth, trim=0 0 30 0]
    {figs/mipsgal-Rc-R90-zoom-annotated}
        & \includegraphics[width=0.45\textwidth, trim=0 0 30 0]
          {figs/mipsgal-Rc-R90-thumbnails} 
  \end{tabular}
  \caption[]{MIPSGAL sources on the bow shock shape diagnostic diagram
    of dimensionless radius of curvature versus perpendicular radius.
    The regions corresponding to different classes of cuadrics are
    shown by shading (see \S~\ref{sec:conic}): oblate spheroids (light
    gray background); prolate spheroids (darker gray background);
    paraboloids (curved black line); hyperboloids (white background).
    (a) Individual sources with bow shock fit quality rating of 3-star
    or above.  All 5-star sources plus those 4-star sources with
    \(R_c/R_0 < 1\) are labelled with their \cite{Kobulnicky:2016a}
    catalog number.  Horizontal error bars do not directly reflect the
    uncertainty in \(R_c/R_0\) but are instead simply the standard
    deviation from the circle fit of bowshock points \(R_k\) within
    \(\pm 45\degr\) of the axis.  Values on the vertical axis
    represent the average of \(R_{90+}\) and \(R_{90-}\), with thin
    vertical error bars showing the difference between \(R_{90+}\) and
    \(R_{90-}\), and thick vertical error bars showing the rms
    dispersion of \(R_k\) about these values for bow shock points
    within \(\pm 10\degr\) of the \(+90\degr\) and \(-90\degr\)
    directions.  (b) Kernel density estimator (KDE) of the
    distribution for 3-star sources (blue, filled contours) and 4-
    plus 5-star sources (orange/brown, unfilled contours).  The KDE
    uses an anisotropic gaussian kernel with bandwidths of
    \(0.18 \times 0.12\). Images of representative 4- and 5-star sources at
    different points on the \(R_c\)--\(R_{90}\) plane are also shown.}
  \label{fig:mipsgal-shapes}
\end{figure*}

In the following analysis, only those sources with a 3-star or higher
rating are used.  These comprise approximately half (227 out of 471)
of all the MIPSGAL arc sources.  In some cases of poor and failed
fits, there is nothing apparently ``wrong'' with the source itself,
and it is likely that minor tweaks to the methodology would improve
matters, but we have elected not to do so, in order to maintain a
uniform methodology across all sources.

The inset of Figure~\ref{fig:mipsgal-examples}d shows histograms of
the difference between the position angle, \(\text{PA}_0\) determined
by our fits and that listed in \citet{Kobulnicky:2016a}.  Although
observational uncertainties undoubtedly contribute in part, the
differences are mainly due to real asymmetries in the bow shocks,
especially for the 4- and 5-star sources.  The
\citeauthor{Kobulnicky:2016a} catalog \(\text{PA}_0\) values are
mostly sensitive to the orientation of the bow shock wings, whereas
our fitted \(\text{PA}_0\) values are determined by the point in the
bow shock head that is closest to the stellar source.  For this
reason, we use the catalog \(\text{PA}_0\) values for defining the
axis when measuring \(R_{90+}\) and \(R_{90-}\). On the other hand,
the fitted values of \(\text{PA}_0\) are better correlated with the
position of the bow shock's brightness peak, as is apparent in the
lower left panels of Figure~\ref{fig:mipsgal-examples}a and c.

The derived bowshock shapes of all the 3-, 4-, and 5-star sources are
shown in Figure~\ref{fig:mipsgal-shapes} on the plane of \(R_c/R_0\)
versus \(R_{90} / R_0\).  Panel a shows the individual points
superimposed on the theoretical results for quadrics of revolution
(see figure caption and \S~\ref{sec:conic}), while panel b shows
contours of the kernel density estimator (KDE, see
\citealp{Leiva-Murillo:2012a, Scott:2015a}) of the two-dimensional
distribution of points on the \(R_c\)--\(R_{90}\) plane.  The
horizontal axis corresponds to the shape of the head of the bow shock
near its apex, ranging from sharper, pointier shapes with
\(R_c/R_0 < 1\) to flatter, snubber shapes with \(R_c/R_0 \gg 1\), where
it must be understood that all judgments of sharpness/flatness are
with respect to the axial separation, \(R_0\), between the source and
the bow shock apex.  The vertical axis corresponds to the shape of the
bow shock wings, ranging from closed ``C'' shapes for smaller values
of \(R_{90} / R_0\) to open ``V'' shapes for larger values of
\(R_{90} / R_0\).  The boundary between closed and open corresponds to
the paraboloids, and is shown by the solid curved line that divides
the shaded from the unshaded regions of the graph.

The KDE contours in Figure~\ref{fig:mipsgal-shapes}b show that the
distribution of 3-star sources is very similar to that of 4- and
5-star sources, although the higher-rated sources are shifted slightly
to the upper right.  This is probably due to higher-rated sources
being on average bigger, as will be discussed further below.  The bulk
of the sources are concentrated around the paraboloid line, with
\(1 < R_c/R_0 < 2.5\), and \(1.2 < R_{90}/R_0 < 2\).  But significant
minorities are found in three other regions: (1)~a clump with
\(R_c/R_0 \la 1\); (2)~a vertical spur towards higher \(R_{90}\) at
\(R_c/R_0 \approx 2\); and (3)~a broad horizontal tail towards higher
\(R_c\) at \(R_{90}/R_0 \la 2\).

\begin{figure*}
  \centering
  \includegraphics[width=0.8\textwidth]{figs/mipsgal-pairplot}
  \caption[]{Matrix of pair plots that illustrate distributions of and
    correlations between the non-shape parameters of all MIPSGAL bow
    shock sources from \citet{Kobulnicky:2016a}.  Plots on the leading
    diagonal show histograms of the following parameters: bow shock
    angular size, \(\log_{10} R_0\); Galactic latitude,
    \(\log_{10}|b|\); Galactic longitude, \(\cos \ell\);
    extinction-corrected \(H\)-band magnitud of the stellar source,
    \(H_0\); \(K\)-band extinction, \(A_K\).  Scatter plots in the
    upper triangle show the joint distribution of each pair of
    parameters.  These are repeated in the lower triangle but showing
    the KDEs of the joint distributions, which are annotated with the
    Pearson linear correlation coefficient, \(r\), for each pair. The
    straight lines shown superimposed on the plots of stellar magnitud
    versus bowshock size correspond to model results for the same star
    at a sequence of distances (dashed lines) and a sequence of
    stellar luminosities at a fixed distance (dotted lines).  See text
    for details. }
  \label{fig:mipsgal-pairplot}
\end{figure*}



\begin{figure*}
  \centering
  \begin{tabular}{ll}
    (a) & (b) \\
    \includegraphics[width=0.45\textwidth]{figs/mipsgal-Rc-R90-R0} &
    \includegraphics[width=0.45\textwidth]{figs/mipsgal-Rc-R90-Mag} 
  \end{tabular}
  \caption[]{Comparison between the distribution of bowshock shapes
    when the sources are divided into two sub-samples according to the
    value of another parameter. (a)~Bow shock angular size, \(R_0\).
    (b)~Extinction-corrected \(H\)-band magnitude of the stellar
    source.}
  \label{fig:mipsgal-correlated}
\end{figure*}

\begin{figure*}
  \centering
  \begin{tabular}{ll}
    (a) & (b) \\
    \includegraphics[width=0.45\textwidth]{figs/mipsgal-Rc-R90-environment} &
    \includegraphics[width=0.45\textwidth]{figs/mipsgal-Rc-R90-candidates} 
  \end{tabular}
  \caption[]{Lack of significant correlation of bow shock shape with
    (a) environment, and (b) uncertainty in stellar source
    identification.}
  \label{fig:mipsgal-uncorrelated}
\end{figure*}


\subsection{Far-infrared arcs around late-type stars}
\label{sec:far-infrared-arcs}

\begin{figure}
  \centering
  \includegraphics[width=\linewidth]{figs/mipsgal-Rc-R90-vs-Herschel}
  \caption[]{Comparison of RSG/AGB arcs with OB star arcs.}
  \label{fig:herschel-compare-mipsgal}
\end{figure}



\subsection{Stationary emission line arcs in M42}
\label{sec:stat-emiss-line}

\begin{figure*}
  \centering
  \includegraphics[width=\textwidth]{figs/annotated-ll-arcs}
  \caption[]{Stationary bow shock arcs in the Orion Nebula.}
  \label{fig:ll-arcs}
\end{figure*}

\begin{figure}
  \centering
  \includegraphics[width=\linewidth]{figs/mipsgal-Rc-R90-vs-Orion}
  \caption[]{Comparison of Orion with OB stars.}
  \label{fig:ll-compare-mipsgal}
\end{figure}




%%% Local Variables:
%%% mode: latex
%%% TeX-master: "quadrics-bowshock.tex"
%%% End:

\section{Conclusions}
\label{sec:conc}

%\newcommand\thC{\(\theta^1\)\,Ori~C}
\defcitealias{Canto:1996}{CRW}

We developed a method to estimate the shape of a generic bow shock product of the
interaction of two winds and was applied to the proplyds in the core of the ONC.
We started measuring the projected characteristic radii $(R'_0,R'_c)$ for each proplyd in our
sample and compare them with the ``conic equivalent'' of a two winds interaction model based 
on \CRW{} work to estimate the intrinsic bow shock shape and get the ionizing flux for ionization balance 
and the stagnation pressure for our sample of proplyds.
Most results are consistent with a proplyd's photoevaporated flow with an anisotropic density
distribution, with different anisotropy degrees. We ound that LV4 has the least anisotropic flow,
while LV2 has the most anisotropic flow. And for the 177-341, 169-348 and 180-331 we found out that 
the stellar wind is not enough to keep their bow socks stationary.

%%% Local Variables:
%%% mode: latex
%%% TeX-master: "proplyd-bowshocks"
%%% End:

\bibliographystyle{mnras}
%% All references should be put in the BibTeX file bowshocks-biblio.bib
\bibliography{bowshocks-biblio}
\clearpage
\appendix
% \appendixpage
% \addappheadtotoc
\section{Parabola of Revolution}
\label{app:parabola}

In this section we develop the particular case of the parabola.
The parametrization is given by:

\begin{align}
x &= - \frac{1}{2}R_ct^2 + R_0 \\
y &= R_c t
\end{align}
Where $R_c$ is the radius of curvature and $R_0$ is the distance of the parabola nose to the
origin.
The respective tridimensional shape is given by:
\begin{align}
x &= -\frac{1}{2}R_ct^2 + R_0 \label{eq:x-par-a}\\
y &= R_c t \cos\phi  \label{eq:y-par-a}\\
z &= R_c t \sin\phi  \label{eq:z-par-a}
\end{align}
The azimutal angle where the surface is tangent to the line of sight in this case is given by:
\begin{align}
\sin\phi_t = -\frac{\tan i}{t} \label{eq:sin-tan-a} 
\end{align}

Subtituting (\ref{eq:sin-tan-a}) into (\ref{eq:x-par-a}), (\ref{eq:y-par-a}) and
(\ref{eq:z-par-a}) we find the apparent shape
of the paraboloid:

\begin{align}
x' &= -\frac{R_c(\frac{1}{2}t^2 \cos^2 i -\sin^2 i)}{\cos i}+R_0\cos i \\
y' &= R_c\left(t^2-\tan^2 i\right)^{1/2} 
\end{align}

Taking the limit of equations (\ref{eq:qprime}) and (\ref{eq:Aprime}) when $\theta_c$ tends to zero we find that:

\begin{align}
\left(\frac{q'}{q}\right)_{\mathrm{parabola}} &= 1+\frac{\tilde{R}_c\tan^2 i}{2}\\
\left(\tilde{R}'_c\right)_{\mathrm{parabola}} &= \frac{\tilde{R}_c}{\cos^2 i + \frac{\tilde{R}_c}{2}\sin^2 i}
\end{align}

\section{Analytic derivation of the radius of curvature in the Thin Shell model}
\label{app:rc-analytic}

For small $\theta$ we may do a polynomial expansion for the shell shape such as:

\begin{align}
R \simeq R_0\left(1+\gamma\theta^2 + \Gamma\theta^4\right) \label{eq:R-exp}
\end{align}

The radius of curvature at the axis for $R$ is given by:

\begin{align}
R_c = R_0\left(1-2\gamma\right)^{-1}
\end{align}

The coefficient gamma may be derived by an expansion at small angles of equation
(\ref{eq:th1th}), as follows:

From the first term of the right side we get:

\begin{align}
\cot\theta &\simeq \theta^{-1}\left[1-\frac{1}{3}\theta^2\right] \\
\cos^k\theta\sin^2\theta &\simeq \theta^2 - \left(\frac{1}{3} + \frac{k}{2}\right)\theta^4 \\
\implies I_k(\theta) &\simeq \frac{1}{3}\theta^3\left[ 1 - \frac{1}{10}(3k+2)\theta^2\right]\\
\implies 2\beta I_k(\theta)\cot\theta &\simeq \frac{2}{3}\beta\theta^2\left[1-\frac{1}{30}
(9k+16)\theta^2\right]\label{eq:AR1} 
\end{align}

For the second term we get:

\begin{align}
-\frac{2\beta}{k+2}\left(1-\cos^{k+2}\theta\right) & \simeq -\beta\theta^2\left[1-\frac{1}{12}
(3k+4)\theta^2\right] \label{eq:AR2}
\end{align}

For the left side we use equation (25) from \CRW{}. Then, equation (\ref{eq:th1th}) results
as follows:

\begin{align}
\theta_1^2\left(1+\frac{1}{15}\theta_1^2\right) = \beta\theta^2\left[1+\frac{1}{60}(4-9k)
\theta^2\right] \label{eq:th1th-app}
\end{align}

And we can use the approximation $\theta_1 \approx \beta\theta^2$ for the correction term in
the left side of (\ref{eq:th1th-app}):

\begin{align}
\theta_1^2 &= \beta\theta^2\left[1+\frac{1}{60}(4-9k)\theta^2\right]
\left(1-\frac{\beta}{15}\theta^2\right) \\
\implies \theta_1^2 &= \beta\theta^2\left[1+ 2C_{k\beta}\theta^2\right]
\label{eq:th1th-small}\\
\mathrm{where:~} C_{k\beta} &\equiv \frac{1}{2}\left(A_k-\frac{\beta}{15}\right) \\
A_k &\equiv \frac{1}{15}-\frac{3k}{20}
\end{align}

Now, using equation (23) from \CRW{} we may estimate $R$ at low angles. To do this, we need to
expand each term as follows (neglecting terms of order four or higher):

\begin{align}
\theta_1 = &= \beta^{1/2}\theta\left[1+ 2C_{k\beta}\theta^2\right]^{1/2} \\
\theta + \theta_1 &= \theta\left[1+\beta^{1/2}\left(1+2C_{k\beta}\theta^2\right)\right]\\
\sin\theta_1 &= \theta_1\left[1-\frac{\theta_1^2}{6}\right] \\
 &= \beta^{1/2}\theta\left[1+\left(C_{k\beta}-\frac{1}{6}\beta\right)\theta^2\right] \\
 \sin(\theta+\theta_1) &= \left[\theta+\theta_1\right]\left[1-\frac{\left(\theta+\theta_1
 \right)^2}{6}\right] \\
 &= \theta\left(1+\beta^{1/2}\right)\left\lbrace 1+\left[\frac{C_{k\beta}\beta^{1/2}}
 {1+\beta^{1/2}}-\frac{1}{6}\left(1+\beta^{1/2}\right)^2\right]\theta^2\right\rbrace
\end{align}


So, combining these terms with equation (23) from \CRW{} we found the final expression for $R$:

\begin{align}
\frac{R}{D}\equiv \frac{\sin\theta_1}{\sin(\theta+\theta_1)} = \frac{\beta^1/2}{1+\beta^{1/2}}
\left\lbrace 1 + \theta^2\left[\frac{C_{k\beta}}{1+\beta^{1/2}}+\frac{1}{6}\left(1+2\beta^{1/2}
\right)\right] \right\rbrace \label{eq:r-small-theta}
\end{align}

Returning to equation (\ref{eq:R-exp}) we see the following:

\begin{align}
R_0 &= \frac{\beta^{1/2}}{1+\beta^{1/2}} \\
\gamma &= \frac{C_{k\beta}}{1+\beta^{1/2}}+\frac{1}{6}\left(1+2\beta^{1/2}\right)
\label{eq:app-gamma}
\end{align}

We recover equation (27) of \CRW{} for $R_0$ and equation (\ref{eq:app-gamma}) is the
needed term to calculate the radius of curvature at the axis.

\section{Analytic derivation of \texorpdfstring{\boldmath $R_{90}$}{R\_90} in the thin shell model}
\label{app:r90-analytic}

To derive $R_{90}$ we need to evaluate equations (23) from \CRW{} and (\ref{eq:th1th})
at $\theta=\frac{\pi}{2}$:

\begin{align}
R_{90} = D\tan\theta_{1,90} \\
\theta_{1,90}\cot\theta_{1,90} -1 = -\frac{2\beta}{k+2} \label{eq:th190}
\end{align}
Where $\theta_{1,90}\equiv \theta_1(\frac{\pi}{2})$. Combining both equations and  introducing
the parameter $\xi\equiv \frac{2}{k+2}$ we have:
\begin{align}
R_{90} &= D\frac{\theta_{1,90}}{1-\xi\beta} \label{eq:r90-incomplete}
\end{align}

Expanding the left side of (\ref{eq:th190}) until fourth order, equation (\ref{eq:th190})
becomes:

\begin{align}
\theta_{1,90}^2\left(1+\frac{\theta_{1,90}^2}{15}\right) \simeq 3\xi\beta
\end{align}

Applying the approximation $\theta_1^2 \approx 3\xi\beta$ we found a solution
for $\theta_{1,90}$:

\begin{align}
\theta_{1,90} = \left(\frac{3\xi\beta}{1+\frac{1}{5}\xi\beta}\right)^{1/2}
\end{align}

And substituting into (\ref{eq:r90-incomplete}) we find the solution for $R_{90}$:

\begin{align}
R_{90} &= \frac{\left(3\xi\beta\right)^{1/2}}{\left(1+\frac{1}{5}\xi\beta\right)^{1/2}
\left(1-\xi\beta\right)} \\
\implies \tilde{R}_{90} &\equiv \frac{R_{90}}{R_0} = \frac{\sqrt{3\xi}\left(1+\beta^{1/2}\right)}
{\left(1+\frac{1}{5}\xi\beta\right)^{1/2}\left(1-\xi\beta\right)}
\end{align}

\section{Derivation of Characteristic Radii in Isotropic Wind/Parallel interaction Problem}
\label{app:ch-rad-Wilkin}

$\tilde{R}_{90}$ is obtained by simply evaluating equation (\ref{eq:R-Wilkin}) at $\theta=\frac{\pi}{2}$.
For the Radius of curvature we follow a similar procedure than the Wind/Wind interaction, but using equation
(\ref{eq:R-Wilkin}) for $R(\theta)$ and inserting the cosecant into the square root.

Expanding the terms of $R(\theta)$ we find the following:

\begin{align}
  \csc^2\theta &\simeq \theta^{-2}\left[1+\frac{\theta^2}{3}\left(1+\frac{\theta^2}{5}\right)\right] \\
  1-\theta\cot\theta &\simeq \frac{\theta^2}{3}\left[1 + \frac{\theta^2}{15}\left(1+\frac{2\theta^2}{21}\right)\right] \\
  \implies \tilde{R}(\theta) \simeq 1 + \frac{\theta^2}{5} + O\left[\theta^4\right]
\end{align}

From equation (\ref{eq:Rcurv}) for the radius of curvature we finally get the numerical value for $\tilde{R_c} = \frac{5}{3}$
\clearpage
\sisetup{detect-weight=true, detect-shape=true, detect-inline-weight=math}
\sisetup{round-mode=figures, round-precision=2}
\begin{landscape}
\begin{table}
  \setlength\tabcolsep{2pt}
  \caption{Big table of \(p\)-values}
  \begin{tabular}{@{} ll @{\ } S[round-mode=places]S[round-mode=places] S[round-mode=places]S[round-mode=places] SS SS @{\quad\quad\quad} SSS @{\quad} S@{}S@{}S @{}}\toprule
     & & \multicolumn{2}{c}{Mean} & \multicolumn{2}{c}{Std.\ Dev.} & \multicolumn{2}{c}{Obs.\ Disp.} & \multicolumn{2}{c @{\quad\quad\quad}}{s.e.m.} & \multicolumn{3}{c @{\quad} }{\dotfill Effect sizes\dotfill } & \multicolumn{3}{c}{\dotfill Non-parametric test \(p\)-values  \dotfill} \\ 
    {Comparison} & {Variable} & {\(\langle \text{A} \rangle\)} & {\(\langle \text{B} \rangle\)} & {\(\sigma_{\text{A}}\)} & {\(\sigma_{\text{B}}\)} & {\(\langle \epsilon_{\text{A}} \rangle\)} & {\(\langle \epsilon_{\text{B}} \rangle\)} & {\((\sigma/\!\sqrt n)_{\text{A}}\)} & {\((\sigma/\!\sqrt n)_{\text{B}}\)} & {\(r_b\)} & {\(d_C\)} & {\(\sigma_{\text{A}}/\sigma_{\text{B}}\)} & \multicolumn{1}{c}{Anderson--Darling} & \multicolumn{1}{c}{Rank biserial} &  \multicolumn{1}{c}{Brown--Forsythe}\\
\midrule
    \multicolumn{10}{@{} l @{\quad\quad\quad}}{\itshape Median split of continuous variables \dotfill}\\
    \addlinespace
    Faint/bright & \(R_{90}\) & 1.677 & 1.768 & 0.269 & 0.316 & 0.231 & 0.236 & 0.025 & 0.030 & \bfseries 0.174 & \bfseries -0.308 & 0.851 & \bfseries 0.0215 & \bfseries 0.0235 & 0.125\\
\(H\) magnitude  & \(\Delta R_{90}\) & 0.183 & 0.198 & 0.161 & 0.163 &  &  & 0.015 & 0.015 & 0.070 & -0.093 & 0.987 & 0.538 & 0.365 & 0.761\\
\(n_{\text{A}} =  n_{\text{B}} = 113\) & \(R_{c}\) & 1.655 & 1.917 & 0.631 & 1.045 & 0.097 & 0.078 & 0.059 & 0.098 & 0.123 & -0.303 & \bfseries 0.604 & 0.123 & 0.111 & \bfseries 0.0335\\
\addlinespace
Low/high  & \(R_{90}\) & 1.707 & 1.739 & 0.250 & 0.336 & 0.256 & 0.212 & 0.024 & 0.031 & 0.061 & -0.110 & \bfseries 0.745 & \bfseries 0.0281 & 0.428 & \bfseries 0.00139\\
\(R_0\) & \(\Delta R_{90}\) & 0.175 & 0.204 & 0.157 & 0.165 &  &  & 0.015 & 0.015 & 0.091 & -0.176 & 0.948 & 0.103 & 0.238 & 0.193\\
\(n_{\text{A}} =  n_{\text{B}} = 113\) & \(R_{c}\) & 1.766 & 1.803 & 0.975 & 0.755 & 0.114 & 0.062 & 0.092 & 0.071 & 0.100 & -0.043 & 1.291 & 0.228 & 0.192 & 0.599\\
\addlinespace
Low/high & \(R_{90}\) & 1.703 & 1.742 & 0.267 & 0.323 & 0.233 & 0.235 & 0.025 & 0.030 & 0.040 & -0.132 & 0.824 & 0.554 & 0.602 & 0.123\\
extinction & \(\Delta R_{90}\) & 0.186 & 0.195 & 0.138 & 0.183 &  &  & 0.013 & 0.017 & -0.039 & -0.057 & 0.754 & 0.301 & 0.61 & 0.112\\
\(n_{\text{A}} =  n_{\text{B}} = 113\)  & \(R_{c}\) & 1.725 & 1.846 & 0.822 & 0.917 & 0.091 & 0.085 & 0.077 & 0.086 & 0.082 & -0.139 & 0.896 & 0.219 & 0.285 & 0.982\\
\addlinespace
Low/high & \(R_{90}\) & 1.722 & 1.724 & 0.328 & 0.261 & 0.234 & 0.234 & 0.031 & 0.024 & 0.020 & -0.008 & 1.256 & 0.308 & 0.795 & 0.0534\\
\(\vert{}b\vert\)  & \(\Delta R_{90}\) & 0.188 & 0.191 & 0.161 & 0.162 &  &  & 0.015 & 0.015 & 0.009 & -0.021 & 0.995 & 0.964 & 0.907 & 0.694\\
\(n_{\text{A}} =  n_{\text{B}} = 113\) & \(R_{c}\) & 1.706 & 1.862 & 0.727 & 0.988 & 0.085 & 0.091 & 0.068 & 0.092 & 0.069 & -0.181 & 0.736 & 0.19 & 0.368 & 0.0842\\
\addlinespace
High/low & \(R_{90}\) & 1.734 & 1.707 & 0.279 & 0.321 & 0.241 & 0.223 & 0.024 & 0.034 & -0.049 & 0.093 & 0.868 & 0.361 & 0.532 & 0.159\\
\(\cos \ell\)  & \(\Delta R_{90}\) & 0.182 & 0.201 & 0.155 & 0.171 &  &  & 0.013 & 0.018 & 0.054 & -0.122 & 0.909 & 0.604 & 0.491 & 0.365\\
\(n_{\text{A}}, n_{\text{B}} = 137, 90\) & \(R_{c}\) & 1.807 & 1.751 & 0.946 & 0.742 & 0.090 & 0.084 & 0.081 & 0.078 & -0.000 & 0.064 & 1.274 & 1.03 & 0.999 & 0.549\\
\midrule
    \multicolumn{10}{@{} l @{\quad\quad\quad}}{\itshape Categorical variables \dotfill}\\
    \addlinespace
Environment:  & \(R_{90}\) & 1.735 & 1.693 & 0.283 & 0.338 & 0.238 & 0.218 & 0.022 & 0.053 & -0.070 & 0.142 & 0.838 & 0.603 & 0.49 & 0.392\\
Facing & \(\Delta R_{90}\) & 0.190 & 0.195 & 0.161 & 0.172 &  &  & 0.012 & 0.027 & -0.019 & -0.034 & 0.938 & 0.507 & 0.85 & 0.438\\
\(n_{\text{A}}, n_{\text{B}} = 170, 41\) & \(R_{c}\) & 1.757 & 1.852 & 0.854 & 0.899 & 0.087 & 0.083 & 0.066 & 0.140 & 0.042 & -0.110 & 0.950 & 0.713 & 0.676 & 0.377\\
\addlinespace
Environment:  & \(R_{90}\) & 1.735 & 1.680 & 0.283 & 0.309 & 0.238 & 0.233 & 0.022 & 0.077 & -0.130 & 0.193 & 0.916 & 0.518 & 0.391 & 0.782\\
\hii & \(\Delta R_{90}\) & 0.190 & 0.175 & 0.161 & 0.138 &  &  & 0.012 & 0.034 & -0.048 & 0.095 & 1.170 & 0.932 & 0.754 & 0.799\\
\(n_{\text{A}}, n_{\text{B}} = 170, 16\) & \(R_{c}\) & 1.757 & 1.907 & 0.854 & 0.955 & 0.087 & 0.105 & 0.066 & 0.239 & 0.024 & -0.174 & 0.894 & 0.496 & 0.875 & 0.255\\
\addlinespace
Single/multiple & \(R_{90}\) & 1.709 & 1.762 & 0.289 & 0.315 & 0.230 & 0.243 & 0.022 & 0.041 & 0.074 & -0.177 & 0.917 & 0.342 & 0.396 & 0.338\\
source candidate & \(\Delta R_{90}\) & 0.184 & 0.206 & 0.162 & 0.160 &  &  & 0.013 & 0.021 & 0.093 & -0.136 & 1.013 & 0.421 & 0.284 & 0.97\\
\(n_{\text{A}}, n_{\text{B}} = 167, 60\) & \(R_{c}\) & 1.767 & 1.833 & 0.825 & 0.988 & 0.090 & 0.080 & 0.064 & 0.128 & 0.027 & -0.076 & 0.835 & 0.999 & 0.756 & 0.605\\
\addlinespace
With/without  & \(R_{90}\) & 1.734 & 1.721 & 0.289 & 0.298 & 0.223 & 0.237 & 0.043 & 0.022 & -0.042 & 0.044 & 0.970 & 0.595 & 0.667 & 0.563\\
\SI{8}{\um} emission & \(\Delta R_{90}\) & 0.203 & 0.186 & 0.205 & 0.149 &  &  & 0.031 & 0.011 & 0.021 & 0.106 & 1.377 & 0.765 & 0.826 & 0.219\\
\(n_{\text{A}}, n_{\text{B}} = 45, 182\) & \(R_{c}\) & 1.714 & 1.802 & 0.598 & 0.926 & 0.091 & 0.087 & 0.089 & 0.069 & -0.012 & -0.101 & 0.646 & 0.824 & 0.904 & 0.2\\
\midrule
    \multicolumn{10}{@{} l @{\quad\quad\quad}}{\itshape Intercomparison with other datasets \dotfill}\\
    \addlinespace
MIPS vs Orion & \(R_{90}\) & 1.723 & 2.418 & 0.297 & 0.811 &  &  & 0.020 & 0.191 & \bfseries 0.700 & \bfseries -1.928 & \bfseries 0.366 & \bfseries 8.02e-06 & \bfseries 7.84e-07 & \bfseries 6.41e-06\\
 & \(\Delta R_{90}\) & 0.190 & 0.506 & 0.162 & 0.551 &  &  & 0.011 & 0.130 & 0.220 & -1.468 & \bfseries 0.294 & \bfseries 3.6e-05 & 0.121 & \bfseries 4.03e-14\\
\(n_{\text{A}}, n_{\text{B}} = 227, 18\) & \(R_{c}\) & 1.784 & 2.639 & 0.871 & 1.302 &  &  & 0.058 & 0.307 & \bfseries 0.516 & \bfseries -0.940 & 0.669 & \bfseries 0.000505 & \bfseries 0.000273 & 0.12\\
\addlinespace
MIPS vs RSG & \(R_{90}\) & 1.723 & 1.386 & 0.297 & 0.122 &  &  & 0.020 & 0.046 & \bfseries -0.756 & \bfseries 1.152 & 2.425 & \bfseries 0.000494 & \bfseries 0.000671 & 0.0573\\
 & \(\Delta R_{90}\) & 0.190 & 0.197 & 0.162 & 0.296 &  &  & 0.011 & 0.112 & -0.242 & -0.045 & 0.546 & 0.156 & 0.276 & 0.397\\
\(n_{\text{A}}, n_{\text{B}} = 227, 7\) & \(R_{c}\) & 1.784 & 1.426 & 0.871 & 0.079 &  &  & 0.058 & 0.030 & -0.201 & 0.418 & 10.969 & 0.0742 & 0.367 & 0.0505\\
\bottomrule
\end{tabular}
\end{table}
\end{landscape}

\section{Distribution of p-values for all correlations tested}
\label{sec:distr-p-values}

\begin{figure}
  (a)\\
  \includegraphics[width=\linewidth]{figs/p-value-histogram-new-linear}\\
  (b)\\
  \includegraphics[width=\linewidth]{figs/p-value-histogram-new}
  \caption{Histogram of \(p\)-values for all non-parametric 2-sample
    tests performed during the analysis of
    section~\ref{sec:comp-with-observ}. (a)~Uniformly spaced linear
    bins and linear vertical axis. (b)~Uniformly spaced logarithmic
    bins and logarithmic vertical axis, with all values
    \(p \le 10^{-6}\) included in the leftmost bin.  The vertical dashed
    lines shows the traditional threshold values for significance:
    \(p = 0.001\) and \(p = 0.05\). The red solid line shows the
    uniform distribution of \(p\)-values that would be expected if the
    null hypothesis were always true, that is, if no significant
    correlations existed.}
  \label{fig:histo-p-values}
\end{figure}



\begin{table}
  \caption{Lower bounds on the Type~I error rate, \(\alpha\)}
  \label{tab:type-I}
  \centering
  \begin{tabular}{lrr} \toprule
    Correlation & \(p\) & \(\alpha\) \\
    \midrule
    \multicolumn{3}{l}{\itshape Quantitative \dotfill}\\
    \(R_c\) vs \(R_0\) & \textit{0.0054} & \textit{0.0712} \\
    \(R_{90}\) vs \(R_0\) & \textbf{0.0001} & \textbf{0.0025} \\
    \(R_c\) vs \(H_0\) & 0.1229 & 0.4119 \\
    \(R_{90}\) vs \(H_0\) & \textit{0.0215} & \textit{0.1833} \\
    \(R_c\) vs \(A_K\) & 0.19 & 0.4617 \\
    \(R_{90}\) vs \(A_K\) & 0.63 & 1 \\
    \(R_c\) vs \(|b|\) & 0.19 & 0.4617 \\
    \(R_{90}\) vs \(|b|\) & 0.31 & 0.4967 \\
    \(R_c\) vs \(\cos(\ell)\) & 1.00 & 1 \\
    \(R_{90}\) vs \(\cos(\ell)\) & 0.36 & 0.4999 \\
    \midrule
    \multicolumn{3}{l}{\itshape Categorical} \dotfill\\
    \(R_c\) vs Facing & 0.71 & 1 \\
    \(R_c\) vs H II & 0.50 & 1 \\
    \(R_{90}\) vs Facing & 0.60 & 1 \\
    \(R_{90}\) vs H II & 0.52 & 1 \\
    \(R_c\) vs Multiple & 1.00 & 1 \\
    \(R_{90}\) vs Multiple & 0.34 & 0.4993 \\
    \(R_c\) vs \SI{8}{\um} & 0.82 & 1 \\
    \(R_{90}\) vs \SI{8}{\um} & 0.6 & 1 \\
    \midrule
    \multicolumn{3}{l}{\itshape Other datasets} \dotfill\\
    \(R_c\) vs Herschel & 0.074 & 0.3437 \\
    \(R_{90}\) vs Herschel & \textbf{0.00052} & \textbf{0.0106} \\
    \(R_c\) vs M42 & \textbf{0.00048} & \textbf{0.0099} \\
    \(R_{90}\) vs M42 & \textbf{0.0000105} & \textbf{0.0003} \\
    \bottomrule
  \end{tabular}
\end{table}


The \(p\)-values are the probability of finding a difference between
two populations as large as (or larger than) what is observed,
\emph{given} that there is no difference in the underlying
distribution from which the two populations are drawn (that is, given
that the null hypothesis is true).  However, what we really want to
know is something else: the probability that the null hypothesis is
true, \emph{given} the observations.  That is, the probability,
\(\alpha\), of a \textit{false positive}, also known as the \textit{Type I
  error rate}.  The common mistake of conflating these two definitions
is known as the ``\(p\)-value fallacy'' \citep{Goodman:1999a}, or``the
error of the transposed conditional'', as discussed in detail by
\citet{Colquhoun:2014a}.  It is possible to derive \(\alpha\) from
\(p\) using Bayes' theorem (e.g., \citealp{Goodman:1999b}), but that
requires an estimate of the prior probability of the null hypothesis,
independent of the observations.  Alternatively, it is also possible
to find a lower bound on \(\alpha\) from a frequentist approach
\citep{Sellke:2001a}:
\begin{equation}
  \label{eq:type-I}
  \alpha(p) \ge \bigg[ 1 - \big(e\, p \ln p\big)^{-1} \bigg]^{-1}
  \quad \text{valid for } p < 1/e.
\end{equation}
This is the approach we adopt here, which also numerically coincides
with the Bayesian approach for the case where the prior probability of
the null hypothesis is 0.5.  The reason that this is only a lower
limit for \(\alpha\) is that if we have overwhelming a priori evidence that
the null hypothesis is true (for instance, from previous empirical
studies, or because it follows from a well-supported theory), then a
Bayesian calculation would give a much higher value of \(\alpha\) than
\eqref{eq:type-I} does.  In our case, however, we have no strong
reasons for favoring any of the null hypotheses, so it is reasonable
to assume \(\alpha\) is close to the lower limit given in \eqref{eq:type-I}.

In order to choose a threshold \(p\)-value that counts as a
``significant'' result, one then needs to balance the risks of false
positives against the risks of \textit{false negatives}.  The false
negative probability, \(\beta\), also known as \textit{Type II error
  rate}, is the probability of failing to reject an untrue null
hypothesis.  That is, in the context of this paper, it is the
probability of failing to detect a real difference between two
sub-samples, or a real correlation between two variables.  The
complementary probability, \(1 - \beta\), is known as the
\textit{statistical power} or sensitivity of the test.  The value of
\(\beta\) depends on three factors:
\begin{enumerate}[1.]
\item The \textit{effect size}, which is a measure of the magnitude of
  the difference in a dependent variable between two sub-samples, or
  the degree of correlation between two continuous variables.  For the
  two sub-sample case, it is common to use a standardised mean
  difference, such as Cohen's \(d\) statistic \citep{Cohen:1988a}:
  \(d = (\bar{X}_A - \bar{X}_B) / s\), where \(\bar{X}_A\),
  \(\bar{X}_B\) are the means of the dependent variable \(X\) for
  samples A and B, while \(s\) is the pooled standard deviation of
  \(X\).  For the case of two continuous variables, the Pearson linear
  correlation coefficient, \(r\), can be used.  In both cases, rules
  of thumb have been developed \citep{Ruscio:2008a} for classifying an
  effect as ``large'' (\(d > 0.8\), \(r > 0.4\)) or ``small''
  (\(d < 0.2\), \(r < 0.1\)).  Alternatively, non-parametric
  statistics can be used, such as the \(A\) measure of stochastic
  superiority \citep{Delaney:2002a}.
\end{enumerate}

Obviously, this depends on the \textit{effect size}, which is the 


All astronomical data analysis is \emph{post hoc} analysis, since the universe was not set up to test a particular hypothesis (as far as we know).  It is therefore important to guard against the ``multiple comparisons problem'', whereby seemingly significant correlations are found where none really exist, simply by virtue of the large number of tests that were carried out.

Under the more conservative Holm--Bonferroni method, only comparisons with \(p < 0.001\) would be significant. 

The p-curve \citep{Head:2015a}

%%% Local Variables:
%%% mode: latex
%%% TeX-master: "quadrics-bowshock.tex"
%%% End:


\end{document}

%%% Local Variables:
%%% mode: latex
%%% TeX-master: t
%%% End:

