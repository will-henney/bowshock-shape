%\RequirePackage{amsmath}
\documentclass[useAMS, usenatbib, a4paper]{mnras}
\pdfsuppresswarningpagegroup=1

% The following is needed to fix the margins if using Letter-size paper
% REMOVE if your LaTeX uses A4 paper by default
%\addtolength\topmargin{-1.8cm}

% Standard LaTeX packages
% \usepackage[varg]{txfonts}
\usepackage{graphicx}
\usepackage{microtype}
\usepackage{xcolor}
\usepackage{fixltx2e}
\usepackage{booktabs}
\usepackage{siunitx}
\usepackage{color}
\usepackage{enumerate}
\usepackage{pdflscape}
\usepackage{rotating}
\usepackage{hyperref}

\usepackage[T1]{fontenc} 
\usepackage[utf8]{inputenc}

% Define a fermata symbol for use inline with text
% 04 Jul 2017 WJH
% Based on info in Secs 18.4.1 and 24.1 of musixdoc.pdf
\usepackage{musixtex}
\makeatletter
\newcommand\textfermata{%
  {\let\extractline\relax
    \setlines10\smallmusicsize \nobarnumbers \nostartrule
    \staffbotmarg0pt \setclefsymbol1\empty \global\clef@skip0pt
    \raisebox{0ex}[0ex][0ex]{
      \startextract\addspace{-\afterruleskip}%
      \notes\fermataup{-2}\en
      \zendextract}}}
\makeatother

% % The \eye command is in the dingbat package, but we need to take care
% % of a naming conflict with the AMS package
% \usepackage{savesym}
% \savesymbol{checkmark}
% \usepackage{dingbat}

% The "eye" symbol is drawn with the \faEye command of the fontawesome
% package
\usepackage{fontawesome}
% The ``irregular'' symbol is drawn from the Icelandic Sorcery and Witchcraft 
\usepackage{staves}

% Fonts 
\usepackage{newtxtext}
% Note: newtxmath must come AFTER newtxtext
\usepackage[varvw,smallerops]{newtxmath}


\hypersetup{colorlinks=True, linkcolor=blue!50!black, citecolor=black,
  urlcolor=blue!50!black}

\usepackage{etoolbox}
\robustify\bfseries
\robustify\itshape


%% Bold italic
\newcommand\hmmax{0}            % we don't need heavy fonts
\newcommand\bmmax{1}            % reduce use of math alphabets for bold
\usepackage{bm}

%% Bundled custom packages
\usepackage{aastex-compat}

% Definitions needed in abstract
\newcommand\hii{\ion{H}{ii}}


\title[Bow shock shapes]{True versus apparent shapes of bow shocks and bow waves}

\newcommand\AddressCRyA{Instituto de Radioastronom\'{\i}a y Astrof\'{\i}sica,
  Universidad Nacional Aut\'onoma de M\'exico, Apartado Postal 3-72,
  58090 Morelia, Michoac\'an, M\'exico}
\author[Tarango Yong \& Henney]{
  Jorge A. Tarango Yong \& William J. Henney\\
  \AddressCRyA
}
\begin{document}
\maketitle
\begin{abstract}
  Astrophysical bow shocks are a common result of the interaction
  between supersonic plasma flows, such as winds or jets from stars or
  active galaxies.  Similarly shaped radiation bow waves are produced
  by the relative motion between dust grains and a luminous source.
  For cylindrically symmetric bow shocks and waves, we develop a
  general theory of the effects of inclination angle, and propose a
  new two-dimensional classification scheme for bow shapes, which is
  based on dimensionless geometric ratios that can be estimated from
  observational images.  The two ratios are related to the
  ``flatness'' of the bow's apex and the ``openness'' of its wings.

  ADD MORE
\end{abstract}

% Force paper II to be before paper III in the reference list
%\nocite{Henney:2018a}

\section{Introduction}
\label{sec:intro}


%%
%% Circumstances when bowshocks arise
%%

The archetypal bow shock is formed when a solid body moves
supersonically through a compressible fluid.  Terrestrial examples
include the atmospheric re-entry of a space capsule, or the sonic boom
produced by a supersonic jet \citep{van-Dyke:1982a}.  In astrophysics
the term bow shock is employed more widely, to refer to many different
types of curved shocks that have approximate cylindrical symmetry.
Instead of a solid body, astrophysical examples usually involve the
interaction of \emph{two} supersonic flows, such as the situation of a
stellar wind emitted by a star that moves supersonically through the
interstellar medium \citep{van-Buren:1988a, Kobulnicky:2010a,
  van-Marle:2011a, Mackey:2012b, Mackey:2015a}.  In such cases, two
shocks are generally produced, one in each flow.  Sometimes,
especially in heliospheric studies \citep{Zank:1999a, Scherer:2014a},
the term ``bow shock'' is reserved for the shock in the ambient
medium, with the other being called the ``wind shock'' or
``termination shock''.  However, in other contexts such as colliding
wind binaries \citep{Stevens:1992a, Gayley:2009a} such a distinction
is not so useful.  

%% 
%% Examples of astrophysical bow shocks
%%

\begin{figure}
  \centering
  \bigskip
  \includegraphics[width=0.9\linewidth]{figs/bow-terminology}
  \caption{Descriptive terminology for a stellar bow shock.  The apex
    is the closest approach of the bow to the star, while the wings
    are the parts of the bow that curve back past the star.}
  \label{fig:bow-terminology}
\end{figure}
A further class of astrophysical bow shock is driven by highly
collimated, supersonic jets of material, such as the Herbig Haro
objects \citep{Schwartz:1978b, Hartigan:1987a} that are powered by
jets from young stars or protostars.  Additional examples are seen in
planetary nebulae \citep{Phillips:2010a, Meaburn:2013a}, active
galaxies \citep{Wilson:1987a}, and in galaxy clusters
\citep{Markevitch:2002a}.  In the jet-driven case, the term ``working
surface'' is often applied to the entire structure comprising the two
shocks plus the shocked gas in between them, separated by a
\textit{contact discontinuity}.  The working surface may be due to the
interaction of the jet with a relatively quiescent medium, or may be an
``internal working surface'' within the jet that is due to
supersonic temporal variations in the flow velocity
\citep{Raga:1990a}.

In empirical studies the relationship between these theoretical
constructs and the observed emission structures is not always clear.
In such cases the term ``bow shock'' is often used in a more general
sense to refer to the entire arc of emission.  In this paper, we will
concentrate on \textit{stellar bow shocks}, in which the position of
the star can serve as a useful reference point for describing the bow
shape.  The empirical terminology that we will employ is illustrated
in Figure~\ref{fig:bow-terminology}.  The \textit{apex} is the point
of closest approach of the bow to the star, which lies on the
approximate symmetry axis, and the region around the apex is sometimes
referred to as the \textit{head} of the bow.  The \textit{wings} are
the swept-back sides of the bow, which lie in a direction from the
star that is orthogonal to the axis, with the \textit{far wings} being
the wing region farthest from the apex. Finally, the \textit{tail} is
the region near the axis but in the opposite direction from the apex.

\begin{figure}
  % \includegraphics[width=\linewidth]{2winds-scheme}
  \includegraphics[width=\linewidth]{figs/generic-bowshock}
  \caption{Quasi-stationary bow shock structure formed by the
    interaction of two supersonic winds.  Lower-left inset box shows
    the case where the inner wind is anisotropic. The streamlines
    (thin lines) are drawn to be qualitatively realistic: they are
    straight in regions of hypersonic flow, but curved in subsonic
    regions, responding to pressure gradients in the shocked
    shells. Streamline slopes are discontinuous across oblique
    shocks.}
\label{fig:2-winds}
\end{figure}
\newcommand\Mach{\ensuremath{\mathcal{M}}} Figure~\ref{fig:2-winds}
shows an idealized schematic of how a double bow-shock shell is formed
from the interaction of two supersonic streams: an \textit{inner wind}
and an \textit{outer wind}, with the inner wind being the weaker of
the two (in terms of momentum), so that the shell curves back around
the inner source.  The outer wind may be from another star, or may be
a larger scale flow of the interstellar medium, such as the
\textit{champagne flow} produced by the expansion of an \hii{} region
away from a molecular cloud \citep{Tenorio-Tagle:1979a, Shu:2002a,
  Medina:2014a}.  Alternatively, it may be due to the supersonic
motion of the inner source through a relatively static medium, in
which case the outer wind will not be divergent as shown in the figure
but rather plane-parallel.  The thickness of the shocked shells at the
apex depends on the Mach number, \Mach{}, of the flows and the
efficiency of the post-shock cooling.  For sufficiently strong
cooling, the post-shock cooling zone thickness is negligible and the
shock can be considered isothermal.  In this case, the shell thickness
is of order \(\Mach^{-2}\) times the source-apex separation
\citep{Henney:2002a}, which can become very small for high Mach
numbers.  The shell thickness will tend to increase towards the wings,
due to the increasing shock obliqueness, which reduces the
perpendicular Mach number.

\begin{figure}
  \centering
  \includegraphics[width=\linewidth]{figs/bowshock-crw-variables}
  \caption[]{Schematic diagram of cylindrically symmetric two-wind
    interaction problem in the thin-shell limit, following
    \citet{Canto:1996}.}
  \label{fig:crw-schema}
\end{figure}
In the extreme thin-shell limit, the entire bow structure can be
treated as a surface.  The bow radius measured from the inner source
(star) is \(R(\theta, \phi)\), where \(\theta\) is the polar angle, measured from
the star-apex axis, and \(\phi\) is the azimuthal angle, measured around
that axis.  Assuming cylindrical symmetry about the axis, this reduces
to \(R(\theta)\), which is illustrated in Figure~\ref{fig:crw-schema},
following \citet{Canto:1996}.  The separation between the two sources
is \(D\) and the complementary angle, as measured at the position of
the outer source, is \(\theta_1\).  The minimum value of \(R(\theta)\) is the
stagnation radius, \(R_0\), which occurs at the apex (\(\theta = 0\)).  In
a steady state, ram-pressure balance on the axis implies that
\begin{equation}
  \label{eq:stagnation-radius}
  \frac{R_0} {D} = \frac{\beta^{1/2}} {1 + \beta^{1/2}} ,
\end{equation}
where \(\beta\) is the momentum ratio between the two winds.  If the winds
are isotropic, with inner wind mass-loss rate \(\dot{M}_{\w}\) and
terminal velocity \(V_{\w}\), while the outer wind has corresponding
values \(\dot{M}_{\w1}\) and \(V_{\w1}\), then the momentum ratio is
\begin{equation}
  \label{eq:beta-definition}
  \beta = \frac{\dot{M}_{\w} V_{\w}} {\dot{M}_{\w1} V_{\w1}} .
\end{equation}
The case where the outer wind is a parallel stream
\citep{Wilkin:1996a} corresponds to the limit \(\beta \to 0\), in which case
\(D\) is no longer a meaningful parameter.

% \TODO{Bow shocks from pulsars and neutron stars
%   \citep{Cordes:1993a, Brownsberger:2014a}.}


The paper is organized as follows.
%
In \S~\ref{sec:plan-alat-bow} we outline the geometric parameters that
are necessary for describing bow shapes and introduce two
dimensionless ratios: planitude and alatude.
%
In \S~\ref{sec:projection} we derive general results for the
projection of bow shapes on to the plane of the sky.
%
In \S~\ref{sec:conic} we apply the results to the simplest possible
class of geometric bow models: the quadrics of revolution, which
comprise spheroids, paraboloids, and hyperboloids, each of which
occupies a distinct region of the planitude--alatude plane.
%
In \S~\ref{sec:crw-scenario} we consider thin-shell hydrodynamic
models for the parallel-stream case (wilkinoids) and wind-wind case
(cantoids), including extension to an anisotropic inner wind
(ancantoids).  We calculate the location of the models in the
planitude--alatude plane as a function of the inclination of the bow
shock axis to the plane of the sky.
%
In \S~\ref{sec:more-realistic-bow} we test our methods against the
results of more realistic numerical simulations of bow shocks,
including the derivation of the shape parameters from maps of infrared
dust emission.
%
In \S~\ref{sec:obs} we apply our methods to example observations of
proplyd bow shocks in the Orion Nebula, paying close attention to the
systematic uncertainties that arise when our algorithms are applied to
real data.
%
In \S~\ref{sec:conc} we summarise our results and outline how
following papers will apply these ideas to a more extensive set of
observations, models and numerical simulations.


% The bow shocks we consider are originated by a source located at the
% origin, emitting a wind with a mass loss rate of $\dot{M}_{\w}$ and a
% terminal (supersonic) velocity $v_{\w}$. This wind interacts with another
% wind originated by another source located at a distance $D$ from the
% first one.  The mass loss rate of the second source is $\dot{M}_{\w1}$
% and the terminal velocity is $v_{\w1}$. The momentum of the wind of the
% second source is higher than the momentum of the first one, and the
% resultant bow shock is stationary due to pressure balance.

\section{Planitude and alatude of bow shapes}
\label{sec:plan-alat-bow}

\begin{figure}
  \centering
  \includegraphics[width=\linewidth]{figs/characteristic-radii}
  \caption[]{Parameters for characterizing a bow shape.  Bow radius
    from the star, measured parallel (\(R_0\)) and perpendicular
    (\(R_{90}\)) to the symmetry axis, together with radius of
    curvature (\(R_{\C}\)) at apex and asymptotic opening angle
    (\(\theta_\infty\)) of the far wings. }
  \label{fig:characteristic-radii}
\end{figure}

The stagnation radius \(R_0\) describes the linear scale of the bow
shock, but in order to characterize its shape more parameters are
required.  To efficiently capture the diversity of bow shapes, we
propose the parameters shown in Figure~\ref{fig:characteristic-radii}.
The perpendicular radius \(R_{90}\) is the value of \(R(\theta)\) at
\(\theta = 90^\circ\), whereas \(R_{\C}\) is the radius of curvature of the bow
at the apex (\(\theta = 0\)).  For a cylindrically symmetric bow, we show
in Appendix~\ref{sec:radius-curvature} that this is given by
\begin{equation}
  \label{eq:radius-curvature}
  R_{\C} 
  = \frac{R_0^2}{R_0 - R_{\theta\theta,0}} \ , 
\end{equation}
where \(R_{\theta\theta,0}\) is \(d^2 \!R / d\theta^2\) evaluated at \(\theta = 0\).

A fourth parameter is the asymptotic opening angle of the far wings,
\(\theta_\infty\), which is useful in the case that the wings are asymptotically
conical.  However, in many bow shocks the wings tend towards the
asymptotic angle only slowly, making \(\theta_\infty\) difficult to measure,
especially since the emission from the far wings is often weak at
best.  In contrast, the three radii, \(R_0\), \(R_{90}\), and
\(R_{\C}\). are straightforward to determine from observations.  One
simple method to estimate the radius of curvature is to make use of
the Taylor expansion\footnote{%
  This method assumes both that \(R(\theta)\) is even (true for a
  cylindrically symmetric bow) and that the orientation of the axis is
  already known.  Generalization to cases where these assumptions do
  not hold is discussed in Appendix~\ref{app:rcurv-empirical}.} %
of \(R(\theta)\) about the apex (with \(\theta\) in radians):
\begin{equation}
  \label{eq:taylor-R-theta}
  R(\theta) = R_0 + \frac12 R_{\theta\theta,0} \,\theta^2 + \mathcal{O}(\theta^4) \ ,
\end{equation}
so that fitting a polynomial in \(\theta^2\) to \(R(\theta)\) for
\(|\theta| < \Delta\theta \) yields \(R_0\) and \(R_{\theta\theta,0}\) from the first two
coefficients, and hence \(R_{\C}\) from
equation~\eqref{eq:radius-curvature}.  Experience has shown that
\(\Delta\theta = 30^\circ\) and three terms in the polynomial are good choices,
where the third term is used only as a monitor (if the co-efficient of
\(\theta^4\) is not small compared with \(R_0\), then it may indicate a
problem with the fit).

Since we have three radii, we can construct two independent
dimensionless parameters:
\begin{align}
  \label{eq:planitude}
  \text{Planitude} \quad \Pi & \equiv  \frac{R_{\C}} {R_0} \\
  \label{eq:alatude}
  \text{Alatude} \quad \Lambda & \equiv  \frac{R_{90}} {R_0}
\end{align}
and these will be the principal shape parameters that we will use in
the remainder of the paper.  The \textit{planitude}, \(\Pi\), is a
measure of the flatness of the head of the bow around the apex, while
the \textit{alatude}, \(\Lambda\), is a measure of the openness of the bow
wings.  Although ``planitude'' can be found in English dictionaries,
``alatude'' is a new word that we introduce here, derived from the
latin \textit{ala} for ``wing''.

Several previous studies have discussed the relation between
\(R_{90}\) and \(R_0\) as a diagnostic of bow shape (for example
\citealp{Robberto:2005a, Cox:2012a, Meyer:2016a}), but as far as we
know, we are the first to include \(R_{\C}\).  \citet{Robberto:2005a}
\S~4.2 use the ratios \(R_0/D\) and \(R_{90}/D\) in analyzing proplyd
bow shapes in the Trapezium cluster in the center of the Orion Nebula
\citep{Hayward:1994a, Garcia-Arredondo:2001a, Smith:2005a}.  In that
case, the source of the outer wind is known, and so \(D\) is
well-determined (at least, in projection), but for many bow shocks
\(D\) is not known, and is not even defined for the moving-star or
parallel-stream case. \citet{Cox:2012a} \S~4.1 compare the observed
shapes of bow shocks around cool giant stars with an analytic model,
and use \(A\) and \(B\) for the projected values of \(R_0\) and
\(R_{90}\), respectively (see next section for discussion of
projection effects).  \citet{Meyer:2016a} \S~3.2 analyze the
distribution of \(R_0 / R_{90}\) (the reciprocal of our \(\Lambda\)) for
hydrodynamic simulations of bow shocks around runaway OB stars.

% In order to contrast different bow shock models, we derive a set of
% measurable radii. Each model used should predict them and these
% predictions can be compared with observations.
% \begin{itemize}
% \item Radius at axis of symmetry. Denoted as $R_0$.
% \item Radius of Curvature at the axis of symmetry. Denoted as $R_{\C}$
% \item Radius at the perpendicular direction to the symmetry
%   axis. Denoted as $R_{90}$
% \item For open bow shocks, the asymptotic angle. Denoted as
%   $\theta_\infty$
% \end{itemize}



%% 
%% Restriction to cylindrical symmetry
%% 

% For simplicity, the current paper is restricted to cylindrically
% symmetric bow shock shapes. 

%%
%% Effects of instabilities
%%


%%% Local Variables:
%%% mode: latex
%%% TeX-master: "quadrics-bowshock"
%%% End:

\section{Generic bow shock model}
\label{sec:generic-model}

\begin{figure}
\includegraphics[width=\linewidth]{2winds-scheme}
\caption{Schematic representation of the two winds problem. Any point in the shell is located with the coordinates
$(R,\theta)$, and $\theta_1$ is measured from the external wind position}
\label{fig:2-winds}
\end{figure}

The bow shocks we consider are originated by a source located at the origin, emitting a wind with a mass loss rate of $\dot{M}_w$ and a terminal
(supersonic) velocity $v_w$. This wind interacts with another wind originated by another source located at a distance $D$ from the first one. 
The mass loss rate of the second source is $\dot{M}_{w1}$ and the terminal velocity is $v_{w1}$. The momentum of the wind of the second source is
higher than the momentum of the first one, and the resultant bow shock is stationary due to pressure balance. 

\subsection{Characteristic Radii}

In order to contrast different bowshock models, we  derive a set of measurable radii. Each model used should predict them and these predictions can be
compared with observations.

\begin{itemize}
\item Radius at axis of symemtry. Denoted as $R_0$. 
\item Radius of Curvature at the axis of symmetry. Denoted as $R_c$
\item Radius at the  perpendicular direction to the symmetry axis. Denoted as $R_{90}$
\item For open bow shocks, the assymptotic angle. Denoted as $\theta_\infty$
\end{itemize} 

\begin{figure}
\includegraphics[width=\linewidth]{ch-radii_ed}
\caption{Schematic representation of the characteristic radii $R_0$, $R_{90}$ and the radius of curvature at the symmetry axis $R_c$}
\end{figure}



%%% Local Variables:
%%% mode: latex
%%% TeX-master: "proplyd-bowshocks"
%%% End:

\section{Projection onto the plane of the sky}
\label{sec:projection}

In this section we calculate the apparent shape on the plane of the
sky of the limb brightened border of a shock or shell that is
idealized as as arbitrary cylindrically symmetric surface.

%Note: I'm aware that some of this material should be moved to an appendix, but I think it will be a future edition.
\subsection{Frames of reference}


Consider body-frame cartesian coordinates $(x,y,z)$, where \(x\) is
the symmetry axis, and spherical polar coordinates
\((R, \theta, \phi)\), where \(\theta\) is the polar angle and
\(\phi\) the azimuthal angle.  Since the surface is cylindrically
symmetric, it is can be specified as $R = R(\theta)$, so that
cartesian coordinates on the surface are:
\begin{equation}
\left(\begin{array}{c}
x \\ y \\ z
\end{array}
\right) = R(\theta)\left(\begin{array}{c}
\cos\theta \\
\sin\theta\cos\phi \\
\sin\theta\sin\phi
\end{array}\right).
\end{equation} 
Suppose that the viewing direction makes an angle \(i\) with the \(z\)
axis, so that we can define observer-frame coordinates
\((x', y', z')\), which are given by the rotating the body-frame
coordinates:
\begin{equation}
\left(\begin{array}{c}
x' \\ y' \\ z'
\end{array}
\right) = \left(\begin{array}{c}
x\cos i - z\sin i\\
y \\
z\cos i + x\sin i
\end{array}\right).
\label{eq:Trans}
\end{equation} 
In what follows, all quantities in the observer's frame are denoted by
attaching a prime to the equivalent quantity in the body frame.  The
coordinates \(x'\) and \(y'\) are in the plane of the sky, with \(x'\)
being the projected symmetry axis of the surface, while the line of
sight lies along \(-z'\).  The inclination angle \(i\) is defined so
that \(i = 0^\circ\) when the surface is viewed perpendicular to its
axis (\textit{side on}) and \(i = 90^\circ\) when it is viewed along
its axis (\textit{end on}).


\subsection{Tangent line}

The normal and tangent vectors to the shell's border in the shell's frame are given by:
\begin{align}
\hat{t} = \left(\begin{array}{c}
-\cos\alpha \\
\sin\alpha\cos\phi\\
\sin\alpha\sin\phi
\end{array}\right)\\
\hat{n} = \left(\begin{array}{c}
\sin\alpha \\
\cos\alpha\cos\phi \\
\cos\alpha\sin\phi
\end{array}\right)
\end{align}
Where:
\begin{align}
\tan\alpha = -\left.\frac{dy}{dx}\right|_{R(\theta)} \\
\label{eq:tanalpha}
\end{align}
Then we apply the transformation (\ref{eq:Trans}) to the normal and tangent vectors to obtain:
\begin{align}
\hat{n}' &= \left(\begin{array}{c}
(\cos\theta+\omega\sin\theta)\cos i -(\sin\theta-\omega\cos\theta)\sin i \sin\phi\\
(\sin\theta-\omega\cos\theta)\cos\phi \\
(\cos\theta+\omega\sin\theta)\sin i + (\sin\theta-\omega\cos\theta)\sin\phi\cos i
\end{array}\right)\\
\hat{t}' &= \left(\begin{array}{c}
-(\sin\theta-\omega\cos\theta)\cos i - (\cos\theta+\omega\sin\theta)\sin\phi\sin i \\
(\cos\theta+\omega\sin\theta)\cos\phi \\
-(\cos\theta+\omega\sin\theta)\sin i + (\sin\theta-\omega\cos\theta)\sin\phi\cos i
\end{array}\right)
\end{align}
Where $\omega = \frac{1}{R}\frac{dR}{d\theta}$

In the thin shell case, the limb brightened border of the shell is such that $\hat{n}'\cdot \hat{z}'$. 
The values for $\phi$ that satisfy this relation for each inclination $i$ are given by:
\begin{equation}
\sin\phi_t = \tan i\tan\alpha = \tan i \frac{1+\omega\tan\theta}{\omega-\tan\theta}
\label{eq:tanphi}
\end{equation}
With this, the coordinates of the limb brightened shell are given by:
\begin{equation}
\left(\begin{array}{c}
x'_t \\ y'_t \\ z'_t
\end{array}\right)= R(\theta)\left(\begin{array}{c}
\cos\theta\cos i - \sin\theta\sin\phi_t \sin i \\
\sin\theta(1-\sin^2\phi_t)^{1/2} \\
\cos\theta\sin i +\sin\theta\sin\phi_t\cos i
\end{array}\right)
\label{eq:tangential}
\end{equation} 

It is important to note that the equation (\ref{eq:tanphi}) does not have a solution for arbitrary values for $\theta$ and $i$, since
it's required that $|\sin\phi_t|<1$. If $i\neq 0$, then, the  allowed values for $\theta$ are such that $\theta > \theta_\parallel$, where
$\theta_\parallel$ is given implicitly by:
\begin{align}
\tan\theta_\parallel = \frac{|\tan i| + \omega(\theta_\parallel)}{1-\omega(\theta_\parallel) |\tan i|}
\label{eq:thetapar}
\end{align}

\subsection{Parallel and perpendicular projected shell radii}

Considering further applications to bow shocks, we will consider open shells. In order to compare the shell shape given by $R(\theta)$ with observations,
it is convenient to define the following apparent radii in the observer frame: $R'_\parallel$ and $R'_\perp$. These are projected distances of the shell tangent line
from the origin. The first is measured in the direction of the symetry axis, and the second in a perpendicular direction. More concretely $R'_\parallel = x'_t(y'_t=0)$
and $R'_\perp = y'_t(x'_t=0)$. From equations (\ref{eq:tanphi}) and (\ref{eq:tangential}) we find that:
\begin{align}
R'_\parallel = R(\theta_\parallel)\cos(\theta + i) \label{eq:Rpar} 
\end{align}
Where $\theta_\parallel$ is the solution of equation (\ref{eq:thetapar}), and
\begin{align}
R'_\perp = R(\theta_\perp)\sin\theta_\perp\left(1-\sin^2(\phi_t(\theta_\perp))\right)^{1/2}
\end{align}
Where $\theta_\perp$ is the solution of the next implicit equation:
\begin{align}
\cot\theta_\perp = \frac{1-\left(1+\omega(\theta_\perp)^2\sin^22i\right)^{1/2}}{2\omega(\theta_\perp\cos^2 i)}
\end{align}


%%% Local Variables:
%%% mode: latex
%%% TeX-master: "proplyd-bowshocks"
%%% End:


\section{Conic section approximation to bow shock shapes}
\label{sec:conic}

%%% Local Variables:
%%% mode: latex
%%% TeX-master: "proplyd-bowshocks"
%%% End:

In this section we will analyze the case where the resultant shape is a conic curve (circle, ellipse, parabola or hyperbola).
These curves are mathematical simple to model and give us a good reference to understand the effects of the projection effects
described in the last section on other bowshocks. Instead of the excentricity, we utilize the parameter $\theta_c$ to characterize the different curves, where
$\tan\theta_c = \frac{b}{a}$,  $b$ and $a$ are the typical parameters of conics. A positive value for $\theta_c$ indicates thethe given curve is a closed one, i.e
an ellipse, while a negative value indicates that is an hyperbola. %Insert figures if neccesary  

Due to the similitudes between the parametrization of the ellipse and the hyperbola, we can do the following parametrization:

\begin{align}
x = au_c(t)-x_0 \\ 
y = bv_c(t)
\end{align}

where:
\begin{align}
u_c(t) = \left\lbrace \begin{array}{c}
\cos t ~\mathrm{if~ellipse} \\
\cosh t ~mathrm{if~hyperbola}
\end{array}\right
\end{align}\\
v_c(t) = \left\lbrace \begin{array}{c}
\sin t ~\mathrm{if~ellipse} \\
\sinh t ~mathrm{if~hyperbola}
\end{array}\right \\
-\pi < t < \pi \\
R_0 = a - x_0 
\end{align}

\subsection{Projection onto the plane of sky} 

Once the parametrization is done, we can find the apparent shape of the shell in the observer's frame, following the procedure explained in section \ref{sec:projection}.

First of all, the intrinsic 3D shape of the shell is given by:

\begin{align}
x = au_c(t)-x_0 \\ 
y = bv_c(t)\cos\phi \\
z =  bv_c(t)\sin\phi
\end{align}

The azimutal angle where the line of sight is tangent to the shell is given by equation (\ref{eq:tanphi}) and (\ref{eq:tanalpha}), then:

\begin{align}
\tan\phi &= \frac{b}{a}w_c(t) 
\end{align}
where:

\begin{align}
w_c(t) = \left\lbrace \begin{array}{c}
\cot t ~\mathrm{if~ellipse} \\
-\coth t ~mathrm{if~hyperbola}
\end{array}\right
\end{align}
\subsection{Characteristic radii}


\newcommand\thC{\(\theta^1\)\,Ori~C}
\defcitealias{Canto:1996}{CRW}
\newcommand\CRW{\citetalias{Canto:1996}}


\section{Thin-shell bow shock models}
\label{sec:crw-scenario}

More physically realistic examples of bow shapes are provided by
steady-state hydrodynamic models for the interaction of hypersonic
flows in the thin-shell limit.  The classic examples are the solutions
for the wind--parallel stream and wind--wind problems (see
\S~\ref{sec:intro}) of \citet[][hereafter \CRW{}]{Canto:1996}, where
it is assumed that the two shocks are highly radiative and that the
post-shock flows are perfectly mixed to form a single shell of
negligible thickness. In this approximation, the shape of the shell is
found algebraically by \CRW{} from conservation of linear and angular
momentum, following an approach first outlined in
\citet{Wilkin:1996a}.  For the wind--stream case, the resulting bow
shape was dubbed \textit{wilkinoid} by \citet{Cox:2012a} and has the
form:
\begin{equation}
  \label{eq:wilkinoid-R-theta}
  R(\theta) = R_0\csc\theta\left( 3(1-\theta\cot\theta) \right)^{1/2} \ .
\end{equation}

For the wind--wind case, a family of solutions are found that depend on
the value of \(\beta\), the wind momentum ratio,\footnote{%
  By always placing the weaker of the two winds at the origin, it is
  only necessary to consider \(\beta \le 1\).  } %
see Figure~\ref{fig:crw-schema},
equations~(\ref{eq:stagnation-radius}, \ref{eq:beta-definition}), and
surrounding discussion in \S~\ref{sec:intro}.  We propose that these
shapes be called \textit{cantoids}.  The exact solution for the
cantoid shapes (eqs.~[23, 24] of \CRW{}) is only obtainable in
implicit form, but an approximate explicit solution (eq.~[26] of
\CRW{}) is very accurate for \(\beta \le 0.1\).  The wilkinoid shape
corresponds to the \(\beta \to 0\) limit of the cantoids.  Note that \CRW{}
employ cylindrical polar coordinates, \(z\) and \(r\), see our
Figure~\ref{fig:crw-schema}, and we follow this usage for the
thin-shell models discussed in this section.  \CRW{}'s \(z\) axis
corresponds to the cartesian \(x\) axis used in
sections~\ref{sec:projection} and \ref{sec:conic} of the current
paper, while the \(r\) axis corresponds to \(y\) when \(\phi = 0\).

A generalization of the cantoids to the case of an
anisotropic\footnote{Note that the wind anisotropy axis must be
  aligned with the star--star axis to maintain cylindrical symmetry.}
inner wind is developed next, giving rise
to what we call \textit{ancantoids}, which depend on an anisotropy
index, \(k\), in addition to \(\beta\).  

\subsection{Bow shocks from anisotropic wind--wind interactions}
\label{sec:ancantoid}
\begin{figure}
  \centering
  \includegraphics[width=\linewidth]{figs/anisotropic-arrows}
  \caption[]{Schematic diagram of wind flow patterns in isotropic and
    non-isotropic cases for different values of the anisotropy index,
    \(k\).  Arrow length represents the wind momentum loss rate per
    solid angle.}
  \label{fig:anisotropic-arrows}
\end{figure}


We wish to generalize the results of \citet[\CRW{}]{Canto:1996} to the
case where the inner wind is no longer isotropic, but instead has a
density that falls off with angle, \(\theta\), away from the symmetry axis.
Specifically, at some fiducial spherical radius, \(R_0\), from the
origin, the wind mass density is given by
\begin{equation}
  \label{eq:ancantoid-density}
  \rho(R_0, \theta) =
  \begin{cases}
    \rho_0 \cos^k \theta & \text{for \(\theta \le 90^\circ\)} \\
    0 & \text{for \(\theta > 90^\circ\)}
  \end{cases}
  \ ,
\end{equation}
where \(\rho_0\) is the density on the symmetry axis and \(k \ge 0\) is an
\textit{anisotropy index}.  The wind velocity is still assumed to be
constant and the wind streamlines to be radial, so the radial
variation of density at each angle is
\(\rho(R, \theta) = \rho(R_0, \theta)\, (R/R_0)^{-2}\) and the wind mass loss rate and
momentum loss rate per solid angle both have the same \(\cos^k\theta\)
dependence as the density.  Examples are shown in
Figure~\ref{fig:anisotropic-arrows} for a variety of different values
of \(k\).  As \(k\) increases, the wind becomes increasingly jet-like.

Our primary motivation for considering such an anisotropic wind is the
case of the Orion Nebula proplyds and their interaction with the
stellar wind of the massive star \thC{}
\citep{Garcia-Arredondo:2001a}.  The inner ``wind'' in this case is
the transonic photoevaporation flow away from a roughly hemispherical
ionization front, where photoionization equilibrium, together with
monodirectional illumination of the front, implies that the ionized
hydrogen density, \(n\), satisfies \(n^2 \propto \cos \theta\), which is
equivalent to \(k = 0.5\) in equation~\eqref{eq:ancantoid-density}.
Since the primary source of ionizing photons is the same star that is
the source of the outer wind, it is natural that the inner wind's axis
should be aligned with the star--star axis in this case.  For other
potential causes of wind anisotropy (for instance, bipolar flow from
an accretion disk), there is no particular reason for the axes to be
aligned, so cylindrical symmetry would be broken.  Nevertheless, we
calculate results for general \(k\) with aligned axes, so as to
provide a richer variety of cylindrically symmetric bow shock shapes
than are seen in the cantoids.

\begin{figure}
\includegraphics[width=\linewidth]{figs/ancantoid-shape}
\caption{Bow shock shapes for interacting winds in the thin-shell
  approximation: cantoids and ancantoids. Coordinates are normalized
  by $D$, the distance between the two wind sources, which are
  indicated by black dots on the axis.  The weaker source is at
  \((0.0, 0.0)\) and the stronger source is at \((1.0, 0.0)\).
  Results are shown for different values of the wind momentum ratio,
  \(\beta\) (different line widths), and for the case where the weaker
  wind is isotropic (black lines) or anisotropic (colored lines).}
\label{fig:r-beta}
\end{figure}


The general solution for the bow shock shape, \(R(\theta)\), in the \CRW{}
formalism is
\begin{equation}
  R(\theta) = \frac {\dot{J}_{\w} + \dot{J}_{\w{}1}}
  {\left(\dot{\Pi}_{\w{}r}+\dot{\Pi}_{\w{}r1}\right)\cos\theta
    - \left(\dot{\Pi}_{\w{}z}+\dot{\Pi}_{\w{}z1}\right)\sin\theta}
  \label{eq:Rmom}
\end{equation}
where \(\dot{\Pi}_{\w{}r}\), \(\dot{\Pi}_{\w{}z}\), \(\dot{J}_{\w}\) are
the accumulated linear radial momentum, linear axial momentum, and
angular momentum, respectively, due to the inner wind emitted between
the axis and \(\theta\). The equivalent quantities for the outer wind have
subscripts appended with ``1''.  The inner wind momenta for our
anisotropic case (replacing \CRW{}'s eqs.~[9, 10]) are:
\begin{gather}
  \label{eq:ancantoid-momenta}
  \begin{aligned}
    \dot{\Pi}_{\w{}z} &= \frac{k + 1}{2(k+2)}\, \dot{M}_{\w}^0 V_{\w}
    \max\left[\bigl(1- \cos^{k+2} \theta\bigr), 1 \right] \\
    \dot{\Pi}_{\w{}r} &= (k + 1)\, \dot{M}^0_{\w} V_{\w}\, I_k (\theta) 
  \end{aligned}
\end{gather}
where
\begin{equation}
  \label{eq:ancantoid-mass-loss}
  \dot{M}^0_{\w} = \frac{2 \pi} {k + 1} r_0^2 \rho_0 V_{\w}
\end{equation}
is the total mass-loss rate of the inner wind. The integral
\begin{equation}
  \label{eq:ancantoid-I-integral}
  I_k(\theta) = \int^{\max(\theta, \pi/2)}_0 \cos^k \theta \sin^2\theta \,d\theta 
\end{equation}
has an analytic solution in terms of the hypergeometric function,
\({}_2 F_1(-\tfrac12; \tfrac{1+k}2; \tfrac{3+k}2; \cos^2 \theta)\), but is
more straightforwardly calculated by numerical quadrature.  The
angular momentum of the inner wind about the origin is
\(\dot{J}_{\w} = 0\) because it is purely radial.  The outer wind
momenta are unchanged from the \CRW{} case, but are given here for
completeness:
\begin{gather}
  \label{eq:ancantoid-momenta-outer}
  \begin{aligned}
    \dot{\Pi}_{\w{}z1} & =
    -\frac{\dot{M}^0_{\w{}1}V_{\w{}1}}{4}\sin^2\theta_1\\
    \dot{\Pi}_{\w{}r1} & =
    \frac{\dot{M}^0_{\w{}1}V_{\w{}1}}{4}\left(\theta_1-\sin\theta_1\cos\theta_1\right)\\
    \dot{J}_{\w{}1} & =
    \frac{\dot{M}^0_{\w{}1}V_{\w{}1}}{4}\left(\theta_1-\sin\theta_1\cos\theta_1\right)D \ .
  \end{aligned} 
\end{gather}

\begin{figure}
  \centering
  \includegraphics[width=\linewidth]{figs/ancantoid-Pi-lambda-true}
  \caption{True shapes of cantoids and ancantoids in the
    \(\Pi\)--\(\Lambda\) plane, calculated according to results of
    App.~\ref{sec:thin-shell-shapes}.  For each line, \(\beta\) varies
    over the range \([0, 1]\) from lower left to upper right (although
    the black and red lines are truncated on the upper right), and
    line colors correspond to different anisotropy indices, matching
    those used in Fig.~\ref{fig:r-beta}. Circle symbols mark
    particular \(\beta\) values: \(0, 0.01, 0.1\), from largest to
    smallest circle.  Square symbols mark \(\beta = 0.5\), but with
    \(\Lambda\) calculated exactly, instead of using the approximation of
    equation~\eqref{eq:Lambda-approx}.  The white plus symbol marks
    the result for the wilkinoid:
    \((\Pi, \Lambda) = (\frac53, \sqrt{3})\).  Background shading indicates
    the domains of different quadric classes: hyperboloids (white),
    prolate spheroids (dark gray), and oblate spheroids (light gray).}
  \label{fig:ancantoid-Pi-lambda-true}
\end{figure}

We define \(\beta\) in this case as the momentum ratio \emph{on the symmetry axis}, which means that 
\begin{equation}
  \label{eq:ancantoid-momentum-ratio}
  \dot{M}^0_{\w{}1}V_{\w{}1} = 2 (k + 1)\, \beta\, \dot{M}^0_{\w} V_{\w} \ . 
\end{equation}
Substituting
equations~(\ref{eq:ancantoid-momenta}--\ref{eq:ancantoid-momentum-ratio})
into equation~\eqref{eq:Rmom} and making use of the geometric relation
between the interior angles of the triangle shown in
Figure~\ref{fig:crw-schema}:
\begin{equation}
  \label{eq:crw-angles}
  R \sin(\theta + \theta_1) = D \sin \theta_1 \ , 
\end{equation}
yields
\begin{equation}
  \label{eq:ancantoid-theta-theta1-implicit}
  \theta_1 \cot \theta_1 = 1 +
  2 \beta \left(
    I_k(\theta) \cot \theta
    - \frac{1 - \cos^{k+2} \theta} {k + 2} \right)   \ , 
\end{equation}
which is the generalization of \CRW{}'s equation~(24) to the
anisotropic case.  Equation~\eqref{eq:ancantoid-theta-theta1-implicit}
is solved numerically to give \(\theta_1(\theta)\), which is then combined with
equations~(\ref{eq:crw-angles}) and (\ref{eq:stagnation-radius}) to
give the dimensionless bow shape, \(R(\theta; \beta, k)/R_0\), where we now
explicitly indicate the dependence of the solution on two parameters:
axial momentum ratio, \(\beta\), and anisotropy index, \(k\).  We refer to
the resultant bow shapes as \textit{ancantoids}.



\begin{figure}
  \centering
  \includegraphics[width=\linewidth]{figs/ancantoid-angles}
  \caption{Equivalent quadric angles, \(\theta_{\Q}\), for ancantoids and
    cantoids.  Solid lines show values of \(\theta_{\Q}\) calculated from
    \((\Pi, \Lambda)\), which is representative of the shape of the head,
    while dashed lines show \(\theta_{\Q}\) calculated from
    \(\theta_\infty\), which is representative of the tail.  Dot-dashed line
    shows the result for cantoids, which differ from the \(k=0\)
    ancantoids in \(\theta_\infty\), but not in \((\Pi, \Lambda)\). Gray shading and
    line colors have the same meaning as in
    Fig.~\ref{fig:ancantoid-Pi-lambda-true}. }
  \label{fig:ancantoid-angles}
\end{figure}



\begin{figure}
  \centering
  \includegraphics[width=\linewidth]{figs/test_xyprime}
  \caption{Apparent bow shapes as a function of inclination angle for
    isotropic thin shell models. (a)~Confocal paraboloid for
    comparison (shape independent of inclination).
    (b)~Wilkinoid. (c)~Cantoid, \(\beta = 0.001\). (d)~Cantoid,
    \(\beta = 0.01\). }
  \label{fig:xyprime}
\end{figure}

\begin{figure}
  \centering
  \includegraphics[width=\linewidth]{figs/test_xyprime_ancantoid}
  \caption{Further apparent bow shapes as a function of inclination
    angle for anisotropic thin shell models (ancantoids).
    (a)~\(\beta = 0.001\), \(k = 0.5\); (b) (a)~\(\beta = 0.1\),
    \(k = 0.5\); (c)~\(\beta = 0.001\) \(k = 3\); (d)
    \(\beta = 0.1\) \(k = 3\).}
  \label{fig:xyprime-ancantoid}
\end{figure}


\subsection{True shapes of cantoids and ancantoids}
\label{sec:true-cantoids-ancantoids}

The shapes of the ancantoid bow shocks are shown in
Figure~\ref{fig:r-beta} for three different values of \(\beta\), and are
compared with the \CRW{} results for cantoids (dashed curves).  The
location of these shapes in the planitude--alatude plane is shown in
Figure~\ref{fig:ancantoid-Pi-lambda-true}, where the gray background
shading indicates the zones of different quadric classes, as in
\S~\ref{sec:conic}, Figures~\ref{fig:quadric-projection-continued}
and~\ref{fig:projected-R90-Rc-snapshots}.  Values of \(\Pi\) and
\(\Lambda\) are calculated via the analytic expressions derived in
Appendix~\ref{sec:ancantoid-planitude} and
\ref{sec:ancantoid-alatude}, respectively, which are only approximate
in the case of \(\Lambda\).  However, the filled square symbols show the
exact results for \(\beta = 0.5\), which can be seen to lie extremely close
to the approximate results, even for the worst case of \(k = 0\). The
leading term in the relative error of
equation~\eqref{eq:Lambda-approx} scales as \((\beta / (k + 2))^2\), so
the approximation is even better for smaller \(\beta\) and larger \(k\).

It is apparent from Figure~\ref{fig:r-beta} that the \(k=0\) ancantoid
is identical to the cantoid for \(\theta \le 90^\circ\) (\(z > 0\), to the right
of vertical dotted line in Fig.~\ref{fig:r-beta}), but is slightly
more swept back in the far wings.\footnote{%
  \label{fn:discontinuity}
  Due to the discontinuity in the inner wind density at
  \(\theta = 90^\circ\) (see Fig.~\ref{fig:anisotropic-arrows}), there is a
  discontinuity in the second derivative of the bow shape.} %
Since the true planitude and alatude depend on \(R(\theta)\) only in the
range \(\theta = [0, 90^\circ]\), the cantoid and the \(k = 0\) ancantoid
behave identically in Figure~\ref{fig:ancantoid-Pi-lambda-true}.
There is a general tendency for the bows to be flatter and more open
with increasing \(\beta\) and decreasing \(k\), with the cantoid being
most open at a given \(\beta\).  All the models cluster close to the
diagonal \(\Lambda \simeq \Pi\) in the planitude--alatude plane, but with a tendency
for \(\Lambda > \Pi\) at higher anisotropy.  There is therefore a degeneracy
between \(\beta\) and \(k\) for higher values of \(\beta\).  The wilkinoid
shape, which corresponds to the \(\beta \to 0\) limit of the cantoids, is
marked by a white plus symbol in
Figure~\ref{fig:ancantoid-Pi-lambda-true}, and lies in the prolate
spheroid region of the plane.  Cantoids lie either in the prolate
spheroid or hyperboloid regions, according to whether \(\beta\) is less
than or greater than about \(0.01\).  For ancantoids of increasing
\(k\), this dividing point moves to higher values of \(\beta\), until
almost the entire range of models with \(k = 8\) are within the
prolate spheroid zone.

However, the true planitude and alatude, which are what would be
observed for a side-on viewing angle (\(i = 0\)), are not at all
sensitive to the behavior of the far wings of the bow shock, which has
a rather different implication as to which variety of quadric best
approximates each shape.  We illustrate this is
Figure~\ref{fig:ancantoid-angles}, which shows two different ways of
estimating the quadric angle, \(\theta_{\Q}\) (see \S~\ref{sec:conic}).
The first is from \((\Pi, \Lambda)\), as in
Figure~\ref{fig:ancantoid-Pi-lambda-true}:
\newcommand\head{^{\text{head}}}
\newcommand\tail{^{\text{tail}}}
\begin{equation}
  \label{eq:thetaQ-head}
  \theta_{\Q}\head =
  \sgn{\bigl(2 \Pi - \Lambda^2\bigr)} \;
  \tan^{-1} \Abs{2\Pi - \Lambda^2}^{1/2} \ ,
\end{equation}
which follows from equations~\eqref{eq:Tq}, \eqref{eq:thetaQ}, and
\eqref{eq:Tq-from-Pi-Lambda}.  The second is from the asymptotic
opening angle of the wings, \(\theta_\infty\) (Fig.~\ref{fig:characteristic-radii}):
\begin{equation}
  \label{eq:thetaQ-tail}
  \theta_{\Q}\tail = \theta_\infty - 180^\circ \ , 
\end{equation}
where \(\theta_\infty\) is calculated from
equation~\eqref{eq:ancantoid-theta-inf} for ancantoids, or
\eqref{eq:cantoid-theta-inf} for cantoids.  If the bow shock shape
were truly a quadric, then these two definitions would agree.
However, as seen in Figure~\ref{fig:ancantoid-angles}, this is not the
case for the cantoids and ancantoids.  While
\(\smash[b]{\theta_{\Q}\head}\) generally corresponds to a prolate spheroid
(except for the largest values of \(\beta\)),
\(\smash[b]{\theta_{\Q}\tail}\) always corresponds to a hyperbola.  This
tension between the shape of the head and the shape of the far wings
has important implications for the projected shapes (as we will see in
the next section), since the far wings influence the projected
planitude and alatude when the inclination is large.


% \subsection{Characteristic Radii}
% $R_0$ is obtained directly from equation (27) of \CRW{} as the distance from the inner source where the RAM pressure of the interacting winds is in equilibrium.
% %Here goes a little introduction
% For the rest of the radii we need a relation between $\theta$ and $\theta_1$ as follows:


% \begin{align}
% \theta_1\cot\theta_1 -1 = 2\beta I_k(\theta) \cot\theta - \frac{2\beta}{k+2}\left(1-\cos^{k+2}\theta\right)
% \label{eq:th1th}
% \end{align}

% Equation (\ref{eq:th1th}) is reduced to equation (24) of \CRW{} when $k=0$.
% We can obtain $R_{90}$ by following the process shown in appendix \ref{app:r90-analytic},
% which lead us to a solution for $B \equiv \frac{R_{90}}{R_0}$:

% \begin{align}
% \tilde{R}_{90} = \frac{\sqrt(3\xi)\left(1+\beta^{1/2}\right)}{(1-\xi\beta)\left(1+\frac{1}{5}\xi\beta\right)^{1/2}}
% \label{eq:B}
% \end{align}

% Now, the solution for $R_c$ is explained in appendix \ref{app:rc-analytic},
% which lead us to derive the radius of curvature at the symmetry axis:

% \begin{align}
% R_c &= R_0\left(1-2\gamma\right)^{-1} \label{eq:Rcurv} \\
% \mathrm{where:~} & \gamma = \frac{C_{k\beta}}{1+\beta^{1/2}}+\frac{1}{6}(1-2\beta^{1/2})
% \end{align}

% Finally, using equations (\ref{eq:Rcurv}) and (\ref{eq:B}) we can estimate the parameter of
% conic curves $\theta_c$ as a function of $(\beta,\xi)$ using equation (\ref{eq:Thc})

% \begin{align}
% \tan^2\theta_c &= \left| \frac{3\xi\left(1+\beta^{1/2}\right)^2}{\left(1-\xi\beta\right)^2\left(1+\frac{1}{5}\xi\beta\right)}-\frac{2}{\left(1-2\gamma\right)}\right| 
% \label{eq:thc-CRW}
% \end{align}

% \begin{figure}
% \begin{tabular}{c}
% \includegraphics[width=\linewidth]{figs/AB-beta-log} \\
% \includegraphics[width=\linewidth]{figs/thc-beta-log}
% \end{tabular}
% \caption{Top: Characteristic radii, $\tilde{R}_c = R_c/R_0$ (solid lines) and $\tilde{R}_{90}
%   = R_{90}/R_0$ (dashed lines)
%   vs $\beta$, calculated from quadric fits to the generalized CRW
%   solutions with varying degrees of isotropy $\xi$.  Bottom: Conic
%   angle $\theta_c$ vs $\beta$ for the bow 
%   shock head (solid lines) and the bow shock tail (dashed lines).}
% \label{fig:rad-beta}
% \end{figure}


% % Observationally, we can measure the projected radii. In order to estimate the model parameters is neccesary to measure at least two of the mentioned radii,
% %being $R_0$ the
% %easiest to measure. Therefore, we may compare both $R_c$ and $R_{90}$ against $R_0$ as shown in figure (\ref{fig:prop-shell-rad}). 

% With this we can use the results of section \ref{sec:conic} to estimate the projected conic shapes for bow shocks with different winds 
% momenta and different density distributions. Figure \ref{fig:rad-beta} shows equations ( \ref{eq:B}), (\ref{eq:Rcurv}) and (\ref{eq:thc-CRW}) for different anisotropy indexes. 

% \subsection{Special Case: Isotropic Wind/Parallel Flow Interaction Problem}

% This problem has already been solved in \citet{Wilkin:1996a} and \CRW{}, which have an explicit form for the shell shape, given by:

% \begin{align}
%   R(\theta) = R_0\csc\theta\left[3(1-\theta\cot\theta)\right]^{1/2} \label{eq:R-Wilkin}
% \end{align}
% Where:
% \begin{align}
%   R_0 = \left(\frac{\dot{M}_wv_w}{4\pi\rho_a v_a^2}\right)^{1/2}
% \end{align}
% And $\rho_a$ and $v_a$ being the density and the velocity of the Parallel wind, respectively.
% The characteristic radii are given by:
% \begin{align}
%   \tilde{R}_{90} &= \sqrt{3} \\
%   \tilde{R}_c &= \frac{5}{3}
% \end{align}
% The derivation of these values is developed in appendix \ref{app:ch-rad-Wilkin}

% \subsection{Fits to the tail}
% \label{sec:fits-tail}

% In most cases, the quadric fit to the head of the bowshock does a very poor job of representing the ``wings'' or ``tail''.  The bowshock tail in all cases is asymptotically hyperbolic, with 

% We therefore 
% We carry out three-level nested fits to determine the 

% For the hyperbola ``center''
% \begin{multline}
%   \label{eq:tail-analytic-x0}
%   x_{0,\mathrm{t}} = 0.7 \beta^{-0.55} \biggl[
%     C_3 \bigl(\log_{10}\beta\bigr)^3 + C_2 \bigl(\log_{10}\beta\bigr)^2 
%   \\ + C_1 \log_{10}\beta + C_0
%   \biggr]
% \end{multline}
% \begin{equation}
%   \label{eq:tail-analytic-x0-minus-a}
%   (x_{0,\mathrm{t}} - a_{\mathrm{t}}) = D_2 (\log_{10}\beta)^2 + D_1 \log_{10}\beta + D_0
% \end{equation}
% where
% \begin{alignat}{2}
%   \label{eq:tail-analytic-coeffs-c}
%   C_k &= c_{2,k} \xi^2 + c_{1,k} \xi + c_{0,k} &\quad \text{for\ } k &= \{0, 1, 2, 3\} \\
%   \label{eq:tail-analytic-coeffs-d}
%   D_k &= d_{2,k} \xi^2 + d_{1,k} \xi + d_{0,k} &\quad \text{for\ } k &= \{0, 1, 2\}
% \end{alignat}


% % \newcommand\iso{\ensuremath{^{\mathrm{iso}}}}

% % \begin{table}
% %   \caption{Coefficients for hyperbola fit to bowshock tails}
% %   \label{tab:tail-fit-coeffs}
% %   \renewcommand\arraystretch{1.2}
% %   \setlength\tabcolsep{0.5\tabcolsep}
% %   \begin{tabular}{@{}llll@{}}
% %     \toprule
% %     Equation~(\ref{eq:tail-analytic-x0}) & 
% %     \multicolumn{3}{l}{
% %     Equation~(\ref{eq:tail-analytic-coeffs-c}) \dotfill
% %     } \\ \midrule
% %     \( C\iso_0 = +1.3195     \)    
% %     & \( c_{0,0} = +2.0758   \)  
% %     & \( c_{1,0} = -0.2309   \)  
% %     & \( c_{2,0} = -0.2532   \)\\
% %       \( C\iso_1 = +0.4229     \)    
% %     & \( c_{0,1} = +0.9571   \)  
% %     & \( c_{1,1} = -0.1530   \)  
% %     & \( c_{2,1} = -0.2487   \)\\
% %       \( C\iso_2 = +0.1092     \)    
% %     & \( c_{0,2} = +0.2528   \)  
% %     & \( c_{1,2} = -0.0360   \)  
% %     & \( c_{2,2} = -0.0794   \)\\
% %       \( C\iso_3 = +0.0051     \)    
% %     & \( c_{0,3} = +0.0171   \)  
% %     & \( c_{1,3} = -0.0010   \)  
% %     & \( c_{2,3} = -0.0095   \)\\ \midrule
% %     Equation~(\ref{eq:tail-analytic-x0-minus-a}) &
% %     \multicolumn{3}{l}{
% %     Equation~(\ref{eq:tail-analytic-coeffs-d}) \dotfill
% %     } \\ \midrule
% %     \( D\iso_0 = +0.7962   \)    
% %     & \( d_{0,0} = +0.8516 \)  
% %     & \( d_{1,0} = -0.0907 \)  
% %     & \( d_{2,0} = -0.2002 \)\\
% %       \( D\iso_1 = -0.2363   \)    
% %     & \( d_{0,1} = -0.7620 \)  
% %     & \( d_{1,1} = +0.1411 \)  
% %     & \( d_{2,1} = -0.0295 \)\\
% %       \( D\iso_2 = -0.0126   \)    
% %     & \( d_{0,2} = -0.0683 \)  
% %     & \( d_{1,2} = +0.0390 \)  
% %     & \( d_{2,2} = -0.0236 \)\\
% %     \bottomrule
% %   \end{tabular}
% % \end{table}




% % \begin{figure*}
% %   \centering
% %   \includegraphics[width=0.8\textwidth]{figs/conic-head-tail-analytic}
% %   \caption{Double quadric fits to thin shell
% %     solutions.}
% %   \label{fig:head-tail}
% % \end{figure*}

% % \begin{figure}
% %   \centering
% %   \includegraphics[width=\linewidth]{figs/two-quadric-th90-vs-i}
% %   \caption{Variation with inclination of the body-frame polar angle
% %     \(\theta_{90}\) that corresponds to the projected perpendicular radius
% %     \(R’_{90}\).  Line colors and thicknesses represent different
% %     quadrics of revolution, as in
% %     Fig.~\ref{fig:projected-R90-Rc-snapshots}.  This quantity is
% %     important because the quadric fits to the bow shock shape must be
% %     accurate for \(\theta < \theta_{90}\) in order to give a reliable estimate
% %     for \(R’_{90}\).}
% %   \label{fig:projected-th90}
% % \end{figure}



% \begin{figure*}
%   \centering
%   \includegraphics[width=0.8\textwidth]{figs/conic-head-tail-residuals}
%   \caption[]{Residuals of double quadric fits to thin shell solutions.
%     Residuals are expressed as the fractional error in the
%     complementary angle \(\theta_1\) (see Fig.~\ref{fig:crw-schema}) for
%     the same parameters as are shown in Fig.~\ref{fig:head-tail}.}
%   \label{fig:head-tail-residuals}
% \end{figure*}

% \subsection{Projection effects in the thin shell model}
% \label{sec:proj-effects-thin}


\begin{figure}
  \centering
  \setkeys{Gin}{width=\linewidth}
  \begin{tabular}{@{}l@{}}
    (a) \\
    \includegraphics{figs/ancantoid-R90-vs-Rc-a} \\
    (b) \\
    \includegraphics{figs/ancantoid-R90-vs-Rc-b}
  \end{tabular}
  \caption{Apparent projected shapes of wilkinoid, cantoids and
    ancantoids in the \(\Pi'\)--\(\Lambda'\) plane.  Colored symbols indicate
    the \(\abs{i} = 0\) position for \(\beta = 0.001\), \(0.003\),
    \(0.01\), \(0.03\), \(0.1\), \(0.3\).  Thin lines show the
    inclination-dependent tracks of each model, with tick marks along
    each track for 20 equal-spaced values of \(\abs{\sin i}\).  Gray
    shaded regions are as in
    Fig.~\ref{fig:quadric-projection-continued}a.  The wilkinoid track
    is shown in white. (a)~Isotropic wind model (cantoid) and
    proplyd-like model (ancantoid, \(k = 0.5\)). (b)~Hemispheric wind
    model (ancantoid, \(k = 0\)) and jet-like model (ancantoid,
    \(k = 3\)).}
  \label{fig:thin-shell-R90-Rc}
\end{figure}

\subsection{Apparent shapes of projected cantoids and ancantoids}
\label{sec:proj-shap-cant}

Figures~\ref{fig:xyprime} and~\ref{fig:xyprime-ancantoid} show the
apparent bow shapes of various thin shell models (wilkinoid, cantoids,
ancantoids) for different inclination angles \(\abs{i}\).  For
comparison, Figure~\ref{fig:xyprime}a shows the confocal paraboloid,
whose apparent shape is independent of inclination (see
Appendix~\ref{app:parabola}).  The wilkinoid (Fig.~\ref{fig:xyprime}b)
shows only subtle changes, with the wings becoming slightly more swept
back as the inclination increases.  The cantoids
(Fig.~\ref{fig:xyprime}c and d) behave in the opposite way, with the
wings becoming markedly more open once \(\abs{i}\) exceeds
\(60^\circ\) (for \(\beta = 0.001\)), or \(45^\circ\) (for
\(\beta = 0.01\)).  The ancantoids (Fig.~\ref{fig:xyprime-ancantoid}) can
show more complex behavior.  For instance, in the \(k = 0.5\),
\(\beta = 0.001\) ancantoid (Fig.~\ref{fig:xyprime-ancantoid}a) the near
wings begin to become more closed with increasing inclination up to
\(\abs{i} = 60^\circ\), at which point they open up again, whereas the
opening angle of the far wings increases monotonically with
\(\abs{i}\).

The inclination-dependent tracks that are traced by the thin-shell
models in the projected planitude--alatude plane are shown in
Figure~\ref{fig:thin-shell-R90-Rc}.  The behavior is qualitatively
different from the quadric shapes shown in
Figure~\ref{fig:quadric-projection-continued}a in that the tracks are
no longer confined within the borders of the region of a single type
of quadric (hyperboloid or spheroid). At low inclinations, many of the
models behave like the prolate spheroids, but then transition to a
hyperboloid behavior at higher inclinations, which is due to the
tension between the shape of the head and the shape of the far wings,
as discussed in the previous section. This can be seen most clearly in
the \(\beta = 0.001\), \(k = 0.5\) ancantoid (lowest red line in
Fig.~\ref{fig:thin-shell-R90-Rc}a, see also zoomed version in
Fig.~\ref{fig:convergence-cantoid-wilkinoid}). The track begins
heading towards \((\Pi', \Gamma') = (1, 1)\), as expected for a spheroid, but
then turns around and crosses the paraboloid line to head out on a
hyperboloid-like track.

Ancantoids with different degrees of inner-wind anisotropy are shown
in Figure~\ref{fig:thin-shell-R90-Rc}b.  In all cases, the tracks
follow hyperboloid-like behavior at high inclinations, tending to
populate the region just above the diagonal \(\Lambda' = \Pi'\).  The
\(k = 0\) ancantoids show a kink in their tracks at the point where
the projected apex passes through \(\theta = 90^\circ\), due to the
discontinuity in the second derivative of \(R(\theta)\) there (see
footnote~\ref{fn:discontinuity}).  The wilkinoid has a much less
interesting track, most clearly seen in the zoomed
Figure~\ref{fig:convergence-cantoid-wilkinoid}, simply moving the
short distance from \((\nicefrac53, \sqrt3)\) to
\((\nicefrac32, \smash{\sqrt{\nicefrac83}})\).  Despite its location
in the ellipsoid region of the plane, the fact that it has
\(\theta_\infty = 180^\circ\) means that it behaves more like a parabola at high
inclination, but converges on
\((\nicefrac32, \smash{\sqrt{\nicefrac83}})\) instead of \((2, 2)\)
since the far wings are asymptotically cubic, rather than quadratic.

\begin{figure}
  \centering
  \includegraphics[width=\linewidth]{figs/ancantoid-R90-vs-Rc-lobeta-a}
  \caption{As Fig.~\ref{fig:thin-shell-R90-Rc}a but zoomed in to show
    the wilkinoid track (white) and the convergence of the cantoid
    tracks (purple) to the wilkinoid as \(\beta \to 0\).}
  \label{fig:convergence-cantoid-wilkinoid}
\end{figure}

The local density of tick marks gives an indication of how likely it
would be to observe each portion of the track, assuming an isotropic
distribution of viewing angles.  It can be seen that the ticks tend to
be concentrated towards the beginning of each track, near the
\(\abs{i} = 0\) point, so the hyperboloid-like portions of the tracks
would be observed for only a relatively narrow range of inclinations.
This concentration becomes more marked as \(\beta\) becomes smaller, which
helps to resolve the apparent paradox that the wilkinoid corresponds
to the \(\beta \to 0\) limit of the cantoids, and yet follows a
qualitatively different track.  The detailed behavior of the
small-\(\beta\) cantoid models is shown in
Figure~\ref{fig:convergence-cantoid-wilkinoid}, which zooms in on the
region around the wilkinoid track.  It can be seen that for
\(\beta < 0.001\) the cantoid tracks begin to develop a downward hook,
similar to the \(k = 0.5\) ancantoids discussed above.  For
\(\beta < 10^{-4}\) this begins to approach the wilkinoid track and the
high inclination, upward portion of the track becomes less and less
important as \(\beta\) decreases.




% \begin{figure*}
%   \centering
%   \includegraphics[width=0.8\textwidth]{figs/two-quadric-R90-Rc-snapshots}
%   \caption{Variation with inclination angle of radii diagnostics}
%   \label{fig:thin-shell-R90-Rc-snapshots}
% \end{figure*}


%%% Local Variables:
%%% mode: latex
%%% TeX-master: "quadrics-bowshock"
%%% End:

\clearpage

\section{Shape of a dusty radiative bow wave}
\label{sec:shape-dust-wave}

As an alternative to hydrodynamic or magnetohydrodynamic bow shocks, it is possible that some observed emission arcs may be

\begin{figure}
  \centering
  \includegraphics[width=\linewidth]{figs/dust-trajectories}
  \caption[Dust grain trajectories]{Dust grain trajectories under
    influence of a repulsive central \(r^{-2}\) radiative force.  Dust
    grains approach from the right at a uniform velocity and with a
    variety of impact parameters (initial \(y\)-coordinate). The
    central source is marked by a red star at the origin, and its
    radiative force deflects the trajectories into a hyperbolic shape,
    each of which reaches a minimum radius marked by a small black
    square.  The incoming hyperbolic trajectories are traced in gray
    and the outgoing trajectories are traced in red.  The locus of
    closest approach of the outgoing trajectories is parabolic in
    shape (traced by the thick, light gray line) and this constitutes
    the inner edge of the bow wave. }
  \label{fig:dust-trajectories}
\end{figure}


%%% Local Variables:
%%% mode: latex
%%% TeX-master: "quadrics-bowshock"
%%% End:

\clearpage

\section{Discussion}
\label{sec:discussion}

In this section, we discuss the physical implications of our empirical
findings regarding bow shock shapes.  Our most reliable result is the
average shape of the OB bow shocks from the 227 MIPSGAL sources with
quality rating of 3~stars or higher.  This yields mean values of
\(\Pi = 1.78 \pm 0.06\) and \(\Lambda = 1.72 \pm 0.02\), or median values of
\(\Pi = 1.57\) and \(\Lambda = 1.69\).  The uncertainty quoted on the mean
values is the ``standard error of the mean'':
\(\text{sem} = \sigma / \sqrt{n}\), where \(\sigma\) is the rms dispersion and
\(n\) is the number of sources.  Note that in the case of the
planitude \(\text{sem}(\Pi) = 0.06\) is considerably smaller than
\(\text{mean}(\Pi) - \text{median}(\Pi) = 0.21\), so the latter would be a
more conservative estimate of the uncertainty.\footnote{This is
  because the distribution of \(\Pi\) is approximately log-normal, which
  yields a significant tail towards high values when converted to
  linear space.}  These values can be compared with the predictions of
the thin-shell wilkinoid model \citep{Wilkin:1996a}, which are
\(\Pi = 1.67\), \(\Lambda = 1.73\) when the bow shock axis lies in the plane
of the sky (following Paper~0, this is defined as the zero point of
the inclination angle, \(i\)).  When the axis is inclined, both
planitude and alatude are predicted to decrease but not by very much,
tending to \(\Pi = 1.5\), \(\Lambda = 1.63\) as
\(\abs{i} \to \ang{90}\) (see \S~5.3 of Paper~0).  The median observed
value\footnote{In the presence of outliers, the median is a more
  robust estimate of the central tendency than is the mean.} falls
squarely inside this range for both the planitude and alatude, which
is a remarkable triumph for the \citet{Wilkin:1996a} model.

On the other hand, turning now to the \emph{variety} of bow shock
shapes, we see that the wilkinoid can no longer explain our results.
The rms dispersions of the planitude and alatude distributions for the
MIPSGAL sources are \(\sigma(\Pi) = 0.87\) and
\(\sigma(\Lambda) = 0.30\) (Tab.~\ref{tab:big-p}), which are respectively 5 times
and 3 times larger than the total range of variation of \(\Pi\) and
\(\Lambda\) predicted for the wilkinoid surface.  Although some of the
dispersion is due to uncertainties in the observations and the fitting
algorithm, this contribution is expected to be small, especially for
the larger, well-resolved sources, for which systematic uncertainties
in the methods for determining \(\Pi\) and \(\Lambda\) will
dominate. Conservative upper limits to the relative size of these
uncertainties were estimated in \S~7 of Paper~0 to be \(< 20\%\) for
\(\Pi\) and \(< 10\%\) for \(\Lambda\), whereas the observed dispersions are
roughly twice as large: \(\sigma(\Pi)/\Pi = 55\%\) and
\(\sigma(\Lambda)/\Lambda = 18\%\).  Furthermore, the variations in planitude and
alatude are readily apparent by eye, as is demonstrated by the example
bow shock images shown in Figure~\ref{fig:mipsgal-shapes}b.  Sources
such as K123 have very typical shapes and fall near the center of the
\(\Pi\)--\(\Lambda\) distribution, whereas high-\(\Pi\) sources such as K447 have
a very flat apex region, while low-\(\Pi\) sources such as K566 have a
pointier, almost triangular apex.  High-\(\Lambda\) sources, such as K517,
have very open wings that bend away from the star, while
low-\(\Lambda\) sources such as K489 have closed wings and a semi-circular
appearance.


In Paper~0 we found that certain bow shock shapes can show a much
greater variation in their projected appearance as a function of
inclination angle than is seen for the wilkinoid.  For example, the
cantoids and ancantoids, which have asymptotically hyperbolic far
wings, can shift towards higher apparent planitude and alatude as the
inclination increases, generally with \(\Lambda \ge \Pi\) (Fig.~20 of Paper~0).
This might plausibly explain the vertical spur towards higher
\(\Lambda\) seen in the empirical distribution (\S~\ref{sec:ob-shapes}). A
different behavior is shown by bow shocks with very flat apex regions,
such as the MHD simulation from \citet{Meyer:2017a} that is analyzed
in the \S~6 of Paper~0.  This shows a high planitude \(\Pi\) when the
orientation is exactly edge-on, but \(\Pi\) decreases sharply along a
roughly horizontal track as the inclination \(\abs{i}\) increases
(Fig.~25 of Paper~0).  This is similar to the principal axis of
variation of the observed shapes (e.g.,
Fig.~\ref{fig:mipsgal-shapes}a).

If such variations in orientation do make a significant contribution
to the observed distribution of bow shock shapes in the
\(\Pi\)-\(\Lambda\) plane, then various predictions follow, which might be
observationally tested.  High-planitude sources with \(\Pi > 3\) would
be expected to have low inclinations, \(\abs{i} < \ang{30}\), whereas
high-alatude sources with \(\Lambda > 2\) would be expected to have high
inclinations, \(\abs{i} > \ang{45}\).  Unfortunately, determination of
the inclination for individual sources requires high resolution
spectroscopy of emission lines in order to map the kinematics of the
flow in the bow shock shell \citep[e.g.,][]{Henney:2013a}.  This is
not currently available for the majority of the MIPSGAL sources, which
are detected only by their dust continuum emission.  A further
prediction for the high-alatude sources is that the environmental flow
should be divergent rather than plane-parallel, in order to give a
cantoid shape instead of a wilkinoid.  This would tend to favor
``weather-vane'' cases, where the interstellar medium is flowing past
the star, and disfavor ``runaway'' cases, where the star is moving
through a static medium.  However, in \S~\ref{sec:corr-shape} we found
no significant difference in the shape distributions as a function of
the bow shock environment.  Figure~\ref{fig:mipsgal-uncorrelated}a
shows that the alatudes of sources that are facing \hii{} regions or
\SI{8}{\um} bright-rimmed clouds (and therefore might be expected to
be immersed in a champagne flow) are no higher than sources that are
isolated.


\begin{figure}
  (a)\\
  \includegraphics[width=\linewidth]
  {figs/wave-R90-vs-Rc-A020-N10}
  (b)\\
  \includegraphics[width=\linewidth]
  {figs/wave-R90-vs-Rc-A010-N20}
  \caption{Diagnostic diagram for perturbed shapes from standing wave
    oscillations.  Each model is characterized by a base shape
    (colored symbols, as described in key) and an amplitude, \(A\),
    and wavenumber, \(N\), of the oscillation: (a)~breathing mode with
    \(N = 1\), \(A = 0.2\); (b)~curling mode with \(N = 2\),
    \(A = 0.1\) (see Fig.~\ref{fig:perturb-shapes} for the
    phase-dependent intrinsic shapes).  The plotted points show the
    varying planitude and alatude of the projected bow shock shapes
    with uniform sampling over an entire period of the oscillation and
    for varying inclinations (sampled according to an isotropic
    distribution of orientations).  Each individual point is plotted
    with a low opacity so that the crowding of points in certain
    regions of the plane can be appreciated.}
  \label{fig:perturb-Rc-R90}
\end{figure}

An alternative explanation for the variety of observed bow shock
shapes is that they are due to time-dependent perturbations to an
underlying base shape.  For instance, multiple studies have shown that
stellar bow shock shells can be unstable \citep{Dgani:1996a,
  Dgani:1996b, Blondin:1998a, Comeron:1998a, Meyer:2014a}, leading to
large amplitude oscillations in the shell shape.  The oscillations are
found to be most vigorous when the post shock cooling is highly
efficient, allowing the formation of a thin shell (see Paper~I).  Even
in cases where the shell is stable, oscillations may be driven by
periodic variations in the stellar wind mass-loss rate or velocity, or
by inhomogeneities in the ambient stream.  Rather than using a
particular dynamical model of these oscillations, we instead crudely
simulate their effect by assuming a constant amplitude standing wave
perturbation to the base shape, as described in
Appendix~\ref{sec:perturbed-bows}.  Example results are shown in
Figure~\ref{fig:perturb-Rc-R90} for an ensemble of bows with different
orientations and phases of oscillation, considering three different
underlying base shapes.  It can be seen that modest amplitudes of 10
to 20\% can give rise to a distribution in \(\Pi\) and \(\Lambda\) similar to
that observed for the MIPSGAL sources when the oscillation wavelength
is of same order as the bow shock size.

An attractive feature of the oscillation hypothesis is that it
naturally explains why we find little correlation between the bow
shock shape and other source parameters (\S~\ref{sec:corr-shape}).
The one significant correlation that we do find is that the alatude
distribution is broader for bow shocks with larger angular sizes.
This might be explained if the relative amplitude of oscillations were
higher for sources with more powerful winds.  However, in that case it
is hard to understand why no such variation is seen in the width of
the planitude distribution.

We now address the difference in shape distribution between the
different classes of source.  Compared with the OB star bow shocks,
the cool star sample from \citet{Cox:2012a} shows a significantly
smaller alatude of \(\Lambda = 1.41 \pm 0.03\) (see
Fig.~\ref{fig:herschel-compare-mipsgal}).\footnote{Although the
  planitude also appears to be slightly smaller, this is not very
  statistically significant due to the small number of cool star
  sources and the large width of the OB star planitude distribution.}
Such a closed shape for the wings is inconsistent with the wilkinoid
value of \(\Lambda = \text{\numrange{1.63}{1.73}}\), which is surprising
given that the emission shells in these sources are relatively narrow
(Figs.~\ref{fig:herschel-arc-fits}
and~\ref{fig:herschel-arc-fits-poor}), so one might have thought that
the thin-shell approximation of \citet{Wilkin:1996a} would be
\emph{more} appropriate than for the OB stars, but this is clearly not
the case.  One possible explanation for this 



%%% Local Variables:
%%% mode: latex
%%% TeX-master: "obs-bowshocks"
%%% End:

\section{Conclusions}
\label{sec:conc}

%\newcommand\thC{\(\theta^1\)\,Ori~C}
\defcitealias{Canto:1996}{CRW}

We developed a method to estimate the shape of a generic bow shock product of the
interaction of two winds and was applied to the proplyds in the core of the ONC.
We started measuring the projected characteristic radii $(R'_0,R'_c)$ for each proplyd in our
sample and compare them with the ``conic equivalent'' of a two winds interaction model based 
on \CRW{} work to estimate the intrinsic bow shock shape and get the ionizing flux for ionization balance 
and the stagnation pressure for our sample of proplyds.
Most results are consistent with a proplyd's photoevaporated flow with an anisotropic density
distribution, with different anisotropy degrees. We ound that LV4 has the least anisotropic flow,
while LV2 has the most anisotropic flow. And for the 177-341, 169-348 and 180-331 we found out that 
the stellar wind is not enough to keep their bow socks stationary.

%%% Local Variables:
%%% mode: latex
%%% TeX-master: "proplyd-bowshocks"
%%% End:

\clearpage
\bibliographystyle{mnras}
%% All references should be put in the BibTeX file bowshocks-biblio.bib
\bibliography{bowshocks-biblio}
\appendix
\section{Paraboloids and their plane-of-sky projection}
\label{app:parabola}

Equation~\eqref{eq:par-xy} for the \(xy\) coordinates of a quadric in the \(\phi = 0\) plane cannot be used in the case of a paraboloid (\(\Q = 0\)).  Instead, a convenient parametrization is
\begin{gather}
  \label{eq:parabola-xy}
  \begin{aligned}
    x &= R_0 \left(1  - \tfrac{1}{2} \Pi\, t^2\right) \\
    y &= R_0\, \Pi\, t \ ,
  \end{aligned}
\end{gather}
where we have ``baked in'' knowledge of the planitude,
\(\Pi = R_c/R_0\) (see \S~\ref{sec:plan-alat-bow}). The projected
plane-of-sky coordinates of the tangent line follow from
equation~\eqref{eq:Trans} as
\begin{gather}
  \label{eq:parabola-xy-prime-phi}
  \begin{aligned}
    x_{\T}' / R_0 &= \left(1 - \tfrac{1}{2} \Pi\, t^2\right) \cos i
      + \Pi\, t \sin\phi_{\T} \sin i\\
    y_{\T}' / R_0 &= \Pi\, t \cos\phi_{\T}\ ,
  \end{aligned}
\end{gather}
The azimuth of the tangent line is found from
equations~(\ref{eq:alpha}, \ref{eq:tanphi}) as
\(\sin\phi_{\T} = -t^{-1} \tan i \), so that
\begin{gather}
  \label{eq:parabola-xy-prime-final}
  \begin{aligned}
    x_{\T}' / R_0 &= \cos i \left[ 1 + \tfrac{1}{2} \Pi \tan^2 i -
      \tfrac{1}{2} \Pi \left( t^2 - \tan^2 i \right) \right]\\
    y_{\T}' / R_0 &= \Pi \left( t^2 - \tan^2 i \right)^{1/2} \ .
  \end{aligned}
\end{gather}
The projected star--apex distance, \(R_0'\), is the value of
\(x_{\T}'\) when \(y_{\T}' = 0\), yielding
\begin{equation}
  \label{eq:parabola-R0-prime}
  R_0' / R_0 = \cos i \left( 1 + \tfrac{1}{2} \Pi \tan^2 i  \right) \ . 
\end{equation}
Note that this same result can be obtained from a Taylor expansion of
equation~\eqref{eq:fQi-factor} substituted into~\eqref{eq:R0-prime} in
the limit \(\Q \to 0\).

Equation~\eqref{eq:parabola-xy-prime-final} can be rewritten in the
form
\begin{gather}
  \label{eq:parabola-xy-all-primes}
  \begin{aligned}
    x_{\T}' &= R_0' \left(1  - \tfrac{1}{2} \Pi' t'^2\right) \\
    y_{\T}' &= R_0' \Pi' t' \ ,
  \end{aligned}
\end{gather}
where
\begin{align}
  \label{eq:parabola-Pi-prime}
  \Pi' &= \frac{2 \Pi} {2 \cos^2 i + \Pi \sin^2 i} \\
  \label{eq:parabola-t-prime}
  t' &= \cos i \left(t^2 - \tan^2 i\right)^{1/2} \ ,
\end{align}
which demonstrates that the projected shape is also a parabola. It is
apparent from \eqref{eq:parabola-Pi-prime} that the projected
planitude obeys
\begin{equation*}
\lim_{i \to 90^\circ} \Pi' = 2 \ ,
\end{equation*}
for all values of the true planitude \(\Pi\), as can be seen in
Figure~\ref{fig:quadric-projection}a.  Finally, the projected alatude
can be found as
\begin{equation}
  \label{eq:parabola-Lambda-prime}
  \Lambda' = \left( 2 \Pi' \right)^{1/2} \ .
\end{equation}
%%% Local Variables:
%%% mode: latex
%%% TeX-master: "quadrics-bowshock"
%%% End:

\section{Analytic derivation of the radius of curvature in the Thin Shell model}
\label{app:rc-analytic}

For small $\theta$ we may do a polynomial expansion for the shell shape such as:

\begin{align}
R \simeq R_0\left(1+\gamma\theta^2 + \Gamma\theta^4\right) \label{eq:R-exp}
\end{align}

The radius of curvature at the axis for $R$ is given by:

\begin{align}
R_c = R_0\left(1-2\gamma\right)^{-1}
\end{align}

The coefficient gamma may be derived by an expansion at small angles of equation
(\ref{eq:th1th}), as follows:

From the first term of the right side we get:

\begin{align}
\cot\theta &\simeq \theta^{-1}\left[1-\frac{1}{3}\theta^2\right] \\
\cos^k\theta\sin^2\theta &\simeq \theta^2 - \left(\frac{1}{3} + \frac{k}{2}\right)\theta^4 \\
\implies I_k(\theta) &\simeq \frac{1}{3}\theta^3\left[ 1 - \frac{1}{10}(3k+2)\theta^2\right]\\
\implies 2\beta I_k(\theta)\cot\theta &\simeq \frac{2}{3}\beta\theta^2\left[1-\frac{1}{30}
(9k+16)\theta^2\right]\label{eq:AR1} 
\end{align}

For the second term we get:

\begin{align}
-\frac{2\beta}{k+2}\left(1-\cos^{k+2}\theta\right) & \simeq -\beta\theta^2\left[1-\frac{1}{12}
(3k+4)\theta^2\right] \label{eq:AR2}
\end{align}

For the left side we use equation (25) from \CRW{}. Then, equation (\ref{eq:th1th}) results
as follows:

\begin{align}
\theta_1^2\left(1+\frac{1}{15}\theta_1^2\right) = \beta\theta^2\left[1+\frac{1}{60}(4-9k)
\theta^2\right] \label{eq:th1th-app}
\end{align}

And we can use the approximation $\theta_1 \approx \beta\theta^2$ for the correction term in
the left side of (\ref{eq:th1th-app}):

\begin{align}
\theta_1^2 &= \beta\theta^2\left[1+\frac{1}{60}(4-9k)\theta^2\right]
\left(1-\frac{\beta}{15}\theta^2\right) \\
\implies \theta_1^2 &= \beta\theta^2\left[1+ 2C_{k\beta}\theta^2\right]
\label{eq:th1th-small}\\
\mathrm{where:~} C_{k\beta} &\equiv \frac{1}{2}\left(A_k-\frac{\beta}{15}\right) \\
A_k &\equiv \frac{1}{15}-\frac{3k}{20}
\end{align}

Now, using equation (23) from \CRW{} we may estimate $R$ at low angles. To do this, we need to
expand each term as follows (neglecting terms of order four or higher):

\begin{align}
\theta_1 = &= \beta^{1/2}\theta\left[1+ 2C_{k\beta}\theta^2\right]^{1/2} \\
\theta + \theta_1 &= \theta\left[1+\beta^{1/2}\left(1+2C_{k\beta}\theta^2\right)\right]\\
\sin\theta_1 &= \theta_1\left[1-\frac{\theta_1^2}{6}\right] \\
 &= \beta^{1/2}\theta\left[1+\left(C_{k\beta}-\frac{1}{6}\beta\right)\theta^2\right] \\
 \sin(\theta+\theta_1) &= \left[\theta+\theta_1\right]\left[1-\frac{\left(\theta+\theta_1
 \right)^2}{6}\right] \\
 &= \theta\left(1+\beta^{1/2}\right)\left\lbrace 1+\left[\frac{C_{k\beta}\beta^{1/2}}
 {1+\beta^{1/2}}-\frac{1}{6}\left(1+\beta^{1/2}\right)^2\right]\theta^2\right\rbrace
\end{align}


So, combining these terms with equation (23) from \CRW{} we found the final expression for $R$:

\begin{align}
\frac{R}{D}\equiv \frac{\sin\theta_1}{\sin(\theta+\theta_1)} = \frac{\beta^1/2}{1+\beta^{1/2}}
\left\lbrace 1 + \theta^2\left[\frac{C_{k\beta}}{1+\beta^{1/2}}+\frac{1}{6}\left(1+2\beta^{1/2}
\right)\right] \right\rbrace \label{eq:r-small-theta}
\end{align}

Returning to equation (\ref{eq:R-exp}) we see the following:

\begin{align}
R_0 &= \frac{\beta^{1/2}}{1+\beta^{1/2}} \\
\gamma &= \frac{C_{k\beta}}{1+\beta^{1/2}}+\frac{1}{6}\left(1+2\beta^{1/2}\right)
\label{eq:app-gamma}
\end{align}

We recover equation (27) of \CRW{} for $R_0$ and equation (\ref{eq:app-gamma}) is the
needed term to calculate the radius of curvature at the axis.


%%% Local Variables:
%%% mode: latex
%%% TeX-master: "quadrics-bowshock"
%%% End:

\section{Analytic derivation of \texorpdfstring{\boldmath $R_{90}$}{R\_90} in the thin shell model}
\label{app:r90-analytic}

To derive $R_{90}$ we need to evaluate equations (23) from \CRW{} and (\ref{eq:th1th})
at $\theta=\frac{\pi}{2}$:

\begin{align}
R_{90} = D\tan\theta_{1,90} \\
\theta_{1,90}\cot\theta_{1,90} -1 = -\frac{2\beta}{k+2} \label{eq:th190}
\end{align}
Where $\theta_{1,90}\equiv \theta_1(\frac{\pi}{2})$. Combining both equations and  introducing
the parameter $\xi\equiv \frac{2}{k+2}$ we have:
\begin{align}
R_{90} &= D\frac{\theta_{1,90}}{1-\xi\beta} \label{eq:r90-incomplete}
\end{align}

Expanding the left side of (\ref{eq:th190}) until fourth order, equation (\ref{eq:th190})
becomes:

\begin{align}
\theta_{1,90}^2\left(1+\frac{\theta_{1,90}^2}{15}\right) \simeq 3\xi\beta
\end{align}

Applying the approximation $\theta_1^2 \approx 3\xi\beta$ we found a solution
for $\theta_{1,90}$:

\begin{align}
\theta_{1,90} = \left(\frac{3\xi\beta}{1+\frac{1}{5}\xi\beta}\right)^{1/2}
\end{align}

And substituting into (\ref{eq:r90-incomplete}) we find the solution for $R_{90}$:

\begin{align}
R_{90} &= \frac{\left(3\xi\beta\right)^{1/2}}{\left(1+\frac{1}{5}\xi\beta\right)^{1/2}
\left(1-\xi\beta\right)} \\
\implies \tilde{R}_{90} &\equiv \frac{R_{90}}{R_0} = \frac{\sqrt{3\xi}\left(1+\beta^{1/2}\right)}
{\left(1+\frac{1}{5}\xi\beta\right)^{1/2}\left(1-\xi\beta\right)}
\end{align}


%%% Local Variables:
%%% mode: latex
%%% TeX-master: "quadrics-bowshock"
%%% End:

\section{Derivation of Characteristic Radii in Isotropic Wind/Parallel interaction Problem}
\label{app:ch-rad-Wilkin}

$\tilde{R}_{90}$ is obtained by simply evaluating equation (\ref{eq:R-Wilkin}) at $\theta=\frac{\pi}{2}$.
For the Radius of curvature we follow a similar procedure than the Wind/Wind interaction, but using equation
(\ref{eq:R-Wilkin}) for $R(\theta)$ and inserting the cosecant into the square root.

Expanding the terms of $R(\theta)$ we find the following:

\begin{align}
  \csc^2\theta &\simeq \theta^{-2}\left[1+\frac{\theta^2}{3}\left(1+\frac{\theta^2}{5}\right)\right] \\
  1-\theta\cot\theta &\simeq \frac{\theta^2}{3}\left[1 + \frac{\theta^2}{15}\left(1+\frac{2\theta^2}{21}\right)\right] \\
  \implies \tilde{R}(\theta) \simeq 1 + \frac{\theta^2}{5} + O\left[\theta^4\right]
\end{align}

From equation (\ref{eq:Rcurv}) for the radius of curvature we finally get the numerical value for $\tilde{R_c} = \frac{5}{3}$


%%% Local Variables:
%%% mode: latex
%%% TeX-master: "quadrics-bowshock"
%%% End:


\end{document}

%%% Local Variables:
%%% mode: latex
%%% TeX-master: t
%%% End:

