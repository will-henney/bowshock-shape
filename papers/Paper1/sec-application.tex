\section{Application to proplyd bowshocks}
\label{sec:application}

Proplyds are comet-like structures observed in HII regions like Orion Nebula Cluster (ONC). 
These objects are interpreted as a D-type Ionization Front (IF) of a photoevaporated flow 
originated in the protoplanetary disk of a nearby low mass YSO \citep{Johnstone:1998}.
The pressure of the surrounding gas is not enough to confine this flow \citep{HA:1998}
may be formed by the interaction of the photoevaporated wind of the proplyds with the stellar wind of $\theta^1~C~Ori$, which is highly supersonic $(M \sim 100)$. 

The density distribution of the photoevaporated flow can be determined using the steady state continuity equation and assuming that almost all ionizing photons are absorbed at the IF \citep{HA:1998} and ignoring dust absorption, can be found that

\begin{align}
N(r_{IF},\theta) = N_0 \cos^{0.5}\theta
\label{eq:nprop}
\end{align}

This scenario can be described with the formalism of \citep{Canto:1996}, who proposes an algebraic solution for the two winds interaction problem in the thin shell approximation, in terms of a free parameter $\beta\equiv\frac{\dot{M}_wv_w}{\dot{M}_{w1}v_{w1}}$,
the winds momentum ratio, which the quantities with subindex 1 corresponds to the strongest wind, therefore $\beta<=1$. 

If we apply the \citep{Canto:1996} formalism for a photoevaporated flow with density given by (\ref{eq:nprop}), we find that the solutions for equations (8) - (11) of \citep{Canto:1996} are the following:

\begin{align}
\dot{M}_w &= \frac{\dot{M}_w^0}{3}\left(1-\cos^{3/2}\theta\right) \label{eq:dotprop} \\
\dot{\Pi}_z &= \frac{1}{5}v_w \dot{M}_w^0\left(1-\cos^{5/2}\theta\right)  \label{eq:pir}\\
\dot{\Pi}_r &= \frac{1}{2}v_w\dot{M}_w^0\left(\frac{4}{5}E\left(\frac{\theta}{2}|2\right) - \frac{1}{5}\sin 2\theta\cos^{1/2}\theta\right) \label{eq:piz}\\
\dot{J}_w = 0 \label{eq:jdot}
\end{align}  
The solution for (\ref{eq:piz}) includes an incomplete elliptic integral of the second kind. Combining equations  (\ref{eq:dotprop}) to (\ref{eq:jdot}) with equation (6) of \citep{Canto:1996}
numerically in python code, we can obtain the bow shock shape $R(\theta)$, as illustred in figure (\ref{fig:r-beta}).


% Combining equations (6),(9)-(11) and (19)-(23) from \citep{Canto:1996} and (\ref{eq:nprop}), we can solve numerically for $R(\theta)$, which gives us the shell's shape. Also we can
%obtain a relation between $\theta$ and $\theta_1$

%\begin{align}
%\theta_1\cot\theta_1 = 1+ 2\beta I(\theta)\cot\theta - \frac{4}{5}\beta\left(1-\cos^{5/2}\theta\right)
%\label{eq:th1th}
%\end{align}

%Where $I(\theta)\equiv \int^\theta_0 \cos^{1/2}\theta\sin^2\theta~d\theta$. The solution of this integral is an incomplete elliptical integral of the second kind:

%\begin{align}
%\int^\theta_0 \cos^{1/2}\theta\sin^2\theta~d\theta = \frac{4}{5}E\left(\frac{\theta}{2}|2\right) - \frac{1}{5}\sin 2\theta\cos^{1/2}\theta
%\end{align}

\begin{figure}
\includegraphics[width=\linewidth]{r-beta}
\caption{Bow shock shapes obtained with the \citep{Canto:1996} algebraic formalism using (\ref{eq:nprop}) to describe the proplyd wind density. The distance are normalized with $D$
to avoid dependency with this parameter.}
\label{fig:r-beta}
\end{figure}

\subsection{Characteristic Radii}

In order to find an analytic form of the characteristic radii $(R_0,R_c,R_{90})$, we may combine equations (\ref{eq:dotprop}) to (\ref{eq:jdot}) with (6) and (23) from \citep{Canto:1996}
and obtain the following relation:


\begin{align}
\theta_1\cot\theta_1 -1 =  \frac{8}{5}\beta E\left(\frac{\theta}{2}|2\right)\cot\theta - \frac{4}{5}\beta
\label{eq:th1th}
\end{align}

$R_0$ is obtained directly from equation (27) of \citep{Canto:1996} as the distance from the inner source where the RAM pressure of the interacting winds is in equilibrium.

We can obtain $R_{90}$ by the following process:
Evaluating equations  (\ref{eq:th1th}) and (23) from \citep{canto:1996} at $\theta=\frac{\pi}{2}$ we obtain the following:

\begin{align}
R_{90} &= D\tan\theta_{1,90} \\
\theta_{1,90}\cot\theta_{1,90} &= 1-\frac{4}{5}\beta \label{eq:th190}
\end{align}
Where $\theta_{1,90}\equiv \theta_�(\frac{\pi}{2})$. Combining both equations we have:

\begin{align}
R_{90} &= D\frac{\theta_{1,90}}{1-\frac{4}{5}\beta} 
\end{align}

Expanding (\ref{eq:th190}) until 3rd order and solving for $\theta_{1,90}$:
\begin{align}
\theta_{1,90} \simeq \left(\frac{2.4\beta}{1-0.8\beta}\right)^{1/2} \\
R_{90} \simeq \left(\frac{2.4\beta}{(1-0.8\beta)^3}\right)^{1/2}
\label{eq:r90}
\end{align}
%%%% Here we go in edition

Now, to obtaining $R_c$ we expand  (\ref{eq:th1th}) until order 2nd order for small $\theta$ values to find that for such values
$\theta_1 = \beta^{1/2}\theta$

Expanding equation (23) from \citep{Canto:1996} and substituting $\theta_1$ we found that:

\begin{align}
R \simeq R_0 \left(1+\frac{1+2\beta^{1/2}}{6}\theta^2\right)
\label{eq:R_approx}
\end{align}

Which lead us to derive the radius of curvature at the symmetry axis:

\begin{equation}
R_c = \frac{3}{2}R_0\left(1-\beta^{1/2}\right)^{-1}
\label{eq:Rcurv}
\end{equation}

Equations (\ref{eq:r90}),  (\ref{eq:Rcurv}) and (27) from \citep{Canto:1996} are labeled as ``analytic i=0'' in figure (\ref{fig:rad-beta}).

\begin{figure*}
\begin{tabular}{ccc}
\includegraphics[width=0.35\linewidth]{R0-b}&
\includegraphics[width=0.35\linewidth]{Rc-b} &
\includegraphics[width=0.35\linewidth]{R90-b}
\end{tabular}
\caption{$R_0$, $R_c$ and $R_{90}$ vsersus $\beta$ for a fixed set of inclinations. Equations (\ref{eq:r90}), (\ref{eq:Rcurv}), and (27) from \citep{Canto:1996}
are labeled as ``analytic i=0''. Note that the analytic solution deviates from the numeric solution when $\beta \gtrsim 0.1$}
\label{fig:rad-beta}
\end{figure*}

Observationally, we can measure the projected radii. In order to estimate the model parameters is neccesary to measure at least two of the mentioned radii, being $R_0$ the
easiest to measure. Therefore, we may compare both $R_c$ and $R_{90}$ against $R_0$ as shown in figure %(\ref{fig:radii-r0})

\subsection{Observational work}

We measured the Characteristic radii in Orion Nebula proplyds from observations in [OIII] from the Hubble Space Telescope (HST).

\subsubsection{Metodology}


Using the DS9 SAO image tools, we marked the positions of the proplyds and $\theta_1^C$. We followed the brightness contours along the 
bow shocks to mark the shell's position. In order to obtain the set of measurements $(R_0, R_c, D)$, we moved to the proplyd's reference frame, (where the proplyd's position)
is $(0,0)$ and measured distances in arcseconds. In this reference, the proplyd-$\theta_1^C$ line is $y=0$. With the shell's positions, we optimize a circle, whose radius is $R_c$, but
this optimization has two variants: the first one with the restriction that the center of the circle must lie in the symmetry axis (y=0), and the second variation is with no restrictions. The figure ()
shows a representation ofthe metodology described.

%\begin{figure}
%\end{figure} 

%\begin{figure*}
%\begin{tabular}{cc}
%\includegraphics[width=0.5\linewidth]{R90-R0-b}&
%\includegraphics[width=0.5\linewidth]{Rc-R0-b}
%\end{tabular}
%\label{fig:radii-r0}
%\caption{Left: $R_c$ normalized with $R_0$ vs $R_0$ normalized with $D$. Right:$R_{90}$ normalized with $R_0$ vs $R_0$ normalized with $D$. Since the bow shock brightness decays 
%at the wings, it is more difficult to measure $R_{90}$. Along every single curve, $\beta$ is constant. The marks in the curves represents jumps of $15^\circ$ in inclination, and the most left 
%mark in each curve represent $i=0^\circ$}
%\end{figure*}


%%% Local Variables:
%%% mode: latex
%%% TeX-master: "proplyd-bowshocks"
%%% End:
