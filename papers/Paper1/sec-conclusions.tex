\section{Conclusions}
\label{sec:conc}

%\newcommand\thC{\(\theta^1\)\,Ori~C}
\defcitealias{Canto:1996}{CRW}

We developed a method to estimate the shape of a generic bow shock product of the
interaction of two winds and was applied to the proplyds in the core of the ONC.
We started measuring the projected characteristic radii $(R'_0,R'_c)$ for each proplyd in our
sample and compare them with the ``conic equivalent'' of a two winds interaction model based 
on \CRW{} work to estimate the intrinsic bow shock shape and get the ionizing flux for ionization balance 
and the stagnation pressure for our sample of proplyds.
Most results are consistent with a proplyd's photoevaporated flow with an anisotropic density
distribution, with different anisotropy degrees. We ound that LV4 has the least anisotropic flow,
while LV2 has the most anisotropic flow. And for the 177-341, 169-348 and 180-331 we found out that 
the stellar wind is not enough to keep their bow socks stationary.

%%% Local Variables:
%%% mode: latex
%%% TeX-master: "proplyd-bowshocks"
%%% End:
