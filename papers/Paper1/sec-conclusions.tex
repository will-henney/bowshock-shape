\section{Conclusions}
\label{sec:conc}

We have shown that the shapes of stellar bow shocks can be usefully
characterized by two dimensionless numbers: the \textit{planitude},
\(\Pi\), or flatness of the bow's apex, and the \textit{alatude},
\(\Lambda\), or openness of the bow's wings.  The planitude and alatude can
be estimated from ratios of lengths that can be straightforwardly
measured from observations or theoretical models.  We develop a
general theory for the projected shape, \((\Pi', \Lambda')\), of a bow shock's
limb-brightened edge, or \textit{tangent line}, as a function of
inclination angle, \(i\), where the emission shell is idealized as a
cylindrically symmetric surface.

We first apply this theory to find inclination-dependent tracks on the
projected planitude--alatude plane for the special case of
\textit{quadric} surfaces, such as hyperboloids, paraboloids, and
spheroids, where the tangent line is a conic section.  The spheroids
and hyperboloids occupy distinct regions of the plane, with the
paraboloids defining the boundary between the two.  As the inclination
is increased from \(\abs{i} = 0\) (side-on) to \(\abs{i} = 90^\circ\)
(end-on), the tracks first tend to approach the diagonal
\(\Lambda' = \Pi'\), corresponding to confocal conics, always remaining within
their own region.  At the highest inclinations, the spheroids all
converge at \(\abs{i} = 90^\circ\) on the point
\((\Pi', \Lambda') = (1, 1)\) and the paraboloids on the point
\((\Pi', \Lambda') = (2, 2)\).  The hyperboloids, on the other hand diverge as
\((\Pi', \Lambda') \to (\infty, \infty)\) for a finite
\(i_{\mathrm{crit}}\), which depends on the hyperbolic eccentricity.
For \(\abs{i} > i_{\mathrm{crit}}\), the tangent line no longer exists
for the hyperbola, and it no longer would appear to be a bow shock.



We have presented a new two-dimensional classification scheme for the
analysis of the shapes of stellar bow shocks, which 


We developed a method to estimate the shape of a generic bow shock product of the
interaction of two winds as well as it apparent shape due to it orientation relative to the line of sight, asuuming
the bow shock is geometrically and optically thin.

In the general case we require numerical techniques, but is possible to handle analytically the apparent shape of
the quadrics of revolution. 

Also we made a generalization of \CRW{} work where the weakest interacing wind is hemispherical and anisotropic where its
density follows a power law of $\cos\theta$. Although other models also may be used.

In any case we developed a form to derive a set of measurable radii which are used as comparison between the \CRW{} model, the quadrics
of revolution and further observations. Our goal is to apply this work to the bow shock shapes observed in the core of the Orion Nebula Cloud
(ONC) due to the interaction of the photoevaporated wind from proplyds and the stellar wind of \thC{}.


How different regions of the \(\Pi\)--\(\Lambda\) plane are populated.
Bottom-right quadrant hard to get to (except for standing wave
oscillations), but may be due to finite shell thickness, which (for
low Mach number) will be more apparent in the wings, which might
decrease \(\Lambda\) more than \(\Pi\).  Fact that thin-shell solutions should
trace the contact discontinuity, but in some cases it may be only the
inner or the outer shell that is visible.


%and was applied to the proplyds in the core of the ONC.
%We started measuring the projected characteristic radii $(R'_0,R'_c)$ for each proplyd in our
%sample and compare them with the ``conic equivalent'' of a two winds interaction model based 
%on \CRW{} work to estimate the intrinsic bow shock shape and get the ionizing flux for ionization balance 
%and the stagnation pressure for our sample of proplyds.
%Most results are consistent with a proplyd's photoevaporated flow with an anisotropic density
%distribution, with different anisotropy degrees. We ound that LV4 has the least anisotropic flow,
%while LV2 has the most anisotropic flow. And for the 177-341, 169-348 and 180-331 we found out that 
%the stellar wind is not enough to keep their bow socks stationary.

%%% Local Variables:
%%% mode: latex
%%% TeX-master: "quadrics-bowshock"
%%% End:
