\section{Conic section approximation to bow shock shapes}
\label{sec:conic}

\newcommand\Sin{\ensuremath{\mathcal{S}}}
\newcommand\Cos{\ensuremath{\mathcal{C}}}
\newcommand\Cot{\ensuremath{\mathcal{T}}}

In this section we will analyze the case where the resultant shape  of the bow shock is a conic curve (circle, ellipse, parabola or hyperbola).
These curves are mathematical simple to model and give us a good reference to understand the effects of the projection effects
described in the last section on other bow shocks. The source of the inner wind is located at the origin, and the center of the conic is located at
a distance $x_0$ from the source.
%Instead of the excentricity, we utilize the parameter $\theta_c$ to characterize the different curves, where
%$\tan\theta_c = \frac{b}{a}$,  $b$ and $a$ are the typical parameters of conics. A positive value for $\theta_c$ indicates thethe given curve is a closed one, i.e
%an ellipse, while a negative value indicates that is an hyperbola. %Insert figures if neccesary  

Due to the similitudes between the parametrization of the ellipse and the hyperbola, we can do the following parametrization:

\begin{align}
x = a\Cos(t)-x_0 \\ 
y = b\Sin(t)
\end{align}

where:
\begin{align}
\Cos(t) = \left\lbrace \begin{array}{c}
\cos t ~\mathrm{if~ellipse} \\
\cosh t ~\mathrm{if~hyperbola}
\end{array}\right.\\
\Sin(t) = \left\lbrace \begin{array}{c}
\sin t ~\mathrm{if~ellipse} \\
\sinh t ~\mathrm{if~hyperbola}
\end{array}\right. \\
-\pi < t < \pi \\
R_0 = a - x_0 \\
a<0 ~ \mathrm{for~hyperbola}
\end{align}

\subsection{Projection onto the plane of sky} 

Once the parametrization is done, we can find the apparent shape of the shell in the observer's frame, following the procedure explained in section \ref{sec:projection}.

First of all, the intrinsic 3D shape of the shell is given by:

\begin{align}
x = a\Cos(t)-x_0 \\ 
y = b\Sin(t)\cos\phi \\
z =  b\Sin(t)\sin\phi
\end{align}

The azimutal angle where the line of sight is tangent to the shell is given by equation (\ref{eq:tanphi}) and (\ref{eq:tanalpha}), then:

\begin{align}
\sin\phi_t &= \frac{b}{a}\tan i\Cot(t) 
\end{align}
where:

\begin{align}
\Cot(t) = \left\lbrace \begin{array}{c}
\cot t ~\mathrm{if~ellipse} \\
-\coth t ~\mathrm{if~hyperbola}
\end{array}\right.
\end{align}

We could work in another frame   where the origin is located at the conic's center: $(X,Y)$, where $X=x-X_0$ and $Y=y$.

In this frame,  we use the transformations (\ref{eq:Trans})  to obtain the apparent shape of the shell:

\begin{align}
X' = \frac{\Cos(t)}{a\cos i}\left(a^2\cos^2 i \pm b^2\sin^2 i\right)  \label{eq:conic-projected-x}\\
Y'= b\Sin(t)\left(1-\frac{b^2}{a^2}\tan^2 i\Cot(t)\right)^{1/2}
\label{eq:conic-projected-y}
\end{align}


We must show that the apparent shape of the shell in the observer's frame is still the same conic. To do that, we can write equations
(\ref{eq:conic-projected-x}) and (\ref{eq:conic-projected-y}) as follows:

\begin{align}
X' = a'\Cos t' \label{eq:conic-projected-x-2} \\
Y' = b'\Sin t' \label{eq:conic-projected-y-2}
\end{align}

Comparing, (\ref{eq:conic-projected-x}), (\ref{eq:conic-projected-x-2}), (\ref{eq:conic-projected-y}) and (\ref{eq:conic-projected-y}) we find that

\begin{align}
a' = \left(a^2\cos^2 i \pm b^2\sin^2 i\right)^{1/2} \\
\Cos t' = \frac{\Cos(t)}{a\cos i} \\
b' = b
\end{align} 

Which proves that effectively the projection of a given conic is the same conic. With this knowledge, we can obtain the characteristic radii in
the observer's frame.

\begin{align}
D' = D\cos i \\
R'_0 = a' - (a-R_0)\cos i \\
R'_0 = \left(a^2\cos^2 i \pm b^2\sin^2 i\right)^{1/2}  - (a-R_0)\cos i
\end{align}

With $\tan\theta_c\equiv \frac{b}{a}$,  $f(i;\theta_c)\equiv\left(1-\tan^2\theta_c\tan^2i\right)^{1/2}$ and introducing the
radius of curvature $R_c\equiv \frac{b^2}{a}$ we find that:

\begin{align}
\frac{R'_0}{D'}={R_0}{D}\left(1-A\cot^2\theta_c(f(i;\theta_c)-1) \right)
\end{align}

Where $A\equiv \frac{R_c}{R_0}$
%%% Local Variables:
%%% mode: latex
%%% TeX-master: "proplyd-bowshocks"
%%% End:

