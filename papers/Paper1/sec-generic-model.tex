\section{Generic bow shock model}
\label{sec:generic-model}

\begin{figure}
\includegraphics[width=\linewidth]{2winds-scheme}
\caption{Schematic representation of the two winds problem. Any point in the shell is located with the coordinates
$(R,\theta)$, and $\theta_1$ is measured from the external wind position}
\label{fig:2-winds}
\end{figure}

The bow shocks we consider are originated by a source located at the origin, emitting a wind with a mass loss rate of $\dot{M}_w$ and a terminal
(supersonic) velocity $v_w$. This wind interacts with another wind originated by another source located at a distance $D$ from the first one. 
The mass loss rate of the second source is $\dot{M}_{w1}$ and the terminal velocity is $v_{w1}$. The momentum of the wind of the second source is
higher than the momentum of the first one, and the resultant bow shock is stationary due to pressure balance.
%%% Local Variables:
%%% mode: latex
%%% TeX-master: "proplyd-bowshocks"
%%% End:
