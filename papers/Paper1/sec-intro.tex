\section{Introduction}
\label{sec:intro}

%%
%% Circumstances when bowshocks arise
%%

The archetypal bow shock is formed when a solid body moves
supersonically through a compressible fluid.  Terrestrial examples
include the atmospheric re-entry of a space capsule, or the sonic boom
produced by a supersonic jet \citep{van-Dyke:1982a}.  In astrophysics
the term bow shock is employed more widely, to refer to many different
types of curved shocks that have approximate cylindrical symmetry.
Instead of a solid body, astrophysical examples usually involve the
interaction of \emph{two} supersonic flows, such as the situation of a
stellar wind emitted by a star that moves supersonically through the
interstellar medium \citep{van-Buren:1988a, Kobulnicky:2010a,
  van-Marle:2011a, Mackey:2012b, Mackey:2015a}.  In such cases, two
shocks are generally produced, one in each flow.  Sometimes,
especially in heliospheric studies \citep{Zank:1999a, Scherer:2014a},
the term ``bow shock'' is reserved for the shock in the ambient
medium, with the other being called the ``wind shock'' or
``termination shock''.  However, in other contexts such as colliding
wind binaries \citep{Stevens:1992a, Gayley:2009a} such a distinction
is not so useful.  

%% 
%% Examples of astrophysical bowshocks
%%

A further class of astrophysical bow shock is driven by highly
collimated, supersonic jets of material, such as the Herbig Haro
objects \citep{Schwartz:1978b, Hartigan:1987a} that are powered by
jets from young stars or protostars.  Additional examples are seen in
planetary nebulae \citep{Phillips:2010a, Meaburn:2013a}, active
galaxies \citep{Wilson:1987a}, and in galaxy clusters
\citep{Markevitch:2002a}.  In the jet-driven case, the term ``working
surface'' is often applied to the entire structure comprising the two
shocks plus the shocked gas in between them, separated by a
\textit{contact discontinuity}.  The working surface may be due to the
interaction of the jet with a relatively quiescent medium, or may be an
``internal working surface'' within the jet that is due to
supersonic temporal variations in the flow velocity
\citep{Raga:1990a}.

In empirical studies the relationship between these theoretical
constructs and the observed emission structures is not always clear.
In such cases the term ``bow shock'' is often used in a more general
sense to refer to the entire arc of emission. 

Bow shocks from pulsars and neutron stars
\citep{Cordes:1993a, Brownsberger:2014a}. 

Terminology: head, wings, nose, apex, limb, etc. 

Radiative versus non-radiative shocks. 

%% 
%% Restriction to cylindrical symmetry
%% 

For simplicity, the current paper is restricted to cylindrically
symmetric bowshock shapes. 

%%
%% Effects of instabilities
%%


%%% Local Variables:
%%% mode: latex
%%% TeX-master: "proplyd-bowshocks"
%%% End:
