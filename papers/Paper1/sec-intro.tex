\section{Introduction}
\label{sec:intro}

%%
%% Circumstances when bowshocks arise
%%

The archetypal bow shock is formed when a solid body moves
supersonically through a compressible fluid.  Terrestrial examples
include the atmospheric re-entry of a space capsule, or the sonic boom
produced by a supersonic jet \citep{van-Dyke:1982a}.  In astrophysics
the term bow shock is employed more widely, to refer to many different
types of curved shocks that have approximate cylindrical symmetry.
Instead of a solid body, astrophysical examples usually involve the
interaction of \emph{two} supersonic flows, such as the situation of a
stellar wind emitted by a star that moves supersonically through the
interstellar medium \citep{van-Buren:1988a, Kobulnicky:2010a,
  van-Marle:2011a, Mackey:2012b, Mackey:2015a}.  In such cases, two
shocks are generally produced, one in each flow.  Sometimes,
especially in heliospheric studies \citep{Zank:1999a, Scherer:2014a}, the
term ``bow shock'' is reserved for the shock in the ambient medium,
with the other being called the ``wind shock'' or ``termination
shock''.  However, in other contexts such as colliding wind binaries
\citep{Stevens:1992a, Gayley:2009a} such a distinction is not so
useful.  In this paper we use the term ``bow shock'' in a more general
sense to refer to either of the two shocks, or to the shocked gas in
between them. 


%% 
%% Examples of astrophysical bowshocks
%%

%% 
%% Restriction to cylindrical symmetry
%% 

For simplicity, the current paper is restricted to cylindrically
symmetric bowshock shapes. 

%%
%% Effects of instabilities
%%


%%% Local Variables:
%%% mode: latex
%%% TeX-master: "proplyd-bowshocks"
%%% End:
