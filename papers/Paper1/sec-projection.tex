\section{Projection onto the plane of the sky}
\label{sec:projection}


%%% Local Variables:
%%% mode: latex
%%% TeX-master: "proplyd-bowshocks"
%%% End:

In this section we calculate the apparent shape of a limb brightened  border of a generic bow shock with cilindrical geometry, observed in the
plane of the sky.


\subsection{Frames of reference}

We can specify the shell's position in cartesian coordinates $(x,y,z)$, where the x axis points towards the symetry axis of the system.
Since the shell is cilindrically symetric, we can describe it's shape fully with the function $R(\theta)$, such that:

\begin{equation}
\left(\begin{array}
x \\ y \\ z
\end{array}
\right) = R(\theta)\left(\begin{array}
\cos\theta \\
\sin\theta\cos\phi \\
\sin\theta\sin\phi
\end{array}\right)
\end{equation} 

Where $\theta$ is the polar angle and $\phi$ the azimutal angle from spherical coordinates

The observer's reference frame is denotated with primes, and is rotated respect y axis by an angle $i$. The transformation between the shell frame and the observed
frame is given by:

\begin{equation}
\left(\begin{array}
x' \\ y' \\ z'
\end{array}
\right) = \left(\begin{array}
x\cos i - z\sin i\\
y \\
z\cos i + x\sin i
\end{array}\right)
\end{equation} 

\subsection{Tangent line}

The normal and tangent vectors to the shell's border in the shell's frame are given by:

\begin{align}
\hat{t} = \left(\begin{array}
\end{array}\right)\\
\hat{n} = \left(\begin{array}
\end{array}\right)
\end{align}

Then we apply the transformation to the observer's frame to obtain:

\begin{align}
\end{align}

In the thin shell case, the limb brightened border of the shell is such that $\hat{n}'\cdot \hat{z}'$. 