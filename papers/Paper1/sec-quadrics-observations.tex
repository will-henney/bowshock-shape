\newcommand\hii{\ion{H}{ii}}


\section{Comparison with observations}
\label{sec:comp-with-observ}


Placing various classes of objects on the \(R_{90}\)--\(R_c\) plane:
\begin{itemize}
\item LL arcs
\item runaway O stars
\item AGB stars
\end{itemize}

\subsection{Mid-infrared arcs around early-type stars}
\label{sec:mid-infrared-arcs}

The most extensive sample of stellar bow shock nebulae to date is a
catalog of 709 arcs \citep{Kobulnicky:2016a} detected in mid-infrared
surveys of the Galactic Plane by the \textit{Spitzer Space Telescope}
(\textit{SST}, \citealp{Werner:2004a}) and \textit{Wide-field Infrared
  Survey Explorer} (\textit{WISE}, \citealp{Wright:2010a}).  These
sources are believed to be powered by the winds of early-type stars,
which are either moving supersonically through the interstellar medium
(runaway stars), or are interacting with a local bulk flow, such as
the champagne flow from a nearby \hii{} region.


\begin{figure*}
  \setlength\tabcolsep{0pt}
  \begin{tabular}{ll}
    (a)
    & (b) \\
    \includegraphics
    [width=0.5\linewidth, trim=20 15 30 10, clip]{figs/0510-3-star}
    & \includegraphics
      [width=0.5\linewidth, trim=20 15 30 10, clip]{figs/0506-4-star} \\
    (c)
    & (d) \\
    \includegraphics
    [width=0.5\linewidth, trim=20 15 30 10, clip]{figs/0517-5-star}
    & \multicolumn{1}{c}
      {\includegraphics[width=0.35\linewidth, trim=20 20 60 60]
      {figs/mipsgal-r0-r0-plus-dPA-edited}}
  \end{tabular}
  \caption[]{Examples of typical fits to the bow shock shapes of
    MIPSGAL sources with different star ratings: (a) K510, 3-star
    rating; (b) K506, 4-star rating; (c) K517, 5-star rating.  Right
    panels of parts (a)--(c) show a 4\('\) square \SI{24}{\um} image,
    centered on each source.  Contours are ten linearly spaced levels
    between the median brightness of the entire image and the maximum
    brightness of the bow shock arc. Grids of galactic coordinates
    (light blue lines, parallel to the box sides) and equatorial
    coordinates (tilted magenta lines) are shown.  The stellar source
    and the bow shock axis, as determined by \citet{Kobulnicky:2016a}
    are indicated by an orange star and an orange line, respectively,
    where the line extends from \(-2 R_0\) to \(+2 R_0\).  The
    automatically traced arc shapes using the ``mean'' and ``peak''
    methods (see text) are shown by blue and red dots, respectively.
    The magenta circle shows the fit to the arc points within
    \(\pm 45^\circ\) of the nominal bowshock axis, with the magenta dot
    showing the center of curvature and the magenta line showing the
    fitted bow shock axis, which is the line passing through the
    source and the center of curvature.  Left panels of parts (a)--(c)
    show the radius measured from the source (upper panel) and
    brightness (lower panel) of the arc points, plotted as a function
    of angle \(\theta\) from the nominal bow shock axis, and with the same
    color coding as used on the image. Angular ranges of
    \(\theta = \pm 45\degr\) and \(\pm 90\degr\) are shown by gray shaded
    boxes In the upper panel, the \(R_0\) value tabulated by
    \citet{Kobulnicky:2016a} is shown by a horizontal blue line. (d)
    Comparison of the bow shock sizes (scatter plot) and position
    angles (inset histograms) determined from our fits with those
    tabulated by \citet{Kobulnicky:2016a} for the MIPSGAL sources.}
  \label{fig:mipsgal-examples}
\end{figure*}



\begin{figure*}
  \centering
  \begin{tabular}{ll}
    (a) & (b) \\
    \includegraphics[width=0.45\textwidth]{figs/mipsgal-Rc-R90-zoom} &
    \includegraphics[width=0.45\textwidth]{figs/mipsgal-Rc-R90-thumbnails} 
  \end{tabular}
  \caption[]{MIPSGAL sources on the bow shock shape diagnostic
    diagram.  (a) Individual sources.  (b) Kernel density estimate of
    the distribution.}
  \label{fig:mipsgal-shapes}
\end{figure*}


\begin{figure*}
  \centering
  \includegraphics[width=\textwidth]{figs/mipsgal-pairplot}
  \caption[]{Correlations between non-shape parameters of MIPSGAL sources.}
  \label{fig:mipsgal-pairplot}
\end{figure*}

\begin{figure*}
  \centering
  \begin{tabular}{ll}
    (a) & (b) \\
    \includegraphics[width=0.45\textwidth]{figs/mipsgal-Rc-R90-R0} &
    \includegraphics[width=0.45\textwidth]{figs/mipsgal-Rc-R90-Mag} 
  \end{tabular}
  \caption[]{Comparison between the distribution of bowshock shapes
    when the sources are divided into two sub-samples according to the
    value of another parameter. (a)~Bow shock angular size, \(R_0\).
    (b)~Extinction-corrected \(H\)-band magnitude of the stellar
    source.}
  \label{fig:mipsgal-correlated}
\end{figure*}

\begin{figure*}
  \centering
  \begin{tabular}{ll}
    (a) & (b) \\
    \includegraphics[width=0.45\textwidth]{figs/mipsgal-Rc-R90-environment} &
    \includegraphics[width=0.45\textwidth]{figs/mipsgal-Rc-R90-candidates} 
  \end{tabular}
  \caption[]{Lack of significant correlation of bow shock shape with
    (a) environment, and (b) uncertainty in stellar source
    identification.}
  \label{fig:mipsgal-uncorrelated}
\end{figure*}


\subsection{Far-infrared arcs around late-type stars}
\label{sec:far-infrared-arcs}

\begin{figure}
  \centering
  \includegraphics[width=\linewidth]{figs/mipsgal-Rc-R90-vs-Herschel}
  \caption[]{Comparison of RSG/AGB arcs with OB star arcs.}
  \label{fig:herschel-compare-mipsgal}
\end{figure}



\subsection{Stationary emission line arcs in M42}
\label{sec:stat-emiss-line}

\begin{figure*}
  \centering
  \includegraphics[width=\textwidth]{figs/annotated-ll-arcs}
  \caption[]{Stationary bow shock arcs in the Orion Nebula.}
  \label{fig:ll-arcs}
\end{figure*}

\begin{figure}
  \centering
  \includegraphics[width=\linewidth]{figs/mipsgal-Rc-R90-vs-Orion}
  \caption[]{Comparison of Orion with OB stars.}
  \label{fig:ll-compare-mipsgal}
\end{figure}




%%% Local Variables:
%%% mode: latex
%%% TeX-master: "quadrics-bowshock.tex"
%%% End:
