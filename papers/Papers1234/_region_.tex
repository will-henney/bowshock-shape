\message{ !name(bs-bw-dw-03.tex)}\documentclass[useAMS, usenatbib, a4paper]{mnras}
\pdfsuppresswarningpagegroup=1

\usepackage{graphicx}
\usepackage{microtype}
\usepackage{xcolor}
\usepackage{fixltx2e}
\usepackage{booktabs}
\usepackage{siunitx}
\sisetup{separate-uncertainty = true}
\usepackage{color}
\usepackage{enumerate}
\usepackage{pdflscape}
\usepackage{rotating}
\usepackage{xr-hyper}
\usepackage{hyperref}

\usepackage[T1]{fontenc} 
\usepackage[utf8]{inputenc}

% Fonts 
\usepackage{newtxtext}
% Note: newtxmath must come AFTER newtxtext
\usepackage[varvw,smallerops]{newtxmath}

\usepackage{chemgreek}
\activatechemgreekmapping{newtx}
\usepackage{listings}

\hypersetup{colorlinks=True, linkcolor=blue!50!black, citecolor=black,
  urlcolor=blue!50!black}

\usepackage{etoolbox}
\robustify\bfseries
\robustify\itshape

%% The following hack solves a problem with
%% ERROR: \pdfendlink ended up in different nesting level than \pdfstartlink.
%% See https://tex.stackexchange.com/a/249743
\makeatletter
\patchcmd\@combinedblfloats{\box\@outputbox}{\unvbox\@outputbox}{}{%
  \errmessage{\noexpand\@combinedblfloats could not be patched}%
}%
\makeatother

%% Bold italic
\newcommand\hmmax{0}            % we don't need heavy fonts
\newcommand\bmmax{1}            % reduce use of math alphabets for bold
\usepackage{bm}

%% Bundled custom packages
\usepackage{aastex-compat}

\title
{Bow shocks, bow waves, and dust waves. III. Diagnostics}

\newcommand\AddressCRyA{Instituto de Radioastronom\'{\i}a y Astrof\'{\i}sica,
  Universidad Nacional Aut\'onoma de M\'exico, Apartado Postal 3-72,
  58090 Morelia, Michoac\'an, M\'exico}
\author[Henney \& Arthur]{
  William J. Henney \& S. J. Arthur\\
  \AddressCRyA
}

% These dates will be filled out by the publisher
\date{Accepted XXX. Received YYY; in original form ZZZ}

% Enter the current year, for the copyright statements etc.
\pubyear{2019}

\DeclareMathOperator{\sgn}{sgn}
\DeclareMathOperator{\Sin}{\mathcal{S}}
\DeclareMathOperator{\Cos}{\mathcal{C}}
\DeclareMathOperator{\Cot}{\mathcal{T}}
\DeclareMathOperator{\GammaFunc}{\Gamma}
\DeclareMathOperator\erf{erf}
\newcommand\w{\ensuremath{\mathrm{w}}}
\newcommand\C{\ensuremath{\mathrm{c}}}
\providecommand{\abs}[1]{\lvert#1\rvert}
\providecommand{\Abs}[1]{\left\lvert#1\right\rvert}
\newcommand\TODO[1]{%
  \begin{center}
    \framebox{\parbox{0.8\linewidth}{
        \texttt{\footnotesize\color{red} #1}}}
  \end{center}}

\newcommand\uvec[1]{\bm{\hat{#1}}}
\newcommand\T{_{\mathrm{\scriptscriptstyle T}}}

\newcommand\Qp{\ensuremath{Q_{\text{p}}}}
\newcommand\Qpbar{\ensuremath{\bar{Q}_{\text{p}}}}
\newcommand{\grain}{\ensuremath{_{\text{d}}}}
\newcommand{\B}{\ensuremath{_{\scriptscriptstyle\text{B}}}}
\newcommand{\alfven}{\ensuremath{_{\scriptscriptstyle\text{A}}}}
\newcommand{\xsec}{\ensuremath{\sigma\grain}}
\newcommand\frad{\ensuremath{f_{\text{rad}}}}
\newcommand\fmax{\ensuremath{f_{\text{max}}}}
\newcommand\thm{\ensuremath{\theta_{\text{m}}}}
\newcommand\drag{\ensuremath{_{\text{drag}}}}
\newcommand{\gas}{\ensuremath{_{\text{gas}}}}
\newcommand{\wind}{\ensuremath{_{\text{w}}}}
\newcommand{\trap}{\ensuremath{_{\text{abs}}}}
\newcommand{\ke}{\ensuremath{_{\text{kin}}}}
\newcommand{\drift}{\ensuremath{_{\text{drift}}}}
\newcommand\rad{\ensuremath{_{\text{rad}}}}
\newcommand\Lya{\ensuremath{_{\text{Ly}\alpha}}}
\newcommand\Rmin{\ensuremath{R_{\scriptscriptstyle\text{min}}}}
% Why do I need both of these?
\newcommand\sound{\ensuremath{c_{\text{s}}}}
\newcommand\soundspeed{\ensuremath{c_{\text{s,gas}}}}
\newcommand\starstar{\ensuremath{_{**}}}
\newcommand\mmp{\ensuremath{_{\text{\tiny MMP83}}}}
\newcommand\Hab{\ensuremath{_{\text{\tiny Habing}}}}
\newcommand\IR{\ensuremath{_{\text{IR}}}}
\newcommand{\thD}{\(\theta^1\)\,Ori~D}
\newcommand\alphaB{\ensuremath{\alpha_{\text{B}}}}
\newcommand\shell{\ensuremath{_{\text{sh}}}}
\newcommand\M{\ensuremath{\mathcal{M}}}
\newcommand\hii{\ion{H}{ii}}

\newcommand\amu{\ensuremath{a_{\si{\um}}}}

%%% Local Variables:
%%% mode: latex
%%% TeX-master: "bs-bw-dw-01"
%%% End:



\defcitealias{Tarango-Yong:2018a}{Paper~I}
\newcommand\PaperI{\citetalias{Tarango-Yong:2018a}}

% External document commands after macro definitions
\externaldocument[Q-]{quadrics-bowshock}
%\externaldocument[I-]{bs-bw-dw-01}
%\externaldocument[II-]{bs-bw-dw-02}


\begin{document}

\message{ !name(sec-observational-diagnostics.tex) !offset(-85) }
\section{Observational diagnostics}
\label{sec:summary-discussion}

% Are there any objects that might not be bow shocks?

% Progression in density:
% \begin{gather*}
%   \text{Increasing density} \longrightarrow \\
%   \text{WBS} \to \text{WBS} + \text{IDW} \to \text{WBS} + \text{DDW} \to \text{(RBW)} \to \text{RBS}
% \end{gather*}

% Chief diagnostic for radiation supported bows (RBW or RBS cases) is
% infrared luminosity of bow.  Favored by high densities.

% Dust waves favored by high velocities and intermediate densities.

\begin{figure*}
  \centering
  \includegraphics[width=0.8\linewidth]{figs/All-sources-eta-tau}
  \caption[Observational diagnostic diagram]{Observational diagnostic
    diagram for bow shocks.  The shell optical depth \(\tau\) (\(x\)
    axis) and momentum efficiency \(\eta\shell\) (\(y\) axis) can be
    estimated from observations of the bolometric stellar luminosity,
    infrared shell luminosity, and shell radius, as described in the
    text.  Results are shown for the 20 sources (circle symbols) from
    \citet{Kobulnicky:2018a} plus three further sources (square
    symbols), where we have obtained the measurements ourselves (see
    Tab.~\ref{tab:observations}).  The color of each symbol indicates
    the stellar luminosity (dark to light) as indicated by the scale
    bar. The shell pressure is determined assuming a gas temperature
    \(T = \SI{e4}{K}\), an absorption opacity
    \(\kappa = \SI{600}{cm^2.g^{-1}}\), and a thickness-to-radius ratio
    \(H/R = 0.25\).  The sensitivity of the results to a
    factor-of-three change in each parameter is shown in the upper
    inset box.  Exceptions are the two Orion Nebula sources, \thD{}
    and LP~Ori, where the small dim squares show the results of
    assuming the standard shell parameters, while the large squares
    show the results of modifications according to the peculiar
    circumstances of each object, as described in the text.  The lower
    inset box shows the sensitivity of the results to a factor-of-two
    uncertainty in each observed quantity: distance to source \(D\);
    stellar luminosity \(L_*\), shell infrared flux \(F\IR\); shell
    angular size \(\theta\).  Lines and shading indicate different
    theoretical bow regimes (see \S\S~\ref{sec:strong-gas-grain} and
    \ref{sec:imperf-coupl-betw}).  The dashed blue diagonal line
    corresponds to radiation-supported bows, while the upper left
    region corresponds to wind-supported bows.  The upper right corner
    (purple) corresponds to optically thick bow shocks, while the
    lower left corner (yellow) is the region where grain--gas
    separation \textit{may} occur, leading to a potential dust wave.
    However, the existence of a dust wave in this region is not
    automatic, since it only includes one of the four necessary
    conditions (\S~\ref{sec:exist-cond-separ} and
    \S~\ref{sec:grain-traj-with}). The lower-right region is strictly
    forbidden, except in case of violation of the assumption that dust
    heating be dominated by stellar radiation. }
  \label{fig:All-sources-eta-tau}
\end{figure*}


In order to provide an empirical anchor to our theoretical
calculations, we now consider how the parameters of our models might
be determined from observations.  The parameter space diagrams, such
as Figures~\ref{fig:zones-v-n-plane} and
\ref{fig:existence-dust-wave}, are not particularly useful in this
regard, since in many cases the ambient density and relative stellar
velocity are not directly measured.  Instead, we aim to construct
diagnostics based on the most common observations, which are of the
infrared dust emission.  Key questions that we wish to address include
\begin{enumerate}[1.]
\item Can we distinguish observationally between radiation support
  (bow waves and dust waves) and wind support (bow shocks)?
\item Are there any clear examples of sources with radiation-supported bows?
\item In the case of wind support, can we reliably determine mass loss
  rates from mid-infrared observations?
\end{enumerate}


\subsection{Energy trapping versus momentum trapping}
\label{sec:energy-trapp-vers}

A fundamental parameter is the optical depth, \(\tau\), of the bow
shell to UV radiation, which determines what fraction of the stellar
photon momentum is available to support the shell (see
\S~\ref{sec:three-bow-regimes}).  But the same photons also heat the
dust grains in the bow, which re-radiate that energy predominantly at
mid-infrared wavelengths (roughly \SIrange{10}{100}{\um}) with
luminosity \(L\IR\).  Assuming that Ly\(\alpha\) and mechanical
heating of the dust shell is negligible (see
\S~\ref{sec:unimp-other-heat}) and that the emitting shell subtends a
solid angle \(\Omega\), as seen from the star, then the optical depth
can be estimated as
\begin{equation}
  \label{eq:tau-empirical}
  \tau = -\ln \left( 1 - \frac{4\pi}{\Omega} \frac{L\IR}{L_*} \right)
  \approx \frac{2 L\IR}{L_*} \ ,
\end{equation}
where the last approximate equality holds if \(\tau \ll 1\) and the
shell emission covers one hemisphere.\footnote{%
  The \(\tau\) of \S~\ref{sec:three-bow-regimes} is not exactly the
  same as the \(\tau\) of equation~\eqref{eq:tau-empirical}, but is
  larger by a factor of
  \(Q_P / Q_{\text{abs}} = 1 + \varpi (1 - g)/(1 - \varpi)\), where
  \(\varpi\) is the grain albedo and \(g\) the scattering asymmetry
  (see App.~\ref{sec:gas-free-bow}).} %

A second important parameter is the thermal plus magnetic pressure in
the shocked shell, which is doubly useful since in a steady state it
is equal to \emph{both} the internal supporting pressure (wind ram
pressure plus absorbed stellar radiation) \emph{and} the external
confining pressure (ram pressure of ambient stream).  The shell
pressure is not given directly by the observations, but can be
determined by the following three steps:
\begin{enumerate}[P1.]
\item \label{P1} The shell mass column (\si{g.cm^{-2}}) can be
  estimated from the optical depth by assuming an effective UV
  opacity: \(\Sigma\shell = \tau / \kappa\)
\item \label{P2} The shell density (\si{g.cm^{-3}}) can be found from
  the mass column if the shell thickness is known:
  \(\rho\shell = \Sigma / h\shell\).  In the absence of other
  information, a fixed fraction of the shell radius can be used.  In
  particular, we normalize by a typical value of
  one~quarter\footnote{%
    This corresponds to a Mach number \(\M_0 = \surd 3\) if the stream
    shock is radiative, or \(\M_0 \gg 1\) if non-radiative (see
    \S~\ref{sec:radi-cool-lengths}).  Further discussion is given in
    Appendix~\ref{app:bow-shock-data}} %
  the star--apex distance: \(h_{1/4} = h\shell / (0.25 R_0)\).
\item \label{P3} Finally, the pressure (\si{dyne.cm^{-2}}) follows by
  assuming values for the sound speed and Alfvén speed:
  \(P\shell = \rho\shell (\sound^2 + \frac12 v\alfven^2) \).
\end{enumerate}
It is natural to normalize this pressure to the stellar radiation
pressure at the shell, so we define a shell momentum efficiency
\newcommand\pc{\ensuremath{_{\text{pc}}}}
\begin{equation}
  \label{eq:eta-shell}
  \eta\shell \equiv \frac{P\shell}{P\rad}
  = \frac{4\pi R_0^2\, (\sound^2 + \frac12 v\alfven^2)\, \tau}{L_*\, \kappa\, h\shell}
  \approx 245 \frac{R\pc \, T_4 \, \tau}{L_4 \, \kappa_{600} \, h_{1/4}} \ , 
\end{equation}
where in the last step we have assumed ionized gas with negligible
magnetic support (\(v\alfven \ll \sound\)) and written the stellar
luminosity and shell parameters in terms of typical values, as in
\S~\ref{sec:depend-stell-type}.  Note that the shell momentum
efficiency is simply the reciprocal of the radiation parameter of
equation~\eqref{eq:Xi-Prad-over-Pgas}:
\(\eta\shell = \Xi\shell^{-1}\), which provides yet a third use for
\(\eta\shell\), since \(\Xi\) is paramount in determining whether the
grains and gas remain well-coupled (see
\S~\ref{sec:exist-cond-separ}).


\subsection[The eta-tau diagnostic diagram]
{\boldmath The \(\eta\shell\)--\(\tau\) diagnostic diagram}
\label{sec:eta-tau-diagnostic}



\begin{table}
  \centering
  \caption[Observational]{Key observational parameters for star/bow systems}
  \label{tab:observations}
  \begin{tabular}{l S S S}
    \toprule
    Star & {\(L_* / \si{L_\odot}\)} & {\(L_{\text{IR}} / \si{L_\odot}\)} & {\(R_0 / \si{pc}\)} \\
    \midrule
    \thD & 2.95e4 & 620 & 0.003 \\
    LP~Ori & 1600 & 240 & 0.01 \\
    \(\sigma\)~Ori & 6e4 & 15 & 0.12 \\[\smallskipamount]
    K18 Sources & \numrange{1.4e4}{8.7e5} & \numrange{8}{2800} & \numrange{0.02}{1.35} \\
    \bottomrule
  \end{tabular}
\end{table}




In Figure~\ref{fig:All-sources-eta-tau} we show the resultant
diagnostic diagram: \(\eta\shell\) versus \(\tau\).  The horizontal
axis shows the fraction of the stellar radiative \emph{energy} that is
reprocessed by the bow shell, while the vertical axis shows the
fraction of stellar radiative \emph{momentum} that is imparted to the
shell, either directly by absorption, or indirectly by the stellar
wind (which is itself radiatively driven).  Radiatively supported bows
(DW, RBW, or RBS, or cases) should lie on the diagonal line
\(\eta\shell = ( Q_P / Q_{\text{abs}}) \tau \approx 1.25 \tau\), where
we have used the ratio of grain radiation pressure efficiency to
absorption efficiency found in the FUV band for the dust mixture shown
in Figure~\ref{fig:cloudy-ism-dust-opacity}.  Wind-supported bows
should lie above this line and no bows should lie below the
\(\eta\shell = \tau\) line, since \(Q_P\) cannot be smaller than
\(Q_{\text{abs}}\).

We have calculated \(\eta\shell\) and \(\tau\) using the
above-described methods for the 20 mid-infrared sources studied by
\citet{Kobulnicky:2018a} (K18) and plotted them on our diagnostic
diagram.  Details of our treatment of this observational material are
provided in Appendix~\ref{app:bow-shock-data}.  In order to expand the
range of physical conditions, we have included three additional
sources (data in Table~\ref{tab:observations}): bows around \thD{}
\citep{Smith:2005a} and LP~Ori \citep{ODell:2001c} in the Orion
Nebula, which show larger optical depths, plus the inner bow around
\(\sigma\)~Ori, which illuminates the Horsehead Nebula and has
previously been claimed to be a dust wave \citep{Ochsendorf:2014b,
  Ochsendorf:2015a}.  Details of the observations of these additional
sources will be published elsewhere.



\subsubsection{Random uncertainties due to observational errors}
\label{sec:rand-syst-uncert}

The fundamental observational quantities that go into determining
\(\tau\) and \(\eta\shell\) for each source are distance, \(D\);
stellar luminosity, \(L_*\); total infrared flux, \(F_{\text{IR}}\);
and bow angular apex distance, \(\theta\).  From these, the shell
radius and infrared luminosity are found as \(R_0 = \theta D\) and
\(L_{\text{IR}} = 4\pi D^2 F_{\text{IR}}\).  Rather than clutter the
diagram with error bars, we instead show the sensitivity to
observational errors in the lower-right box, where each arrow
corresponds to a factor of two increase (0.3~dex) in each quantity:
\(D\), \(L_*\), \(F_{\text{IR}}\), and \(\theta\).  In
\S~\ref{sec:uncert-estim-observ} we estimate the uncertainty in each
observational quantity for the K18 sources.  We then combine these in
\S~\ref{sec:comb-uncert-covar} to find the \(\pm 1~\sigma\) error
ellipse, shown in blue in the figure.  It can be seen that
observational uncertainties in \(\tau\) and \(\eta\shell\) are highly
correlated: the dispersion is \SI{0.7}{dex} in the product
\(\eta\shell \tau\) but only \SI{0.16}{dex} in the ratio
\(\eta\shell/\tau\), with stellar luminosity errors dominating in both
cases.  Observational uncertainties are therefore relatively
unimportant in determining whether a given source is wind-driven or
radiation-driven, which depends only on \(\eta\shell/\tau\).  On the
other hand, they do significantly effect the question of whether a
source has a sufficiently high radiation parameter \(\Xi\) to possibly
be a dust wave.

\subsubsection{Systematic uncertainties due to assumed shell parameters}
\label{sec:syst-uncert-due}

A further source of uncertainty arises from the parameters of the
shocked shell that are assumed in steps P\ref{P1}--P\ref{P3}, namely
the relative shell thickness, \(h\shell/R_0\), the ultraviolet grain
opacity per mass of gas, \(\kappa\), and the shell temperature, \(T\).
These parameters effect only \(\eta\shell\), not \(\tau\), with a
sensitivity shown by arrows in the upper left box of
Figure~\ref{fig:All-sources-eta-tau}.

\subparagraph{Shell thickness}
We do not expect a great deal of variation in the shell thickness,
except for in the case of fast runaway stars
(\(v > \SI{100}{km.s^{-1}}\)), for which the shell may be dramatically
thinner if the post-shock cooling is sufficiently rapid
(\(h / R_0 \sim M_0^{-2}\)).  For ambient densities less than about
\SI{10}{cm^{-3}}, the minimum thickness is about ten times smaller
than we are assuming.  This occurs at
\(v \approx \SI{60}{km.s^{-1}}\), corresponding to the peak in the
cooling curve at \SI{e5}{K} (see \S~\ref{sec:radi-cool-lengths}),
since the thickness is set by the cooling length at higher speeds.  In
principle, the shell thickness can be measured observationally if the
source is sufficiently well resolved \citep{Kobulnicky:2017a},
although this is complicated by projection effects.

\subparagraph{Dust opacity}
The dust opacity will depend on the total dust-gas ratio and on the
composition and size distribution of the grains.  Our adopted value of
\SI{600}{cm^2.g^{-1}}, or \SI{1.3e-21}{cm^2.H^{-1}}, is appropriate
for average Galactic interstellar grains in the EUV and FUV (e.g.,
\citealp{Weingartner:2001a}), but there is ample evidence for
substantial spatial variations in grain extinction properties
\citep{Fitzpatrick:2007a}, both on Galactic scales
\citep{Schlafly:2016a} and within a single star forming region
\citep{Beitia-Antero:2017a}.  The properties of grains within
photoionized regions are very poorly constrained observationally
because the optical depth is generally much lower than in overlying
neutral material.  In the Orion Nebula, there is some evidence
\citep{Salgado:2016a} that the FUV dust opacity in the ionized gas may
be as low as \SI{90}{cm^2.g^{-1}}, although the uncertainties in this
estimate are large and different results are obtained in other
regions, such as W3(A) \citep{Salgado:2012a}.   

\subparagraph{Shell gas temperature} For bows around O~stars, the
shell temperature should be close to the photoionization equilibrium
value of \(\approx \SI{e4}{K}\), since the post-shock cooling length
is short in ambient densities above \SI{0.1}{cm^{-3}}
(\S~\ref{sec:radi-cool-lengths}) and the shell does not trap the
ionization front for ambient densities below \SI{e4}{cm^{-3}}
(\S~\ref{sec:trapp-ioniz-front}).  For B~stars, on the other hand,
these two density limits move closer together (cf.~the smaller gap
between the blue and red lines in Figs.~\ref{fig:zones-v-n-plane}a and
\ref{fig:B-supergiant}, as compared with \ref{fig:zones-v-n-plane}b
and~c), making it more likely that a bow will lie in a different
temperature regime.  The only source for which we have evidence that
this occurs is LP~Ori, as discussed in the next section.



\subsubsection{Special treatment of particular sources}
\label{sec:spec-treatm-part}

For two of the additional bows listed in Table~\ref{tab:observations},
we are forced to deviate from the default values for the shell
parameters.  For LP~Ori, the bow shell appears to be formed from
neutral gas \citep{ODell:2001c} and its relatively high \(\tau\) value
is more than sufficient to trap the weak ionizing photon output of a
B3 star.  We therefore move its point in
Figure~\ref{fig:All-sources-eta-tau} downward to reflect a temperature
of \SI{1000}{K} (or, equivalently, a magnetically supported shell with
\(v\alfven \approx \SI{3}{km.s^{-1}}\)).  For the case of the Orion
Trapezium star \thD{}, we find that using the default parameters
results in a placement well inside the forbidden zone of
Figure~\ref{fig:All-sources-eta-tau} (indicated by fainter symbol).
For this object there is no reason to suspect anything but the usual
photoionized temperature of \SI{e4}{K}, but its placement could be
resolved either by decreasing the shell thickness, or decreasing the
UV dust opacity, or both. Given the moderate limb brightening seen in
the highest resolution images of the Ney--Allen nebula
\citep{Robberto:2005a, Smith:2005a}, the shell thickness is unlikely
to be less than half our default value.  But, if this were combined
with a factor 5 decrease in \(\kappa\), as suggested by
\citet{Salgado:2016a}, then this would be sufficient to move the
source up to the RBW line, or slightly above.


\subsection{Candidate radiation-supported bows}
\label{sec:cand-radi-supp}

Four sources are sufficiently close to the diagonal line
\(\eta\shell = 1.25 \tau\) in Figure~\ref{fig:All-sources-eta-tau}
that they should be treated as potential candidates for
radiation-supported bows. These are K18 sources~380 (HD~53367,
V750~Mon) and 407 (HD~93249 in Carina) plus \thD{} and LP~Ori.  Of the
four, source 380 is the only one that is also a candidate for
grain-gas decoupling.  Details of some of these sources are presented
in \S~\ref{sec:notes-part-sourc}, where for source~380 we show that
reducing both luminosity and distance by a factor of roughly 2 with
respect to the values used by K18 would provide a better fit to the
totality of observational data.  However, the ratio
\(\eta\shell /\tau\) is proportional to \(D / L_*\) so it would not be
affected by such an adjustment and the bow remains
radiation-supported.  On the other hand \(\eta\shell\) (proportional
to \(D / L_*^2\)) would increase by 2, making the classification as
dust wave candidate more marginal.



\subsection{Stellar wind mass-loss rates}
\label{sec:stellar-wind-mass}

Third method used by \citet{Gvaramadze:2012a}, relies on measurements
of \hii{} region and bow. Only works for isolated stars.

Mass loss rates - starting from \citet{Kobulnicky:2010a}

\begin{figure*}
  \centering
  \includegraphics[width=\linewidth]{figs/Mdot-vs-lum-combo-edited}
  \caption{Wind mass-loss rates as a function of stellar luminosity,
    derived from (a)~our trapped energy/momentum method and (b)~the
    grain emissivity method of \citet{Kobulnicky:2018a}, with
    corrections as described in our Appendix~\ref{app:bow-shock-data}.
    Circle symbols show the sources from K18, colored according to the
    stellar effective temperature (see key at far right). In panel a,
    squares show two of our additional sources
    (Tab.~\ref{tab:observations}). Upper limits to the mass loss are
    given for sources that lie close to the radiation-supported line
    in Fig.~\ref{fig:All-sources-eta-tau}, represented by faint
    symbols and downward-pointing arrows.  Sources that lie less than
    a factor of three above the line have enhanced downward
    uncertainties and are also shown by slightly fainter symbols.  For
    \(\sigma\)~Ori, the large symbol corresponds to the sum of the
    luminosities of the triple OB system Aa, Ab, and B
    \citep{Simon-Diaz:2015a}, while the small symbol corresponds to
    the luminosity of only the most massive component Aa.  Lines show
    the predictions of stellar wind models: red lines are the commonly
    used recipes from \citet{Vink:2000a} for dwarfs (solid), giants
    (dashed), and supergiants (dotted), while black lines show
    eq.~(11) of \citet{Krticka:2017a} for O~stars (solid) and models
    of \citet{Krticka:2014a} for B stars.  Orange plus symbols show
    mass-loss measurements from NUV lines for weak-wind O~dwarfs
    \citep{Marcolino:2009a}, while the green plus symbol shows the
    measurement from infrared H recombination lines for \(\sigma\)~Ori
    \citep{Najarro:2011a}.  Boxes show the sensitivity of the results
    to observational uncertainties (lower right) and assumed shell
    parameters (upper left).}
  \label{fig:mass-loss-vs-luminosity}
\end{figure*}


Source~342 is anomalous in both methods.

Evidence for small grains around Arches cluster near Galactic center \citep{Hankins:2017a}



Different scenarios for producing velocities: dynamic ejection from
young clusters \citep{Hoogerwerf:2001a, Oh:2016c} produce high
velocities, dissolution of binary systems following core-collapse SN
\citep{Renzo:2018a} tend to produce lower velocities for the unbound
MS companion (walkaways, slower than 30 km/s).  Also, champagne flows
have low velocities.

How different regions of the \(\Pi\)--\(\Lambda\) plane are populated.
Bottom-right quadrant hard to get to (except for standing wave
oscillations), but may be due to finite shell thickness, which (for
low Mach number) will be more apparent in the wings, which might
decrease \(\Lambda\) more than \(\Pi\).  Fact that thin-shell solutions should
trace the contact discontinuity, but in some cases it may be only the
inner or the outer shell that is visible.

Justification for standing waves: Fig.~3 of \citet{Meyer:2016a} shows
a time sequence of thin-shell instability, which looks a bit like a
standing wave. But much larger amplitude than we are considering.

Deviations from axisymmetry as an alternative to oscillations. 

\subsection{Diagnosing bow type from shell emission profile}
\label{sec:diagnosing-bow-type}

Given that the predicted shell density profile varies between
wind-supported and radiation-supported bows (Paper~I) it is worth
considering whether it may be possible to use observed emission
profiles to distinguish between the two.



\subsection{The case of inside-out bows}
\label{sec:case-inside-out}

So far, we have considered the case where the inner source dominates
the radiation, while dust is present only in the outer stream, which
applies to hot stars interacting with the ISM.  However, in the case
of cool stars, the inner wind will also be dusty.  Examples are the
red supergiant (RSG) phase of high-mass evolution, or the asymptotic
giant branch (AGB) stage of low/intermediate-mass evolution.  In both
these cases, it is still the inner source that provides the radiation
field.  However, not all winds are radiatively driven and in those
cases it is conceivable that it is the outer source that dominates the
radiation field.  An example is the case of photoevaporating
protoplanetary disks (proplyds) in the Orion Nebula and other \hii{}
regions \citep{ODell:1994a}.  In the proplyds, the inner wind is a
thermally driven photoevaporation flow \citep{Henney:1998b, Henney:1999a},
while the outer stream is the stellar wind from an O~star
\citep{Garcia-Arredondo:2001a}.


\section{Summary and conclusions}
\label{sec:conclusions}



%%% Local Variables:
%%% mode: latex
%%% TeX-master: "bs-bw-dw-03"
%%% End:

\message{ !name(bs-bw-dw-03.tex) !offset(-415) }

\end{document}


%%% Local Variables:
%%% mode: latex
%%% TeX-master: t
%%% End:
