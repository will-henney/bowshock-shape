
\section{Further details on ionization front trapping}
\label{sec:furth-deta-ioniz}

This appendix will probably be dropped from the paper.  It contains
details of the derivation of the ionization front trapping that I now
think are too verbose to be included, given that this is not the main
point of the paper.  They are collected here for completeness.

We wish to calculate whether the star is capable of photoionizing the
entire bow shock shell, or whether the ionization front will be
trapped within it.  The number of hydrogen recombinations\footnote{%
  The diffuse field is treated in the on-the-spot approximation,
  assuming all emitted Lyman continuum photons are immediately
  re-absorbed locally, so the case~B recombination co-efficient,
  \(\alphaB = \num{2.6e-13}\, T_4^{-0.7}\, \si{cm^3.s^{-1}}\), is
  used, where \(T_4 = T/\SI{e4}{K}\).} %
per unit time per unit area in a fully ionized shell is
\begin{equation}
  \label{eq:shell-recombination-rate-app}
  \mathcal{R} = \alphaB n\shell^2 h\shell \ ,
\end{equation}
while the flux of hydrogen-ionizing photons
(\(h \nu > \SI{13.6}{eV}\)) incident on the inner edge of the shell is
\begin{equation}
  \label{eq:shell-ionizing-flux-app}
  \mathcal{F} = \frac{S} {4 \pi R_0^2} \ , 
\end{equation}
where \(S\) is the ionizing photon luminosity of the star.  Any shell
with \(\mathcal{R} > \mathcal{F}\) cannot be entirely photoionized by
the star, and so must have trapped the ionization front.  The column
density of the shocked shell can be found, for example, from
equations~(10) and~(12) of \citet{Wilkin:1996a} in the limit
\(v_\infty/V \to 0\) (Wilkin's parameter \(\alpha\)) and \(\theta \to 0\).  This yields
\begin{equation}
  \label{eq:shocked-shell-column-app}
  n\shell h\shell = \tfrac34 n R_0 \ .
\end{equation}
Assuming strong cooling behind the shock,\footnote{%
  This is shown to be justified in \S~\ref{sec:radi-cool-lengths}.
} %
the shell density is \(n\shell = \mathcal{M}_0^2 n\), where
\(\mathcal{M}_0 = v_\infty / \sound\) is the isothermal Mach number of the
external stream.\footnote{%
  \label{fn:temperature-dependence}
  The sound speed depends on the temperature and hydrogen and helium
  ionization fractions, \(y\) and \(y_{\text{He}}\) as
  \(\sound^2 = (1 + y + z_{\text{He}} y_{\text{He}}) (k T /
  \bar{m})\), where \(z_{\text{He}}\) is the helium nucleon abundance
  by number relative to hydrogen and
  \(k = \SI{1.3806503e-16}{erg.K^{-1}}\) is Boltzmann's constant.  We
  assume \(y = 1\), \(y_{\text{He}} = 0.5\), \(z_{\text{He}} = 0.09\),
  so that \(\sound = \num{11.4}\, T_4^{1/2}\, \si{km.s^{-1}}\). } %
Putting these together with equations~\eqref{eq:Rstar} and
~\eqref{eq:tau-star}, one finds that \(\mathcal{R} > \mathcal{F}\)
implies
\begin{equation}
  \label{eq:ifront-trap-x-cubed-taustar-app}
  x^3 \tau_* > \frac{4 S c \sound \bar{m}^2 \kappa}{3 \alpha L} \ .
\end{equation}
From equation~\eqref{eq:rad-full-x}, it can be seen that \(x\) depends
on the external stream parameters, \(n\), \(v_\infty\) only via
\(\tau_*\), and so equation~\eqref{eq:ifront-trap-x-cubed-taustar} is a
condition for \(\tau_*\).  In the radiation bow shock case,
\(x = (1 + \eta)^{1/2}\), and the condition can be written:
\begin{equation}
  \label{eq:ifront-trap-taustar-bow-shock}
  \tau_* > 145.0 \frac{S_{49} T_4^{1.7} \kappa_{600}}{L_4 (1 + \eta)^{3/2}} \ , 
\end{equation}
where
\begin{equation*}
  S_{49} = S / \bigl( \SI{e49}{s^{-1}} \bigr) \ .
\end{equation*}
Numerical values of \(S_{49}\) for our three example stars are given
in Table~\ref{tab:stars}.  In the radiation bow wave case,
\(x = 2\tau_*\), and the condition can be written:
\begin{equation}
  \label{eq:ifront-trap-taustar-bow-wave}
  \tau_* > \left(  18.1 \frac{S_{49} T_4^{1.7} \kappa_{600}}{L_4}\right)^{1/4} \ , 
\end{equation}



\(\tau_* \sim n^{1/2} / v_\infty\) 
This simple
criterion is shown by the dark red line in
Figure~\ref{fig:zones-v-n-plane}.  If 
\(n / v_{10}^2 > \text{\numrange{1000}{5000}}\), depending
only weakly on the stellar parameters, then the outer parts of the
shocked shell are neutral, instead of ionized. 


Assuming photoionization equilibrium, the
hydrogen photoabsorption optical depth of the shell is
\begin{equation}
  \label{eq:ion-tau-gas}
  \tau\gas = - \ln(1 - \mathcal{R} / \mathcal{F}) \ ,
\end{equation}
so long as \(\mathcal{R} < \mathcal{F}\).

We will assume
a typical photoionized temperature of \SI{8000}{K}, so that
\(\sound \approx \SI{10}{km.s^{-1}}\) and \(M_0 = v_{10}\), yielding
\begin{equation}
  \label{eq:ion-tau-gas-expanded}
  \tau\gas = -\ln\bigl(1 -
  \num{8.42e-6}\, v_{10}^2 n^2 R_{0,\text{pc}}^3 S_{49}^{-1}\bigr) \ , 
\end{equation}
where 
\begin{align*}
  R_{0,\text{pc}} &= R_0 / \bigl( \SI{1}{pc} \bigr)
\end{align*}
The dust opacity is approximately constant
at FUV to EUV wavelengths, so the dust optical depth of the shocked
shell to ionizing photons follows from equations~\eqref{eq:tau-thin}
and~\eqref{eq:shocked-shell-column} as \(\tau\grain = \tfrac38 \tau\).

The hydrogen ionization fraction, \(y\), at the outer edge of the shocked
shell then follows as
\begin{equation}
  \label{eq:outer-shell-ionization-balance}
  \frac{y^2}{1 - y} = \frac{\sigma \mathcal{F}}{\alphaB n} e^{-(\tau\grain + \tau\gas)} \ ,
\end{equation}
where \(\sigma\) is the effective hydrogen photoionization cross section,
averaged over the local ionizing spectrum.  Since the
frequency-dependent cross section, \(\sigma_\nu \sim \nu^{-3}\), is strongly
peaked at the threshold, the local EUV spectrum becomes harder with
increasing \(\tau\gas\), as the lower frequency photons are selectively
absorbed,\footnote{} leading to a reduction in the effective
\(\sigma\).  An approximate fit to the results in Appendix~A of
\citet{Henney:2005b} is
\begin{equation}
  \label{eq:sigma-vs-tau}
  \sigma = 0.5 \sigma_0 e^{-\tau\gas/3}
\end{equation}
where \(\sigma_0 = \SI{6e-18}{cm^2}\) is the threshold cross-section.
Although this was derived for a particular ionizing spectrum (\SI{40
  000}{K} black body), we adopt it for all our hot stars.



%%% Local Variables:
%%% mode: latex
%%% TeX-master: "dusty-bow-wave"
%%% End:
