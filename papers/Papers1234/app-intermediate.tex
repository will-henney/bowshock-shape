%% \section{Derivation of selected results from \S\S~2 and 3}

\section{Radius of curvature}
\label{sec:radius-curvature}

The radius of curvature of a general curve can be written (e.g.,
eq.~[2-5] of \citealp{Guggenheimer:2012a}):
\begin{equation}
  \label{eq:Rcurv-general}
  R_{\C} \equiv \frac{1}{\abs{\kappa}} = \Abs{\frac{d s}{d \alpha}} \ ,  
\end{equation}
where \(\kappa\) is the \textit{curvature}, \(s\) is the path length along
the curve and \(\alpha\) is the tangent angle (see Fig.~\ref{fig:unitvec}).
In spherical polar coordinates, this becomes \citep{Weisstein:2018a}:
\begin{equation}
  \label{eq:Rcurv-polar}
  R_{\C} = \frac{\left( R^2 + R_\theta^2 \right)^{3/2}}
  {\Abs{R^2 + 2 R_\theta^2 - R R_{\theta\theta}}} \ , 
\end{equation}
where \(R_\theta = d R / d \theta\) and
\(R_{\theta\theta} = d^2 R / d \theta^2\).  At the apex,
\(R_\theta = 0\) by symmetry, which yields
equation~\eqref{eq:radius-curvature} of
\S~\ref{sec:plan-alat-bow}. Note that \(\theta\) is dimensionless and
should be measured in radians \citep{Mohr:2015a, Quincey:2017a}.


\section{Rotation matrices and plane of sky projection}
\label{sec:plane-sky-projection}

The transformation from the body frame (unprimed) to observer-frame
(primed) coordinates is a rotation about the \(y\) axis by an angle
\(i\), which is described by the rotation matrix:
\begin{equation}
  \label{eq:Rotation-matrix-y}
  \mathbfss{A}_y(i) = 
   \begin{pmatrix}
    \cos i & 0 & -\sin i \\
    0 & 1 & 0 \\
    \sin i & 0 &\cos i 
  \end{pmatrix} \ .
\end{equation}
This is used in equation~\eqref{eq:Trans}.

Since we are considering cylindrically symmetric bows, all azimuths
\(\phi\) are equivalent, so it is sufficient to work with two-dimensional
curves in the plane \(z = 0\) (which is also \(\phi = 0\)) and then find
the three-dimensional surface by rotating about the \(x\)-axis via the
rotation matrix:
\begin{equation}
  \label{eq:Rotation-matrix-x}
  \mathbfss{A}_x(\phi) = 
   \begin{pmatrix}
    1 & 0 & 0 \\
    0 &\cos\phi & -\sin\phi \\
    0 &\sin\phi & \cos\phi 
  \end{pmatrix} \ .
\end{equation}
This is used in \S~\ref{sec:unit-vectors-normal} to find the general
form of the body-frame normal and tangent unit vectors:
\begin{align}
  \uvec{n} &= \mathbfss{A}_x(\phi) \, \uvec{n}_0 =
  \begin{Vector}
    \sin\alpha \\ \cos\alpha\cos\phi \\ \cos\alpha\sin\phi
  \end{Vector} \\
  \uvec{t} &= \mathbfss{A}_x(\phi) \, \uvec{t}_0 =
  \begin{Vector}
    -\cos\alpha \\ \sin\alpha\cos\phi \\ \sin\alpha\sin\phi
  \end{Vector} \ . 
\end{align}



%%% Local Variables:
%%% mode: latex
%%% TeX-master: "quadrics-bowshock"
%%% End:
