\documentclass[useAMS, usenatbib, a4paper]{mnras}
\pdfsuppresswarningpagegroup=1

\usepackage{graphicx}
\usepackage{microtype}
\usepackage{xcolor}
\usepackage{fixltx2e}
\usepackage{booktabs}
\usepackage{siunitx}
\sisetup{separate-uncertainty = true}
\usepackage{color}
\usepackage{enumerate}
\usepackage{pdflscape}
\usepackage{rotating}
\usepackage{xr-hyper}
\usepackage{hyperref}

\usepackage[T1]{fontenc} 
\usepackage[utf8]{inputenc}

% Fonts 
\usepackage{newtxtext}
% Note: newtxmath must come AFTER newtxtext
\usepackage[varvw,smallerops]{newtxmath}

\usepackage{chemgreek}
\activatechemgreekmapping{newtx}
\usepackage{listings}

\hypersetup{colorlinks=True, linkcolor=blue!50!black, citecolor=black,
  urlcolor=blue!50!black}

\usepackage{etoolbox}
\robustify\bfseries
\robustify\itshape

%% The following hack solves a problem with
%% ERROR: \pdfendlink ended up in different nesting level than \pdfstartlink.
%% See https://tex.stackexchange.com/a/249743
\makeatletter
\patchcmd\@combinedblfloats{\box\@outputbox}{\unvbox\@outputbox}{}{%
  \errmessage{\noexpand\@combinedblfloats could not be patched}%
}%
\makeatother

%% Bold italic
\newcommand\hmmax{0}            % we don't need heavy fonts
\newcommand\bmmax{1}            % reduce use of math alphabets for bold
\usepackage{bm}

%% Bundled custom packages
\usepackage{aastex-compat}

\title
{Bow shocks, bow waves, and dust waves. II. Beyond the rip point}

\newcommand\AddressCRyA{Instituto de Radioastronom\'{\i}a y Astrof\'{\i}sica,
  Universidad Nacional Aut\'onoma de M\'exico, Apartado Postal 3-72,
  58090 Morelia, Michoac\'an, M\'exico}
\author[Henney \& Arthur]{
  William J. Henney \& S. J. Arthur\\
  \AddressCRyA
}

% These dates will be filled out by the publisher
\date{Accepted XXX. Received YYY; in original form ZZZ}

% Enter the current year, for the copyright statements etc.
\pubyear{2019}

\DeclareMathOperator{\sgn}{sgn}
\DeclareMathOperator{\Sin}{\mathcal{S}}
\DeclareMathOperator{\Cos}{\mathcal{C}}
\DeclareMathOperator{\Cot}{\mathcal{T}}
\DeclareMathOperator{\GammaFunc}{\Gamma}
\DeclareMathOperator\erf{erf}
\newcommand\w{\ensuremath{\mathrm{w}}}
\newcommand\C{\ensuremath{\mathrm{c}}}
\providecommand{\abs}[1]{\lvert#1\rvert}
\providecommand{\Abs}[1]{\left\lvert#1\right\rvert}
\newcommand\TODO[1]{%
  \begin{center}
    \framebox{\parbox{0.8\linewidth}{
        \texttt{\footnotesize\color{red} #1}}}
  \end{center}}

\newcommand\uvec[1]{\bm{\hat{#1}}}
\newcommand\T{_{\mathrm{\scriptscriptstyle T}}}

\newcommand\Qp{\ensuremath{Q_{\text{p}}}}
\newcommand\Qpbar{\ensuremath{\bar{Q}_{\text{p}}}}
\newcommand{\grain}{\ensuremath{_{\text{d}}}}
\newcommand{\B}{\ensuremath{_{\scriptscriptstyle\text{B}}}}
\newcommand{\alfven}{\ensuremath{_{\scriptscriptstyle\text{A}}}}
\newcommand{\xsec}{\ensuremath{\sigma\grain}}
\newcommand\frad{\ensuremath{f_{\text{rad}}}}
\newcommand\fmax{\ensuremath{f_{\text{max}}}}
\newcommand\thm{\ensuremath{\theta_{\text{m}}}}
\newcommand\drag{\ensuremath{_{\text{drag}}}}
\newcommand{\gas}{\ensuremath{_{\text{gas}}}}
\newcommand{\wind}{\ensuremath{_{\text{w}}}}
\newcommand{\trap}{\ensuremath{_{\text{abs}}}}
\newcommand{\ke}{\ensuremath{_{\text{kin}}}}
\newcommand{\drift}{\ensuremath{_{\text{drift}}}}
\newcommand\rad{\ensuremath{_{\text{rad}}}}
\newcommand\Lya{\ensuremath{_{\text{Ly}\alpha}}}
\newcommand\Rmin{\ensuremath{R_{\scriptscriptstyle\text{min}}}}
% Why do I need both of these?
\newcommand\sound{\ensuremath{c_{\text{s}}}}
\newcommand\soundspeed{\ensuremath{c_{\text{s,gas}}}}
\newcommand\starstar{\ensuremath{_{**}}}
\newcommand\mmp{\ensuremath{_{\text{\tiny MMP83}}}}
\newcommand\Hab{\ensuremath{_{\text{\tiny Habing}}}}
\newcommand\IR{\ensuremath{_{\text{IR}}}}
\newcommand{\thD}{\(\theta^1\)\,Ori~D}
\newcommand\alphaB{\ensuremath{\alpha_{\text{B}}}}
\newcommand\shell{\ensuremath{_{\text{sh}}}}
\newcommand\M{\ensuremath{\mathcal{M}}}
\newcommand\hii{\ion{H}{ii}}

\newcommand\amu{\ensuremath{a_{\si{\um}}}}

%%% Local Variables:
%%% mode: latex
%%% TeX-master: "bs-bw-dw-01"
%%% End:


\externaldocument[Q-]{quadrics-bowshock}
%\externaldocument[I-]{bs-bw-dw-01}
%\externaldocument[III-]{bs-bw-dw-03}

\defcitealias{Tarango-Yong:2018a}{Paper~I}
\newcommand\PaperI{\citetalias{Tarango-Yong:2018a}}


\begin{document}
\label{firstpage}
\pagerange{\pageref{firstpage}--\pageref{lastpage}}
\maketitle
\begin{abstract}
  Dust waves are a result of gas--grain decoupling in a stream of
  dusty plasma that flows past a luminous star.  The radiation field
  is sufficiently strong to overcome the collisional coupling between
  grains and gas at a \textit{rip-point}, where the ratio of radiation
  pressure to gas pressure exceeds a critical value of roughly 1000.
  When the rip point occurs outside the hydrodynamic bow shock, a
  separate dust wave may form, decoupled from the gas shell, which can
  either be drag-confined or inertia-confined, depending on the stream
  density and relative velocity.  In the drag-confined case, there is
  a minimum stream velocity of roughly \SI{60}{km.s^{-1}} that allows
  a steady-state stagnant drift solution for the dust wave apex.  For
  lower relative velocities, the dust dynamics close to the axis
  exhibit a limit cycle behavior (rip and snap back) between two
  different radii.  Strong coupling of charged grains to the plasma's
  magnetic field can modify these effects, but for a quasi-parallel
  field orientation the results are qualitatively similar to the
  non-magnetic case. For a quasi-perpendicular field, on the other
  hand, the formation of a decoupled dust wave is strongly suppressed.
\end{abstract}

\begin{keywords}
  circumstellar matter -- radiation: dynamics -- stars: winds, outflows
\end{keywords}


\section{Introduction}
\label{sec:rip-introduction}
Stellar bow shocks are predicted to occur whenever a star moves
supersonically relative to its surrounding medium, which may either be
due to the star's own motion \citetext{runaways, e.g.,
  \citealp{Blaauw:1961a}}, or due to an independent environmental flow
\citetext{weather vanes, e.g., \citealp{Povich:2008a}}.  Bow shocks
are associated with a large variety of different types of stars, for
example: AGB stars and red super-giants \citep{Cox:2012a}; OB stars
\citep{Kobulnicky:2017a}; T~Tauri stars \citep{Gull:1979a,
  Henney:2013a}; photoevaporating protoplanetary disks
\citep{Garcia-Arredondo:2001a, Smith:2005a}; neutron stars
\citep{Cordes:1993a, Brownsberger:2014a}; planetary nebula halos
\citep{Ali:2012a}; and Galactic Center sources \citep{Geballe:2004a,
  Sanchez-Bermudez:2014a}.  The dominant emission mechanism can vary
considerably between the different classes of sources, but
recombination line radiation such as H\(\alpha\), infrared continuum
radiation from warm dust, and free-free radio continuum are common
\citep{Canto:2005a, Acreman:2016a, Meyer:2016a}.  In nearly all cases,
stellar bow shocks are easily spatially resolved and mapped, allowing
their shapes to be compared with theoretical predictions
\citep{Wilkin:1996a, Tarango-Yong:2018a}.  A different class of
stellar bow shocks are found in interacting binary systems
\citep{Stevens:1992a}, but these typically emit by non-thermal
mechanisms and are only resolvable using radio interferometry
techniques \citep{Contreras:1999a, Dzib:2013a}.

In \citet{Henney:2019a} (hereafter, Paper~I) we studied the formation
of bows around OB stars by the combined action of their winds and
radiation.  For the case of strong collisional coupling between gas
and dust grains, we showed that three different regimes of interaction
are possible, according to the relative values of stellar wind
radiative momentum efficiency \(\eta\wind\) and the UV optical depth of
the bow shell \(\tau\):
\begin{enumerate}[{\(\star\)}]
\item Wind-supported bow shocks (WBS) where \(\tau < \eta\wind\).  These are
  purely hydrodynamic (or MHD) interactions, in which the stellar
  radiation is dynamically unimportant.  This regime dominates for
  fast-moving stars and for low-density environments.
\item Radiation-supported bow waves (RBW) where
  \(\eta\wind < \tau < 1\).  In these optically thin bows, the stellar
  radiative momentum gradually decelerates the oncoming stream in a
  broad shell.  This regime is important for B~stars and weak-wind
  O~stars in moderate density environments (\(> \SI{100}{cm^{-3}}\)).
\item Radiation-supported bow shocks (RBS) where \(\tau > 1\).  These are
  optically thick shells, internally supported by the trapped stellar
  radiation pressure.  This regime applies to slow-moving O~stars in
  dense environments.
\end{enumerate}
In this paper, we investigate under what circumstances the strong
gas--grain coupling might break down, leading to a fourth regime: a
separate dust wave outside of the hydrodynamic bow shock (see Fig.~1
of Paper~I).

In neutral and molecular regions, the drag forces are relatively weak
and so divergent dynamics of gas and grains are frequently found in
the context of molecular clouds \citetext{\citealp{Hopkins:2016a,
    Lee:2017a, Mattsson:2019a}, but see \citealp{Tricco:2017a}} and
protoplanetary disks \citep{Weidenschilling:1977b, Birnstiel:2010a,
  Dipierro:2018a}.  In ionized regions, the electrostatic forces
between charged dust grains and charged gas particles (principally
protons) leads to greatly increased drag forces, which tend to
maintain a tight coupling between dust and gas \citep{Draine:2011a}.
However, in a photoionized \hii{} region around a high-mass star, the
dust feels a much larger radiation force than does the gas. This leads
to a slow relative drift between the two components in the outer
regions of the nebula \citep{Gail:1979a, Akimkin:2015a, Akimkin:2017a,
  Ishiki:2018a} and a total decoupling very close to the central star
\citep[Fig.~8 of][]{Draine:2011a}.

With respect to stellar bow shocks, gas--grain decoupling has been
previously studied in the context of post-shock flow, where it is the
sudden deceleration of the gas that is the primary impetus for the
dust to separate.  The case of a cool red supergiant was simulated by
\citet{van-Marle:2011a}, where large grains present in the stellar
wind are not stopped in the inner termination shock, but are carried
through the contact discontinuity by their inertia, penetrating into
the interstellar medium.  \citet{Katushkina:2017a} simulated the
opposite case, where grains in the ambient medium become decoupled
from the gas after passing the outer bow shock, and subsequently
gyrate about the magnetic field, forming filamentary structures (see
also \citealp{Katushkina:2018a}).  In this paper, we study a different
decoupling mechanism, where it is the stellar radiation force acting
on the grains that causes a separation from the gas \emph{before the
  stream reaches the bow shock}.  This mechanism was first proposed by
\citet{Ochsendorf:2014b} to explain the infrared emission arc around
the high-mass multiple system \(\sigma\)~Ori.

The plan of the paper is as follows.
%
In \S~\ref{sec:recap-paper-i} we give some results from Paper~I that
we will need in later sections.
%
In \S~\ref{sec:gas-free-bow} we calculate analytical models of the
shape of dust waves in the drag-free limit.
% %
% In \S\ref{sec:cloudy-models-dust} we discuss the relevant dust physics
% and the Cloudy models that we employ.
%
In \S~\ref{sec:imperf-coupl-betw} we calculate in detail the grain
charging and dynamics, subject to radiation and drag forces, in order
to determine the \textit{rip point}, which is where gas--grain coupling
catastrophically breaks down.  This is used to determine existence
conditions for the presence of a decoupled dust wave.
%
In \S~\ref{sec:magn-effects-grain} we consider the coupling between
the grains and the plasma's magnetic field, and how this effects the
existence and structure of dust waves.
%
In \S~\ref{sec:discussion} we discuss our results in the context of
previous studies.
%
In \S~\ref{sec:summary} we summarise our findings.
%
In Appendix~\ref{sec:equat-moti-grains} we provide additional
technical information on the numerical calculation of grain
trajectories.

\section{Recapitulation of Paper~I}
\label{sec:recap-paper-i}

In Paper~I, we found approximate expressions for the bow radius in each of the three regimes discussed in the introduction: 
%% XREF \ref{eq:x-cases}
\begin{equation}
\begin{aligned}
  \text{RBS} \quad& (\tau_*^2 > 1): & R_0 &\approx  (1 + \eta\wind)^{1/2} \, R_*  \\
  \text{RBW} \quad& (\eta\wind < \tau_*^2 < 1): & R_0 &\approx 2 \, \tau_* \, R_* \\
  \text{WBS} \quad& (\tau_*^2 < \eta): & R_0 &\approx \eta\wind^{1/2} \, R_*  
\end{aligned}\label{eq:x-cases}
\end{equation}
In these expressions, we use a fiducial radius
%% XREF \eqref{eq:Rstar}
\begin{equation}
  \label{eq:Rstar}
  R_* = \left(\frac{L}{4\pi c \rho v_\infty^2}\right)^{1/2} \ ,
\end{equation}
a fiducial optical depth
%% XREF \eqref{eq:tau-star},
\begin{equation}
  \label{eq:tau-star}
  \tau_* = \rho \kappa R_* \ ,
\end{equation}
and a factor that describes the efficiency of the stellar wind as the fraction of the stellar radiation momentum that is converted to wind momentum
\begin{equation}
  \label{eq:wind-efficiency}
  \eta\wind  = \frac{c\, \dot{M} V\wind}{L} \ .
\end{equation}
Paper~I's equation~(11) allows a more exact value for \(R_0\) to be
found numerically in intermediate cases.  Note that \(R_0\) is the
star--apex distance measured along the symmetry axis of the bow
\citetext{see \citealp{Tarango-Yong:2018a} for explanation of
  nomenclature and discussion of bow shapes}, and is calculated in the
limit that the momentum transfer occurs at a surface.  In cases where
the shell's finite width is significant, \(R_0\), should correspond
approximately to the astropause (contact discontinuity) in the WBS
regime, or a UV optical depth of unity (as measured from the star) in
the RBS regime.  The total perpendicular optical depth of the shell to
stellar radiation can be found as
\begin{equation}
  \label{eq:actual-tau}
  \tau = \frac{R_0}{R_*} \, \tau_* \ .
\end{equation}

In the foregoing, we employ the stellar bolometric luminosity, \(L\),
wind mass-loss rate, \(\dot{M}\), and terminal velocity, \(V\wind\),
together with the ambient stream's mass density, \(\rho\), relative
velocity \(v_\infty\), and effective dust opacity, \(\kappa\).  It is convenient
to define dimensionless versions of these parameters by normalizing to
typical values:
\begin{align*}
  % \label{eq:stellar-parameters}
  \dot{M}_{-7} &= \dot{M} / \bigl(\SI{e-7}{M_\odot.yr^{-1}}\bigr) \\
  V_3 &= V / \bigl(\SI{1000}{km.s^{-1}}\bigr) \\
  L_4 &= L / \bigl(\SI{e4}{L_\odot}\bigr) \\
  v_{10} &= v_\infty / \bigl( \SI{10}{km.s^{-1}} \bigr) \\
  n &= (\rho / \bar{m}) / \bigl( \SI{1}{cm^{-3}} \bigr) \\
  \kappa_{600} &= \kappa / \bigl( \SI{600}{cm^2.g^{-1}} \bigr) \ ,
\end{align*}
where \(\bar{m}\) is the mean mass per hydrogen nucleon
(\(\bar{m} \approx 1.3 m_{\text{p}} \approx \SI{2.17e-24}{g}\) for solar
abundances).  In terms of these dimensionless parameters,
equations~(\ref{eq:Rstar}--\ref{eq:wind-efficiency}) take the
convenient forms:
%% XREF \ref{eq:wind-eta-typical}
%% XREF \ref{eq:taustar-typical}
%% XREF \ref{eq:Rstar-typical}
\begin{align}
  \label{eq:Rstar-typical}
  R_* / \si{pc} &= \num{2.21} \, (L_4 / n)^{1/2} \,v_{10}^{-1} \\
  \label{eq:taustar-typical}
  \tau_* &= \num{0.0089} \,\kappa_{600} \, (L_4 \,n)^{1/2} \,v_{10}^{-1} \\
  \label{eq:wind-eta-typical}
  \eta\wind &= \num{0.495} \,\dot{M}_{-7} \,V_3  \,L_4^{-1} \ .
\end{align}



\section{Shape of a dusty radiative bow wave}
\label{sec:shape-dust-wave}

As an alternative to hydrodynamic or magnetohydrodynamic bow shocks, it is possible that some observed emission arcs may be

\begin{figure}
  \centering
  \includegraphics[width=\linewidth]{figs/dust-trajectories}
  \caption[Dust grain trajectories]{Dust grain trajectories under
    influence of a repulsive central \(r^{-2}\) radiative force.  Dust
    grains approach from the right at a uniform velocity and with a
    variety of impact parameters (initial \(y\)-coordinate). The
    central source is marked by a red star at the origin, and its
    radiative force deflects the trajectories into a hyperbolic shape,
    each of which reaches a minimum radius marked by a small black
    square.  The incoming hyperbolic trajectories are traced in gray
    and the outgoing trajectories are traced in red.  The locus of
    closest approach of the outgoing trajectories is parabolic in
    shape (traced by the thick, light gray line) and this constitutes
    the inner edge of the bow wave. }
  \label{fig:dust-trajectories}
\end{figure}


%%% Local Variables:
%%% mode: latex
%%% TeX-master: "quadrics-bowshock"
%%% End:




\section{Imperfect coupling between gas and dust}
\label{sec:imperf-coupl-betw}

\begin{figure}
  \includegraphics[width=\linewidth]{figs/test-Fdrag-components}
  \caption{Contributions of different collider species to the
    dimensionless drag force, \(f\drag / f_*\), as a function of
    gas--grain slip velocity, \(w\).  Solid lines show the Coulomb
    (electrostatic) drag, while dashed lines show the Epstein
    (solid-body) drag.  Results are shown for dimensionless grain
    potential \(\phi = 10\).  All Coulomb forces scale with
    \(\phi^2\), while the Epstein forces are independent of \(\phi\).  The
    species labelled ``CNO++'' represents the combined effect of all
    metals (see footnote~\ref{fn:metal-drag}).}
  \label{fig:drag-components}
\end{figure}


\begin{figure}
  \includegraphics[width=\linewidth]{figs/test-Fdrag-param-space}
  \caption{Regimes of gas--grain drag as a function of slip velocity
    and grain potential.  The different regimes are indicated by bold
    roman numerals, as explained in
    Table~\ref{tab:fdrag-regimes}. Blue shading indicates regions
    dominated by Epstein (solid-body) drag, whereas red and green
    shading indicate regions dominated by Coulomb drag due to protons
    and electrons, respectively.  In each case, the saturated color
    represents a contribution \(> 70\%\) of the relevant component to
    the total drag force, while progressively lighter shading
    represents the \(> 60\%\) and \(> 50\%\) levels.  The thick white
    dotted line indicates the transition between the subthermal and
    superthermal regimes for protons, while the thin white dotted line
    indicates the corresponding transition for electrons.  Contours
    show the total drag force in units of \(f_*\) (see
    eq.~[\ref{eq:fstar}]) in decade intervals from \(0.1\) to
    \(10^4\), as labelled.  Results are shown for
    \(T = \SI{8000}{K}\) and \(n = \SI{100}{cm^{-3}}\), but the
    differences are very slight throughout the ranges
    \(T = \text{\SIrange{5000}{15000}{K}}\) and
    \(n = \text{\SIrange{e-3}{e6}{cm^{-3}}}\).}
  \label{fig:drag-v-phi-plane}
\end{figure}


\begin{figure*}
  \includegraphics[width=\linewidth]{figs/gas-grain-drag-photoionized}
  \caption{Dimensionless drag force, \(f\drag / f_*\), as a function
    of gas--grain slip velocity, \(w\), for different values of the
    grain potential in thermal units, \(\phi\).  Contributions from
    proton and electron Coulomb (electrostatic) drag, as well as
    Epstein (solid-body) drag are indicated.  Examples of subsonic and
    highly supersonic stable drift velocities are shown (thin dark
    blue arrows), where the drag force is in equilibrium with the
    radiation force (thick dark blue dashed lines), while blue shading
    indicates the unstable, mildly supersonic velocity regime, where
    no stable drift equilibrium exists.  Inset graph shows \(\phi\) as a
    function of the radiation parameter,
    \(\Xi = P_{\mathrm{rad}} / P_{\mathrm{gas}}\) on a log--linear scale
    for a collection of Cloudy models (see
    Appendix~\ref{sec:cloudy-models-dust}).}
  \label{fig:gas-grain-drag-photoionized}
\end{figure*}


% \begin{figure}
%   \includegraphics[width=\linewidth]{figs/decouple-v-n-plane}
%   \caption{As Fig.~\ref{fig:zones-v-n-plane}(a), but accounting for
%     gas-grain decoupling with constant efficiency \(\xi = 0.07\). }
%   \label{fig:decouple-v-n-plane}
% \end{figure}



If the radiation field is sufficiently strong, then the collisional
coupling between grains and gas will break down.  In this section, we
calculate the regions of star+stream parameter space where this might
occur, leading to a separation of the bow into an outer dust wave and
an inner, dust-free bow shock.

\subsection{Drag force on grains}
\label{sec:drag-force-grains}

\begin{table}
  \centering
  \caption{Regimes of drag force as function of grain potential and slip speed}
  \label{tab:fdrag-regimes}
  \renewcommand\arraystretch{1.3}
  \resizebox{\linewidth}{!}{%
    \begin{tabular}{@{}r l l l@{}}
    \toprule
      & Regime & Approximate criteria & \(f\drag / f_*\) \\ \midrule
      I & Epstein subsonic & \(\phi^2 \ll 1\)
                             and \(w_{10} < 1\) & \(1.5\, w_{10}\) \\
      II & Epstein supersonic & \(w_{10} > 1\)
                                and \(w_{10} > 5\,\abs{\phi}\)& \( w_{10}^2\) \\
      III & Coulomb p\(^+\) subthermal & \(\phi^2 > 1\)
                                         and \(w_{10} < 1\) & \((1 + 20\, \phi^2)\,
                                                              w_{10}\) \\
    % & Coulomb p\(^+\) peak & \(\phi^2 > 1\)
    %                          and \(w_{10} \approx 1\) & \(1 + 10\, \phi^2\) \\
      IV & Coulomb p\(^+\) superthermal & \(\phi^2 > 1\)
                                          and \(1 < w_{10} < 5\) & \(w_{10}^2
                                                                   + 10\, \phi^2/w_{10}^2 \) \\
      V & Coulomb e\(^-\) subthermal & \(\phi^2 > 20\)
                                 and \(5 < w_{10} < 42\) & \(0.48\, \phi^2 \,
                                                           w_{10}\) \\
    \bottomrule
  \end{tabular}
  }
\end{table}

The drag force on a charged dust grain moving at a relative speed
\(w\) through a plasma has contributions from both direct collisions
and from electrostatic Coulomb interactions with ions and electrons.
We use the expressions in \citet{Draine:1979a}, equations~(4)--(6),
considering the contributions from protons, electrons, helium
ions,\footnote{Helium is assumed to be singly ionized, leading to only
  a small contribution to the drag force.  For much hotter stars, such
  as the central stars of planetary nebulae, helium may be doubly
  ionized, which leads to a fourfold increase in its Coulomb drag
  contribution, which is significant for \(w < \SI{5}{km.s^{-1}}\).} %
and metal ions.\footnote{%
  \label{fn:metal-drag}
  All metals are lumped together as a single species, assuming
  standard \hii{} region gas-phase abundances.  They are dominated by
  C and O, with minor contributions from N and Ne.  The total
  abundance is \num{8.5e-4} and the effective atomic weight is
  \num{15.3}.  All are assumed to be doubly ionized.  Their largest
  relative contribution to the drag force is for
  \(w < \SI{2}{km.s^{-1}}\), but is less than 1\% even there.} %
Results are shown in Figure~\ref{fig:drag-components}, where dashed
lines correspond to direct solid body collisions and solid lines to
electrostatic interactions.  The latter depend on the grain potential,
which is described in dimensionless terms by \(\phi\), which is the
electrostatic potential energy of a unit charge at the surface of a
grain of charge \(z\grain\) and radius \(a\), in units of the
characteristic thermal energy of a gas particle:
\begin{equation}
  \label{eq:phi-potential}
  \phi = \frac{e^2 z\grain}{a kT} \ .
\end{equation}
The electrostatic contributions to \(f\drag\) are proportional to
\(\phi^2\) (results are shown for \(\abs{\phi} = 10\)), whereas the
solid-body contributions are independent of \(\phi\).  The drag force is
put in dimensionless units by dividing by a characteristic force:
\begin{equation}
  \label{eq:fstar}
  f_* = 2 n k T \cdot \pi a^2 \ , 
\end{equation}
which is approximately\footnote{%
  To simplify the exposition, the gas pressure in this section is
  calculated assuming a fully ionized, pure hydrogen plasma, yielding
  \(P\gas = 2 n k T\). For typical ISM abundances, the contribution of
  helium and its corresponding electrons yield a correction to this of
  order 5\%.  The required modifications when a cool star interacts
  with a predominantly neutral gas stream are discussed later.  } %
the ionized gas pressure multiplied by the grain geometric cross
section.

For grains with low electric charge, \(\phi^2 \ll 1\), the drag force is
dominated by direct collisions of protons with the grain (dashed red
line in Fig.~\ref{fig:drag-components}).  The gas collisional mean
free path is much larger than the grain size, so the drag is in the
Epstein regime \citep{Weidenschilling:1977b}.
% This is illustrated by
% the \(\phi = 0.25\) case (blue line) in
% Figure~\ref{fig:gas-grain-drag-photoionized}.
As the relative gas--grain slip speed, \(w\), increases, \(f\drag\)
first increases linearly with \(w\) reaching \(f\drag \approx f_*\) at
\(w = \sound \approx \SI{10}{km.s^{-1}}\), then transitions to a quadratic
increase in the supersonic regime.

As \(\abs{\phi}\) increases, long-range electrostatic interactions with
protons within the Debye radius (Coulomb drag) become increasingly
important at subsonic relative velocities, as shown by the solid lines
in Figure~\ref{fig:drag-components}).
% in
% Figure~\ref{fig:gas-grain-drag-photoionized} by the orange
% (\(\phi = 1\)), green (\(\phi = 4\)), and red (\(\phi = 16\)) lines.
However, the Coulomb drag has a peak when \(w\) is equal to the
thermal speed of the colliders, which is
\(\approx \SI{10}{km.s^{-1}}\) for protons, giving a maximum strength of
\begin{equation}
  \label{eq:fdrag-maximum}
  f_{\mathrm{max}} = 0.5\, (\ln\Lambda)\, \phi^2 f_* \approx 10\, \phi^2 f_* \ , 
\end{equation}
where \(\Lambda\) is the plasma parameter (number of particles within a
Debye volume), such that
\(\ln\Lambda = 23.267 + 1.5 \ln T_4 - 0.5 \ln n\).  At highly super-thermal
speeds, the Coulomb drag falls asymptotically as
\(f\drag \propto 1 / w^{2}\).  The thermal speed of electrons is higher than
that of the protons by a factor of \((m_p / m_e)^{1/2}\), so that the
electron Coulomb drag (solid light blue line) gives a second peak of
similar strength, but at \(w \approx \SI{430}{km.s^{-1}}\).  The behavior of
\(f\drag\) in all these different regimes is summarised in
Table~\ref{tab:fdrag-regimes}, in terms of \(\phi\) and
\(w_{10} = w / \SI{10}{km.s^{-1}}\).  This is further illustrated in
Figure~\ref{fig:drag-v-phi-plane}, where each of the drag regimes is
located on the \((w, \abs{\phi})\) plane.


% \begin{equation}
%   \label{eq:fdrag-regimes}
%   f\drag \approx
%   \begin{cases}
%     \text{Epstein subsonic (\(\phi^2 \ll 1\) and \(w_{10} < 1\)):}
%     & w_{10}\, f_* \\
%     \text{Epstein supersonic (\(\phi^2 \ll 1\) and \(w_{10} > 1\)):}
%     & w_{10}^2\, f_* \\
%     \text{Coulomb p\(^+\) subthermal (\(\phi^2 > 1\) and \(w_{10} < 1\)):}
%     & (1 + 20 \phi^2)\, w_{10}\, f_* \\
%     \text{Coulomb p\(^+\) peak (\(\phi^2 > 1\) and \(w_{10} \approx 1\)):}
%     & (1 + 10 \phi^2)\, f_* \\
%     \text{Coulomb p\(^+\) superthermal (\(\phi^2 > 1\) and \(1 < w_{10} < 5\)):}
%     & (w_{10}^2 + 10 \phi^2/w_{10}^2) \, f_* \\
%     \text{Coulomb e\(^-\) subthermal (\(\phi^2 > 20\) and \(5 < w_{10} < 42\)):}
%     & 0.48 \phi^2 \, w_{10}\, f_* \\
%   \end{cases}
% \end{equation}

% Or \Lambda = (4 pi / 3) n r_D^3?
% Where Debye length is r_D^2 = k T / 4 pi n e^2
% => \Lambda = n r_D (4 pi / 3) k T / 4 pi n e^2
% = k T r_D / 3 e^2
% This is 9 times less than the other expression
% Anyway, from kappa notes I have
% \ln\Lambda = 9.452 + 1.5 ln(T) - 0.5 ln(n)

% If I am going to use log10, then the coefficients get divided by
% ln(10) = 2.30258509299, and if we use T4, then we add 1.5 ln(1e4) =
% 13.815.  So we get 23.267 + 0.651 log10(T4) - 0.217 log10(n).  Nope,
% best with natural log

\subsection{Gas--grain separation: drift and rip}
\label{sec:gas-grain-separ}

In Appendix~\ref{sec:gas-free-bow} we calculate the behaviour of an
incoming stream of dust grains, subject only to the repulsive
radiation force from a star.  For an initial inward radial trajectory,
the dust grain motion is decelerated and turned around, reaching a
minimum radius \(R\starstar\), given by equation~\eqref{eq:dust-r0}.
This drag-free radiative turnaround radius, \(R\starstar\), is smaller
for higher initial inward velocities, but is independent of the
density of the incoming stream.  We are now in a position to see how
gas--grain drag will modify this picture.

From equations~\eqref{eq:dust-rad-force} and~\eqref{eq:fstar}, we can
write the radiation force acting on a grain as
\begin{equation}
  \label{eq:frad-Xi}
  f\rad = \Qp\, \Xi\, f_* \ ,
\end{equation}
where \(\Qp\) is the grain's radiation pressure efficiency (see
footnote~\ref{fn:Qp} in Appendix~\ref{sec:gas-free-bow}) and \(\Xi\) is
the local radiation parameter, defined as the ratio of direct stellar
radiation pressure to gas pressure:
\begin{equation}
  \label{eq:Xi-Prad-over-Pgas}
  \Xi \equiv \frac{P\rad}{P\gas} \approx \frac{L}{4 \pi R^2 c\, (2 n k T)} \ ,
\end{equation}
where the last expression corresponds to the optically thin limit.
The grain potential \(\phi\) is also primarily determined by \(\Xi\), as
shown in Appendix~\ref{sec:cloudy-models-dust} and the inset graph of
Figure~\ref{fig:gas-grain-drag-photoionized}, but with a slow
dependence, which can be approximated as
\begin{equation}
  \label{eq:phi-vs-Xi}
  \phi(\Xi) \approx 1.5 \bigl( 2.3 +  \ln \Xi \bigr) \ .
\end{equation}
There are also slight secondary dependencies on the grain composition and
stellar spectrum.  The relationship given in eq.~\eqref{eq:phi-vs-Xi}
is appropriate for graphite grains and for stellar effective
temperatures in the range \SIrange{20}{30}{kK}.  For hotter stars than
this, \(\phi\) should be multiplied by a further factor of \(1.5\), while
for silicate grains it should be divided by \(1.5\).

In the outer regions of the photoionized volume around an OB star,
close to the ionization front, the radiation parameter is low, with
typical value \(\Xi \sim 0.1\).  In this regime, the negative charge
current at the grain surface due to electron collisions is roughly in
balance with the positive current due to the ultraviolet photoelectric
effect \citep{Weingartner:2001b}, leading to a low grain potential,
\(\abs{\phi} < 1\), which may be positive or negative.  The low
\(\Xi\) means that the radiative force is also weak:
\(f\rad \sim 0.1 f_*\) from equation~\eqref{eq:frad-Xi} if
\(\Qp \sim 1\) at UV wavelengths, which is true for all but the smallest
grains.  Thus, from the equations for \(f\drag\) given in
Table~\ref{tab:fdrag-regimes}, the radiative force can be balanced by
Epstein drag if \(w_{10} \sim 0.1\), leading to a small equilibrium drift
velocity, \(w\drift < \SI{1}{km.s^{-1}}\), of the grains with respect
to the gas.  This drift is much smaller than the inward stream
velocities that we are considering
(\(v_\infty > \SI{10}{km.s^{-1}}\)), so the dust follows the gas stream at
a slightly reduced velocity (\(< 10\%\)), and (by mass conservation) a
slightly increased density.  Each grain exerts an exactly opposite
force to \(f\drag\) upon the gas, but since the dust-gas mass ratio,
\(Z\grain\), is small, this produces a negligible acceleration of the
gas.

\begin{table}
  \caption{Critical values of radiation parameter at the rip point: \(\Xi_\dag\)}
  \centering
  \begin{tabular*}{0.75\columnwidth}{l @{\quad\quad\quad\quad} S S} \toprule
    & \multicolumn{2}{c}{Grain composition} \\
    Spectrum & {Graphite} & {Silicate}
    \\ \midrule
    B star & 1000 +- 400 & 350 +- 150 \\
    O star & 3000 +- 500 & 2500 +- 500 \\
    \bottomrule
    \addlinespace
    \multicolumn{3}{@{}p{0.75\columnwidth}@{}}{
    Calculated from the Cloudy models shown in Figure~\ref{fig:drift-gn}. 
    Uncertainties represent variations with grain size and gas density. 
    See Appendix~\ref{sec:cloudy-models-dust} for further details.}
  \end{tabular*}
  \label{tab:Xi-rip}
\end{table}

As the dusty stream approaches the star, the radiation parameter
\(\Xi\) will increase, with a dependence of \(R^{-2}\) once the stream
is well inside the ionization front.  This increases \(f\rad\)
(eq.~[\ref{eq:frad-Xi}]), but also increases the grain potential,
\(\phi\) (eq.~[\ref{eq:phi-vs-Xi}]) due to the increasing dominance of
grain charging by photoelectric ejection.  Initially, this results in
a lowering of the equilibrium drift velocity to \(w_{10} \sim 0.01\) as
the Coulomb drag kicks in (see Appendix~\ref{sec:cloudy-models-dust}).
However, at smaller radii the slow logarithmic increase in
\(\phi(\Xi)\) means that the drift velocity must start increasing again to
accommodate the linear increase of \(f\rad(\Xi)\).  Eventually,
\(f\rad\) exceeds \(f_{\mathrm{max}}\), the maximum drag force that
proton Coulomb interactions can provide
(eq.~[\ref{eq:fdrag-maximum}]).  This occurs at a critical value of
the radiation parameter, which we denote the \textit{rip point}:
\(\Xi_\dag \sim 1000\).  The variations in \(\Xi_\dag\) with star and grain
parameters, which are of order \SI{+- 0.5}{dex}, are listed in
Table~\ref{tab:Xi-rip} and illustrated graphically in
Figure~\ref{fig:drift-gn}.  The radius of the rip point, \(R_\dag\), can
be expressed in terms of \(R_*\), the fiducial optically thick bow
shock radius introduced in Paper~I's \S~2.1:
%% XREF \ref{sec:three-bow-regimes}
\begin{equation}
  \label{eq:Rdag-over-Rstar}
  R_\dag = \frac{v_\infty}{\sound}\, \Xi_\dag^{-1/2} R_* \approx v_{10}\, \Xi_\dag^{-1/2} R_* \ ,
\end{equation}
where we have made use of equation~\eqref{eq:Xi-Prad-over-Pgas} and
Paper~I's equation~(3).
%% XREF \eqref{eq:Rstar}

What happens to the dust grain following this catastrophic breakdown
of gas--grain coupling depends on the relation between the rip point
radius, \(R_\dag\), and the drag-free radiative turnaround radius,
\(R\starstar\).  If \(R_\dag > R\starstar\), then the grain's inertia
will still carry it in as far as \(R\starstar\) and the initial
trajectory will be almost identical to that described in
Appendix~\ref{sec:gas-free-bow} for the drag-free case. However, after
being turned around by the radiation field and pushed out past
\(R_\dag\) again, the grain will \emph{recouple} to the gas and be
dragged back for a second approach.\footnote{If the initial impact
  parameter of the trajectory is not strictly \(b = 0\), then the
  resultant lateral component of \(f\rad\) will mean that \(b\) will
  be much increased for the second approach.} %
We will refer to this as an \textit{inertia-confined dust wave}
(IDW).  From equations~(\ref{eq:Rdag-over-Rstar},
\ref{eq:Rstarstar-over-Rstar},
\ref{eq:dust-wave-high-density-condition}), the condition
\(R_\dag > R\starstar\) corresponds to
\(\tau_* < (\kappa/\kappa\grain) \tau_{*,\text{max}}\), which is indicated by dashed
lines in the left panel of Figure~\ref{fig:existence-dust-wave}.  If,
on the other hand, \(R_\dag < R\starstar\), then the tail wind provided
by the gas carries the grain closer to the star than its inertia would
naturally take it.  When the grain finally decouples at \(R_\dag\) it
experiences a much higher unbalanced \(f\rad\), which can initially
accelerate it to outward velocities significantly higher than the
inflow velocity if \(R_\dag \ll R\starstar\).  We will refer to this case
as a \textit{drag-confined dust wave} (DDW).


\begin{figure*}
  \centering
  \includegraphics[width=\linewidth]{figs/existence-dust-wave}
  \caption{Regions of stream parameter space \((v, n)\) where dust
    waves may form around main-sequence OB stars of
    \SIlist{10;20;40}{M_\odot} (see Paper~I's Tab.~1).  This figure is
    similar to Paper~I's Fig.~2,
    %% XREF\ref{fig:zones-v-n-plane},
    except that the velocity axis is logarithmic and extends out to
    \SI{1000}{km.s^{-1}}.  Overlapping colored shapes show parameters
    where dust waves may be allowed in the cases of large
    (\(a = \SI{0.2}{\um}\)) and small (\(a = \SI{0.02}{\um}\))
    graphite and silicate grains, as labeled in the left panel.  For
    \((v, n)\) outside of these shapes, dust waves cannot occur for
    the reasons indicated by labeled orange arrows in the center
    panel.  Labeled dashed lines in the right panel show the
    correspondence between the region boundaries and each dust wave
    existence condition given in
    equations~(\ref{eq:dust-wave-velocity-condition},
    \ref{eq:dust-wave-low-density-condition},
    \ref{eq:dust-wave-high-density-condition}). Heavy dashed lines in
    the left panel show where the rip point and the drag-free
    turnaround radius coincide.  Dust waves above these lines are drag
    confined, while dust waves below the lines are inertia confined.
  }
  \label{fig:existence-dust-wave}
\end{figure*}

\subsection{Existence conditions for dust waves}
\label{sec:exist-cond-separ}

In order for a separate outer dust wave to exist, it is necessary for
the grains to decouple from the incoming gas stream before the stream
hits the hydrodynamic bow shock caused by the stellar wind.  The wind
bow shock radius is \(R_0 = \eta\wind^{1/2} R_*\) (Paper~I's eq.~[12]),
%% XREF \ref{eq:x-cases}
where \(\eta\wind\) is the wind momentum efficiency
(Paper~I's eq.~[13]).
%% XREF \ref{eq:wind-eta-typical}
Therefore, the condition
\(R_\dag > R_0\) becomes from equation~\eqref{eq:Rdag-over-Rstar}:
\begin{equation}
  \label{eq:dust-wave-velocity-condition}
  v_{10} > v_{10,\text{min}} = \bigl( \Xi_\dag \, \eta\wind \bigr)^{1/2} \ . 
\end{equation}
For early O main-sequence stars and OB supergiants, the wind
efficiency is generally high (\(\eta\wind > 0.1\)) and
\(\Xi_\dag > 2000\) (Tab.~\ref{tab:Xi-rip}), so that dust waves can only
exist when the stream velocity is very high
(\(v_\infty > \SI{150}{km.s^{-1}}\)).  For main-sequence B~stars, in
contrast, the wind can be much weaker (\(\eta\wind < 0.01\)) and
\(\Xi_\dag\) is also smaller, so that dust waves are permitted by this
criterion for much lower stream velocities:
(\(v_\infty > \SI{30}{km.s^{-1}}\)).  The same will be true of the
``weak-wind'' class of late O main-sequence stars, which also show
\(\eta\wind < 0.01\) (see Paper~I, \S~4).

However, there are other conditions that need to be satisfied in order
for the dust wave to exist.  For instance, the drag-free turnaround
radius must also be outside the bow shock: \(R\starstar > R_0\),
otherwise the radiation is incapable of repelling the grain
opportunely, even once it has decoupled from the gas.  From
equation~\eqref{eq:dust-r0}, together with Paper~I's equations~(3, 9),
%% XREF \eqref{eq:Rstar}
%% XREF \eqref{eq:tau-star},
we find
\begin{equation}
  \label{eq:Rstarstar-over-Rstar}
  \frac{R\starstar}{R_*} = \frac{2 \kappa\grain \tau_*}{\kappa} \ , 
\end{equation}
so the condition becomes
\begin{equation}
  \label{eq:dust-wave-low-density-condition}
  \tau_* >  \tau_{*,\text{min}} = 0.5\, \frac{\kappa}{\kappa\grain}\, \eta\wind^{1/2} 
  % \frac{\kappa \eta\wind^{1/2}}{2 \kappa\grain}
  \ . 
\end{equation}
The average value of the factor \(\kappa / \kappa\grain\) over the entire grain
population must be equal to the dust--gas mass ratio,
\(Z\grain \approx 0.01\), but the factor will vary between grains, according
to their size and composition.\footnote{%
  Remember that \(\kappa\) is the opacity per unit mass of gas, while
  \(\kappa\grain\) is the opacity per unit mass of a particular grain. In
  both cases, averaged over the stellar spectrum.} %
In particular, it will be relatively larger for the largest grains
(\(a \approx \SI{0.2}{\um}\)), which dominate the total dust mass, and
smaller for the smaller grains (\(a \approx \SI{0.02}{\um}\)), which
dominate the UV opacity.  Given the dependence of \(\tau_*\) on the
stream parameters (Paper~I's eq.~[15]),
%% XREF \ref{eq:taustar-typical}
for a given
stellar luminosity this condition corresponds to a minimum value for
\(n / v_\infty^2\).

A third condition comes from requiring \(R_\dag > R_0\) in the radiation
bow wave regime (see Paper~I's \S~2.1),
%% XREF \ref{sec:three-bow-regimes}
where
\(R_0 \approx 2 \tau_* R_*\).  This yields
\begin{equation}
  \label{eq:dust-wave-high-density-condition}
  \tau_* < \tau_{*,\text{max}} = 0.5\, v_{10}\, \Xi_\dag^{-1/2} \ , 
\end{equation}
which, for a given stellar luminosity, corresponds to a maximum value
for \(n / v_\infty^4\).  Thus, for a given stream velocity that satisfies
equation~\eqref{eq:dust-wave-velocity-condition},
equations~(\ref{eq:dust-wave-low-density-condition},
\ref{eq:dust-wave-high-density-condition}) determine respectively the
minimum and maximum stream density for which a dust wave can exist.

The combined effects of the three conditions are illustrated in
Figure~\ref{fig:existence-dust-wave} for each of the three example
main sequence stars from Paper~I's Table~1.
%% XREF \ref{tab:stars}
Further restrictions
on the existence of dust waves arise when the effects of magnetic
fields are considered, as will be discussed in
\S~\ref{sec:magn-effects-grain} below.  Note that the three conditions
are restrictions solely on the formation of an \textit{outer} dust
wave, that is, outside of the wind-supported hydrodynamic bow shock.
In the case of the equation~\eqref{eq:dust-wave-low-density-condition}
condition, there is a further possibility: if the gas--grain coupling
(and magnetic coupling) is so weak that it is still unimportant at the
higher densities found in the bow shock shell, then an
inertia-confined \textit{inner} dust wave may form inside the bow
shock, even when \(\tau_* < \tau_{*,\text{min}}\).  Although the same might
be thought to apply to the condition of
equation~\eqref{eq:dust-wave-velocity-condition}, this is not the
case, since the density compression in the bow shock will reduce the
radiation parameter, \(\Xi\), which moves the rip point, \(R_\dag\), to an
even smaller radius.  Therefore, if radiation has not managed to
decouple a grain before it passes through the shock, it is unlikely to
be able to do it afterwards.

\begin{figure*}
  \centering
  \includegraphics[width=\linewidth]{figs/dust-wave-phase-trajectories-annotate}
  \caption{Trajectories of small graphite grains
    (\(a = \SI{0.02}{\um}\)) at impact parameter \(b = 0\) for two
    example cases (see yellow ``+'' symbols in left panel of
    Fig.~\ref{fig:existence-dust-wave}), which differ only in the
    stream velocity: \(v = \SI{40}{km.s^{-1}}\) (left panels) and
    \SI{80}{km.s^{-1}} (right panels).  In both cases, the stream
    density is \(n = \SI{1}{cm^{-3}}\) and the central star is a
    \SI{10}{M_\odot} main-sequence B star (see Tab.~\ref{tab:stars}).
    Upper panels show the evolution of grain radius, \(R\) (blue
    curve, normalized by the rip point radius, \(R_\dag\)), and grain
    velocity, \(v\) (orange curve, normalized by the gas stream
    velocity).  The origin of the time axis is set to the moment of
    closest approach of the grain to the star: \(R = \Rmin\).  Lower
    panels show the trajectories in phase space: position versus
    gas--grain relative slip velocity (\(w = \abs{v - v_\infty}\)).  Filled
    contours show the net force on the grain: \(f\rad - f\drag\), with
    positive values in blue and negative values in red.  The heavy
    dotted line shows where there is no net force: \(f\rad = f\drag\).
    The grain trajectory (thick, solid black line with arrows)
    initially follows this line, but departs from it after the rip
    point. In the left panel, the grain enters a limit cycle between
    decoupling (rip) and re-coupling (snap back).  In the right panel,
    the grain spirals in on the stagnant drift point.  See text for
    further details.}
    \label{fig:phase-space-trajectories}
\end{figure*}

\subsection{Grain trajectories along the symmetry axis}
\label{sec:grain-traj-along}

The post-decoupling behavior of the grain depends on the sign of
\(d f\drag / d w\) when \(w = \abs{v_\infty}\).  If this derivative is
positive, as is the case in drag regimes~II and V (see
Tab.~\ref{tab:fdrag-regimes} and Fig.~\ref{fig:drag-v-phi-plane}),
then the grain can reach a stable equilibrium drift at rest with
respect to the star\footnote{%
  Again, this is only strictly true when the impact parameter is zero.
  However, as we show below, it is a reasonable approximation over a
  range of impact parameters in the case where the angle between the
  magnetic field direction and the stream velocity is not too
  large.} %
at a point \(R_\ddag\), which we call the \textit{stagnant drift
  radius}. If the stream velocity is not excessively high
(\(v_\infty < \SI{150}{km.s^{-1}}\) when \(\phi = 4\), or
\(< \SI{300}{km.s^{-1}}\) when \(\phi = 16\)), then the equilibrium
\(f\rad\) is less than the value at the rip point, requiring a lower
value of the radiation parameter: \(\Xi_\ddag < \Xi_\dag\).  The resultant
stagnant drift radius is therefore outside the rip point:
\(R_\ddag > R_\dag\).

On the other hand, if \(d f\drag / d w < 0\) when
\(w = \abs{v_\infty}\), then the equilibrium is unstable and no stagnant
drift is possible.  This occurs for drag regime~IV, which applies when
\(\phi > 1\) and
\(\SI{10}{km.s^{-1}} < v_\infty < \SI{50}{km.s^{-1}}\).  There is also a
second unstable regime (partially visible in the upper-right corner of
Fig.~\ref{fig:drag-v-phi-plane}), which is related to the thermal peak
in the electron Coulomb drag when \(\phi > 30\) and
\(\SI{400}{km.s^{-1}} < v_\infty < \SI{2000}{km.s^{-1}}\), but this is not
relevant to bow shocks around OB~stars.\footnote{%
  It may apply in other contexts, such as outflows from AGN, since
  detailed modeling of grain charging around quasars
  \citep{Weingartner:2006a} implies that grain potentials as high as
  \(\phi \sim 100\) can be achieved.} %

An example of each of these two behaviors is illustrated in
Figure~\ref{fig:phase-space-trajectories}.  The left panels show the
case where \(v_\infty = \SI{40}{km.s^{-1}}\), which is in the unstable
regime, resulting in periodic ``limit-cycle'' behavior (the parameters
of this model correspond to the yellow ``plus'' symbol labeled ``40''
in the left panel of Fig.~\ref{fig:existence-dust-wave}).  During the
grain's first approach, it starts to follow a phase trajectory (lower
left panel) along the \(f\rad - f\drag = 0\) contour, corresponding to
equilibrium drift, in which the grain begins to move a few
\si{km.s^{-1}} slower than the gas stream.  Then, when it reaches the
rip point (\(R = R_\dag\), \(w \approx \SI{10}{km.s^{-1}}\)) it suddenly
experiences a large unbalanced outward radiation force (blue region of
phase space in Fig.~\ref{fig:phase-space-trajectories}). The grain's
inward momentum carries it to the point \(\Rmin \approx 0.85 R_\dag\), before
it is expelled at roughly twice the inflow speed.  However, after
moving outward, it finds itself in a drag-dominated region of phase
space (red in the figure), and so recouples to the inflowing gas
stream.  The recoupling initiates gradually, as the grain's outward
motion is slowed and it begins to move inward again, but is completed
suddenly once \(w\) again falls below \SI{10}{km.s^{-1}}, in what we
term \textit{snap back}. The net result is that the grain has returned
to exactly the same phase track that it started in on, and so repeats
the cycle indefinitely.

The right panels of Figure~\ref{fig:phase-space-trajectories} show the
case where the stream velocity is doubled to
\(v_\infty = \SI{80}{km.s^{-1}}\), but all other parameters remain the
same.  At this velocity, the equilibrium drift is stable and so the
grain can achieve a stagnant drift solution, where it is stationary
with respect to the star.  The trajectory during the first approach is
similar to the previous case, except that the overshoot of the rip
point is greater, so that \(\Rmin \approx 0.65 R_\dag\) in this case.  This is
a consequence of the fact that the rip point is closer to the
drag-free turnaround radius (\(R_\dag / R\starstar\) is larger than in
the lower velocity case), so that the grain inertia is relatively more
important.  A second consequence of this is that the speed of the
initial expulsion is not so large, being only a little higher than the
inflow velocity.  The qualitative difference between the two cases
emerges after the first recoupling: instead of the snap back and
endless limit cycle, the grain oscillates about the stagnant drift
radius with ever decreasing amplitude, so that after a few oscillation
periods it has come to almost a complete rest.

\begin{figure}
  \includegraphics[width=\linewidth]{figs/onaxis-stats-plot-MS10-v080-gra002}
  \caption{Bow radius as a function of stream density for a stream of
    initial velocity \SI{80}{km.s^{-1}}, which interacts with a
    \SI{10}{M_\odot} main-sequence B~star.  This corresponds to a vertical
    slice through the left panel of
    Fig.~\ref{fig:existence-dust-wave}.  At low densities, the
    hydrodynamic bow shock (blue line) is larger than the drag-free
    turnaround radius for small carbon grains, meaning that a grain's
    inertia carries it into the bow shock along with the gas, even
    though the gas--grain coupling is not particularly strong.  At
    densities above about \SI{0.05}{cm^{-3}}, however, this is no
    longer true and a separate dust wave forms outside of the
    hydrodynamic bow shock, which is now dust-free (green dashed
    line).  The grains in the dust wave will occupy a range of radii
    (pale orange shading) between \(\Rmin\) (solid orange line) and
    \(R_\ddag\), the stagnant drift radius.  At densities above about
    \SI{1000}{cm^{-3}}, the gas stream starts to feel the effect of
    passing through the dust wave, and above \SI{3e4}{cm^{-3}}, the
    dust wave and bow shock merge to form a radiative bow wave (red
    line), which becomes an optically thick radiative bow shock
    (purple line) above \SI{e6}{cm^{-3}}.}
  \label{fig:decouple-vertical-cut}
\end{figure}

\subsection{Back reaction on the gas flow}
\label{sec:back-reaction-gas}

So far we have ignored the effect of the drag force on the gas stream
itself, but it is clear that this must become important as \(\tau_*\)
approaches \(\tau_{*,\text{max}}\), since that is the point where the
dust wave transitions to a bow wave, in which the dust and gas are
perfectly coupled.  A full treatment of this problem would require
solving the hydrodynamic equations simultaneously with the equations
of motion of the dust grains, which is beyond the scope of this paper.
Instead, we outline a heuristic approach that qualitatively captures
the physics involved.

The maximum drag force experienced by a grain is at the rip point.
Since the grain follows a zero-net-force phase track up until that
point, this can be written with the help of
equations~(\ref{eq:dust-rad-force}, \ref{eq:dust-r0}) as
\begin{equation}
  \label{eq:fdrag-max}
  f\drag (R_\dag) = f\rad(R_\dag) =   \frac{m\grain v_\infty^2 R\starstar}{ 2 R_\dag^2} 
\end{equation}
The timescale of the flow can be characterized by the crossing time
\(R_\dag / v_\infty\), but the residence time of the grain at the bow apex
will be several times larger than this (see previous section).  On the
other hand, the average drag force during this residence will be
several times smaller than \(f\drag (R_\dag)\) if
\(R_\ddag > R_\dag\), which is typically the case.  We therefore parameterize
our ignorance via a dimensionless factor, \(\alpha\), which we expect to be
of order unity, and write the total impulse imparted to the grain by
drag as
\begin{equation}
  \label{eq:grain-impulse}
  J\drag \equiv \int \!f\drag \, dt \approx \alpha f\drag (R_\dag) \frac{R_\dag}{v_\infty}
  = \tfrac12 \alpha \, m\grain v_\infty \, \frac{R\starstar}{R_\dag} \ .
\end{equation}

By Newton's Third Law, an equal and opposite impulse is imparted to
the gas, which will act to decelerate the gas stream as it decouples
from the grains.  Realistically, \(J\drag\) should be summed over the
grain size distribution, but for simplicity we assume that all grains
are identical, so that the mass of gas that accompanies each grain is
given by
\begin{equation}
  \label{eq:gas-mass}
  m_{\text{gas}} = \frac{m\grain}{Z\grain} =  m\grain \, \frac{\kappa\grain}{\kappa} \ . 
\end{equation}
If the gas remains supersonic after decoupling, then thermal pressure
can be ignored and the gas will suffer a change in momentum equal to
\(J\drag\), so that its velocity is reduced by
\(\Delta v = J\drag / m_{\text{gas}}\), which by
equations~(\ref{eq:Rdag-over-Rstar}, \ref{eq:Rstarstar-over-Rstar},
\ref{eq:dust-wave-high-density-condition}, \ref{eq:grain-impulse},
\ref{eq:gas-mass}) is
\begin{equation}
  \label{eq:gas-dv}
  \Delta v = \tfrac12 \alpha \frac{\tau_*}{\tau_{*, \text{max}}} v_\infty\ .
\end{equation}
This deceleration reduces the gas stream's ram pressure before it
interacts with the central star's stellar wind.  The radius of the
dust-free bow shock formed by this interaction is therefore increased
by a factor \((1 - \Delta v / v_\infty)^{-1}\) with respect to the case
calculated in \S~\ref{sec:three-bow-regimes}, yielding
\begin{equation}
  \label{eq:gas-free-bow-shock}
  R_{\text{dfbs}} \approx \frac{\eta\wind^{1/2} R_*}{1 - \tfrac12 \alpha \tau_* / \tau_{*, \text{max}}} \ .
\end{equation}

An example is illustrated in Figure~\ref{fig:decouple-vertical-cut},
where the dust-free bow shock radius is shown by the green dashed line
as a function of stream density, \(n\).  This is calculated for fixed
stream velocity and grain and star properties, so that
\(\tau_* \propto n^{1/2}\) (eq.~[\ref{eq:taustar-typical}]).  In order for
\(R_{\text{dfbs}}\) to match the dust-wave and bow-wave radii at the
point where they cross at \(\tau_* = \tau_{*, \text{max}}\), we find
\(\alpha \approx 1.5\) is required.  It can be seen that the gas deceleration is
negligible over most of the density range for which a separate dust
wave arises.  Only for \(n > \SI{e3}{cm^{-3}}\) does
\(R_{\text{dfbs}}\) begin to curve up from the general \(n^{-1/2}\)
trend, becoming essentially flat at a value
\(R_{\text{dfbs}} \approx (\kappa/\kappa\grain) R\starstar\) until full-coupling is
established at \(n > \SI{3e4}{cm^{-3}}\).  Note, however, that the
treatment described here is very approximate: it does not take into
account the shock that will form once \(J\drag\) reaches an
appreciable fraction of \(m_{\text{gas}} v_\infty\) and, additionally, it
includes a factor, \(\alpha\), whose value has not been rigorously
justified.  More detailed modeling is required to fully understand the
bow behavior in this transition regime.



\section{Magnetic coupling of grains}
\label{sec:magn-effects-grain}


It remains to calculate in detail the effects on grain dynamics of the
plasma's magnetic field, in order to justify the approach taken in
\S~\ref{sec:tight-magn-coupl} and extend those results to include the
effects of gas--grain collisions as in \S~\ref{sec:imperf-coupl-betw}.
The Lorentz force on charged grains due to a magnetic field is
\begin{equation}
  \label{eq:f-lorentz}
  \bm{f}\!\B = \frac{z\grain e}{c} \, \bm{w} \times \bm{B} \ . 
\end{equation}
The direction of the force is perpendicular both to the magnetic
field, \(\bm{B}\), and to the relative velocity, \(\bm{w}\), of the
grain with respect to the plasma.  If \(\bm{w}\) and \(\bm{B}\) (as
seen by the grain) are changing slowly, compared with the
gyrofrequency, \(\omega\B = z\grain e B / m\grain c\), then the grain
motion perpendicular to \(\bm{B}\) is constrained to be a circle of
radius equal to the Larmor radius:
\begin{equation}
  \label{eq:Larmor}
  r\B = \frac{m\grain c w_\perp} {\abs{z\grain} e B} \ ,
\end{equation}
where \(B = \abs{\bm{B}}\) and \(w_\perp\) is the perpendicular component
of \(\bm{w}\).  The component of \(\bm{w}\) parallel to \(\bm{B}\) is
unaffected by \(\bm{f}\!\B\), so the resultant trajectory is helical.

The relative importance of the magnetic field can be characterized by
the ratio of the Larmor radius to the minimum radius, \(\Rmin\),
reached by the grain in the dust wave (see
\S~\ref{sec:grain-traj-along}), where \(\Rmin \approx R_\dag\) for
drag-confined dust waves (DDW), or \(\Rmin \approx R\starstar\) for
inertia-confined dust waves (IDW).  We write the field strength in
terms of the Alfvén speed,
\begin{equation}
  \label{eq:alfven}
  v\alfven = \frac{B}{(4\pi\rho\gas)^{1/2}}
  = 1.9\, \frac{B}{\si{\micro G}} n^{-1/2} \, \si{km.s^{-1}} \ ,
\end{equation}
and the grain charge \(z\grain e\) in terms of the potential \(\phi\) (eq.~[\ref{eq:phi-potential}]) to obtain
\begin{equation}
  \label{eq:larmor-over-Rdag}
  \text{DDW:}\quad \frac{r\B}{\Rmin} =  
  \frac{r\B}{R_\dag} = 0.0140 \,
  a_{\si{\um}}^2 \,
  \frac{w_\perp}{v\alfven} \,
  \left( \frac{\Xi_\dag}{L_4 T_4} \right)^{1/2}
  \frac{\rho\grain}{\phi_\dag}
\end{equation}
and
\begin{equation}
  \label{eq:larmor-over-Rstarstar}
  \text{IDW:}\quad \frac{r\B}{\Rmin} =  
  \frac{r\B}{R\starstar} = 0.0544 \,
  a_{\si{\um}}^3 \,
  \frac{w_\perp}{v\alfven} \,
  \frac{v_{10}^2 }{n^{1/2}} \,
  \frac{1}{L_4 T_4} \,
  \frac{\rho\grain^2}{\Qp \phi\starstar} \ ,
\end{equation}
where \(a_{\si{\um}} = a / \SI{1}{\um}\), \(\rho\grain\) is the grain
material density in \si{g.cm^{-3}}, and we have made use of
equations~(\ref{eq:dust-r0}, \ref{eq:Rdag-over-Rstar}), together with Paper~I's equation~(14).
%% XREF \ref{eq:Rstar-typical}

If \(r\B / \Rmin \ll 1\), then the grains are so strongly coupled to the
field that they can be treated in the guiding-center approximation, in
which the trajectory is decomposed into a tight circular gyromotion
around the field lines, plus a sliding of the guiding center along the
field lines, which is governed by the radiation and drag forces (the
slow \(\bm{f} \times \bm{B}\) drift across the field lines is negligible in
this case, see App.~\ref{sec:tight-magn-coupl}).  In the opposite
limit, \(r\B / \Rmin \gg 1\), magnetic coupling is so weak that the
non-magnetic results of \S~\ref{sec:imperf-coupl-betw} are scarcely
modified.  Assuming \(w_\perp \sim v_\infty\) and adopting a threshold of
\(r\B / \Rmin < 0.1\), equations~(\ref{eq:larmor-over-Rdag},
\ref{eq:larmor-over-Rstarstar}) can be transformed into conditions on
the stream velocity (in \si{km.s^{-1}}) where tight magnetic coupling
will apply: \newcommand\freeze{\ensuremath{_{\text{tight}}}}
\begin{equation}
  \label{eq:velocity-strong-B-coupling}
  v_{\infty} < v\freeze \approx
  \begin{cases}
    \text{drag-confined:}
    & 0.8 \, a_{\si{\um}}^{-2} \, v\alfven \, L_4^{1/2}\\
    \text{inertia-confined:}
    & 6 \, a_{\si{\um}}^{-1} \, v\alfven^{1/3} \, n^{1/6} \, L_4^{1/3} \\
  \end{cases} \ ,
\end{equation}
where we have substituted typical values of the minor parameters
\(\Xi_\dag\), \(\phi_{\dag}\), \(\phi_{**}\), \(\rho\grain\),
\(T_4\).\footnote{%
  The most significant systematic variation in \(v\freeze\) from these
  suppressed parameters is due to grain composition, yielding slightly
  higher values for graphite than for silicate (\SI{+-0.15}{dex}).} %
It is apparent that \(v\freeze\) is very sensitive to the grain size.
For instance, taking a typical \hii{} region value of
\(v\alfven = \SI{2}{km.s^{-1}}\) \citep{Arthur:2011a} and
\(L_4 = 0.63\) (Tab.~\ref{tab:stars}, B1.5~V star), then for the
drag-confined case \(v\freeze \approx \SI{30}{km.s^{-1}}\) for \SI{0.2}{\um}
grains but \(v\freeze \approx \SI{3000}{km.s^{-1}}\) for \SI{0.02}{\um}
grains. Thus, for typical stream velocities of
\SIrange{20}{100}{km.s^{-1}}, the small grains are always tightly
coupled to the magnetic field, but the large grains are only loosely
coupled for the faster streams.

\subsection{Grain trajectories with tight magnetic coupling}
\label{sec:grain-traj-with}

\begin{figure*}
  \centering
  \includegraphics[width=\linewidth]{figs/frozen-stream-map-multi}
  \caption{Drag-confined dust waves with tight magnetic coupling.
    Upper panels show a quasi-parallel field, \(\theta\B = \ang{10}\),
    while lower panels show a quasi-perpendicular field,
    \(\theta\B = \ang{75}\).  Left panels show an incident stream velocity of
    \(v_\infty = \SI{40}{km.s^{-1}}\), while right panels show
    \(v_\infty = \SI{80}{km.s^{-1}}\).  In all cases, the stream density is
    \(n = \SI{10}{cm^{-3}}\) and the calculations are performed for
    small graphite grains, \(a = \SI{0.02}{\um}\), and the
    \SI{10}{M_\odot} main-sequence B~star.  Continuous black lines show
    grain trajectories, with triplets of colored symbols indicating
    the progress of individual cohorts, which entered from the right
    edge at a particular time.  Continuous blue lines show the
    magnetic field, which flows from right to left along with the
    incident stream.  The radius of the rip point, \(R_\dag\), and the
    stagnant drift point, \(R_\ddag\), are shown respectively by red and
    blue dashed lines.  The approximate shape of the wind-supported
    bow shock is shown by the dotted gray line.  The calculations are
    no longer valid after trajectories cross this surface.}
  \label{fig:frozen-stream}
\end{figure*}

\begin{figure}
  \centering
  \includegraphics[width=\linewidth]{figs/frozen-trajectories-multi}
  \caption{Sample grain trajectories for drag-confined dust waves with
    tight magnetic coupling and a quasi-parallel field
    orientation. These are the same models as in the upper row of
    Fig.~\ref{fig:frozen-stream}. (a)~Incident stream velocity of
    \(v_\infty = \SI{40}{km.s^{-1}}\), showing quasi-limit-cycle behavior
    (\numrange{0}{4000} years).  (b)~Incident stream velocity of
    \(v_\infty = \SI{80}{km.s^{-1}}\), showing quasi-stagnation behavior
    (\numrange{500}{2000} years).  In both cases, the streamline with
    initial impact parameter \(y = -0.4 R_\dag\) is shown.}
  \label{fig:frozen-trajectories}
\end{figure}

We can now investigate how the results of the previous sections are
modified by magnetic fields in the tight coupling limit.  For
simplicity, we assume a uniform field in the incoming stream, with
field lines oriented at an angle \(\theta\B\) to the velocity vector that
defines the bow axis.  We also assume a super-alfvénic stream,
\(v_\infty > v\alfven\), so that the radius, \(R_0\), of the wind bow shock
is unaffected by the magnetic field, and additionally assume
\(\tau_* \ll \tau_{*, \text{max}}\), so that the back-reaction of the grain
drag on the plasma is negligible (see previous section) and \(B\)
remains uniform in magnitude and direction in the dust wave region,
outside of the bow shock.

In Appendix~\ref{sec:tight-magn-coupl}, we derive analytic and
semi-analytic results in the limit of zero gas--grain drag, which is
appropriate for inertia-confined dust waves. We find very different
structures, depending on the orientation of the field.  For a parallel
field (Fig.~\ref{fig:inertia-thB0}), the apex of the dust wave occurs
at the same point, \(R\starstar\), as in the non-magnetic case, but
the shape of the dust wave wings is more closed, being hemispherical
rather than parabolic in shape.  For a perpendicular field
(Fig.~\ref{fig:inertia-thB90}), on the other hand, grains in the apex
region are dragged very close to the star and no dust wave forms
there.  A dust wave can form in the wings, with impact parameter
\(> R\starstar\), which is roughly parabolic in shape, but more
swept-back than in the non-magnetic case.  Whether such a dust wave
will exist in practice depends critically on the size and shape of the
MHD wind-supported bow shock.

In Figures~\ref{fig:frozen-stream} and~\ref{fig:frozen-trajectories}
we show example results for drag-confined dust waves, which are
calculated by numerically integrating the grain's equation of motion,
as described in Appendix~\ref{sec:equat-moti-grains}. Apart from the
inclusion of the magnetic field, the model parameters are the same as
used in Figure~\ref{fig:phase-space-trajectories}, with the exception
that the stream density is increased to \SI{10}{cm^{-3}}.\footnote{The
  reason for using a higher density is to decrease the amplitude of
  the radial oscillations of the trajectories, which allows the dust
  wave structure to be more clearly perceived in the figures. } This
time, we use quasi-parallel (\(\theta\B = \ang{10}\)) and
quasi-perpendicular (\(\theta\B = \ang{75}\)) field orientations.  The
quasi-parallel field is most similar to the non-magnetic case, and the
models shown in the two upper panels of Figure~\ref{fig:frozen-stream}
closely mirror the cases shown in
Figure~\ref{fig:phase-space-trajectories}, with a limit cycle behavior
when \(v_\infty = \SI{40}{km.s^{-1}}\) and stagnant drift when
\(v_\infty = \SI{80}{km.s^{-1}}\).

The principle difference from the 1-D axial trajectories discussed in
\S~\ref{sec:grain-traj-along} is that the small \ang{10} misalignment
of \(\bm{B}\) from the incident stream direction causes a slow
sideways migration, which puts a finite limit on the time a grain can
reside in front of the star.  This can be appreciated more clearly in
Figure~\ref{fig:frozen-trajectories}, which shows the grain position
and velocity along a sample streamline with initial impact parameter
\(b = -0.4 R_\dag\) for the two quasi-parallel models.  In panel~a,
corresponding to \(v_\infty = \SI{40}{km.s^{-1}}\), we see the same
rip-and-snap-back cycle of de-coupling and re-coupling that was
discussed previously, but only two periods of the cycle are completed
before the grain's lateral migration takes it as far as
\(y \approx +R_\dag\), at which point nothing can stop the gas stream from
dragging it past the star and away.  All told, the grain remains in
the apex region for about \(1/\sin\theta\B\) times longer than the crossing
time, \(R_\dag / v_\infty\).

In panel~b, corresponding to \(v_\infty = \SI{80}{km.s^{-1}}\), the grain
settles for a while around the stagnant drift radius, \(R_\ddag\), after
the initial rip and turn around.  Again, it slowly migrates sideways,
and eventually recouples to the incident stream, but this time after
reaching \(y \approx +R_\ddag\).  The grain residence time in the apex region,
measured in crossing times, is slightly longer than in panel~a, but is
of the same order.  In both these cases, a second, exterior dense
shell is formed in addition to the hemispherical one produced by the
initial turn around inside \(R_\dag\). In the limit-cycle case, it forms
at the snap-back point, while in the stagnant-drift case, it forms at
\(R_\ddag\) and is significantly denser than the interior shell (upper
right panel of Fig.~\ref{fig:frozen-stream}).

The assumptions behind these models break down when the grain
trajectory intersects the outer shock of the dust-free wind-supported
bow shock.  The increased gas density in the bow shock shell will
reduce \(\Xi\), which is likely to cause gas--grain recoupling in the
case of drag-confined dust waves (see also discussion in the final
paragraph of \S~\ref{sec:exist-cond-separ}).  However, detailed
modeling of this requires magnetohydrodynamical simulations, which are
beyond the scope of this paper.  In Figure~\ref{fig:frozen-stream} we
show in gray dotted lines the Wilkinoid surface \citep{Wilkin:1996a},
which is an approximation to the shape of the inner bow shock, see
\PaperI.  It can be seen that the majority of dust streamlines do not
cross this surface until well into the wings of the bow, so that a
separate exterior dust wave can exist for this quasi-parallel
magnetic field orientation.

This is no longer the case for the quasi-perpendicular field
orientation, as illustrated in the lower panels of
Figure~\ref{fig:frozen-stream}, where the behavior is similar to the
perpendicular inertia-confined case studied in
Appendix~\ref{sec:perp-magn-field}.  Although the grains decouple from
the gas inside the rip point, for small impact parameters the magnetic
geometry does not allow the radiation field to expel them until they
are much nearer to the star.  Since the inner wind-supported bow shock
radius is only a few times smaller than \(R_\dag\), it is possible that
they will pass through the shock surface and re-couple before
expulsion can occur.  For the \SI{40}{km.s^{-1}} model (lower left
panel), all of the streamlines with impact parameters
\(\abs{b} < R_\dag\) intersect the Wilkinoid surface before they are
deflected, and so no separate dust wave would form in this case.  For
the \SI{80}{km.s^{-1}} model (lower left panel), some of the
streamlines (mainly those with \(b > 0\)) do manage to avoid crossing
the bow shock surface, so it is possible that a dust wave may still
form, although it would only be on one side of the axis.


%%% Local Variables:
%%% mode: latex
%%% TeX-master: "bs-bw-dw-02"
%%% End:



\section{Discussion}
\label{sec:discussion}

In molecular clouds:  \citet{Hopkins:2016a, Lee:2017a} say they
decouple. \citet{Tricco:2017a} says they don't.  But then
\citet{Mattsson:2019a} say they do.

In \citep{Akimkin:2015a} they used the drift approximation (ignored
grain inertia effects) and also ignore back reaction on gas. In
\citet{Akimkin:2017a} they do include back reaction.


Instabilities: Resonant Drag Instability \citep{Squire:2018a, Hopkins:2018a}

Survival of dust grains.  Sputtering \citep{Draine:2011a}.  Radiative Torque Disruption \citep{Hoang:2018a}. 

\section{Summary}
\label{sec:summary}


%%% Local Variables:
%%% mode: latex
%%% TeX-master: "bs-bw-dw-02"
%%% End:


\section*{Acknowledgements}
We are grateful for financial support provided by Dirección General de
Asuntos del Personal Académico, Universidad Nacional Autónoma de
México, through grants Programa de Apoyo a Proyectos de Investigación
e Inovación Tecnológica IN112816 and IN107019.  This work has made
extensive use of Python language libraries from the SciPy
\citep{Jones:2001a} and AstroPy \citep{Astropy-Collaboration:2013a,
  Astropy-Collaboration:2018a} projects.  We thank Olga Katushkina for useful discussions. 


\bibliographystyle{mnras}
\bibliography{bowshocks-biblio}
\appendix
\section{Equation of motion for grains with radiation, gas drag, and
  magnetic field}
\label{sec:equat-moti-grains}

Following \citet{Draine:1979a}, the drag force on a grain that is
moving at relative velocity \(\bm{w} = \bm{v}\grain - \bm{v}\gas\)
through a partially ionized gas can be written as a sum over each
collider species, \(k\), with mass \(m_k\), abundance relative to
hydrogen \(\alpha_k\) and charge \(z_k\). If the relative speed,
\(w = \abs{\bm{w}}\), is normalized to the thermal speed of each
species:
\begin{equation}
  \label{eq:s-velocity}
  s_k = \left( m_k w^2 / 2 k T  \right)^{1/2} \ ,
\end{equation}
then the magnitude of the force is 
\begin{equation}
  \label{eq:ds79}
  f\drag = f_* \sum_k \alpha_k \left[ G_0(s_k) + z_k^2 \phi^2 \ln(\Lambda/z_k) G_2(s_k) \right],
\end{equation}
where \(f_*\) is a characteristic thermal force on the grain (see
eq.~[\ref{eq:fstar}]). The dimensionless functions of normalized speed
\(G_0(s)\) and \(G_2(s)\) are given by
\begin{align}
  \label{eq:G0}
  G_0(s) & = \left( s^2 + 1 - \frac{1}{4 s^2} \right) \erf(s)
  +  \left( s + \frac{1}{2 s} \right) \frac{e^{-s^2}}{\sqrt{\pi}}\\
  \label{eq:G2}
  G_2(s) & = \frac{\erf(s)}{s^2} - \frac{2 e^{-s^2}}{s \sqrt{\pi}} \ .
\end{align}
The \(G_0\) term is due to inelastic solid-body collisions in the
Epstein limit, and is derived in \S~4 of \citet{Baines:1965a}.  The
\(G_2\) term is due to electrostatic Coulomb interactions, with
\(\phi\) being the grain potential in thermal units
(eq.~[\ref{eq:phi-potential}]) and \(\Lambda\) the plasma parameter.  It was
first derived in the different context of dynamical friction in
stellar systems by \citet{Chandrasekhar:1941a}.
Figures~\ref{fig:drag-components}--\ref{fig:gas-grain-drag-photoionized}
show example applications to gas--grain drag in a photoionized region.

The grain trajectories presented in \S~\ref{sec:grain-traj-along} and
\ref{sec:grain-traj-with} are calculated by numerically solving the
grain equation of motion:
\begin{equation}
  \label{eq:grain-equation-motion}
  m\grain \frac{d^2 \bm{r}}{d t^2} = \bm{f} \ .
\end{equation}
The total force \(\bm{f}\) is the sum of radiation, drag, and Lorentz
terms:
\begin{equation}
  \label{eq:total-force}
  \bm{f} = \frac{\sigma\grain \Qp L}{4\pi R^2 c} \hat{\bm{r}}
  - f\drag \hat{\bm{w}}
  + \frac{z\grain e}{c} \bm{w} \times \bm{B} \ ,
\end{equation}
with \(f\drag\) given by equation~\eqref{eq:ds79} and where
\(\hat{\bm{r}}\) is the unit vector in the radial direction and
\(\hat{\bm{w}} = \bm{w} / w\) is the unit vector along the direction
of gas--grain relative motion.  In the strong magnetic coupling limit
(see \S~\ref{sec:grain-traj-with} and \ref{sec:tight-magn-coupl}), the
Lorentz term is not included explicitly, but instead the equation of
motion is solved for the guiding center by replacing \(\bm{f}\) by its
projection along the magnetic field: 
\begin{equation}
  \label{eq:projected-force}
  \widetilde{\bm{f}} = (\bm{f} \cdot \hat{\bm{b}})  \, \hat{\bm{b}} \ ,
\end{equation}
where \(\hat{\bm{b}} = \bm{B} / B\).

If distances are measured in units of the radiative turnaround radius,
\(R\starstar\) (eq.~[\ref{eq:dust-r0}]), and times in units of
\(R\starstar / v_\infty\), then the grain acceleration
\(\bm{a}\grain = \bm{f} / m\grain\) in the non-magnetic case can be
written in non-dimensional form as
\begin{equation}
  \label{eq:grain-acceleration}
  \frac{\bm{a}\grain}{a\starstar}
  = \frac{R\starstar^2}{2 R^{2}} \hat{\bm{r}}
  - C\drag \frac{f\drag}{f_*} \hat{\bm{w}} \, 
\end{equation}
where \(a\starstar = v_\infty^2 / R\starstar\) is a characteristic
acceleration scale and the dimensionless drag constant is
\begin{equation}
  \label{eq:drag-constant}
  C\drag = \frac{4}{\Qp} \left(\frac{\sound \tau_* \kappa\grain}{v_\infty \kappa}\right)^2 \ .
\end{equation}

A collection of python programs that implement the equations of this
appendix is available at
\url{https://github.com/div-B-equals-0/dust-trajectories}, including
programs to generate all the grain trajectory figures of this paper
plus additional figures and movies.  The integration of
equation~\eqref{eq:grain-acceleration} is carried out using the
python library function \texttt{scipy.integrate.odeint}, which wraps
the Fortran ODEPACK library \citep{Hindmarsh:1983a, Jones:2001a}.

% \begin{figure}
%   \includegraphics[width=\linewidth]{figs/dust-couple-div-stream}
%   \caption{Divergent dragoids}
%   \label{fig:divergent-dragoids}
% \end{figure}

% To find the dust grain trajectories \(R\grain(\theta)\) in the presence of
% radiation and drag forces (\S~\ref{sec:bow-wave-drag}), we numerically
% integrate the equations of motion. We define dimensionless cylindrical
% polar coordinates,
% \begin{equation}
%   \label{eq:dust-XY}
%   (X, Y) = \left(\frac{R\grain(\theta) \cos\theta } {R_0}, \ 
%     \frac{R\grain(\theta) \sin\theta } {R_0}\right)
%   \ ,
% \end{equation}
% and dust grain velocities,
% \begin{equation}
%   \label{eq:dust-UV}
%   (U, V) = \left( \frac{\bm{v}\grain \cdot \uvec{x}} {v_\infty}, \ 
%   \frac{\bm{v}\grain \cdot \uvec{y}} {v_\infty}\right) \ ,
% \end{equation}
% where \(\uvec{x}\) and \(\uvec{y}\) are unit vectors along the \(X\)
% and \(Y\) axes (parallel and perpendicular, respectively, to the
% symmetry axis).  The grain equation of motion then follows from
% equations~(\ref{eq:dust-rad-force}, \ref{eq:dust-r0},
% \ref{eq:dust-fdrag}--\ref{eq:dust-alpha}) as the following set of
% coupled differential equations:
% \begin{gather}
%   \label{eq:dust-motion}
%   \begin{aligned}
%     \frac{d X}{d t} &= U \quad\quad
%     \frac{d Y}{d t} = V \\
%     \frac{d U}{d t} &= \frac12 \left[  
%       X \left(X^2 + Y^2\right)^{-3/2} - \alpha\drag^2 D_1 \left(U - U_1\right)
%     \right] \\
%     \frac{d V}{d t} &= \frac12 \left[  
%       Y \left(X^2 + Y^2\right)^{-3/2} - \alpha\drag^2 D_1 \left(V - V_1\right)
%     \right] \ ,
%   \end{aligned}
% \end{gather}
% where \((U_1, V_1)\) are the components of the gas velocity (assumed
% fixed), given by
% \begin{equation}
%   \label{eq:dust-gas-velocities}
%   (U_1, V_1) = 
%   \begin{cases}
%     \text{parallel stream} & (-1, 0)\\
%     \text{divergent stream} &
%     \left( \dfrac{X - \mu^{-1}}{R_1},\ \dfrac{Y}{R_1}\right) \ ,
%   \end{cases}
% \end{equation}
% where
% \begin{equation}
%   \label{eq:dust-R1}
%   R_1 = \left( \bigl(X - \mu^{-1}\bigr)^2 + Y^2 \right)^{1/2}
% \end{equation}
% is the distance from the second source, located at
% \((X, Y) = (\mu^{-1}, 0)\).  The dimensionless gas density, \(D_1\),
% normalized by the value at \((X, Y) = (1, 0)\), is
% \begin{equation}
%   \label{eq:dust-gas-density}
%   D_1 = 
%   \begin{cases}
%     \text{parallel stream} & 1\\
%     \text{divergent stream} & \dfrac{\bigl(\mu^{-1} - 1\bigr)^2} {R_1^{2}} \ .
%   \end{cases}
% \end{equation}

% Equations~\eqref{eq:dust-motion} are integrated using the python
% wrapper \texttt{scipy.integrate.odeint} to the Fortran ODEPACK library
% \citep{Hindmarsh:1983a, Jones:2001a}, with results shown in
% Figure~\ref{fig:dust-wave-coupling} for parallel-stream cases and
% Figure~\ref{fig:divergent-dragoids} for divergent-stream cases. 

%%% Local Variables:
%%% mode: latex
%%% TeX-master: "dusty-bow-wave"
%%% End:


% Don't change these lines
\bsp	% typesetting comment
\label{lastpage}
\end{document}


%%% Local Variables:
%%% mode: latex
%%% TeX-master: t
%%% End:
