\documentclass[useAMS, usenatbib, a4paper]{mnras}
\pdfsuppresswarningpagegroup=1

\usepackage{graphicx}
\usepackage{microtype}
\usepackage{xcolor}
\usepackage{fixltx2e}
\usepackage{booktabs}
\usepackage{siunitx}
\sisetup{separate-uncertainty = true}
\usepackage{color}
\usepackage{enumerate}
\usepackage{pdflscape}
\usepackage{rotating}
\usepackage{xr-hyper}
\usepackage{hyperref}
\externaldocument[Q-]{quadrics-bowshock}

\usepackage[T1]{fontenc} 
\usepackage[utf8]{inputenc}

% Fonts 
\usepackage{newtxtext}
% Note: newtxmath must come AFTER newtxtext
\usepackage[varvw,smallerops]{newtxmath}

\usepackage{chemgreek}
\activatechemgreekmapping{newtx}

\hypersetup{colorlinks=True, linkcolor=blue!50!black, citecolor=black,
  urlcolor=blue!50!black}

\usepackage{etoolbox}
\robustify\bfseries
\robustify\itshape

%% The following hack solves a problem with
%% ERROR: \pdfendlink ended up in different nesting level than \pdfstartlink.
%% See https://tex.stackexchange.com/a/249743
\makeatletter
\patchcmd\@combinedblfloats{\box\@outputbox}{\unvbox\@outputbox}{}{%
  \errmessage{\noexpand\@combinedblfloats could not be patched}%
}%
\makeatother

%% Bold italic
\newcommand\hmmax{0}            % we don't need heavy fonts
\newcommand\bmmax{1}            % reduce use of math alphabets for bold
\usepackage{bm}

%% Bundled custom packages
\usepackage{aastex-compat}

\title
{Bow shocks, bow waves, and dust waves}

\newcommand\AddressCRyA{Instituto de Radioastronom\'{\i}a y Astrof\'{\i}sica,
  Universidad Nacional Aut\'onoma de M\'exico, Apartado Postal 3-72,
  58090 Morelia, Michoac\'an, M\'exico}
\author[Henney, Arthur, \& Tarango Yong]{
  William J. Henney, S. Jane Arthur, \& Jorge A. Tarango-Yong\\
  \AddressCRyA
}

% These dates will be filled out by the publisher
\date{Accepted XXX. Received YYY; in original form ZZZ}

% Enter the current year, for the copyright statements etc.
\pubyear{2017}
\DeclareMathOperator{\sgn}{sgn}
\DeclareMathOperator{\Sin}{\mathcal{S}}
\DeclareMathOperator{\Cos}{\mathcal{C}}
\DeclareMathOperator{\Cot}{\mathcal{T}}
\DeclareMathOperator{\GammaFunc}{\Gamma}
\newcommand\w{\ensuremath{\mathrm{w}}}
\newcommand\C{\ensuremath{\mathrm{c}}}
\providecommand{\abs}[1]{\lvert#1\rvert}
\providecommand{\Abs}[1]{\left\lvert#1\right\rvert}
\newcommand\TODO[1]{%
  \begin{center}
    \framebox{\parbox{0.8\linewidth}{
        \texttt{\footnotesize\color{red} #1}}}
  \end{center}}

\newcommand\uvec[1]{\bm{\hat{#1}}}
\newcommand\T{_{\mathrm{\scriptscriptstyle T}}}

\newcommand\Qp{\ensuremath{Q_{\text{p}}}}
\newcommand{\grain}{\ensuremath{_{\text{d}}}}
\newcommand{\B}{\ensuremath{_{\scriptscriptstyle\text{B}}}}
\newcommand{\alfven}{\ensuremath{_{\scriptscriptstyle\text{A}}}}
\newcommand{\xsec}{\ensuremath{\sigma\grain}}
\newcommand\frad{\ensuremath{f_{\text{rad}}}}
\newcommand\fmax{\ensuremath{f_{\text{max}}}}
\newcommand\thm{\ensuremath{\theta_{\text{m}}}}
\newcommand\drag{\ensuremath{_{\text{drag}}}}
\newcommand{\gas}{\ensuremath{_{\text{gas}}}}
\newcommand{\drift}{\ensuremath{_{\text{drift}}}}
\newcommand\rad{\ensuremath{_{\text{rad}}}}
\newcommand\Rmin{\ensuremath{R_{\scriptscriptstyle\text{min}}}}
% Why do I need both of these?
\newcommand\sound{\ensuremath{c_{\text{s}}}}
\newcommand\soundspeed{\ensuremath{c_{\text{s,gas}}}}
\newcommand\starstar{\ensuremath{_{**}}}

\defcitealias{Tarango-Yong:2018a}{Paper~I}
\newcommand\PaperI{\citetalias{Tarango-Yong:2018a}}


\begin{document}
\label{firstpage}
\pagerange{\pageref{firstpage}--\pageref{lastpage}}
\maketitle
\begin{abstract}
  Dust waves and bow waves result from the action of a star's
  radiation pressure on a stream of dust grains that is flowing past
  it.  They are an alternative mechanism to hydrodynamic bow shocks
  for explaining the curved arcs of infrared emission seen around some
  stars.  We develop simple models for the combined effects of
  radiation and stellar winds on an ambient flow in the case where gas
  and grains are perfectly coupled. We find that wind-supported bow
  shocks predominate when the ambient density is low
  (\(\ltae \SIrange{100}{1000}{cm^{-3}}\)) for a broad class of
  stellar parameters.  At higher densities, radiation-supported bows
  can form, tending to be optically thin bow waves in the case of B
  stars, or optically thick bow shocks in the case of early O stars.
  We further investigate the dynamics of dust grains when the
  radiation field is sufficiently strong to overcome the collisional
  coupling between grains and gas, showing that this occurs at a
  \textit{rip-point}, where the ratio of radiation pressure to gas
  pressure exceeds a critical value (\(P\rad/\Pgas > 1000\)).

\end{abstract}

\begin{keywords}
  circumstellar matter -- radiation: dynamics -- stars: winds, outflows
\end{keywords}

\defcitealias{Tarango-Yong:2018a}{Paper~I}
\newcommand\PaperI{\citetalias{Tarango-Yong:2018a}}

\section{Introduction}
\label{sec:introduction}
\newcommand\hii{\ion{H}{ii}}

Curved emission arcs around stars \citep[e.g.,][]{Gull:1979a} are
often interpreted as \textit{bow shocks}, due to a supersonic
hydrodynamic interaction between the star's wind and an external
stream. This stream may be due to the star's own motion or to an
independent flow, such as an \hii{} region in the champagne phase
\citep{Tenorio-Tagle:1979a}, or another star's wind
\citep{Canto:1996}. However, an alternative interpretation in some
cases may be a radiation-pressure driven bow wave, as first proposed
by \citet[\S\textsc{vi}]{van-Buren:1988a}.  In this scenario, photons
emitted by the star are absorbed by dust grains in the incoming
stream, with the resultant momentum transfer being sufficient to
decelerate and deflect the grains within a certain distance from the
star, forming a dust-free, bow-shaped cavity with an enhanced dust
density at its edge.  Two regimes are possible, depending on the
strength of coupling between the gas (or plasma) and the dust.  In the
strong-coupling regime, gas--grain drag decelerates the gas along with
the dust, forming a shocked gas shell in a similar fashion to the
wind-driven bow shock case.  In the weak-coupling regime, the gas
stream is relatively unaffected and the dust temporarily decouples to
form a dust-only shell.  This second case has recently been studied in
detail in the context of the interaction of late O-type stars (which
have only weak stellar winds) with dusty photoevaporation flows inside
\hii{} regions \citep{Ochsendorf:2014a, Ochsendorf:2014b,
  Ochsendorf:2015a}.  We follow the nomenclature proposed by
\citet{Ochsendorf:2014b}, in which \textit{dust wave} refers to the
weak coupling case and \textit{bow wave} to the strong coupling case.
More complex, hybrid scenarios are also possible, such as that studied
by \citet{van-Marle:2011a}, where a hydrodynamic bow shock forms, but
the larger dust grains that accompany the stellar wind pass right
through the shocked gas shell, and form their own dust wave at a
larger radius.

\begin{figure*}
  \centering
  \includegraphics[width=\linewidth]{figs/bows-and-waves}
  \caption{Bow shocks, bow waves, and dust waves}
  \label{fig:3-types-bow}
\end{figure*}

In \citet[][hereafter \PaperI{}]{Tarango-Yong:2018a}, we proposed a
new two-dimensional classification scheme for bow shapes: the
projected planitude--alatude, or \(\Pi'\)--\(\Lambda'\), diagram.  Planitude
measures the flatness of the bow's apex, while alatude measures the
openness of the bow's wings.  Both are dimensionless ratios of lengths
that can be estimated from observational images.  We have analyzed the
inclination-dependent tracks on the \(\Pi'\)--\(\Lambda'\) plane for simple
geometric shapes (spheroids, paraboloids, hyperboloids) and for
thin-shell hydrodynamic bow shock models (wilkinoid, cantoids,
ancantoids).  In this paper, we will do the same for simple models of
radiation-driven dust waves (dragoids) and bow waves (trapoids).

The paper is organized as follows.
%
In \S~\ref{sec:shape-dust-wave} we do the same for simple models of a
dusty radiation bow wave (dragoids), including the effects of
gas-grain drag.
%
In \S~\ref{sec:perturbed-bows} we investigate the effects on the
planitude--alatude plane of small-amplitude perturbations to the bow
shape.
%


\section{Different types of bow}
\label{sec:different-types-bow}

\begin{figure*}
  \includegraphics[width=\linewidth]{figs/zones-v-n-plane}
  \caption{Bow regimes in parameter space (\(v, n\)) of the external
    stream for main-sequence OB stars of different masses:
    (a)~\SI{10}{M_\odot}, (b)~\SI{20}{M_\odot}, (c)~\SI{40}{M_\odot}.  In all
    cases, \(\kappa = \SI{600}{cm^2.g^{-1}}\) and efficient gas-grain
    coupling is assumed. Solid black lines of varying width show the
    bow size (star-apex separation, \(R_0\)), while gray shading shows
    the radiation bow wave regime, with lower border \(\tau = \eta\) and
    upper border \(\tau = 1\), where \(\tau = 2 \kappa \rho R_0\) is the optical
    depth through the bow.  For bows above the red solid line, the
    ionization front is trapped inside the bow.  Blue lines delineate
    different cooling regimes.  Above the thin blue line
    (\(d_{\text{cool}} = h_0\)), the bow shock radiates efficiently,
    forming a thin shocked shell.  Below the thick blue line
    (\(d_{\text{cool}} = R_0\)), the bow shock is essentially
    non-radiative.}
  \label{fig:zones-v-n-plane}
\end{figure*}

In this section, we investigate the different types of bow interaction
that will occur in different regions of parameter space. We will
mainly treat the canonical case\footnote{%
  Variant cases with differing arrangements of dust and radiation
  sources are treated in \S~\ref{sec:case-inside-out}.} %
of a bow around a star of bolometric luminosity, \(L\), with a
radiatively driven wind, which is immersed in an external stream of
gas and dust with density, \(\rho\), and velocity, \(v\).  The size and
shape of the bow is determined by a generalized balance of pressure
(or, equivalently, momentum) between internal and external sources.
We assume that the stream is supersonic and super-alfvenic, so that
the external pressure is dominated by the ram pressure: \(\rho v^2\).

\subsection{Strong gas-grain coupling}
\label{sec:strong-gas-grain}

We first consider the case where the dust grains and gas are perfectly
coupled by collisions.\footnote{%
  Cases where this assumption does not hold are investigated below in
  \S~\ref{sec:imperf-coupl-betw}.} %
Although dust grains typically constitute only a small fraction
\(Z\grain \sim 0.01\) of the mass of the external stream, they
nevertheless dominate the broad-band opacity at FUV, optical and IR
wavelengths if they are present.\footnote{%
  At EUV wavelengths (\(\lambda < \SI{912}{\angstrom}\)), gas opacity
  dominates if the hydrogen neutral fraction is larger than
  \(\approx 0.001\), see discussion of ionization front trapping
  below.} %
The strong coupling assumption means that all the radiative forces
applied to the dust grains are directly felt by the gas also.

The internal pressure is the sum of wind ram pressure and the
effective radiation pressure that acts on the bow shell.  The
radiative momentum loss rate of the star is \(L/c\) and the wind
momentum loss rate can be expressed as
\begin{equation}
  \label{eq:wind-efficiency}
  \dot{M} V = \eta L / c \ , 
\end{equation}
where \(\eta\) is the momentum efficiency of the wind, which is typically
\(< 1\) \citep{Lamers:1999b}. If the optical depth is very large, then
all of the stellar radiative momentum, emitted with rate \(L/c\), is
trapped by the bow shell.  In the single scattering limit,\footnote{%
  Although it may seem inconsistent to assume single scattering in the
  case of high optical depths, this is defensible for the following
  reasons. (1)~The grain albedo is not that high (typically
  \(\sim 0.5\) at ultraviolet through optical wavelengths). (2)~The
  scattered radiation field is more isotropic than the stellar field,
  leading to cancellation in the radiative
  flux. (3)~Absorbed radiation is re-emitted at infrared
  wavelengths, where the dust opacity is very much lower.} %
and temporarily neglecting the wind, then pressure balance at the bow
apex, a distance \(R_0\) along the symmetry axis from the star is
given by
\begin{equation}
  \label{eq:rad-press-balance-thick}
  \frac{L}{4 \pi c R_0^2} = \rho v^2 \ ,
\end{equation}
which yields a fiducial bow shock radius in this optically thick limit
as
\begin{equation}
  \label{eq:Rstar}
  R_* = \left(\frac{L}{4\pi c \rho v^2}\right)^{1/2} \ .
\end{equation}

We now consider the opposite, optically thin limit.  If the total
opacity (gas plus dust) per total mass (gas plus dust) is \(\kappa\) (with
units of \si{cm^2.g^{-1}}), then the radiative acceleration is
\begin{equation}
  \label{eq:rad-accel}
  a_{\text{rad}} = \frac{\kappa L}{4 \pi c R^2} \ .
\end{equation}
Therefore, an incoming stream with initial velocity, \(v_\infty\), can be
brought to rest by radiation alone\footnote{%
  For simplicity, we here ignore the effects of gravity, which are
  important for low ratios of \(\kappa L / M\), see
  \S~\ref{sec:effects-gravity}.  We also ignore pressure gradients and
  shocks, which are important as the velocity approaches the sound
  speed, \(\sound\) (in \S~XX below, we show that the resultant
  corrections to \(R_0\) are of order \(\sound / v_\infty\)).} %
at a distance \(R_0\) where
\begin{equation}
  \label{eq:rad-poten}
  \int_{R_0}^\infty a_{\text{rad}} \, dr = \tfrac12 v_\infty^2 \ , 
\end{equation}
yielding
\begin{equation}
  \label{eq:rad:R0}
  R_0 = \frac{\kappa L}{2\pi c v_\infty^2} \ .
\end{equation}
On the other hand, we can also argue as in the optically thick case
above by approximating the bow shell as a surface, and balancing
stellar radiation pressure against the ram pressure of the incoming
stream.  The important difference when the shell is not optically
thick is that only a fraction \(1 - e^{-\tau}\) of the radiative momentum
is absorbed by the bow, so that
equation~\eqref{eq:rad-press-balance-thick} is replaced with
\begin{equation}
  \label{eq:rad-press-balance-tau}
  \frac{L (1 - e^{-\tau})}{4 \pi c R_0^2} = \rho v^2 \ .
\end{equation}
In the optically thin limit, \(1 - e^{-\tau} \approx \tau\), so these two
descriptions can be seen to agree so long as
\begin{equation}
  \label{eq:tau-thin}
  \tau = 2 \kappa \rho R_0 \ ,
\end{equation}
which we will assume to hold generally.

Then, defining a fiducial optical depth,
\begin{equation}
  \label{eq:tau-star}
  \tau_* = \rho \kappa R_* \ ,
\end{equation}
and adding in the stellar wind ram pressure\footnote{%
  We implicitly assume that the interaction of the stellar wind with
  the external stream can always be treated in the continuum limit.
  This will be true if either the collisional mean free path or the
  ion Larmor radius is much smaller than \(R_0\), which is almost
  always the case.} %
from equation~\eqref{eq:wind-efficiency}, we find that the general bow
radius can be written in terms of the fiducial radius as
\(R_0 = x R_*\), where \(x\) is the solution of
\begin{equation}
  \label{eq:rad-full-x}
  x^2 - \bigl(1 - e^{-2 \tau_* x} \bigr) - \eta = 0 \ .
\end{equation}
Since this is a transcendental equation, \(x\) must be found
numerically, but we can write explicit expressions for three limiting
cases:
\begin{equation}
  \label{eq:x-cases}
  x \approx
  \begin{cases}
    \text{if \(\tau_* \gg 1\):} & (1 + \eta)^{1/2}  \\
    \text{if \(\tau_*^2 \ll 1\):} & \tau_* + \bigl( \tau_*^2 + \eta \bigr)^{1/2} \approx
    \begin{cases}
      \text{if \(\tau_*^2 \gg \eta\):} & 2 \tau_*  \\
      \text{if \(\tau_*^2 \ll \eta\):} & \eta^{1/2} 
    \end{cases}
  \end{cases}
\end{equation}
The first case, \(x \approx (1 + \eta)^{1/2}\), corresponds to a
\textit{radiation bow shock}; the second case,
\(x \approx 2 \tau_* \), corresponds to a \textit{radiation bow wave}; and the
third case, \(x \approx \eta^{1/2}\), corresponds to a \textit{wind bow shock}.
The two bow shock cases are similar in that the external stream is
oblivious to the presence of the star until it suddenly hits the bow
shock shell, differing only in whether it is radiation or wind that is
providing the internal pressure.  In the intermediate bow wave case,
on the other hand, the external stream is gradually decelerated by
absorption of photons as it approaches the bow.\footnote{A shock does
  still form in this case, but shocked material constitutes only a
  fraction of the total column density of the shell, see
  \S~\ref{sec:shape-bow-wave}.}

\begin{table*}
  \centering
  \caption{Stellar parameters for example stars}
  \label{tab:stars}
  \begin{tabular}{l S S S S S S S S S l}
    \toprule
    & {\(M / \si{M_\odot}\)} & {\(L_4\)}
    & {\(\dot{M}_{-7}\)} & {\(V_3\)} & {\( \eta \)}
    & {Sp.~Type} 
    & {\(T_{\text{eff}} / \si{kK}\)} & {\(\lambda_{\text{eff}}\) / \si{\um}}
    & {\(S_{49}\)} & Figures 
    \\
    \midrule
    & 10 & 0.63 & 0.0034 & 2.47 & 0.0066 & {B1.5\,V} & 25.2 & 0.115 & 0.00013
                   & \ref{fig:zones-v-n-plane}a,
                     \ref{fig:decouple-v-n-plane},
                     \ref{fig:decouple-v40-versus-n} \\
    Main-sequence OB stars
    & 20 & 5.45 & 0.492 & 2.66 & 0.1199 & {O9\,V} & 33.9 & 0.086 & 0.16
                   & \ref{fig:zones-v-n-plane}b\\
    & 40 & 22.2 & 5.1 & 3.31 & 0.4468 & {O5\,V} & 42.5 & 0.068 & 1.41
                   & \ref{fig:zones-v-n-plane}c\\[\smallskipamount]
    Blue supergiant star
    & 33 & 30.2 & 20.2 & 0.93 & 0.3079 & {B0.7\,Ia} & 23.5 & 0.123 & 0.016
                   & \ref{fig:B-supergiant} \\[\smallskipamount]
    Red supergiant star
    & 20 & 15.6 & 100 & 0.015 & 0.0476 & {M1\,Ia} & 3.6 & 0.805 & 0
                   & \ref{fig:M-supergiant} \\ 
    \bottomrule
  \end{tabular}
\end{table*}

We now consider the application to bow shocks around main sequence OB
stars, expressing stellar and ambient parameters in terms of typical
values as follows:
\begin{align*}
  \label{eq:stellar-parameters}
  \dot{M}_{-7} &= \dot{M} / \bigl(\SI{e-7}{M_\odot.yr^{-1}}\bigr) \\
  V_3 &= V / \bigl(\SI{1000}{km.s^{-1}}\bigr) \\
  L_4 &= L / \bigl(\SI{e4}{L_\odot}\bigr) \\
  v_{10} &= v_\infty / \bigl( \SI{10}{km.s^{-1}} \bigr) \\
  n &= (\rho / \bar{m}) / \bigl( \SI{1}{cm^{-3}} \bigr) \\
  \kappa_{600} &= \kappa / \bigl( \SI{600}{cm^2.g^{-1}} \bigr) \ ,
\end{align*}
where \(\bar{m}\) is the mean mass per hydrogen nucleon
(\(\bar{m} \approx 1.3 m_{\text{p}} \approx \SI{2.17e-24}{g}\) for solar
abundances).  Note that \(\kappa = \SI{600}{cm^2.g^{-1}}\) corresponds to a
cross section of \(\approx \SI{e-21}{cm^2}\) per hydrogen nucleon, which is
typical for interstellar medium dust \citep{Bertoldi:1996a} at far
ultraviolet wavelengths, where OB stars emit most of their radiation.
In terms of these parameters, we can express the stellar wind momentum
efficiency as
\begin{equation}
  \label{eq:wind-eta-typical}
  \eta = \num{0.495} \,\dot{M}_{-7} \,V_3  \,L_4^{-1}
\end{equation}
and the fiducial radius and optical depth as
\begin{align}
  \label{eq:Rstar-typical}
  R_* / \si{pc} &= \num{2.21} \, (L_4 / n)^{1/2} \,v_{10}^{-1} \\
  \label{eq:taustar-typical}
  \tau_* &= \num{0.0089} \,\kappa_{600} \, (L_4 \,n)^{1/2} \,v_{10}^{-1} \ .
\end{align}
In Figure~\ref{fig:zones-v-n-plane}, we show results for the bow size
(apex distance, \(R_0\)) as a function of the density, \(n\), and
relative velocity, \(v_\infty\), of the external stream, with each panel
corresponding to a particular star, with parameters as shown in
Table~\ref{tab:stars}.  To facilitate comparison with previous work,
we choose stellar parameters similar to those used in the
hydrodynamical simulations of \citet{Meyer:2014b, Meyer:2016a,
  Meyer:2017a}, based on stellar evolution tracks for stars of
\SIlist{10;20;40}{M_\odot} \citep{Brott:2011a} and theoretical wind
prescriptions \citep{de-Jager:1988a, Vink:2000a}.  Although the
stellar parameters do evolve with time, they change relatively little
during the main-sequence lifetime of several million years.\footnote{%
  \label{fn:meyer-velocities-too-low}
  Note that we have recalculated the stellar wind terminal velocities,
  since the values given in the \citeauthor{Meyer:2014b} papers are
  troublingly low.  We have used the prescription
  \(V = 2.6 V_{\text{esc}}\), where
  \(V_{\text{esc}} = \left( 2 G M (1 - \Gamma_e)/ R \right)^{1/2}\) is the
  photospheric escape velocity, which is appropriate for strong
  line-driven winds with \(T_{\text{eff}} > \SI{21 000}{K}\)
  \citep{Lamers:1995a}.  We find velocities of
  \SIrange{2500}{3300}{km.s^{-1}}, which are consistent with
  observations and theory \citep{Vink:1999a} for O~stars, but at least
  two times higher than those cited by \citet{Meyer:2014b}. For
  main-sequence B~stars, wind column densities are too low to reliably
  measure the terminal velocity from near ultraviolet P~Cygni profiles
  \citep{Prinja:1989a}, and so the values are theory-dependent
  \citep{Krticka:2014a} and hence more uncertain.  A further
  complication is the existence of a subset of OB stars with
  anomalously weak winds \citep{Puls:2008a}, which in some cases is
  related to the presence of strong (\(\sim \SI{1}{kG}\)) magnetic fields
  \citep{Oskinova:2011b}.} %
The three examples are an early B~star (\SI{10}{M_\odot}), a late O~star
(\SI{20}{M_\odot}), and an early O~star (\SI{40}{M_\odot}), which cover the
range of luminosities and wind strengths expected from bow-producing
hot main sequence stars.  The luminosity is a steep function of
stellar mass (\(L \sim M^{2.5}\)) and the wind mass-loss rate is a steep
function of luminosity (\(\dot{M} \sim L^{2.2}\)), which means that the
wind momentum efficiency is also a steep function of mass
(\(\eta \sim M^3\)), approaching unity for early O~stars, but falling to
less than 1\% for B~stars.

\begin{figure}
  \includegraphics[width=\linewidth]{figs/zones-v-n-plane-RSG}
  \caption{As Fig.~\ref{fig:zones-v-n-plane}, but for a cool M-type
    supergiant instead of hot main sequence stars.  A smaller dust
    opacity is used, \(\kappa = \SI{60}{cm^2.g^{-1}}\), because of the
    reduced extinction efficiency at the optical/infrared wavelengths
    emitted by this star.}
  \label{fig:M-supergiant}
\end{figure}

It can be seen from Figure~\ref{fig:zones-v-n-plane} that the onset of
the radiation bow wave regime is very similar for the three
main-sequence stars, occurring at
\(n > \text{\numrange{20}{40}} \, v_{10}^2\).  An important
difference, however, is that for the \SI{40}{M_\odot} star, which has a
powerful wind, the radiation bow wave regime only occurs for a very
narrow range of densities, whereas for the \SI{10}{M_\odot} star, with a
much weaker wind, the regime is much broader, extending to
\(n < \num{e4} \, v_{10}^2\).  Another difference is the size scale of
the bows in this regime, which is
\(R_0 = \text{\SIrange{0.001}{0.003}{pc}}\) for the \SI{10}{M_\odot} star
if \(v_\infty = \SI{40}{km.s^{-1}}\), but \(R_0 \approx \SI{0.1}{pc}\) for the
\SI{40}{M_\odot} star, assuming the same inflow velocity.

Figure~\ref{fig:M-supergiant} shows results for a cool M-type
super-giant star with stellar parameters inspired by Betelgeuse
(\chemalpha~Orionis), as listed in Table~\ref{tab:stars}.  Unlike the
UV-dominated spectrum of the hot stars, this star emits predominantly
in the near-infrared, where the dust extinction efficiency is lower,
so we adopt a lower opacity of \SI{60}{cm^2.g^{-1}}.  This has the
effect of shifting the radiation bow wave regime to higher densities:
\(n = \text{\numrange{1000}{30 000}}\, v_{10}^2\) in this case.


\subsubsection{Effects of stellar gravity}
\label{sec:effects-gravity}

In principle, gravitational attraction from the star, of mass \(M\),
will partially counteract the radiative acceleration.  This can be
accounted for by replacing \(L\) with an effective luminosity
\newcommand\Edd{\ensuremath{_{\text{E}}}}
\begin{equation}
  \label{eq:effective-luminosity}
  L_{\text{eff}} = L \bigl(1 - \Gamma\Edd^{\,-1}\bigr) \ ,
\end{equation}
in which \(\Gamma\Edd\) is the Eddington factor:
\begin{equation}
  \label{eq:eddington-factor}
  \Gamma\Edd = \frac{\kappa L}{4\pi c G M} = 458.5 \, \frac{\kappa_{600} L_4}{ M } \ ,
\end{equation}
where, in the last expression, \(M\) is measured in solar masses.  For
the stars in Table~\ref{tab:stars}, we find
\(\Gamma\Edd \approx \text{\numrange{30}{400}}\), so gravity can be safely
ignored.  The only exception is when the optical depth of the bow is
very large: \(\tau > \ln\Gamma\Edd \sim 5\), in which case gravity may be
important in the outer parts of the shell (see
\citealt{Rodriguez-Ramirez:2016b}).



\subsubsection{Ionization state of the bow shell}
\label{sec:trapp-ioniz-front}

\newcommand\alphaB{\ensuremath{\alpha_{\text{B}}}}
\newcommand\shell{\ensuremath{_{\text{sh}}}}

\begin{figure}
  \includegraphics[width=\linewidth]{figs/zones-v-n-plane-BSG}
  \caption{As Fig.~\ref{fig:zones-v-n-plane}, but for an evolved
    B-type supergiant instead of main sequence stars.  This is similar
    to the early O MS star of Fig.~\ref{fig:zones-v-n-plane}\textit{c}
    in many respects, except for the trapping of the ionization front,
    which occurs for much lower outer stream densities.}
  \label{fig:B-supergiant}
\end{figure}

In this section we calculate whether the star is capable of
photoionizing the entire bow shock shell, or whether the ionization
front will be trapped within it.  The number of hydrogen
recombinations\footnote{%
  The diffuse field is treated in the on-the-spot approximation,
  assuming all emitted Lyman continuum photons are immediately
  re-absorbed locally, so the case~B recombination co-efficient,
  \(\alphaB = \num{2.6e-13}\, T_4^{-0.7}\, \si{cm^3.s^{-1}}\), is
  used, where \(T_4 = T/\SI{e4}{K}\).} %
per unit time per unit area in a fully ionized shell is
\begin{equation}
  \label{eq:shell-recombination-rate}
  \mathcal{R} = \alphaB n\shell^2 h\shell \ ,
\end{equation}
while the advective flux of hydrogen nuclei through the shock is 
\begin{equation}
  \label{eq:shell-advective-flux}
  \mathcal{A} = n v \ ,
\end{equation}
and the flux of hydrogen-ionizing photons
(\(h \nu > \SI{13.6}{eV}\)) incident on the inner edge of the shell is
\begin{equation}
  \label{eq:shell-ionizing-flux}
  \mathcal{F} = \frac{S} {4 \pi R_0^2} \ , 
\end{equation}
where \(S\) is the ionizing photon luminosity of the star.  Any shell
with \(\mathcal{R} + \mathcal{A} > \mathcal{F}\) cannot be entirely
photoionized by the star, and so must have trapped the ionization
front.

The ratio of advective particle flux to ionizing flux is, from equations~\eqref{eq:Rstar}, \eqref{eq:shell-advective-flux}, \eqref{eq:shell-ionizing-flux},
\begin{equation}
  \label{eq:advective-over-ionizing-flux}
  \frac{\mathcal{A}}{\mathcal{F}} = \num{5.86e-5} \frac{x^2 L_4}{v_{10} S_{49}} \ , 
\end{equation}
which is nearly always small.  For clarity of exposition, we therefore
ignore \(\mathcal{A}\) in the following discussion, although it is
included in quantitative calculations.  The column density of the
shocked shell can be found, for example, from equations~(10) and~(12)
of \citet{Wilkin:1996a} in the limit \(v_\infty/V \to 0\) (Wilkin's parameter
\(\alpha\)) and \(\theta \to 0\).  This yields
\begin{equation}
  \label{eq:shocked-shell-column}
  n\shell h\shell = \tfrac34 n R_0 \ .
\end{equation}
Assuming strong cooling behind the shock,\footnote{%
  This is shown to be justified in \S~\ref{sec:radi-cool-lengths}.
} %
the shell density is
\begin{equation}
  \label{eq:isothermal-shell-density}
  n\shell = \mathcal{M}_0^2 n \,
\end{equation}
where
\(\mathcal{M}_0 = v_\infty / \sound\) is the isothermal Mach number of the
external stream.\footnote{%
  \label{fn:temperature-dependence}
  The sound speed depends on the temperature and hydrogen and helium
  ionization fractions, \(y\) and \(y_{\text{He}}\) as
  \(\sound^2 = (1 + y + z_{\text{He}} y_{\text{He}}) (k T /
  \bar{m})\), where \(z_{\text{He}}\) is the helium nucleon abundance
  by number relative to hydrogen and
  \(k = \SI{1.3806503e-16}{erg.K^{-1}}\) is Boltzmann's constant.  We
  assume \(y = 1\), \(y_{\text{He}} = 0.5\), \(z_{\text{He}} = 0.09\),
  so that \(\sound = \num{11.4}\, T_4^{1/2}\, \si{km.s^{-1}}\). } %
Putting these together with equations~\eqref{eq:Rstar} and
~\eqref{eq:tau-star}, one finds that \(\mathcal{R} > \mathcal{F}\)
implies
\begin{equation}
  \label{eq:ifront-trap-x-cubed-taustar}
  x^3 \tau_* > \frac{4 S c \sound \bar{m}^2 \kappa}{3 \alpha L} \ .
\end{equation}
From equation~\eqref{eq:rad-full-x}, it can be seen that \(x\) depends
on the external stream parameters, \(n\), \(v_\infty\) only via
\(\tau_*\), and so equation~\eqref{eq:ifront-trap-x-cubed-taustar} is a
condition for \(\tau_*\), which, by using equation~\eqref{eq:taustar-typical}, becomes a condition on \(n / v_{10}^2\).  In the radiation bow shock case,
\(x = (1 + \eta)^{1/2}\), and the condition can be written:
\begin{equation}
  \label{eq:ifront-trap-density-RBS}
  \frac{n}{v_{10}^2} > \num{2.65e8} \, \frac{S_{49}^2 T_4^{3.4}}{L_4^3 (1 + \eta)^3} \ , 
\end{equation}
where
\begin{equation*}
  S_{49} = S / \bigl( \SI{e49}{s^{-1}} \bigr) \ .
\end{equation*}
Numerical values of \(S_{49}\) for our three example stars are given
in Table~\ref{tab:stars}, taken from Figure~4 of
\citet{Sternberg:2003a}.  In the radiation bow wave case,
\(x = 2\tau_*\), and the condition can be written:
\begin{equation}
  \label{eq:ifront-trap-taustar-RBW}
  \frac{n}{v_{10}^2} > \num{5.36e4} \, \frac{S_{49}^{1/2} T_4^{0.85} }{\kappa_{600}^{3/2} L_4^{3/2}} \ . 
\end{equation}
In the wind bow shock case, the result is the same as
equation~\eqref{eq:ifront-trap-density-RBS}, but changing the factor
\((1 + \eta)^3\) to \(\eta^3\).  For the example hot stars in
Table~\ref{tab:stars}, and assuming \(\kappa_{600} = 1\),
\(T_4 = 0.8\), the resulting density threshold is
\(n > (\text{\numrange{1000}{5000}})\, v_{10}^2\), depending only
weakly on the stellar parameters, which is shown by the red lines in
Figure~\ref{fig:zones-v-n-plane}.  For the \SI{10}{M_\odot} star, this is
in the radiation bow wave regime, whereas for the higher mass stars it
is in the radiation bow shock regime.  When the external stream is
denser than this, then the outer parts of the shocked shell may be
neutral instead of ionized, giving rise to a cometary compact \hii{}
region \citep{Mac-Low:1991a, Arthur:2006a}.  This is only necessarily
true, however, when the star is isolated.  If the star is in a cluster
environment, then the contribution of other nearby massive stars to
the ionizing radiation field must be considered.

Quite different results are obtained for a B-type supergiant star (see
Tab.~\ref{tab:stars} and Fig.~\ref{fig:B-supergiant}), which has a
similar bolometric luminosity and wind strength to the \SI{40}{M_\odot}
main-sequence star, but a hundred times lower ionizing luminosity.
This results in a far lower threshold for trapping the ionization
front of \(n > 40 v_{10}^2\).  The advective flux, \(\mathcal{A}\), is
relatively stronger for this star than for the main-sequence stars, but
even for \(v_{10} < 2\), where the effect is strongest, the change is
only of order the width of the dark red line in
Figure~\ref{fig:B-supergiant}.


In principle, when the ionization front trapping occurs in the bow
wave regime, then the curves for \(R_0\) will be modified in the
region above the red line because all of the ionizing radiation is
trapped in the shell due to gas opacity, which is not included in
equation~\eqref{eq:tau-thin}.  However, this only happens for our
\SI{10}{M_\odot} star, which has a relatively soft spectrum.
Table~\ref{tab:stars} gives the peak wavelength of the stellar
spectrum for this star as \(\lambda_{\text{eff}} = \SI{0.115}{\um}\), which
is significantly larger than the hydrogen ionization threshold at
\SI{0.0912}{\um}, meaning that only a small fraction of the total
stellar luminosity is in the EUV band and affected by the gas opacity.
The effect on \(R_0\) is therefore small.  For the higher mass stars,
\(\lambda_{\text{eff}} < \SI{0.0912}{\um}\), so the majority of the
luminosity is in the EUV band, but in these cases the ionization front
trapping occurs well inside the radiation bow shock zone, where the
dust optical depth is already sufficient to trap all of the radiative
momentum.

\subsubsection{Radiative cooling lengths}
\label{sec:radi-cool-lengths}
\newcommand\M{\ensuremath{\mathcal{M}}}
In this section, we calculate whether the radiative cooling is
sufficiently rapid behind the bow shock to allow the formation of a
thin, dense shell.  Since cooling is least efficient at low densities,
we will assume that the wind bow shock regime applies unless otherwise
specified. We label quantities just outside the shock by the subscript
``0'', quantities just inside the shock (after thermalization, but
before any radiative cooling) by the subscript ``1'', and quantities
after the gas has cooled back to the photoionization equilibrium
temperature by the subscript ``2''.  Assuming a ratio of specific
heats, \(\gamma = 5/3\), the relation between the pre-shock and immediate
post-shock quantities is
\begin{align}
  % \M_1 &= \left(\frac{\M_0^2 + 3} {5\M_0^2 - 1}\right)^{1/2} \\
  \label{eq:shock-n-jump}
  \frac{n_1}{n_0} &= \frac{4 \M_0^2} {\M_0^2 + 3} \\
  \label{eq:shock-T-jump}
  \frac{T_1}{T_0} &= \tfrac1{16} \bigl( 5\M_0^2 - 1 \bigr) \bigl( 1 + 3/\M_0^2 \bigr) \\
  \label{eq:shock-v-jump}
  \frac{v_1}{v_0} &= \left(\frac{n_1}{n_0}\right)^{-1} \ ,
\end{align}
where \(\M_0 = v_0 / \sound\).  The cooling length of the post-shock
gas can be written as
\newcommand\cool{\ensuremath{_{\text{cool}}}}
\begin{equation}
  \label{eq:dcool}
  d\cool = \frac{3 P_1 v_1} { 2 \bigl(  \mathcal{L}_1 - \mathcal{G}_1 \bigr) }\ ,  
\end{equation}
where \(P_1\) is the thermal pressure and \(\mathcal{L}_1\),
\(\mathcal{G}_1\) are the volumetric radiative cooling and heating
rates.  For fully photoionized gas, we have
\(P_1 \approx 2 n_1 k T_1\), \(\mathcal{L}_1 = n_1^2 \Lambda(T_1)\), and
\(\mathcal{G}_1 = n_1^2 \Gamma(T_1)\), where \(\Lambda(T)\) is the cooling
coefficient, which is dominated by metal emission lines that are
excited by electron collisions, and \(\Gamma(T)\) is the heating
coefficient, which is dominated by hydrogen photo-electrons
\citep{Osterbrock:2006a}. The cooling coefficient has a maximum around
\SI{e5}{K}, and for typical ISM abundances can be approximated as
follows:
\begin{align}
  \label{eq:cooling-coefficient}
  \Lambda_{\text{warm}} &= \num{3.3e-24} \, T_4^{2.3} \, \si{erg.cm^{-3}.s^{-1}}\\
  \Lambda_{\text{hot}} &= \num{e-20} \, T_4^{-1}\, \si{erg.cm^{-3}.s^{-1}} \\
  \Lambda &= \left( \Lambda_{\text{warm}}^{-k} +  \Lambda_{\text{hot}}^{-k} \right)^{-1/k}
      \quad \text{with} \quad k = 3 \ ,
\end{align}
which is valid in the range \(0.7 < T_4 < 1000\).  We approximate the heating coefficient as
\begin{equation}
  \label{eq:heating-coefficient}
  \Gamma = \num{1.77e-24} \, T_4^{-1/2} \, \si{erg.cm^{-3}.s^{-1}} \ ,
\end{equation}
where the coefficient is chosen so as to give \(\Gamma = \Lambda\) at
an equilibrium temperature of \(T_4 = 0.8\).

In Figure~\ref{fig:zones-v-n-plane} we show curves calculated from
equations~\eqref{eq:shock-n-jump} to~\eqref{eq:heating-coefficient},
corresponding to \(d\cool = R_0\) (thick blue line) and
\(d\cool = h_0\) (thin blue line), where \(h_0\) is the shell
thickness in the efficient cooling case.  In this context, \(n_0 = n\)
and \(n_2 = n\shell\), so that \(h_0\) follows from
equations~\eqref{eq:shocked-shell-column}
and~\eqref{eq:isothermal-shell-density} as
\begin{equation}
  \label{eq:strong-cooling-h0}
  h_0 = \tfrac34 \M_0^{-2} R_0 \ .
\end{equation}
The bends in the curves at \(v \approx \SI{50}{km.s^{-1}}\) are due to the
maximum in the cooling coefficient \(\Lambda(T)\) around
\(\SI{e5}{K}\).  For bows with outer stream densities above the thin
blue line, radiative cooling is so efficient that the bow shock can be
considered isothermal, and so the shell is dense and thin (at least,
in the apex region).  It can be seen that the ionization front
trapping always occurs at densities larger than this, which justifies
the use of equation~\eqref{eq:isothermal-shell-density} in the
previous section.  For bows with outer stream densities below the
thick blue line, cooling is unimportant and the bow shock can be
considered non-radiative.  In this case the shell is thicker than in
the radiative case,
\(h\shell/R_0 \approx \text{\numrange{0.2}{0.3}}\).\footnote{%
  An approximate value can be found from
  equation~\eqref{eq:shocked-shell-column} by substituting \(n = n_0\)
  and \(n\shell \approx n_1\), then using equation~\eqref{eq:shock-n-jump}.
  Consideration of the slight increase in density between the shock
  and the contact discontinuity reduces this value by 5--10\%.} %
For bows with outer stream densities between the two blue lines,
cooling does occur, albeit inefficiently, so that the shell thickness
is set by \(d\cool\) rather than \(h_0\).

\subsection{Imperfect coupling between gas and dust}
\label{sec:imperf-coupl-betw}

\begin{figure}
  \includegraphics[width=\linewidth]{figs/test-Fdrag-components}
  \caption{Contributions of different collider species to the
    dimensionless drag force, \(f\drag / f_*\), as a function of
    gas--grain slip velocity, \(w\).  Solid lines show the Coulomb
    (electrostatic) drag, while dashed lines show the Epstein
    (solid-body) drag.  Results are shown for dimensionless grain
    potential \(\phi = 10\).  All Coulomb forces scale with
    \(\phi^2\), while the Epstein forces are independent of \(\phi\).  The
    species labelled ``CNO++'' represents the combined effect of all
    metals (total abundance \num{8.5e-4}, effective atomic weight
    \num{15.3}), which are assumed to be doubly ionized.}
  \label{fig:drag-components}
\end{figure}


\begin{figure}
  \includegraphics[width=\linewidth]{figs/test-Fdrag-param-space}
  \caption{Regimes of gas--grain drag as a function of slip velocity
    and grain potential.  The different regimes are indicated by bold
    roman numerals, as explained in
    Table~\ref{tab:fdrag-regimes}. Blue shading indicates regions
    dominated by Epstein (solid-body) drag, whereas red and green
    shading indicate regions dominated by Coulomb drag due to protons
    and electrons, respectively.  In each case, the saturated color
    represents a contribution \(> 70\%\) of the relevant component to
    the total drag force, while progressively lighter shading
    represents the \(> 60\%\) and \(> 50\%\) levels.  The thick white
    dotted line indicates the transition between the subthermal and
    superthermal regimes for protons, while the thin white dotted line
    indicates the corresponding transition for electrons.  Contours
    show the total drag force in units of \(f_*\) (see
    eq.~\eqref{eq:fstar}) in decade intervals from \(0.1\) to
    \(10^4\), as labelled.  Results are shown for
    \(T = \SI{8000}{K}\) and \(n = \SI{100}{cm^{-3}}\), but the
    differences are very slight throughout the ranges
    \(T = \text{\SIrange{5000}{15000}{K}}\) and
    \(n = \text{\SIrange{e-3}{e6}{cm^{-3}}}\).}
  \label{fig:drag-v-phi-plane}
\end{figure}


\begin{figure*}
  \includegraphics[width=\linewidth]{figs/gas-grain-drag-photoionized}
  \caption{Dimensionless drag force, \(f\drag / f_*\), as a function
    of gas--grain slip velocity, \(w\), for different values of the
    grain potential in thermal units, \(\phi\).  Contributions from
    proton and electron Coulomb (electrostatic) drag, as well as
    Epstein (solid-body) drag are indicated.  Examples of subsonic and
    highly supersonic stable drift velocities are shown (thin dark
    blue arrows), where the drag force is in equilibrium with the
    radiation force (thick dark blue dashed lines), while blue shading
    indicates the unstable, mildly supersonic velocity regime, where
    no stable drift equilibrium exists.  Inset graph shows \(\phi\) as a
    function of the radiation parameter,
    \(\Xi = P_{\mathrm{rad}} / P_{\mathrm{gas}}\) on a log--linear scale
    for a collection of Cloudy models (see
    Appendix~\ref{sec:cloudy-models-dust}).}
  \label{fig:gas-grain-drag-photoionized}
\end{figure*}


\begin{figure}
  \includegraphics[width=\linewidth]{figs/decouple-v-n-plane}
  \caption{As Fig.~\ref{fig:zones-v-n-plane}(a), but accounting for
    gas-grain decoupling with constant efficiency \(\xi = 0.07\). }
  \label{fig:decouple-v-n-plane}
\end{figure}

\begin{figure}
  \includegraphics[width=\linewidth]{figs/decouple-v40-versus-n}
  \caption{Vertical cut through Fig.~\ref{fig:decouple-v-n-plane},
    showing bow radius and different regimes for a fixed inflow
    velocity of \SI{40}{km.s^{-1}}.}
  \label{fig:decouple-v40-versus-n}
\end{figure}


If the radiation field is sufficiently strong, then the collisional
coupling between grains and gas will break down.  In this section, we
calculate the regions of star+stream parameter space where this might
occur, leading to a separation of the bow into an outer dust wave and
an inner, dust-free bow shock.

\subsubsection{Drag force on grains}
\label{sec:drag-force-grains}

\begin{table}
  \centering
  \caption{Regimes of drag force as function of grain potential and slip speed}
  \label{tab:fdrag-regimes}
  \renewcommand\arraystretch{1.3}
  \resizebox{\linewidth}{!}{%
    \begin{tabular}{@{}r l l l@{}}
    \toprule
      & Regime & Approximate criteria & \(f\drag / f_*\) \\ \midrule
      I & Epstein subsonic & \(\phi^2 \ll 1\)
                             and \(w_{10} < 1\) & \(1.5\, w_{10}\) \\
      II & Epstein supersonic & \(w_{10} > 1\)
                                and \(w_{10} > 5\,\abs{\phi}\)& \( w_{10}^2\) \\
      III & Coulomb p\(^+\) subthermal & \(\phi^2 > 1\)
                                         and \(w_{10} < 1\) & \((1 + 20\, \phi^2)\,
                                                              w_{10}\) \\
    % & Coulomb p\(^+\) peak & \(\phi^2 > 1\)
    %                          and \(w_{10} \approx 1\) & \(1 + 10\, \phi^2\) \\
      IV & Coulomb p\(^+\) superthermal & \(\phi^2 > 1\)
                                          and \(1 < w_{10} < 5\) & \(w_{10}^2
                                                                   + 10\, \phi^2/w_{10}^2 \) \\
      V & Coulomb e\(^-\) subthermal & \(\phi^2 > 20\)
                                 and \(5 < w_{10} < 42\) & \(0.48\, \phi^2 \,
                                                           w_{10}\) \\
    \bottomrule
  \end{tabular}
  }
\end{table}

The drag force on a charged dust grain moving at a relative speed
\(w\) through a plasma has contributions from both direct collisions
and from electrostatic Coulomb interactions with ions and electrons.
We use the expressions in \citet{Draine:1979a}, equations~(4)--(6),
summing over protons, electrons, and helium ions.\footnote{Helium is
  assumed to be 50\% neutral and 50\% singly ionized, leading to only
  a small contribution to the drag force.  For much hotter stars, such
  as the central stars of planetary nebulae, helium may be doubly
  ionized, which leads to a more significant drag contribution,
  especially around \(w \approx \SI{5}{km.s^{-1}}\).} %
Results are shown in Figure~\ref{fig:gas-grain-drag-photoionized} for
different values of the dimensionless grain potential:
\(\phi = e^2 Z\grain / a kT\), which is the electrostatic potential
energy of a unit charge at the surface of a grain of charge
\(Z\grain\) and radius \(a\), in units of the characteristic thermal
energy of a gas particle.  The drag force \(f\drag\) is put in dimensionless units by dividing by
a characteristic force:
\begin{equation}
  \label{eq:fstar}
  f_* = 2 n k T \cdot \pi a^2 \ , 
\end{equation}
which is approximately\footnote{%
  To simplify the exposition, the gas pressure in this section is
  calculated assuming a fully ionized, pure hydrogen plasma, yielding
  \(P\gas = 2 n k T\). For typical ISM abundances, the contribution of
  helium and its corresponding electrons yield a correction to this of
  order 5\%.  The required modifications when a cool star interacts
  with a predominantly neutral gas stream are discussed later.  } %
the ionized gas pressure multiplied by the grain geometric cross
section.

For grains with low electric charge, \(\phi^2 \ll 1\), the drag force is
dominated by direct collisions of protons with the grain.  The gas
collisional mean free path is much larger than the grain size, so the
drag is in the Epstein regime \citep{Weidenschilling:1977b}.  This is
illustrated by the \(\phi = 0.25\) case (blue line) in
Figure~\ref{fig:gas-grain-drag-photoionized}. As the relative
gas--grain slip speed, \(w\), increases, \(f\drag\) first increases
linearly with \(w\) reaching \(f\drag \approx f_*\) at
\(w = \sound \approx \SI{10}{km.s^{-1}}\), then transitions to a quadratic
increase in the supersonic regime.

As \(\abs{\phi}\) increases, long-range electrostatic interactions with
protons within the Debye radius (Coulomb drag) become increasingly
important at subsonic relative velocities, as shown in
Figure~\ref{fig:gas-grain-drag-photoionized} by the orange
(\(\phi = 1\)), green (\(\phi = 4\)), and red (\(\phi = 16\)) lines.  However,
the Coulomb drag has a peak when \(w\) is equal to the thermal
speed of the colliders, which is \(\approx \SI{10}{km.s^{-1}}\) for protons,
giving a maximum strength of
\begin{equation}
  \label{eq:fdrag-maximum}
  f_{\mathrm{max}} = 0.5\, (\ln\Lambda)\, \phi^2 f_* \approx 10\, \phi^2 f_* \ , 
\end{equation}
where \(\Lambda\) is the plasma parameter (number of particles within a
Debye volume), such that
\(\ln\Lambda = 23.267 + 1.5 \ln T_4 - 0.5 \ln n\).  At highly super-thermal
speeds, the Coulomb drag falls asymptotically as
\(f\drag \propto 1 / w^{2}\).  The thermal speed of electrons is higher than
that of the protons by a factor of \((m_p / m_e)^{1/2}\), so that the
electron Coulomb drag gives a second peak fof similar strength, but at
\(w \approx \SI{430}{km.s^{-1}}\).  The behavior of \(f\drag\) in all these
different regimes is summarised in Table~\ref{tab:fdrag-regimes}, in
terms of \(\phi\) and \(w_{10} = w / \SI{10}{km.s^{-1}}\).  This is
further illustrated in Figure~\ref{fig:drag-v-phi-plane}, where the
drag regimes are located on the \((w, \abs{\phi})\) plane.


% \begin{equation}
%   \label{eq:fdrag-regimes}
%   f\drag \approx
%   \begin{cases}
%     \text{Epstein subsonic (\(\phi^2 \ll 1\) and \(w_{10} < 1\)):}
%     & w_{10}\, f_* \\
%     \text{Epstein supersonic (\(\phi^2 \ll 1\) and \(w_{10} > 1\)):}
%     & w_{10}^2\, f_* \\
%     \text{Coulomb p\(^+\) subthermal (\(\phi^2 > 1\) and \(w_{10} < 1\)):}
%     & (1 + 20 \phi^2)\, w_{10}\, f_* \\
%     \text{Coulomb p\(^+\) peak (\(\phi^2 > 1\) and \(w_{10} \approx 1\)):}
%     & (1 + 10 \phi^2)\, f_* \\
%     \text{Coulomb p\(^+\) superthermal (\(\phi^2 > 1\) and \(1 < w_{10} < 5\)):}
%     & (w_{10}^2 + 10 \phi^2/w_{10}^2) \, f_* \\
%     \text{Coulomb e\(^-\) subthermal (\(\phi^2 > 20\) and \(5 < w_{10} < 42\)):}
%     & 0.48 \phi^2 \, w_{10}\, f_* \\
%   \end{cases}
% \end{equation}

% Or \Lambda = (4 pi / 3) n r_D^3?
% Where Debye length is r_D^2 = k T / 4 pi n e^2
% => \Lambda = n r_D (4 pi / 3) k T / 4 pi n e^2
% = k T r_D / 3 e^2
% This is 9 times less than the other expression
% Anyway, from kappa notes I have
% \ln\Lambda = 9.452 + 1.5 ln(T) - 0.5 ln(n)

% If I am going to use log10, then the coefficients get divided by
% ln(10) = 2.30258509299, and if we use T4, then we add 1.5 ln(1e4) =
% 13.815.  So we get 23.267 + 0.651 log10(T4) - 0.217 log10(n).  Nope,
% best with natural log

\subsubsection{Gas--grain separation: drift and rip}
\label{sec:gas-grain-separ}

In Appendix~\ref{sec:gas-free-bow} we calculate the behaviour of an
incoming stream of dust grains, subject only to the repulsive
radiation force from a star.  For an initial inward radial trajectory,
the dust grain motion is decelerated and turned around, reaching a
minimum radius \(R\starstar\), given by equation~\eqref{eq:dust-r0}.
This drag-free radiative turnaround radius, \(R\starstar\), is smaller
for higher initial inward velocities, but is independent of the
density of the incoming stream.  We are now in a position to see how
gas--grain drag will modify this picture.

From equations~\eqref{eq:dust-rad-force} and~\eqref{eq:fstar}, we can
write the radiation force acting on a grain as
\begin{equation}
  \label{eq:frad-Xi}
  f\rad = \Qp\, \Xi\, f_* \ ,
\end{equation}
where \(\Qp\) is the grain's radiation pressure efficiency (see
footnote~\ref{fn:Qp} in Appendix~\ref{sec:gas-free-bow}) and \(\Xi\) is
the local radiation parameter, defined as the ratio of direct stellar
radiation pressure to gas pressure:
\begin{equation}
  \label{eq:Xi-Prad-over-Pgas}
  \Xi \equiv \frac{P\rad}{P\gas} \approx \frac{L}{4 \pi R^2 c\, (2 n k T)} \ ,
\end{equation}
where the last expression corresponds to the optically thin limit.
The grain potential \(\phi\) is also primarily determined by \(\Xi\), as
shown in Appendix~\ref{sec:cloudy-models-dust} and the inset graph of
Figure~\ref{fig:gas-grain-drag-photoionized}, but with a slow
dependence, which can be approximated as
\begin{equation}
  \label{eq:phi-vs-Xi}
  \phi \approx 1.5 \bigl( 2.3 +  \ln \Xi \bigr) \ .
\end{equation}
There are also slight secondary dependencies on the grain composition and
stellar spectrum.  The relationship given in eq.~\eqref{eq:phi-vs-Xi}
is appropriate for graphite grains and for stellar effective
temperatures in the range \SIrange{20}{30}{kK}.  For hotter stars than
this, \(\phi\) should be multiplied by a further factor of \(1.5\), while
for silicate grains it should be divided by \(1.5\).

In the outer regions of the photoionized volume around an OB star,
close to the ionization front, the radiation parameter is low, with
typical value \(\Xi \sim 0.1\).  In this regime, the negative charge
current at the grain surface due to electron collisions is roughly in
balance with the positive current due to the ultraviolet photoelectric
effect \citep{Weingartner:2001b}, leading to a low grain potential,
\(\abs{\phi} < 1\), which may be positive or negative.  The low
\(\Xi\) means that the radiative force is also weak:
\(f\rad \sim 0.1 f_*\) from equation~\eqref{eq:frad-Xi} if
\(\Qp \sim 1\) at UV wavelengths, which is true for all but the smallest
grains.  Thus, from the equations for \(f\drag\) given in
Table~\ref{tab:fdrag-regimes}, the radiative force can be balanced by
Epstein drag if \(w_{10} \sim 0.1\), leading to a small equilibrium drift
velocity, \(w\drift < \SI{1}{km.s^{-1}}\), of the grains with respect
to the gas.  This drift is much smaller than the inward stream
velocities that we are considering
(\(v_\infty > \SI{10}{km.s^{-1}}\)), so the dust follows the gas stream at
a slightly reduced velocity (\(< 10\%\)), and (by mass conservation) a
slightly increased density.  Each grain exerts an exactly opposite
force to \(f\drag\) upon the gas, but since the dust-gas mass ratio,
\(Z\grain\), is small, this produces a negligible acceleration of the
gas.

As the dusty stream approaches the star, the radiation parameter
\(\Xi\) will increase, with a dependence of \(R^{-2}\) once the stream
is well away from the ionization front.  This increases \(f\rad\)
(eq.~\eqref{eq:frad-Xi}), but also increases the grain potential,
\(\phi\) (eq.~\eqref{eq:phi-vs-Xi}) due to the increasing dominance of
grain charging by photoelectric ejection.  Initially, this results in
a lowering of the equilibrium drift velocity to \(w_{10} \sim 0.01\) as
the Coulomb drag kicks in (see Appendix~\ref{sec:cloudy-models-dust}).
However, at smaller radii the slow logarithmic increase in
\(\phi(\Xi)\) means that the drift velocity must start increasing again to
accommodate the linear increase of \(f\rad(\Xi)\).  Eventually,
\(f\rad\) exceeds \(f_{\mathrm{max}}\), the maximum drag force that
proton Coulomb interactions can provide
(eq.~\eqref{eq:fdrag-maximum}).  This occurs at a critical value of
the radiation parameter, which we denote the \textit{rip point}:
\(\Xi_\dag \sim 1000\).  The variations in \(\Xi_\dag\) with star and grain
parameters, which are of order \SI{+- 0.5}{dex}, are listed in
Table~\ref{tab:Xi-rip}.


\begin{table}
  \caption{Critical values of radiation parameter at the rip point: \(\Xi_\dag\)}
  \centering
  \begin{tabular}{l @{\quad\quad\quad} S S} \toprule
    & \multicolumn{2}{c}{Grain composition} \\
    Spectrum & {Graphite} & {Silicate}
    \\ \midrule
    B star & 1000 +- 400 & 350 +- 150 \\
    O star & 3000 +- 500 & 2500 +- 500 \\
    \bottomrule
    \addlinespace
    \multicolumn{3}{@{}p{0.6\columnwidth}@{}}{
    Uncertainties represent variations with grain size and gas density
    (see Appendix~\ref{sec:cloudy-models-dust}).}
  \end{tabular}
  \label{tab:Xi-rip}
\end{table}

\subsection{The case of inside-out bows}
\label{sec:case-inside-out}

So far, we have considered the case where the inner source dominates
the radiation, while dust is present only in the outer stream, which
applies to hot stars interacting with the ISM.  However, in the case
of cool stars, the inner wind will also be dusty.  Examples are the
red supergiant (RSG) phase of high-mass evolution, or the asymptotic
giant branch (AGB) stage of low/intermediate-mass evolution.  In both
these cases, it is still the inner source that provides the radiation
field.  However, not all winds are radiatively driven and in those
cases it is conceivable that it is the outer source that dominates the
radiation field.  An example is the case of photoevaporating
protoplanetary disks (proplyds) in the Orion Nebula and other \hii{}
regions \citep{ODell:1994a}.  In the proplyds, the inner wind is a
thermally driven photoevaporation flow \citep{HA:1998, Henney:1999a},
while the outer stream is the stellar wind from an O~star
\citep{Garcia-Arredondo:2001a}.


%%% Local Variables:
%%% mode: latex
%%% TeX-master: "dusty-bow-wave"
%%% End:

\section{Summary and discussion}
\label{sec:summary-discussion}

How different regions of the \(\Pi\)--\(\Lambda\) plane are populated.
Bottom-right quadrant hard to get to (except for standing wave
oscillations), but may be due to finite shell thickness, which (for
low Mach number) will be more apparent in the wings, which might
decrease \(\Lambda\) more than \(\Pi\).  Fact that thin-shell solutions should
trace the contact discontinuity, but in some cases it may be only the
inner or the outer shell that is visible.

Justification for standing waves: Fig.~3 of \citet{Meyer:2016a} shows
a time sequence of thin-shell instability, which looks a bit like a
standing wave. But much larger amplitude than we are considering.

Deviations from axisymmetry as an alternative to oscillations. 


%%% Local Variables:
%%% mode: latex
%%% TeX-master: "dusty-bow-wave"
%%% End:

\bibliographystyle{mnras}
\bibliography{bowshocks-biblio}
\appendix

\section{Grain charging and gas--grain coupling around OB stars}
\label{sec:cloudy-models-dust}

We calculate models of the physical properties of dust grains using
the plasma physics code Cloudy \citep{Ferland:2013a, Ferland:2017a},
which self-consistently solves the multi-frequency radiative transfer
together with thermal, ionization, and excitation balance of all
plasma constituents.  Cloudy incorporates grain charging as described
in \citet{Baldwin:1991a} and \citet{van-Hoof:2004a} with photoelectric
emission theory from \citet{Weingartner:2001b, Weingartner:2006a}.  We
use the default ``ISM'' dust mixture included in Cloudy, which
comprises ten size bins each for spherical silicate and graphite
grains in the range \num{0.005} to \SI{0.25}{\um}, and which is
designed to reproduce the average Galactic extinction curve
\citep{Weingartner:2001a, Abel:2008a}.  The optical properties of each
grain species are calculated using Mie theory \citep{Bohren:1983a},
assuming solid spheres.  The resultant wavelength-dependent extinction
properties of the mixture are summarised in
Figure~\ref{fig:cloudy-ism-dust-opacity}.

\begin{figure}
  \centering
  \includegraphics[width=\linewidth]{figs/cloudy-ism-dust-opacity}
  \caption{Extinction properties of Cloudy's standard ``ISM'' dust
    mixture. %
    (a)~Wavelength dependence of mean values over the entire mixture
    of three dimensionless quantities related to scattering: albedo,
    \(\varpi\) (solid line); scattering asymmetry,
    \(g = \langle \cos\theta \rangle\) (dashed line); ratio of radiation pressure
    efficiency to absorption efficiency, \(Q_P / Q_{\text{abs}}\)
    (dotted line).
    %
    (b)~Wavelength dependence of mass opacity (cross section per unit
    mass of gas) for the whole mixture (heavy black line) and broken
    down by size bin and grain composition (colored lines, see key). }
  \label{fig:cloudy-ism-dust-opacity}
\end{figure}

To ascertain the expected variation in grain properties in the
circumstellar environs of luminous stars, we calculate a series of
spherically symmetric, steady-state, constant density Cloudy simulations,
illuminated by the stars listed in Table~\ref{tab:stars}, with stellar
spectra taken from the OSTAR2002 and BSTAR2006 grids, calculated with
the TLUSTY model atmosphere code \citep{Lanz:2003a, Lanz:2007a}.
Simulations are run for hydrogen densities of
\numlist{1;10;100;e3;e4} \si{cm^{-3}} and assuming standard \hii{} region
gas phase abundances.  The calculation is stopped when the ionization
front is reached and the inner radius is chosen to be roughly 1\% of this. 



\begin{figure*}
  \includegraphics[width=\linewidth]{figs/multi-dustprops}
  \caption{Dust properties as a function of radius from star for three
    selected Cloudy simulations. (a)~\SI{40}{M_\odot} main-sequence star in
    medium of density \SI{e4}{cm^{-3}}. (b)~\SI{10}{M_\odot} main-sequence
    star in medium of density \SI{1}{cm^{-3}}. (c)~Blue supergiant
    star in medium of density \SI{1}{cm^{-3}}}.
  \label{fig:multi-dustprops}
\end{figure*}
Figure~\ref{fig:multi-dustprops} shows resultant radial profiles of
dust properties for representative simulations: grain temperature, grain
abundance, grain potential, and grain drift velocity.  Line types
correspond to the different size bins of graphite and silicate grains,
as indicated in the key from smallest to largest. The left hand panels
show results for a high-density (\(n = \SI{e4}{cm^{-3}}\)), compact
(\(R \approx \SI{0.1}{pc}\)) region around an early O~star, where the grain
temperature is very high, especially for the smaller silicate grains,
and sublimation significantly reduces the grain abundance in the inner
regions.  The remaining columns show low-density
(\(n = \SI{1}{cm^{-3}}\)), extended (\(R \sim \SI{10}{pc}\)) regions
around main-sequence and supergiant B-type stars, in which the grain
temperatures are much lower, ranging from \SIrange{20}{50}{K} in the
outer parts up to \SIrange{100}{200}{K} in the inner parts.

 
\begin{figure*}
  \includegraphics[width=\linewidth]{figs/drift-pratio-4panel}
  \caption{Drift velocity \(w\drift\) versus radiation parameter
    \(\Xi\). Each line represents a simulation with ambient density and
    stellar type as indicated in the key.  Results are shown for
    graphite and silicate grains of two different sizes.  The rip
    point, which corresponds to gas--grain decoupling, is the
    discontinuity in the curves at
    \(w\drift \approx \SI{10}{km.s^{-1}}\), indicated by the upper
    horizontal dashed line.  The vertical dashed lines show the narrow
    range of radiation parameter, \(\Xi = 1000 \pm \SI{0.5}{dex}\), that
    encompasses the rip point for all simulations. }
  \label{fig:drift-gn}
\end{figure*}

Unlike the strong differences in thermal properties, the radial
dependence of grain electrostatic potential (third row in
Fig.~\ref{fig:multi-dustprops}) is qualitatively similar for all the
simulations.  The grains are predominantly positively charged, with high
potentials (\(> 10\) times the thermal energy of gas particles) close
to the star due to the strong EUV and FUV photo-ejection.  The
potential falls to much lower values in the outer ionized region, as
the EUV flux falls off, and then climbs again at the ionization front
due to the fall in electron density, while the FUV photo-ejection
persists well into the neutral region.  There are small differences
between the simulations due to the increasing relative importance of the
EUV radiation for hotter stars, which leads to a deeper dip in the
potential just inside the ionization front for the \SI{40}{M_\odot} case,
even reaching negative values for some grain species.

Equilibrium drift velocity for each grain species is calculated in the
Cloudy simulations using the same theory \citep{Draine:1979a} as outlined
in \S~\ref{sec:drag-force-grains}.  The way that this is implemented
by default in Cloudy means that if the only solution at the inner
radius is a superthermal one, then the superthermal solution branch
(upper right corner of Fig.~\ref{fig:gas-grain-drag-photoionized}) is
followed as far as possible through the outer spatial zones.  We have
modified the code so as to instead always prefer the slower subthermal
branch whenever multiple solutions are available.  This makes the most
sense in our context, where the grains are moving towards the star and
so the radiative force is gradually increasing from an initial low
value.  Example results are shown in the bottom row of
Figure~\ref{fig:multi-dustprops} and again they are qualitatively
similar for all the simulations.  Close to the star, the radiation force is
higher than the upper limit on the Coulomb drag force
(eq.~[\ref{eq:fdrag-maximum}]), so that the equilibrium drift velocity
is exceedingly high.\footnote{Note that such high drift velocities are
  much higher than any realistic true relative velocity between grains
  and gas, since they are based on the assumption that the radiation
  force remains constant while the grain is accelerated, which can
  never be the case.  Instead, they are simply an indication that the
  gas and grains have completely decoupled.}.  As the radial distance
from the star increases, the radiation field is increasingly diluted
but the grain potential falls only slowly, so eventually one reaches a
point where an equilibrium between Coulomb drag and radiation force
can be established, which corresponds to a discontinuity in the drift
velocity.  This is the \textit{rip point} discussed in
\S~\ref{sec:gas-grain-separ}.  The drift velocity carries on falling
towards the outside of the \hii{} region, but then increases again
just inside the ionization front due to the drop in grain potential
there.

Both the charge balance and the force balance are essentially due to
competition between the photons and the charged particles that
interact with the grain.  It is therefore reasonable to surmise that
the gas--grain decoupling that occurs at the rip point might be
determined principally by the ratio of photons to gas particles.  We
test this hypothesis in Figure~\ref{fig:drift-gn}, where we
characterize the photon-gas ratio by a dimensionless radiation
parameter, \(\Xi\), equal to the radiation pressure divided by the gas
pressure.  Results are shown for four different grain types and for
all combinations of stellar parameters and ambient densities for which
we have run simulations.  It can be seen that the rip point does indeed
always occur at a similar value of \(\Xi \sim 1000\) for all simulations, albeit
with some variation according to the spectral type of the star and the
grain composition, as given in Table~\ref{tab:Xi-rip}.  The gas
density and grain size have very little influence on this critical
value \(\Xi_\dag\), with the only exception being the very smallest grains
(\(a < \SI{0.006}{\um}\), not illustrated), which show
\(\Xi_\dag \approx \num{e4}\), but such grains are only minor contributors to the
UV opacity (\(< 10\%\) in EUV and \(< 1\%\) in FUV, see
Fig.~\ref{fig:cloudy-ism-dust-opacity}).

\begin{figure}
  \centering
  \includegraphics[width=\linewidth]{figs/phi-versus-xi-annotate}
  \caption{Grain potential in thermal units (linear scale) versus
    radiation parameter (logarithmic scale). All densities and stellar
    types are shown, with line colors as in Fig.~\ref{fig:drift-gn}.
    Solid lines show silicate grains and dashed lines show graphite
    grains.  Line width increases with grain size (to reduce clutter,
    only every second size bin is shown).  The solid and dashed lines
    show the logarithmic fits discussed in the text.}
  \label{fig:phi-vs-Xi}
\end{figure}
Finally, Figure~\ref{fig:phi-vs-Xi} shows the slow dependence of grain
potential on radiation parameter for all the simulations on a linear versus
logarithmic scale.  The logarithmic fit of
equation~\eqref{eq:phi-vs-Xi}, most appropriate for carbon grains
around cooler stars, is shown by the solid line.  The dashed lines
show the modifications for silicate grains and for hotter stars (see
\S~\ref{sec:gas-grain-separ}).

\begin{figure}
  \centering
  \includegraphics[width=\linewidth]{figs/grain-j70-vs-U-edited}
  \caption{Grain emissivity at \SI{70}{\um} for all Cloudy models
    (lines colored as in Fig.~\ref{fig:drift-gn}), compared with grain
    models from \citet{Draine:2007a} (dark gray symbols), which assume
    illumination by a scaled interstellar radiation field, which has a
    SED with a very different shape from that of an OB star.  }
  \label{fig:grain-j70}
\end{figure}

\begin{figure}
  \centering
  \includegraphics[width=\linewidth]{figs/sed-comparison}
  \caption{Comparison between the spectral energy distribution (SED)
    of a typical OB star (blue line) and the interstellar radiation
    field in the solar neighborhood (orange line).  The OB star is the
    \SI{20}{M_\odot} model from Table~\ref{tab:stars} and is plotted for a
    distance from the star of \SI{1}{pc}.  The interstellar SED is
    from \citet{Mathis:1983a} and is multiplied by \num{100} so that
    the total FUV-to-NIR flux is similar for the two SEDs.}
  \label{fig:sed-comparison}
\end{figure}



%%% Local Variables:
%%% mode: latex
%%% TeX-master: "dusty-bow-wave"
%%% End:


\section{Shape of a dusty radiative bow wave}
\label{sec:shape-dust-wave}

As an alternative to hydrodynamic or magnetohydrodynamic bow shocks, it is possible that some observed emission arcs may be

\begin{figure}
  \centering
  \includegraphics[width=\linewidth]{figs/dust-trajectories}
  \caption[Dust grain trajectories]{Dust grain trajectories under
    influence of a repulsive central \(r^{-2}\) radiative force.  Dust
    grains approach from the right at a uniform velocity and with a
    variety of impact parameters (initial \(y\)-coordinate). The
    central source is marked by a red star at the origin, and its
    radiative force deflects the trajectories into a hyperbolic shape,
    each of which reaches a minimum radius marked by a small black
    square.  The incoming hyperbolic trajectories are traced in gray
    and the outgoing trajectories are traced in red.  The locus of
    closest approach of the outgoing trajectories is parabolic in
    shape (traced by the thick, light gray line) and this constitutes
    the inner edge of the bow wave. }
  \label{fig:dust-trajectories}
\end{figure}


%%% Local Variables:
%%% mode: latex
%%% TeX-master: "quadrics-bowshock"
%%% End:

\section{Equation of motion for grains with radiation, gas drag, and
  magnetic field}
\label{sec:equat-moti-grains}

Following \citet{Draine:1979a}, the drag force on a grain that is
moving at relative velocity \(\bm{w} = \bm{v}\grain - \bm{v}\gas\)
through a partially ionized gas can be written as a sum over each
collider species, \(k\), with mass \(m_k\), abundance relative to
hydrogen \(\alpha_k\) and charge \(z_k\). If the relative speed,
\(w = \abs{\bm{w}}\), is normalized to the thermal speed of each
species:
\begin{equation}
  \label{eq:s-velocity}
  s_k = \left( m_k w^2 / 2 k T  \right)^{1/2} \ ,
\end{equation}
then the magnitude of the force is 
\begin{equation}
  \label{eq:ds79}
  f\drag = f_* \sum_k \alpha_k \left[ G_0(s_k) + z_k^2 \phi^2 \ln(\Lambda/z_k) G_2(s_k) \right],
\end{equation}
where \(f_*\) is a characteristic thermal force on the grain (see
eq.~[\ref{eq:fstar}]). The dimensionless functions of normalized speed
\(G_0(s)\) and \(G_2(s)\) are given by
\begin{align}
  \label{eq:G0}
  G_0(s) & = \left( s^2 + 1 - \frac{1}{4 s^2} \right) \erf(s)
  +  \left( s + \frac{1}{2 s} \right) \frac{e^{-s^2}}{\sqrt{\pi}}\\
  \label{eq:G2}
  G_2(s) & = \frac{\erf(s)}{s^2} - \frac{2 e^{-s^2}}{s \sqrt{\pi}} \ .
\end{align}
The \(G_0\) term is due to inelastic solid-body collisions in the
Epstein limit, and is derived in \S~4 of \citet{Baines:1965a}.  The
\(G_2\) term is due to electrostatic Coulomb interactions, with
\(\phi\) being the grain potential in thermal units
(eq.~[\ref{eq:phi-potential}]) and \(\Lambda\) the plasma parameter.  It was
first derived in the different context of dynamical friction in
stellar systems by \citet{Chandrasekhar:1941a}.
Figures~\ref{fig:drag-components}--\ref{fig:gas-grain-drag-photoionized}
show example applications to gas--grain drag in a photoionized region.

The grain trajectories presented in \S~\ref{sec:grain-traj-along} and
\ref{sec:grain-traj-with} are calculated by numerically solving the
grain equation of motion:
\begin{equation}
  \label{eq:grain-equation-motion}
  m\grain \frac{d^2 \bm{r}}{d t^2} = \bm{f} \ .
\end{equation}
The total force \(\bm{f}\) is the sum of radiation, drag, and Lorentz
terms:
\begin{equation}
  \label{eq:total-force}
  \bm{f} = \frac{\sigma\grain \Qp L}{4\pi R^2 c} \hat{\bm{r}}
  - f\drag \hat{\bm{w}}
  + \frac{z\grain e}{c} \bm{w} \times \bm{B} \ ,
\end{equation}
with \(f\drag\) given by equation~\eqref{eq:ds79} and where
\(\hat{\bm{r}}\) is the unit vector in the radial direction and
\(\hat{\bm{w}} = \bm{w} / w\) is the unit vector along the direction
of gas--grain relative motion.  In the strong magnetic coupling limit
(see \S~\ref{sec:grain-traj-with} and \ref{sec:tight-magn-coupl}), the
Lorentz term is not included explicitly, but instead the equation of
motion is solved for the guiding center by replacing \(\bm{f}\) by its
projection along the magnetic field: 
\begin{equation}
  \label{eq:projected-force}
  \widetilde{\bm{f}} = (\bm{f} \cdot \hat{\bm{b}})  \, \hat{\bm{b}} \ ,
\end{equation}
where \(\hat{\bm{b}} = \bm{B} / B\).

If distances are measured in units of the radiative turnaround radius,
\(R\starstar\) (eq.~[\ref{eq:dust-r0}]), and times in units of
\(R\starstar / v_\infty\), then the grain acceleration
\(\bm{a}\grain = \bm{f} / m\grain\) in the non-magnetic case can be
written in non-dimensional form as
\begin{equation}
  \label{eq:grain-acceleration}
  \frac{\bm{a}\grain}{a\starstar}
  = \frac{R\starstar^2}{2 R^{2}} \hat{\bm{r}}
  - C\drag \frac{f\drag}{f_*} \hat{\bm{w}} \, 
\end{equation}
where \(a\starstar = v_\infty^2 / R\starstar\) is a characteristic
acceleration scale and the dimensionless drag constant is
\begin{equation}
  \label{eq:drag-constant}
  C\drag = \frac{4}{\Qp} \left(\frac{\sound \tau_* \kappa\grain}{v_\infty \kappa}\right)^2 \ .
\end{equation}

A collection of python programs that implement the equations of this
appendix is available at
\url{https://github.com/div-B-equals-0/dust-trajectories}, including
programs to generate all the grain trajectory figures of this paper
plus additional figures and movies.  The integration of
equation~\eqref{eq:grain-acceleration} is carried out using the
python library function \texttt{scipy.integrate.odeint}, which wraps
the Fortran ODEPACK library \citep{Hindmarsh:1983a, Jones:2001a}.

% \begin{figure}
%   \includegraphics[width=\linewidth]{figs/dust-couple-div-stream}
%   \caption{Divergent dragoids}
%   \label{fig:divergent-dragoids}
% \end{figure}

% To find the dust grain trajectories \(R\grain(\theta)\) in the presence of
% radiation and drag forces (\S~\ref{sec:bow-wave-drag}), we numerically
% integrate the equations of motion. We define dimensionless cylindrical
% polar coordinates,
% \begin{equation}
%   \label{eq:dust-XY}
%   (X, Y) = \left(\frac{R\grain(\theta) \cos\theta } {R_0}, \ 
%     \frac{R\grain(\theta) \sin\theta } {R_0}\right)
%   \ ,
% \end{equation}
% and dust grain velocities,
% \begin{equation}
%   \label{eq:dust-UV}
%   (U, V) = \left( \frac{\bm{v}\grain \cdot \uvec{x}} {v_\infty}, \ 
%   \frac{\bm{v}\grain \cdot \uvec{y}} {v_\infty}\right) \ ,
% \end{equation}
% where \(\uvec{x}\) and \(\uvec{y}\) are unit vectors along the \(X\)
% and \(Y\) axes (parallel and perpendicular, respectively, to the
% symmetry axis).  The grain equation of motion then follows from
% equations~(\ref{eq:dust-rad-force}, \ref{eq:dust-r0},
% \ref{eq:dust-fdrag}--\ref{eq:dust-alpha}) as the following set of
% coupled differential equations:
% \begin{gather}
%   \label{eq:dust-motion}
%   \begin{aligned}
%     \frac{d X}{d t} &= U \quad\quad
%     \frac{d Y}{d t} = V \\
%     \frac{d U}{d t} &= \frac12 \left[  
%       X \left(X^2 + Y^2\right)^{-3/2} - \alpha\drag^2 D_1 \left(U - U_1\right)
%     \right] \\
%     \frac{d V}{d t} &= \frac12 \left[  
%       Y \left(X^2 + Y^2\right)^{-3/2} - \alpha\drag^2 D_1 \left(V - V_1\right)
%     \right] \ ,
%   \end{aligned}
% \end{gather}
% where \((U_1, V_1)\) are the components of the gas velocity (assumed
% fixed), given by
% \begin{equation}
%   \label{eq:dust-gas-velocities}
%   (U_1, V_1) = 
%   \begin{cases}
%     \text{parallel stream} & (-1, 0)\\
%     \text{divergent stream} &
%     \left( \dfrac{X - \mu^{-1}}{R_1},\ \dfrac{Y}{R_1}\right) \ ,
%   \end{cases}
% \end{equation}
% where
% \begin{equation}
%   \label{eq:dust-R1}
%   R_1 = \left( \bigl(X - \mu^{-1}\bigr)^2 + Y^2 \right)^{1/2}
% \end{equation}
% is the distance from the second source, located at
% \((X, Y) = (\mu^{-1}, 0)\).  The dimensionless gas density, \(D_1\),
% normalized by the value at \((X, Y) = (1, 0)\), is
% \begin{equation}
%   \label{eq:dust-gas-density}
%   D_1 = 
%   \begin{cases}
%     \text{parallel stream} & 1\\
%     \text{divergent stream} & \dfrac{\bigl(\mu^{-1} - 1\bigr)^2} {R_1^{2}} \ .
%   \end{cases}
% \end{equation}

% Equations~\eqref{eq:dust-motion} are integrated using the python
% wrapper \texttt{scipy.integrate.odeint} to the Fortran ODEPACK library
% \citep{Hindmarsh:1983a, Jones:2001a}, with results shown in
% Figure~\ref{fig:dust-wave-coupling} for parallel-stream cases and
% Figure~\ref{fig:divergent-dragoids} for divergent-stream cases. 

%%% Local Variables:
%%% mode: latex
%%% TeX-master: "dusty-bow-wave"
%%% End:

%
\section{Further details on ionization front trapping}
\label{sec:furth-deta-ioniz}

This appendix will probably be dropped from the paper.  It contains
details of the derivation of the ionization front trapping that I now
think are too verbose to be included, given that this is not the main
point of the paper.  They are collected here for completeness.

We wish to calculate whether the star is capable of photoionizing the
entire bow shock shell, or whether the ionization front will be
trapped within it.  The number of hydrogen recombinations\footnote{%
  The diffuse field is treated in the on-the-spot approximation,
  assuming all emitted Lyman continuum photons are immediately
  re-absorbed locally, so the case~B recombination co-efficient,
  \(\alphaB = \num{2.6e-13}\, T_4^{-0.7}\, \si{cm^3.s^{-1}}\), is
  used, where \(T_4 = T/\SI{e4}{K}\).} %
per unit time per unit area in a fully ionized shell is
\begin{equation}
  \label{eq:shell-recombination-rate-app}
  \mathcal{R} = \alphaB n\shell^2 h\shell \ ,
\end{equation}
while the flux of hydrogen-ionizing photons
(\(h \nu > \SI{13.6}{eV}\)) incident on the inner edge of the shell is
\begin{equation}
  \label{eq:shell-ionizing-flux-app}
  \mathcal{F} = \frac{S} {4 \pi R_0^2} \ , 
\end{equation}
where \(S\) is the ionizing photon luminosity of the star.  Any shell
with \(\mathcal{R} > \mathcal{F}\) cannot be entirely photoionized by
the star, and so must have trapped the ionization front.  The column
density of the shocked shell can be found, for example, from
equations~(10) and~(12) of \citet{Wilkin:1996a} in the limit
\(v_\infty/V \to 0\) (Wilkin's parameter \(\alpha\)) and \(\theta \to 0\).  This yields
\begin{equation}
  \label{eq:shocked-shell-column-app}
  n\shell h\shell = \tfrac34 n R_0 \ .
\end{equation}
Assuming strong cooling behind the shock,\footnote{%
  This is shown to be justified in \S~\ref{sec:radi-cool-lengths}.
} %
the shell density is \(n\shell = \mathcal{M}_0^2 n\), where
\(\mathcal{M}_0 = v_\infty / \sound\) is the isothermal Mach number of the
external stream.\footnote{%
  \label{fn:temperature-dependence}
  The sound speed depends on the temperature and hydrogen and helium
  ionization fractions, \(y\) and \(y_{\text{He}}\) as
  \(\sound^2 = (1 + y + z_{\text{He}} y_{\text{He}}) (k T /
  \bar{m})\), where \(z_{\text{He}}\) is the helium nucleon abundance
  by number relative to hydrogen and
  \(k = \SI{1.3806503e-16}{erg.K^{-1}}\) is Boltzmann's constant.  We
  assume \(y = 1\), \(y_{\text{He}} = 0.5\), \(z_{\text{He}} = 0.09\),
  so that \(\sound = \num{11.4}\, T_4^{1/2}\, \si{km.s^{-1}}\). } %
Putting these together with equations~\eqref{eq:Rstar} and
~\eqref{eq:tau-star}, one finds that \(\mathcal{R} > \mathcal{F}\)
implies
\begin{equation}
  \label{eq:ifront-trap-x-cubed-taustar-app}
  x^3 \tau_* > \frac{4 S c \sound \bar{m}^2 \kappa}{3 \alpha L} \ .
\end{equation}
From equation~\eqref{eq:rad-full-x}, it can be seen that \(x\) depends
on the external stream parameters, \(n\), \(v_\infty\) only via
\(\tau_*\), and so equation~\eqref{eq:ifront-trap-x-cubed-taustar} is a
condition for \(\tau_*\).  In the radiation bow shock case,
\(x = (1 + \eta)^{1/2}\), and the condition can be written:
\begin{equation}
  \label{eq:ifront-trap-taustar-bow-shock}
  \tau_* > 145.0 \frac{S_{49} T_4^{1.7} \kappa_{600}}{L_4 (1 + \eta)^{3/2}} \ , 
\end{equation}
where
\begin{equation*}
  S_{49} = S / \bigl( \SI{e49}{s^{-1}} \bigr) \ .
\end{equation*}
Numerical values of \(S_{49}\) for our three example stars are given
in Table~\ref{tab:stars}.  In the radiation bow wave case,
\(x = 2\tau_*\), and the condition can be written:
\begin{equation}
  \label{eq:ifront-trap-taustar-bow-wave}
  \tau_* > \left(  18.1 \frac{S_{49} T_4^{1.7} \kappa_{600}}{L_4}\right)^{1/4} \ , 
\end{equation}



\(\tau_* \sim n^{1/2} / v_\infty\) 
This simple
criterion is shown by the dark red line in
Figure~\ref{fig:zones-v-n-plane}.  If 
\(n / v_{10}^2 > \text{\numrange{1000}{5000}}\), depending
only weakly on the stellar parameters, then the outer parts of the
shocked shell are neutral, instead of ionized. 


Assuming photoionization equilibrium, the
hydrogen photoabsorption optical depth of the shell is
\begin{equation}
  \label{eq:ion-tau-gas}
  \tau\gas = - \ln(1 - \mathcal{R} / \mathcal{F}) \ ,
\end{equation}
so long as \(\mathcal{R} < \mathcal{F}\).

We will assume
a typical photoionized temperature of \SI{8000}{K}, so that
\(\sound \approx \SI{10}{km.s^{-1}}\) and \(M_0 = v_{10}\), yielding
\begin{equation}
  \label{eq:ion-tau-gas-expanded}
  \tau\gas = -\ln\bigl(1 -
  \num{8.42e-6}\, v_{10}^2 n^2 R_{0,\text{pc}}^3 S_{49}^{-1}\bigr) \ , 
\end{equation}
where 
\begin{align*}
  R_{0,\text{pc}} &= R_0 / \bigl( \SI{1}{pc} \bigr)
\end{align*}
The dust opacity is approximately constant
at FUV to EUV wavelengths, so the dust optical depth of the shocked
shell to ionizing photons follows from equations~\eqref{eq:tau-thin}
and~\eqref{eq:shocked-shell-column} as \(\tau\grain = \tfrac38 \tau\).

The hydrogen ionization fraction, \(y\), at the outer edge of the shocked
shell then follows as
\begin{equation}
  \label{eq:outer-shell-ionization-balance}
  \frac{y^2}{1 - y} = \frac{\sigma \mathcal{F}}{\alphaB n} e^{-(\tau\grain + \tau\gas)} \ ,
\end{equation}
where \(\sigma\) is the effective hydrogen photoionization cross section,
averaged over the local ionizing spectrum.  Since the
frequency-dependent cross section, \(\sigma_\nu \sim \nu^{-3}\), is strongly
peaked at the threshold, the local EUV spectrum becomes harder with
increasing \(\tau\gas\), as the lower frequency photons are selectively
absorbed,\footnote{} leading to a reduction in the effective
\(\sigma\).  An approximate fit to the results in Appendix~A of
\citet{Henney:2005b} is
\begin{equation}
  \label{eq:sigma-vs-tau}
  \sigma = 0.5 \sigma_0 e^{-\tau\gas/3}
\end{equation}
where \(\sigma_0 = \SI{6e-18}{cm^2}\) is the threshold cross-section.
Although this was derived for a particular ionizing spectrum (\SI{40
  000}{K} black body), we adopt it for all our hot stars.



%%% Local Variables:
%%% mode: latex
%%% TeX-master: "dusty-bow-wave"
%%% End:


% Don't change these lines
\bsp	% typesetting comment
\label{lastpage}
\end{document}


%%% Local Variables:
%%% mode: latex
%%% TeX-master: t
%%% End:
