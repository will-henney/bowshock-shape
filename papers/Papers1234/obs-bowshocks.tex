%\RequirePackage{amsmath}
\documentclass[useAMS, usenatbib, a4paper]{mnras}
\pdfsuppresswarningpagegroup=1

% Standard LaTeX packages
\usepackage{graphicx}
\usepackage{microtype}
\usepackage{xcolor}
\usepackage{fixltx2e}
\usepackage{booktabs}
\usepackage{siunitx}
\usepackage{color}
\usepackage{enumerate}
\usepackage{pdflscape}
\usepackage{rotating}
\usepackage{hyperref}

\usepackage[T1]{fontenc} 
\usepackage[utf8]{inputenc}

% Fonts 
\usepackage{newtxtext}
% Note: newtxmath must come AFTER newtxtext
\usepackage[varvw,smallerops]{newtxmath}

\usepackage{chemgreek}
\activatechemgreekmapping{newtx}


% Define a fermata symbol for use inline with text
% 04 Jul 2017 WJH
% Based on info in Secs 18.4.1 and 24.1 of musixdoc.pdf
\let\ORIGna\na
\usepackage{musixtex}
\makeatletter
\newcommand\textfermata{%
  {\let\extractline\relax
    \setlines10\smallmusicsize \nobarnumbers \nostartrule
    \staffbotmarg0pt \setclefsymbol1\empty \global\clef@skip0pt
    \raisebox{0ex}[0ex][0ex]{
      \startextract\addspace{-\afterruleskip}%
      \notes\fermataup{-2}\en
      \zendextract}}}
\makeatother
\let\na\ORIGna

% % The \eye command is in the dingbat package, but we need to take care
% % of a naming conflict with the AMS package
% \usepackage{savesym}
% \savesymbol{checkmark}
% \usepackage{dingbat}

% The "eye" symbol is drawn with the \faEye command of the fontawesome
% package
\usepackage{fontawesome}
% The ``irregular'' symbol is drawn from the Icelandic Sorcery and Witchcraft
\usepackage{staves}

\hypersetup{colorlinks=True, linkcolor=blue!50!black, citecolor=black,
  urlcolor=blue!50!black}

\usepackage{etoolbox}
\robustify\bfseries
\robustify\itshape

%% The following hack solves a problem with
%% ERROR: \pdfendlink ended up in different nesting level than \pdfstartlink.
%% See https://tex.stackexchange.com/a/249743
\makeatletter
\patchcmd\@combinedblfloats{\box\@outputbox}{\unvbox\@outputbox}{}{%
  \errmessage{\noexpand\@combinedblfloats could not be patched}%
}%
\makeatother

%% Bold italic
\newcommand\hmmax{0}            % we don't need heavy fonts
\newcommand\bmmax{1}            % reduce use of math alphabets for bold
\usepackage{bm}

%% Bundled custom packages
\usepackage{aastex-compat}

\defcitealias{Tarango-Yong:2018a}{Paper~0}
\newcommand\PaperI{\citetalias{Tarango-Yong:2018a}}


% \title[Stellar bow shocks]{Weather vanes and runaways: Bow
%   shocks across the HR diagram}
\title[Bow shocks, bow waves, and dust waves. IV.] %
{Bow shocks, bow waves, and dust waves. IV. Shell shape statistics}

\newcommand\AddressCRyA{Instituto de Radioastronom\'{\i}a y Astrof\'{\i}sica,
  Universidad Nacional Aut\'onoma de M\'exico, Apartado Postal 3-72,
  58090 Morelia, Michoac\'an, M\'exico}
\author[Henney et al.]{
  William J. Henney,\thanks{Email: w.henney@irya.unam.mx}
  Jorge A. Tarango-Yong,
  Luis \'Angel Guti\'errez-Soto,\thanks{%
    Current address: Observatório do Valongo,
    Universidade Federal do Rio de Janeiro,
    Ladeira Pedro Antonio 43, 20080-090 Rio de Janeiro, Brazil}
  \& S. J. Arthur
  \\
  \AddressCRyA
}

% These dates will be filled out by the publisher
\date{Accepted XXX. Received YYY; in original form ZZZ}

% Enter the current year, for the copyright statements etc.
\pubyear{2019}

\DeclareMathOperator{\sgn}{sgn}
\DeclareMathOperator{\Sin}{\mathcal{S}}
\DeclareMathOperator{\Cos}{\mathcal{C}}
\DeclareMathOperator{\Cot}{\mathcal{T}}
\DeclareMathOperator{\GammaFunc}{\Gamma}
\DeclareMathOperator\erf{erf}
\newcommand\w{\ensuremath{\mathrm{w}}}
\newcommand\C{\ensuremath{\mathrm{c}}}
\providecommand{\abs}[1]{\lvert#1\rvert}
\providecommand{\Abs}[1]{\left\lvert#1\right\rvert}
\newcommand\TODO[1]{%
  \begin{center}
    \framebox{\parbox{0.8\linewidth}{
        \texttt{\footnotesize\color{red} #1}}}
  \end{center}}

\newcommand\uvec[1]{\bm{\hat{#1}}}
\newcommand\T{_{\mathrm{\scriptscriptstyle T}}}

\newcommand\Qp{\ensuremath{Q_{\text{p}}}}
\newcommand\Qpbar{\ensuremath{\bar{Q}_{\text{p}}}}
\newcommand{\grain}{\ensuremath{_{\text{d}}}}
\newcommand{\B}{\ensuremath{_{\scriptscriptstyle\text{B}}}}
\newcommand{\alfven}{\ensuremath{_{\scriptscriptstyle\text{A}}}}
\newcommand{\xsec}{\ensuremath{\sigma\grain}}
\newcommand\frad{\ensuremath{f_{\text{rad}}}}
\newcommand\fmax{\ensuremath{f_{\text{max}}}}
\newcommand\thm{\ensuremath{\theta_{\text{m}}}}
\newcommand\drag{\ensuremath{_{\text{drag}}}}
\newcommand{\gas}{\ensuremath{_{\text{gas}}}}
\newcommand{\wind}{\ensuremath{_{\text{w}}}}
\newcommand{\trap}{\ensuremath{_{\text{abs}}}}
\newcommand{\ke}{\ensuremath{_{\text{kin}}}}
\newcommand{\drift}{\ensuremath{_{\text{drift}}}}
\newcommand\rad{\ensuremath{_{\text{rad}}}}
\newcommand\Lya{\ensuremath{_{\text{Ly}\alpha}}}
\newcommand\Rmin{\ensuremath{R_{\scriptscriptstyle\text{min}}}}
% Why do I need both of these?
\newcommand\sound{\ensuremath{c_{\text{s}}}}
\newcommand\soundspeed{\ensuremath{c_{\text{s,gas}}}}
\newcommand\starstar{\ensuremath{_{**}}}
\newcommand\mmp{\ensuremath{_{\text{\tiny MMP83}}}}
\newcommand\Hab{\ensuremath{_{\text{\tiny Habing}}}}
\newcommand\IR{\ensuremath{_{\text{IR}}}}
\newcommand{\thD}{\(\theta^1\)\,Ori~D}
\newcommand\alphaB{\ensuremath{\alpha_{\text{B}}}}
\newcommand\shell{\ensuremath{_{\text{sh}}}}
\newcommand\M{\ensuremath{\mathcal{M}}}
\newcommand\hii{\ion{H}{ii}}

\newcommand\amu{\ensuremath{a_{\si{\um}}}}

%%% Local Variables:
%%% mode: latex
%%% TeX-master: "bs-bw-dw-01"
%%% End:


\begin{document}
\label{firstpage}
\pagerange{\pageref{firstpage}--\pageref{lastpage}}
\maketitle
\begin{abstract}
  Stellar bow shocks result from relative motions between stars and
  their environment. The interaction of the stellar wind and radiation
  with gas and dust in the interstellar medium produces curved arcs of
  emission at optical, infrared, and radio wavelengths.  We recently
  proposed a new two-dimensional classification scheme for the shape
  of such bow shocks, which we here apply to three very different
  observational datasets: mid-infrared arcs around hot OB stars;
  far-infrared arcs around luminous cool stars; and H\(\alpha\)
  emission-line arcs around proplyds and other young stars in the
  Orion Nebula.  For OB stars, the average shape is consistent with
  simple thin-shell models for the interaction of a spherical wind
  with a parallel stream, but the diversity of observed shapes is many
  times larger than such models predict.  We propose that this may be
  caused by time-dependent oscillations in the bow shocks, due to
  either instabilities or wind variability.  Cool star bow shocks have
  markedly more closed wings than hot star bow shocks, which may be
  due to the dust emission arising in the shocked stellar wind instead
  of the shocked interstellar medium.  The Orion Nebula arcs, on the
  other hand, have both significantly more open wings and
  significantly flatter apexes than the hot star bow shocks.  We test
  several possible explanations for this difference (divergent ambient
  stream, low Mach number, observational biases, and influence of
  collimated jets), but the evidence for each is inconclusive.
\end{abstract}

\begin{keywords}
  circumstellar matter -- methods: statistical -- stars: winds, outflows
\end{keywords}


\section{Introduction}
\label{sec:introduction}

Stellar bow shocks are produced by the relative motion between a star
and its surrounding medium, and are commonly detected as curved arcs
of emission at optical \citep{Gull:1979a, Brown:2005a}, infrared
\citep{van-Buren:1988a, Kobulnicky:2016a}, or radio
\citep{van-Buren:1990a, Benaglia:2010a} wavelengths.  The canonical
theory for these objects is that they are formed by a two-shock
interaction between the stellar wind and the interstellar medium
\citep{Pikelner:1968a, Dyson:1972a}, which is distorted due to the
supersonic motion of the star \citep{Baranov:1970a, Wilkin:1996a}.  In
some instances, however, the absorbed stellar radiation pressure may
be more important than the stellar wind in providing the inner support
for the bow shell \citep[Paper~I]{Henney:2019a} and this may even be
sufficient to break the collisional coupling between gas and dust
grains \citep[Paper~II]{Henney:2019b}.  In other cases, the appearance
of an infrared emission arc may be due to the illumination of the
inner wall of an asymmetrical cavity \citep{Mackey:2016a}, rather than
the formation of a dense shell, in which case the relative velocity of
the star may be subsonic with respect to its surroundings
\citep{Mackey:2015a}.

The largest number of bow shocks have been detected around OB stars,
via their mid-infrared dust emission \citep{van-Buren:1995a,
  Smith:2005a, Povich:2008a, Kobulnicky:2010a, Peri:2012a, Peri:2015a,
  Sexton:2015b, Kobulnicky:2016a, Bodensteiner:2018a}, and these have
sizes ranging from \SIrange{0.01}{1}{pc}.  

In the Orion Nebula, at least three different classes of stellar bow
shocks have been identified. As well as small number of OB bow shocks \citep{Smith:2005a, ODell:2001c}, 

Interaction of wind with photoevaporation flow \citep{Dyson:1975a}

HMXRB in external galaxies, possible bowshock in LMC~X-1 \citep{Hyde:2017a}.

Wolf-Rayet bow shock nebulae \citep{Dyson:1989a}

Non-thermal radio emission from BD+43~3654 \citep{Benaglia:2010a}, has
been searched for but

Cooler red supergiant and asymptotic giant branch stars
\citep{Ueta:2008a, Sahai:2010a, Cox:2012a}. 
Bowshocks from red supergiants \citep{Meyer:2014a}


Instability of bow \citep{Blondin:1998a}

%%% Local Variables:
%%% mode: latex
%%% TeX-master: "obs-bowshocks"
%%% End:

\newcommand\hii{\ion{H}{ii}}

\section{Comparison with observations}
\label{sec:comp-with-observ}


Placing various classes of objects on the \(R_{90}\)--\(R_c\) plane:
\begin{itemize}
\item LL arcs
\item runaway O stars
\item AGB stars
\end{itemize}

\subsection{Mid-infrared arcs around early-type stars}
\label{sec:mid-infrared-arcs}

\begin{figure*}
  \setlength\tabcolsep{0pt}
  \begin{tabular}{ll}
    (a)
    & (b) \\
    \includegraphics
    [width=0.5\linewidth, trim=20 15 30 10, clip]{figs/0510-3-star}
    & \includegraphics
      [width=0.5\linewidth, trim=20 15 30 10, clip]{figs/0506-4-star} \\
    (c)
    & (d) \\
    \includegraphics
    [width=0.5\linewidth, trim=20 15 30 10, clip]{figs/0517-5-star}
    & \multicolumn{1}{c}
      {\includegraphics[width=0.35\linewidth, trim=20 20 20 20]
      {figs/mipsgal-r0-r0-plus-dPA-edited}}
  \end{tabular}
  \caption[]{Examples of typical fits to the bow shock shapes of
    MIPSGAL sources with different star ratings: (a) K510, 3-star
    rating; (b) K506, 4-star rating; (c) K517, 5-star rating.  Right
    panels of parts (a)--(c) show a 4\('\) square \SI{24}{\um} image,
    centered on each source.  Contours are ten linearly spaced levels
    between the median brightness of the entire image and the maximum
    brightness of the bow shock arc. Grids of galactic coordinates
    (light blue lines, parallel to the box sides) and equatorial
    coordinates (tilted magenta lines) are shown.  The stellar source
    and the bow shock axis, as determined by \citet{Kobulnicky:2016a}
    are indicated by an orange star and an orange line, respectively,
    where the line extends from \(-2 R_0\) to \(+2 R_0\).  The
    automatically traced arc shapes using the ``mean'' and ``peak''
    methods (see text) are shown by blue and red dots, respectively.
    The magenta circle shows the fit to the arc points within
    \(\pm 45^\circ\) of the nominal bowshock axis, with the magenta dot
    showing the center of curvature and the magenta line showing the
    fitted bow shock axis, which is the line passing through the
    source and the center of curvature.  Left panels of parts (a)--(c)
    show the radius measured from the source (upper panel) and
    brightness (lower panel) of the arc points, plotted as a function
    of angle \(\theta\) from the nominal bow shock axis, and with the same
    color coding as used on the image. Angular ranges of
    \(\theta = \pm 45\degr\) and \(\pm 90\degr\) are shown by gray shaded
    boxes In the upper panel, the \(R_0\) value tabulated by
    \citet{Kobulnicky:2016a} is shown by a horizontal blue line. (d)
    Comparison of the bow shock sizes (scatter plot) and position
    angles (inset histograms) determined from our fits with those
    tabulated by \citet{Kobulnicky:2016a} for the MIPSGAL sources.
    The horizontal dotted line on the main plot shows the MIPS
    \SI{24}{\um} point spread function FWHM of \(5.5\arcsec\).  The
    standard deviation (s.d.) of the position angle differences is
    shown on each inset histogram.}
  \label{fig:mipsgal-examples}
\end{figure*}


The most extensive observational sample of stellar bow shock nebulae
to date is a catalog of 709 arcs \citep{Kobulnicky:2016a} detected in
mid-infrared surveys of the Galactic Plane by the \textit{Spitzer
  Space Telescope} (\textit{SST}, \citealp{Werner:2004a}) and
\textit{Wide-field Infrared Survey Explorer} (\textit{WISE},
\citealp{Wright:2010a}).  These sources are believed to be powered by
the winds of early-type stars, which are either moving supersonically
through the interstellar medium (runaway stars,
\citealp{Gvaramadze:2008a}), or are interacting with a local bulk
flow, such as the champagne flow from a nearby \hii{} region (weather
vanes, \citealp{Povich:2008a}).

In order to study the shapes of these bow shocks, we downloaded data
from the NASA/IPAC Infrared Science Archive archive\footnote{
  \url{http://irsa.ipac.caltech.edu/docs/program_interface/api_images.html}}
and extracted 4\arcmin{} square images in the \SI{24}{\um} bandpass of
the Multiband Imaging Photometer for \textit{Spitzer} (MIPS) centered
on each of the 471 \citet{Kobulnicky:2016a} sources that are covered
by the MIPSGAL \citep{Carey:2009a} survey, which includes most of the
sources with Galactic longitude within \(\pm 60\degr\) of the Galactic
center.

We developed a methodology for automatically tracing the arcs as follows:
\begin{enumerate}[1.]
\item Calculate arrays of celestial coordinates, \(C\), for each pixel
  of the image.
\item Using the central source coordinates, \(C_0\) and nominal
  bowshock radius, \(R_0\) from \citet{Kobulnicky:2016a}, construct a
  pixel mask that includes only those pixels with separations from the
  source that satisfy \(\frac12 R_0 \le |C - C_0| \le 3 R_0\).  This mask
  will be used for all subsequent operations, which serves to help
  avoid confusion from the star itself and other bright sources in the
  field of view.
\item Define a ``step-back'' point, \(C_1\), which is located at a
  separation \(2 R_0\) from the source, but in the opposite direction
  from the apex of the bow shock. That is, along a position angle
  180\degr{} from the nominal position angle, \(\text{PA}_0\), of the
  bow shock axis.  This point is at one end of the orange line shown
  superimposed on the bow shock images in
  Figure~\ref{fig:mipsgal-examples}.
\item Looping over a grid of 50 position angles, \(\text{PA}_k\),
  within \(\pm 60\degr\) of \(\text{PA}_0\), estimate the location of
  the arc along rays cast from the step-back point, using two
  different methods:
  \begin{enumerate}[(a)]
  \item The pixel with the peak brightness, with coordinates
    \(C_{k,\text{peak}}\) (red dots in
    Fig.~\ref{fig:mipsgal-examples}).
  \item The mean brightness-weighted separation from \(C_1\), with
    coordinates \(C_{k,\text{mean}}\) (light blue dots in
    Fig.~\ref{fig:mipsgal-examples}).
  \end{enumerate}
  For each \(\text{PA}_k\) in the grid, the calculation is performed
  over only those pixels that satisfy
  \(|\text{PA}(C, C_1) - \text{PA}_k| < \frac12 \delta\theta\), where
  \(\delta\theta = 120/50 = 2.4\degr\), which defines a thin radial wedge from
  \(C_1\).  The results are shown as red and blue dots superimposed
  on the images in Figure~\ref{fig:mipsgal-examples}. Each of the two
  methods, ``peak'' and ``mean'', works better in some objects and
  worse in others (according to the subjective judgment of
  ``correctly'' tracing the bow shock shape).  We therefore take the
  average by amalgamating all the \(C_{k,\text{peak}}\) and
  \(C_{k,\text{mean}}\) points into a single set, \(C_{k}\), for
  the following steps.
\item For each of the points \(C_{k}\), determine the radial
  separation from the central source, \(R_k = |C_k - C_0|\) and the
  angle from the bow shock axis about the central source
  \(\theta_k = \text{PA}(C_k, C_0) - \text{PA}_0\).  These are plotted in
  the upper left panels of Figure~\ref{fig:mipsgal-examples}.  Note
  that, even though the rays are cast from the step-back point \(C_1\)
  within \(\pm 60\degr\) of \(\text{PA}_0\), the angles \(\theta_k\) are
  measured from the source, \(C_0\), which is closer to the bow shock
  than \(C_1\) and therefore \(|\theta_k|\) can be much larger than
  \(60\degr\).
\item Make our own estimate of the axial size, \(R_0\), of the bow
  shock by calculating the mean and standard deviation of \(R_k\) over
  all points \(C_k\) with \(|\theta_k| \le 10\degr\).  Note that this is
  distinct from the nominal value of \(R_0\) given in the
  \citet{Kobulnicky:2016a} catalog, which was ``measured by eye''.
\item Estimate the radius of curvature, \(R_c\), by fitting a circle
  to all those points within \(\pm 45\degr\) of the nominal axis
  (\(|\theta_k| < 45\degr\)), but after excluding any point with
  \(R_k < \frac12 R_m\) or \(R_k > 2 R_m\), where \(R_m\) is the median
  \(R_k\) for \(|\theta_k| < 45\degr\).
\item Determine two separate estimates, \(R_{90+}\) and \(R_{90-}\),
  of the perpendicular radius, \(R_{90}\), by taking the mean and
  standard deviation of \(R_k\) over all points \(C_k\) with
  \(|\theta_k - 90\degr| \le 10\degr\) for \(R_{90+}\), and with
  \(|\theta_k + 90\degr| \le 10\degr\) for \(R_{90-}\).
\end{enumerate}

After these automatic steps, we subjectively evaluate the results by giving a star rating to each source:

\paragraph*{0 stars} The fitting algorithm failed for some reason. 

\paragraph*{1 star} The fit was formally successful, but the results
for \(R_c\) or \(R_{90}\) are far removed from what a human would
predict by looking at the image.  For example, in the smallest
bowshocks, which are only marginally resolved by Spitzer's 6\arcsec{}
beam, the dispersion in \(R_k\) can be a significant fraction of
\(R_0\), in which case our algorithm tends to erroneously favor
\(R_c < R_0\).

\paragraph*{2 stars} The fit results are not totally outlandish, but
nonetheless some problem is apparent that casts doubt on their
reliability.  For example, a double-shell structure to the bow shock
that leads to large differences between the ``peak'' and ``mean''
methods, or point sources near to the bow shock that interfere with
the tracing procedure.
  
\paragraph*{3 stars} A good fit, but where the dispersion in \(R_k\)
and/or the asymmetry in the bow shock reduces the precision in the
determination of \(R_c\) and \(R_{90}\), giving subjectively estimated
uncertainties around the 20\% level.  An example of a 3-star fit is
shown in Figure~\ref{fig:mipsgal-examples}a.

\paragraph*{4 stars} A high quality fit, with subjectively estimated
uncertainties in \(R_c\) and \(R_{90}\) around the 10\% level. An
example of a 4-star fit is shown in
Figure~\ref{fig:mipsgal-examples}b.

\paragraph*{5 stars} The highest-quality fit, usually corresponding to
large, sharply defined bow shocks, whose shape is determined with high
precision. An example of a 5-star fit is shown in
Figure~\ref{fig:mipsgal-examples}c.

\bigskip
%
Figure~\ref{fig:mipsgal-examples}d compares the bow shock size,
\(R_0\), determined by our fits (vertical axis) with the corresponding
value given in the \citet{Kobulnicky:2016a} catalog (horizontal axis).
For most sources with 3-star or higher rating, the two estimates agree
to within \(\pm 20\%\), but there are a small number of sources with a
discrepancy of more than a factor of two.  In all cases that we
checked, we believe that our estimates of \(R_0\) are more accurate
than those in the catalog.  It is apparent that the star ratings are
correlated with the bow shock size, with larger bow shocks tending to
receive higher ratings, although there is considerable overlap.  In
particular, most of the 1- and 2-star sources are close to the
resolution limit of the MIPSGAL \SI{24}{\um} images (\(6\arcsec\),
indicated by the dotted horizontal line in te figure).

\begin{figure*}
  \centering
  \begin{tabular}{ll}
    (a) & (b) \\
    \includegraphics[width=0.45\textwidth, trim=0 0 30 0]
    {figs/mipsgal-Rc-R90-zoom-annotated}
        & \includegraphics[width=0.45\textwidth, trim=0 0 30 0]
          {figs/mipsgal-Rc-R90-thumbnails} 
  \end{tabular}
  \caption[]{MIPSGAL sources on the bow shock shape diagnostic diagram
    of dimensionless radius of curvature versus perpendicular radius.
    The regions corresponding to different classes of cuadrics are
    shown by shading (see \S~\ref{sec:conic}): oblate spheroids (light
    gray background); prolate spheroids (darker gray background);
    paraboloids (curved black line); hyperboloids (white background).
    (a) Individual sources with bow shock fit quality rating of 3-star
    or above.  All 5-star sources plus those 4-star sources with
    \(R_c/R_0 < 1\) are labelled with their \cite{Kobulnicky:2016a}
    catalog number.  Horizontal error bars do not directly reflect the
    uncertainty in \(R_c/R_0\) but are instead simply the standard
    deviation from the circle fit of bowshock points \(R_k\) within
    \(\pm 45\degr\) of the axis.  Values on the vertical axis
    represent the average of \(R_{90+}\) and \(R_{90-}\), with thin
    vertical error bars showing the difference between \(R_{90+}\) and
    \(R_{90-}\), and thick vertical error bars showing the rms
    dispersion of \(R_k\) about these values for bow shock points
    within \(\pm 10\degr\) of the \(+90\degr\) and \(-90\degr\)
    directions.  (b) Kernel density estimator (KDE) of the
    distribution for 3-star sources (blue, filled contours) and 4-
    plus 5-star sources (orange/brown, unfilled contours).  The KDE
    uses an anisotropic gaussian kernel with bandwidths of
    \(0.18 \times 0.12\). Images of representative 4- and 5-star sources at
    different points on the \(R_c\)--\(R_{90}\) plane are also shown.}
  \label{fig:mipsgal-shapes}
\end{figure*}

In the following analysis, only those sources with a 3-star or higher
rating are used.  These comprise approximately half (227 out of 471)
of all the MIPSGAL arc sources.  In some cases of poor and failed
fits, there is nothing apparently ``wrong'' with the source itself,
and it is likely that minor tweaks to the methodology would improve
matters, but we have elected not to do so, in order to maintain a
uniform methodology across all sources.

The inset of Figure~\ref{fig:mipsgal-examples}d shows histograms of
the difference between the position angle, \(\text{PA}_0\) determined
by our fits and that listed in \citet{Kobulnicky:2016a}.  Although
observational uncertainties undoubtedly contribute in part, the
differences are mainly due to real asymmetries in the bow shocks,
especially for the 4- and 5-star sources.  The
\citeauthor{Kobulnicky:2016a} catalog \(\text{PA}_0\) values are
mostly sensitive to the orientation of the bow shock wings, whereas
our fitted \(\text{PA}_0\) values are determined by the point in the
bow shock head that is closest to the stellar source.  For this
reason, we use the catalog \(\text{PA}_0\) values for defining the
axis when measuring \(R_{90+}\) and \(R_{90-}\). On the other hand,
the fitted values of \(\text{PA}_0\) are better correlated with the
position of the bow shock's brightness peak, as is apparent in the
lower left panels of Figure~\ref{fig:mipsgal-examples}a and c.

The derived bowshock shapes of all the 3-, 4-, and 5-star sources are
shown in Figure~\ref{fig:mipsgal-shapes} on the plane of \(R_c/R_0\)
versus \(R_{90} / R_0\).  Panel a shows the individual points
superimposed on the theoretical results for quadrics of revolution
(see figure caption and \S~\ref{sec:conic}), while panel b shows
contours of the kernel density estimator (KDE, see
\citealp{Leiva-Murillo:2012a, Scott:2015a}) of the two-dimensional
distribution of points on the \(R_c\)--\(R_{90}\) plane.  The
horizontal axis corresponds to the shape of the head of the bow shock
near its apex, ranging from sharper, pointier shapes with
\(R_c/R_0 < 1\) to flatter, snubber shapes with \(R_c/R_0 \gg 1\), where
it must be understood that all judgments of sharpness/flatness are
with respect to the axial separation, \(R_0\), between the source and
the bow shock apex.  The vertical axis corresponds to the shape of the
bow shock wings, ranging from closed ``C'' shapes for smaller values
of \(R_{90} / R_0\) to open ``V'' shapes for larger values of
\(R_{90} / R_0\).  The boundary between closed and open corresponds to
the paraboloids, and is shown by the solid curved line that divides
the shaded from the unshaded regions of the graph.

The KDE contours in Figure~\ref{fig:mipsgal-shapes}b show that the
distribution of 3-star sources is very similar to that of 4- and
5-star sources, although the higher-rated sources are shifted slightly
to the upper right.  This is probably due to higher-rated sources
being on average bigger, as will be discussed further below.  The bulk
of the sources are concentrated around the paraboloid line, with
\(1 < R_c/R_0 < 2.5\), and \(1.2 < R_{90}/R_0 < 2\).  But significant
minorities are found in three other regions: (1)~a clump with
\(R_c/R_0 \la 1\); (2)~a vertical spur towards higher \(R_{90}\) at
\(R_c/R_0 \approx 2\); and (3)~a broad horizontal tail towards higher
\(R_c\) at \(R_{90}/R_0 \la 2\).

\begin{figure*}
  \centering
  \includegraphics[width=0.8\textwidth]{figs/mipsgal-pairplot}
  \caption[]{Matrix of pair plots that illustrate distributions of and
    correlations between the non-shape parameters of all MIPSGAL bow
    shock sources from \citet{Kobulnicky:2016a}.  Plots on the leading
    diagonal show histograms of the following parameters: bow shock
    angular size, \(\log_{10} R_0\); Galactic latitude,
    \(\log_{10}|b|\); Galactic longitude, \(\cos \ell\);
    extinction-corrected \(H\)-band magnitud of the stellar source,
    \(H_0\); \(K\)-band extinction, \(A_K\).  Scatter plots in the
    upper triangle show the joint distribution of each pair of
    parameters.  These are repeated in the lower triangle but showing
    the KDEs of the joint distributions, which are annotated with the
    Pearson linear correlation coefficient, \(r\), for each pair. The
    straight lines shown superimposed on the plots of stellar magnitud
    versus bowshock size correspond to model results for the same star
    at a sequence of distances (dashed lines) and a sequence of
    stellar luminosities at a fixed distance (dotted lines).  See text
    for details. }
  \label{fig:mipsgal-pairplot}
\end{figure*}



\begin{figure*}
  \centering
  \begin{tabular}{ll}
    (a) & (b) \\
    \includegraphics[width=0.45\textwidth]{figs/mipsgal-Rc-R90-R0} &
    \includegraphics[width=0.45\textwidth]{figs/mipsgal-Rc-R90-Mag} 
  \end{tabular}
  \caption[]{Comparison between the distribution of bowshock shapes
    when the sources are divided into two sub-samples according to the
    value of another parameter. (a)~Bow shock angular size, \(R_0\).
    (b)~Extinction-corrected \(H\)-band magnitude of the stellar
    source.}
  \label{fig:mipsgal-correlated}
\end{figure*}

\begin{figure*}
  \centering
  \begin{tabular}{ll}
    (a) & (b) \\
    \includegraphics[width=0.45\textwidth]{figs/mipsgal-Rc-R90-environment} &
    \includegraphics[width=0.45\textwidth]{figs/mipsgal-Rc-R90-candidates} 
  \end{tabular}
  \caption[]{Lack of significant correlation of bow shock shape with
    (a) environment, and (b) uncertainty in stellar source
    identification.}
  \label{fig:mipsgal-uncorrelated}
\end{figure*}


\subsection{Far-infrared arcs around late-type stars}
\label{sec:far-infrared-arcs}

\begin{figure}
  \centering
  \includegraphics[width=\linewidth]{figs/mipsgal-Rc-R90-vs-Herschel}
  \caption[]{Comparison of RSG/AGB arcs with OB star arcs.}
  \label{fig:herschel-compare-mipsgal}
\end{figure}



\subsection{Stationary emission line arcs in M42}
\label{sec:stat-emiss-line}

\begin{figure*}
  \centering
  \includegraphics[width=\textwidth]{figs/annotated-ll-arcs}
  \caption[]{Stationary bow shock arcs in the Orion Nebula.}
  \label{fig:ll-arcs}
\end{figure*}

\begin{figure}
  \centering
  \includegraphics[width=\linewidth]{figs/mipsgal-Rc-R90-vs-Orion}
  \caption[]{Comparison of Orion with OB stars.}
  \label{fig:ll-compare-mipsgal}
\end{figure}




%%% Local Variables:
%%% mode: latex
%%% TeX-master: "quadrics-bowshock.tex"
%%% End:


\section{Discussion}
\label{sec:discussion}

In this section, we discuss the physical implications of our empirical
findings regarding bow shock shapes.  Our most reliable result is the
average shape of the OB bow shocks from the 227 MIPSGAL sources with
quality rating of 3~stars or higher.  This yields mean values of
\(\Pi = 1.78 \pm 0.06\) and \(\Lambda = 1.72 \pm 0.02\), or median values of
\(\Pi = 1.57\) and \(\Lambda = 1.69\).  The uncertainty quoted on the mean
values is the ``standard error of the mean'':
\(\text{sem} = \sigma / \sqrt{n}\), where \(\sigma\) is the rms dispersion and
\(n\) is the number of sources.  Note that in the case of the
planitude \(\text{sem}(\Pi) = 0.06\) is considerably smaller than
\(\text{mean}(\Pi) - \text{median}(\Pi) = 0.21\), so the latter would be a
more conservative estimate of the uncertainty.\footnote{This is
  because the distribution of \(\Pi\) is approximately log-normal, which
  yields a significant tail towards high values when converted to
  linear space.}  These values can be compared with the predictions of
the thin-shell wilkinoid model \citep{Wilkin:1996a}, which are
\(\Pi = 1.67\), \(\Lambda = 1.73\) when the bow shock axis lies in the plane
of the sky (following Paper~0, this is defined as the zero point of
the inclination angle, \(i\)).  When the axis is inclined, both
planitude and alatude are predicted to decrease but not by very much,
tending to \(\Pi = 1.5\), \(\Lambda = 1.63\) as
\(\abs{i} \to \ang{90}\) (see \S~5.3 of Paper~0).  The median observed
value\footnote{In the presence of outliers, the median is a more
  robust estimate of the central tendency than is the mean.} falls
squarely inside this range for both the planitude and alatude, which
is a remarkable triumph for the \citet{Wilkin:1996a} model.

On the other hand, turning now to the \emph{variety} of bow shock
shapes, we see that the wilkinoid can no longer explain our results.
The rms dispersions of the planitude and alatude distributions for the
MIPSGAL sources are \(\sigma(\Pi) = 0.87\) and
\(\sigma(\Lambda) = 0.30\) (Tab.~\ref{tab:big-p}), which are respectively 5 times
and 3 times larger than the total range of variation of \(\Pi\) and
\(\Lambda\) predicted for the wilkinoid surface.  Although some of the
dispersion is due to uncertainties in the observations and the fitting
algorithm, this contribution is expected to be small, especially for
the larger, well-resolved sources, for which systematic uncertainties
in the methods for determining \(\Pi\) and \(\Lambda\) will
dominate. Conservative upper limits to the relative size of these
uncertainties were estimated in \S~7 of Paper~0 to be \(< 20\%\) for
\(\Pi\) and \(< 10\%\) for \(\Lambda\), whereas the observed dispersions are
roughly twice as large: \(\sigma(\Pi)/\Pi = 55\%\) and
\(\sigma(\Lambda)/\Lambda = 18\%\).  Furthermore, the variations in planitude and
alatude are readily apparent by eye, as is demonstrated by the example
bow shock images shown in Figure~\ref{fig:mipsgal-shapes}b.  Sources
such as K123 have very typical shapes and fall near the center of the
\(\Pi\)--\(\Lambda\) distribution, whereas high-\(\Pi\) sources such as K447 have
a very flat apex region, while low-\(\Pi\) sources such as K566 have a
pointier, almost triangular apex.  High-\(\Lambda\) sources, such as K517,
have very open wings that bend away from the star, while
low-\(\Lambda\) sources such as K489 have closed wings and a semi-circular
appearance.


In Paper~0 we found that certain bow shock shapes can show a much
greater variation in their projected appearance as a function of
inclination angle than is seen for the wilkinoid.  For example, the
cantoids and ancantoids, which have asymptotically hyperbolic far
wings, can shift towards higher apparent planitude and alatude as the
inclination increases, generally with \(\Lambda \ge \Pi\) (Fig.~20 of Paper~0).
This might plausibly explain the vertical spur towards higher
\(\Lambda\) seen in the empirical distribution (\S~\ref{sec:ob-shapes}). A
different behavior is shown by bow shocks with very flat apex regions,
such as the MHD simulation from \citet{Meyer:2017a} that is analyzed
in the \S~6 of Paper~0.  This shows a high planitude \(\Pi\) when the
orientation is exactly edge-on, but \(\Pi\) decreases sharply along a
roughly horizontal track as the inclination \(\abs{i}\) increases
(Fig.~25 of Paper~0).  This is similar to the principal axis of
variation of the observed shapes (e.g.,
Fig.~\ref{fig:mipsgal-shapes}a).

If such variations in orientation do make a significant contribution
to the observed distribution of bow shock shapes in the
\(\Pi\)-\(\Lambda\) plane, then various predictions follow, which might be
observationally tested.  High-planitude sources with \(\Pi > 3\) would
be expected to have low inclinations, \(\abs{i} < \ang{30}\), whereas
high-alatude sources with \(\Lambda > 2\) would be expected to have high
inclinations, \(\abs{i} > \ang{45}\).  Unfortunately, determination of
the inclination for individual sources requires high resolution
spectroscopy of emission lines in order to map the kinematics of the
flow in the bow shock shell \citep[e.g.,][]{Henney:2013a}.  This is
not currently available for the majority of the MIPSGAL sources, which
are detected only by their dust continuum emission.  A further
prediction for the high-alatude sources is that the environmental flow
should be divergent rather than plane-parallel, in order to give a
cantoid shape instead of a wilkinoid.  This would tend to favor
``weather-vane'' cases, where the interstellar medium is flowing past
the star, and disfavor ``runaway'' cases, where the star is moving
through a static medium.  However, in \S~\ref{sec:corr-shape} we found
no significant difference in the shape distributions as a function of
the bow shock environment.  Figure~\ref{fig:mipsgal-uncorrelated}a
shows that the alatudes of sources that are facing \hii{} regions or
\SI{8}{\um} bright-rimmed clouds (and therefore might be expected to
be immersed in a champagne flow) are no higher than sources that are
isolated.


\begin{figure}
  (a)\\
  \includegraphics[width=\linewidth]
  {figs/wave-R90-vs-Rc-A020-N10}
  (b)\\
  \includegraphics[width=\linewidth]
  {figs/wave-R90-vs-Rc-A010-N20}
  \caption{Diagnostic diagram for perturbed shapes from standing wave
    oscillations.  Each model is characterized by a base shape
    (colored symbols, as described in key) and an amplitude, \(A\),
    and wavenumber, \(N\), of the oscillation: (a)~breathing mode with
    \(N = 1\), \(A = 0.2\); (b)~curling mode with \(N = 2\),
    \(A = 0.1\) (see Fig.~\ref{fig:perturb-shapes} for the
    phase-dependent intrinsic shapes).  The plotted points show the
    varying planitude and alatude of the projected bow shock shapes
    with uniform sampling over an entire period of the oscillation and
    for varying inclinations (sampled according to an isotropic
    distribution of orientations).  Each individual point is plotted
    with a low opacity so that the crowding of points in certain
    regions of the plane can be appreciated.}
  \label{fig:perturb-Rc-R90}
\end{figure}

An alternative explanation for the variety of observed bow shock
shapes is that they are due to time-dependent perturbations to an
underlying base shape.  For instance, multiple studies have shown that
stellar bow shock shells can be unstable \citep{Dgani:1996a,
  Dgani:1996b, Blondin:1998a, Comeron:1998a, Meyer:2014a}, leading to
large amplitude oscillations in the shell shape.  The oscillations are
found to be most vigorous when the post shock cooling is highly
efficient, allowing the formation of a thin shell (see Paper~I).  Even
in cases where the shell is stable, oscillations may be driven by
periodic variations in the stellar wind mass-loss rate or velocity, or
by inhomogeneities in the ambient stream.  Rather than using a
particular dynamical model of these oscillations, we instead crudely
simulate their effect by assuming a constant amplitude standing wave
perturbation to the base shape, as described in
Appendix~\ref{sec:perturbed-bows}.  Example results are shown in
Figure~\ref{fig:perturb-Rc-R90} for an ensemble of bows with different
orientations and phases of oscillation, considering three different
underlying base shapes.  It can be seen that modest amplitudes of 10
to 20\% can give rise to a distribution in \(\Pi\) and \(\Lambda\) similar to
that observed for the MIPSGAL sources when the oscillation wavelength
is of same order as the bow shock size.

An attractive feature of the oscillation hypothesis is that it
naturally explains why we find little correlation between the bow
shock shape and other source parameters (\S~\ref{sec:corr-shape}).
The one significant correlation that we do find is that the alatude
distribution is broader for bow shocks with larger angular sizes.
This might be explained if the relative amplitude of oscillations were
higher for sources with more powerful winds.  However, in that case it
is hard to understand why no such variation is seen in the width of
the planitude distribution.

We now address the difference in shape distribution between the
different classes of source.  Compared with the OB star bow shocks,
the cool star sample from \citet{Cox:2012a} shows a significantly
smaller alatude of \(\Lambda = 1.41 \pm 0.03\) (see
Fig.~\ref{fig:herschel-compare-mipsgal}).\footnote{Although the
  planitude also appears to be slightly smaller, this is not very
  statistically significant due to the small number of cool star
  sources and the large width of the OB star planitude distribution.}
Such a closed shape for the wings is inconsistent with the wilkinoid
value of \(\Lambda = \text{\numrange{1.63}{1.73}}\), which is surprising
given that the emission shells in these sources are relatively narrow
(Figs.~\ref{fig:herschel-arc-fits}
and~\ref{fig:herschel-arc-fits-poor}), so one might have thought that
the thin-shell approximation of \citet{Wilkin:1996a} would be
\emph{more} appropriate than for the OB stars, but this is clearly not
the case.  One possible explanation for this 



%%% Local Variables:
%%% mode: latex
%%% TeX-master: "obs-bowshocks"
%%% End:


\section{Conclusions}
\label{sec:conclusion}

Difference in SEDs between FIR and MIR samples?  \citet{Meyer:2016a}
say emission peaks at 3--50 micron for O star bow shocks.

Al other things being equal, larger bows will be at higher
inclinations (we should estimate how much variation in size we should
get due to inclination -- it will be more for flatter bows).  Higher
inclinations means less variation from the standing wave oscillations,
which goes against our result that larger bows have greater dispersion
in \(\Lambda'\).  Although the standing waves mainly give dispersion in
\(\Pi'\) anyhow.

%%% Local Variables:
%%% mode: latex
%%% TeX-master: "obs-bowshocks"
%%% End:


\section*{Acknowledgements}
We are grateful for financial support provided by Dirección General de
Asuntos del Personal Académico, Universidad Nacional Autónoma de
México, through grants Programa de Apoyo a Proyectos de Investigación
e Inovación Tecnológica IN111215 and IN107019.  

\bibliographystyle{mnras}
%% All references should be put in the BibTeX file bowshocks-biblio.bib
\bibliography{bowshocks-biblio}
\appendix
\clearpage

% Landscape table needs to go on its own page, so we need to manually
% adjust the section counter so that it will ``belong'' to the next
% section
\addtocounter{section}{1}
\sisetup{detect-all=true, detect-inline-weight=math}
\sisetup{round-mode=figures, round-precision=2}
\begin{landscape}
\begin{table}
  \setlength\tabcolsep{3pt}
  \caption{Results of all statistical tests performed on observed bow shock shape parameters. Significant correlations are shown in \textbf{bold}, marginally significant correlations in \textit{italic}}
  \label{tab:big-p}
  %%% MIPSGAL summary statistics
%%% Table automatically generated from mipsgal-summary-stats.tab
%%% 2020-02-05 22:21:40.747825
%%%
% \Width is used to align number under col header in first column
	\newlength\Width\settowidth\Width{Comparison}
	\begin{tabular}{
	  @{} ll @{\quad }
	  % Mean
	  S[round-mode=places]S[round-mode=places]
	  % Std Dev
	  S[round-mode=places]S[round-mode=places]
	  % Obs Disp
	  SS
	  % Effect sizes
	  @{\quad} SSS[round-mode=places]
	  % p values
	  @{\quad}
	  S[table-format = +1.2e1]
	  S[table-format = +1.2e1]
	  S[table-format = +1.2e1] @{}
	  }\toprule
	  & {Dependent}
	  & \multicolumn{2}{c}{Mean}
	  & \multicolumn{2}{c}{Std.\ Dev.}
	  & \multicolumn{2}{c}{Obs.\ Disp.}
	  & \multicolumn{3}{c @{} }{\dotfill Effect sizes\dotfill }
	  & \multicolumn{3}{c @{}}{Non-parametric test \(p\)-values} \\ 
	  {Comparison} & {Variable}
	  & {\(\langle \text{A} \rangle\)} & {\(\langle \text{B} \rangle\)}
	  & {\(\sigma_{\text{A}}\)} & {\(\sigma_{\text{B}}\)}
	  & {\(\langle \epsilon_{\text{A}} \rangle\)} & {\(\langle \epsilon_{\text{B}} \rangle\)}
	  & {\(r_b\)} & {Cohen \(d\)} & {\(\sigma_{\text{A}}/\sigma_{\text{B}}\)}
	  & {K--S} & {Rank} &  {B--F}\\
	  {\makebox[\Width]{(1)}} & \multicolumn{1}{c@{\quad}}{(2)}
	  & {(3)} & {(4)}
	  & {(5)} & {(6)}
	  & {(7)} & {(8)}
	  & {(9)} & {(10)} & {(11)}
	  & {(12)} & {(13)} & {(14)}  \\  
	  \midrule\multicolumn{10}{@{} l @{}}{\itshape Median split of continuous independent variables}\\
\addlinespace
Faint/bright & \(\Pi\) & 1.701 & 1.872 & 0.605 & 0.806 & 0.101 & 0.084 & 0.103 & 0.24 & 1.333 & 0.379 & 0.182 & 0.0588\\
\(H\) magnitude & \(\Lambda\) & 1.655 & 1.724 & 0.255 & 0.322 & 0.238 & 0.232 & 0.147 & 0.237 & 1.262 & \itshape 0.018 & 0.057 & 0.0884\\
\(n_{\text{A}} =  n_{\text{B}} = 113\) & \(\Delta \Lambda\) & 0.176 & 0.196 & 0.152 & 0.155 &   &   & 0.087 & 0.13 & 1.019 & 0.921 & 0.261 & 0.833\\
\addlinespace
Low/high & \(\Pi\) & 1.696 & 1.874 & 0.668 & 0.75 & 0.115 & 0.07 & \itshape 0.171 & \itshape 0.251 & 1.123 & 0.431 & \itshape 0.0261 & 0.66\\
bow shock size, \(R_0\) & \(\Lambda\) & 1.67 & 1.71 & 0.266 & 0.314 & 0.251 & 0.22 & 0.103 & 0.135 & 1.18 & \itshape 0.0351 & 0.18 & 0.184\\
\(n_{\text{A}} =  n_{\text{B}} = 113\) & \(\Delta \Lambda\) & 0.174 & 0.197 & 0.148 & 0.159 &   &   & 0.049 & 0.148 & 1.073 & 0.345 & 0.521 & 0.201\\
\addlinespace
Low/high & \(\Pi\) & 1.769 & 1.805 & 0.761 & 0.672 & 0.096 & 0.088 & 0.067 & 0.05 & 0.883 & 0.771 & 0.388 & 0.477\\
extinction, \(A_K\) & \(\Lambda\) & 1.681 & 1.697 & 0.269 & 0.314 & 0.234 & 0.236 & 0.021 & 0.055 & 1.168 & 0.771 & 0.789 & 0.415\\
\(n_{\text{A}} =  n_{\text{B}} = 113\) & \(\Delta \Lambda\) & 0.189 & 0.184 & 0.139 & 0.168 &   &   & -0.069 & -0.032 & 1.207 & 0.572 & 0.373 & 0.306\\
\addlinespace
Low/high & \(\Pi\) & 1.769 & 1.804 & 0.713 & 0.719 & 0.089 & 0.095 & 0.01 & 0.049 & 1.009 & 0.974 & 0.896 & 0.771\\
\(\vert{}b\vert\) & \(\Lambda\) & 1.69 & 1.691 & 0.315 & 0.266 & 0.242 & 0.228 & 0.015 & 0.004 & 0.844 & 0.653 & 0.85 & 0.534\\
\(n_{\text{A}} =  n_{\text{B}} = 113\) & \(\Delta \Lambda\) & 0.182 & 0.19 & 0.151 & 0.157 &   &   & 0.045 & 0.05 & 1.04 & 0.415 & 0.556 & 0.683\\
\addlinespace
High/low & \(\Pi\) & 1.797 & 1.771 & 0.714 & 0.719 & 0.094 & 0.089 & -0.027 & -0.035 & 1.007 & 0.724 & 0.728 & 0.999\\
\(\cos \ell\) & \(\Lambda\) & 1.7 & 1.676 & 0.284 & 0.302 & 0.236 & 0.234 & -0.021 & -0.081 & 1.063 & 0.973 & 0.792 & 0.937\\
\(n_{\text{A}}, n_{\text{B}} = 137, 90\) & \(\Delta \Lambda\) & 0.174 & 0.204 & 0.146 & 0.164 &   &   & 0.088 & 0.197 & 1.124 & 0.871 & 0.262 & 0.327\\
\midrule
\multicolumn{10}{@{} l @{}}{\itshape Categorical independent variables}\\
\addlinespace
Environment: & \(\Pi\) & 1.802 & 1.784 & 0.705 & 0.738 & 0.093 & 0.085 & -0.035 & -0.026 & 1.046 & 0.259 & 0.727 & 0.671\\
Isolated vs Facing & \(\Lambda\) & 1.699 & 1.674 & 0.281 & 0.304 & 0.239 & 0.223 & -0.078 & -0.088 & 1.081 & 0.883 & 0.439 & 0.919\\
\(n_{\text{A}}, n_{\text{B}} = 170, 41\) & \(\Delta \Lambda\) & 0.182 & 0.201 & 0.153 & 0.16 &   &   & 0.061 & 0.12 & 1.042 & 0.69 & 0.549 & 0.549\\
\addlinespace
Environment: & \(\Pi\) & 1.802 & 1.62 & 0.705 & 0.756 & 0.093 & 0.105 & -0.265 & -0.257 & 1.072 & \itshape 0.0201 & 0.0804 & 0.422\\
Isolated vs \hii & \(\Lambda\) & 1.699 & 1.636 & 0.281 & 0.36 & 0.239 & 0.22 & -0.224 & -0.216 & 1.279 & 0.11 & 0.14 & 0.0601\\
\(n_{\text{A}}, n_{\text{B}} = 170, 16\) & \(\Delta \Lambda\) & 0.182 & 0.185 & 0.153 & 0.147 &   &   & 0.067 & 0.02 & 0.963 & 0.465 & 0.658 & 0.323\\
\addlinespace
Single/multiple & \(\Pi\) & 1.738 & 1.919 & 0.668 & 0.82 & 0.094 & 0.087 & 0.134 & 0.254 & 1.228 & 0.779 & 0.125 & 0.258\\
source candidate & \(\Lambda\) & 1.678 & 1.723 & 0.291 & 0.292 & 0.234 & 0.239 & 0.099 & 0.153 & 1.005 & 0.554 & 0.257 & 0.613\\
\(n_{\text{A}}, n_{\text{B}} = 167, 60\) & \(\Delta \Lambda\) & 0.182 & 0.197 & 0.157 & 0.145 &   &   & 0.067 & 0.097 & 0.927 & 0.544 & 0.439 & 0.379\\
\addlinespace
With/without & \(\Pi\) & 1.749 & 1.796 & 0.545 & 0.753 & 0.099 & 0.09 & -0.039 & 0.065 & 1.382 & 0.882 & 0.688 & 0.233\\
\SI{8}{\um} emission & \(\Lambda\) & 1.687 & 1.691 & 0.298 & 0.29 & 0.222 & 0.238 & 0.012 & 0.014 & 0.973 & 0.91 & 0.9 & 0.89\\
\(n_{\text{A}}, n_{\text{B}} = 45, 182\) & \(\Delta \Lambda\) & 0.198 & 0.183 & 0.193 & 0.142 &   &   & 0.028 & -0.098 & 0.738 & 0.771 & 0.772 & 0.226\\
\addlinespace
3-star vs (4+5)-star & \(\Pi\) & 1.615 & 2.026 & 0.618 & 0.773 & 0.108 & 0.07 & \bfseries 0.39 & \bfseries 0.599 & 1.25 & \bfseries 0.000411 & \bfseries 5.64e-07 & 0.128\\
 & \(\Lambda\) & 1.621 & 1.787 & 0.276 & 0.286 & 0.244 & 0.223 & \bfseries 0.369 & \bfseries 0.592 & 1.036 & \bfseries 0.000183 & \bfseries 2.24e-06 & 0.852\\
\(n_{\text{A}}, n_{\text{B}} = 133, 94\) & \(\Delta \Lambda\) & 0.183 & 0.189 & 0.153 & 0.156 &   &   & -0.0 & 0.039 & 1.017 & 0.479 & 0.999 & 0.4\\
\midrule
\multicolumn{10}{@{} l @{}}{\itshape Intercomparison with other datasets}\\
\addlinespace
MIPS vs Orion & \(\Pi\) & 1.787 & 3.09 & 0.716 & 1.666 &   &   & \bfseries 0.557 & \bfseries 1.581 & \bfseries 2.325 & \bfseries 0.000456 & \bfseries 8.48e-05 & \bfseries 2.42e-05\\
 & \(\Lambda\) & 1.69 & 2.532 & 0.292 & 0.749 &   &   & \bfseries 0.735 & \bfseries 2.428 & \bfseries 2.568 & \bfseries 4.28e-06 & \bfseries 2.13e-07 & \bfseries 1.26e-08\\
\(n_{\text{A}}, n_{\text{B}} = 227, 18\) & \(\Delta \Lambda\) & 0.186 & 0.329 & 0.154 & 0.257 &   &   & \itshape 0.349 & \itshape 0.874 & \itshape 1.666 & \itshape 0.0185 & \itshape 0.0137 & \itshape 0.0199\\
\addlinespace
MIPS vs RSG & \(\Pi\) & 1.787 & 1.479 & 0.716 & 0.227 &   &   & -0.238 & -0.443 & \itshape 0.317 & \bfseries 0.0011 & 0.112 & \itshape 0.0108\\
 & \(\Lambda\) & 1.69 & 1.41 & 0.292 & 0.101 &   &   & \bfseries -0.648 & \bfseries -0.99 & \itshape 0.345 & \bfseries 2.19e-05 & \bfseries 1.5e-05 & \itshape 0.00549\\
\(n_{\text{A}}, n_{\text{B}} = 227, 16\) & \(\Delta \Lambda\) & 0.186 & 0.073 & 0.154 & 0.038 &   &   & \itshape -0.442 & \itshape -0.743 & \itshape 0.246 & \itshape 0.013 & \itshape 0.0181 & \itshape 0.00698
\\
\bottomrule
\multicolumn{14}{@{}p{\linewidth}@{}}{
  \textit{Description of columns:}
  (Col.~1)~How the two A/B source sub-samples are defined, also giving the size of each sub-sample, \(n_{\text{A}}\) and~\(n_{\text{B}}\).
  (Col.~2)~Dependent variable whose distribution is compared between the two sub-samples.
  (Cols.~3--6)~Mean and standard deviation, \(\sigma\), of the dependent variable for each of the two sub-samples.
  (Cols.~7--8)~Mean over each sub-sample of the observational dispersion (\(\epsilon\), standard deviation) of radii that contribute to the dependent variable for each individual source, as in steps~\ref{step:R0} and \ref{step:R90} of \S~\ref{sec:autom-trac-fitt}.  Note that in the case of \(\Pi\), this is \(\epsilon(R_0)\), and so is not a direct measure of the observational uncertainty in \(\Pi\). 
  % (Cols.~9--10)~``Standard error of the mean'' (s.e.m.) of the dependent variable for each of the two sub-samples. 
  (Cols.~9--11)~Standardized ``effect sizes'', which are dimensionless measures of the difference in the distribution of the dependent variable between the two sub-samples.
  (Col.~9)~Rank biserial correlation coefficient \citep{Cureton:1956a}, which is obtained by considering all \(n_{\text{A}} n_{\text{B}}\) pair-wise comparisons of the dependent variable between a source in sub-sample~A and a source in sub-sample~B.  It is the difference between the fraction of such comparisons ``won'' by sub-sample~A and those ``won'' by sub-sample~B, and thus may vary between \(-1\) and \(+1\). 
  (Col.~10)~Cohen's \(d\), which is a dimensionless mean difference: \(d = (\langle \text{A} \rangle - \langle \text{B} \rangle) / \sigma_{\text{pool}} \), where \(\sigma_{\text{pool}} = (n_{\text{A}} \sigma_{\text{A}}^2 + n_{\text{B}} \sigma_{\text{B}}^2)^{1/2} / \sqrt{n_{\text{A}} + n_{\text{B}}}\) is the pooled standard deviation.
  (Col.~11)~Ratio of standard deviations between the two sub-samples.
  (Cols.~12--14)~Probabilities (\(p\)-values) of the two sub-samples being as different as observed if they were to be drawn from the same population, according to three different non-parametric tests.
  (Col.~12)~Anderson--Darling 2-sample test, which is a general test of similarity between two distributions that is designed to retain sensitivity to differences in the tails of the distributions.
  (Col.~13)~Mann--Whitney--Wilcoxon \(U\) test \citep{Mann:1947a}, which is sensitive to differences in the central value of the distributions.
  (Col.~14)~Brown--Forsythe test for equality of variance \citep{Brown:1974a}
}
\end{tabular}

\end{table}
\end{landscape}
\begin{table}
  \contcaption{Results of all statistical tests performed on observed
    bow shock shape parameters.}
  \begin{tabular}{p{0.9\linewidth}}
    \toprule
    \textit{Description of columns:}
    (Col.~1)~How the two A/B source sub-samples are defined, also giving the size of each sub-sample, \(n_{\text{A}}\) and~\(n_{\text{B}}\).
    (Col.~2)~Dependent variable whose distribution is compared between the two sub-samples.
    (Cols.~3--6)~Mean and standard deviation, \(\sigma\), of the dependent variable for each of the two sub-samples.
    (Cols.~7--8)~Mean over each sub-sample of the observational dispersion (\(\epsilon\), standard deviation) of radii that contribute to the dependent variable for each individual source, as in steps~\ref{step:R0} and \ref{step:R90} of \S~\ref{sec:autom-trac-fitt}.  Note that in the case of \(R_c\), this is \(\epsilon(R_0)\), and so is not a direct measure of the observational uncertainty in \(R_c\). 
    (Cols.~9--10)~``Standard error of the mean'' (s.e.m.) of the dependent variable for each of the two sub-samples. 
    (Cols.~11--13)~Standardized ``effect sizes'', which are dimensionless measures of the difference in the distribution of the dependent variable between the two sub-samples.
    (Col.~11)~Rank biserial correlation coefficient \citep{Cureton:1956a}, which is obtained by considering all \(n_{\text{A}} n_{\text{B}}\) pair-wise comparisons of the dependent variable between a source in sub-sample~A and a source in sub-sample~B.  It is the difference between the fraction of such comparisons ``won'' by sub-sample~A and those ``won'' by sub-sample~B, and thus may vary between \(-1\) and \(+1\). 
    (Col.~12)~Cohen's \(d\), which is a dimensionless mean difference: \(d = (\langle \text{A} \rangle - \langle \text{B} \rangle) / \sigma_{\text{pool}} \), where \(\sigma_{\text{pool}} = (n_{\text{A}} \sigma_{\text{A}}^2 + n_{\text{B}} \sigma_{\text{B}}^2)^{1/2} / \sqrt{n_{\text{A}} + n_{\text{B}}}\) is the pooled standard deviation.
    (Col.~13)~Ratio of standard deviations between the two sub-samples.
    (Cols.~14--16)~Probabilities (\(p\)-values) of the two sub-samples being as different as observed if they were to be drawn from the same population, according to three different non-parametric tests.
    (Col.~14)~Anderson--Darling 2-sample test, which is a general test of similarity between two distributions that is designed to retain sensitivity to differences in the tails of the distributions.
    (Col.~15)~Mann--Whitney--Wilcoxon \(U\) test \citep{Mann:1947a}, which is sensitive 
    (Col.~16)~Brown--Forsythe test for equality of variance \citep{Brown:1974a}
    \\
    \bottomrule
  \end{tabular}
\end{table}

\addtocounter{section}{-1}

\section{Distribution of p-values for all correlations tested}
\label{sec:distr-p-values}
\addtocounter{table}{1}


\begin{figure}
  (a)\\
  \includegraphics[width=\linewidth]{figs/p-value-histogram-new-linear}\\
  (b)\\
  \includegraphics[width=\linewidth]{figs/p-value-histogram-new}
  \caption{Histogram of \(p\)-values for all non-parametric 2-sample
    tests listed in Table~\ref{tab:big-p}. (a)~Uniformly spaced linear
    bins and linear vertical axis. (b)~Uniformly spaced logarithmic
    bins and logarithmic vertical axis, with all values
    \(p \le 10^{-6}\) included in the leftmost bin.  Short thin vertical
    lines above the horizontal axis show the individual value.  The
    thick vertical dashed lines shows the traditional threshold values
    for significance: \(p = 0.003\) (\(\approx 3 \sigma\)) and
    \(p = 0.05\) (\(\approx 2 \sigma\)). The red solid line shows the uniform
    distribution of \(p\)-values that would be expected if the null
    hypothesis were always true, that is, if no significant
    correlations existed.}
  \label{fig:histo-p-values}
\end{figure}


Results from all the statistical tests discussed in \S~\ref{sec:corr-shape} are given in Table~\ref{tab:big-p}.  


\begin{table}
  \caption{Lower bounds on the Type~I error rate, \(\alpha\)}
  \label{tab:type-I}
  \centering
  \begin{tabular}{lrr} \toprule
    Correlation & \(p\) & \(\alpha\) \\
    \midrule
    \multicolumn{3}{l}{\itshape Quantitative \dotfill}\\
    \(R_c\) vs \(R_0\) & \textit{0.0054} & \textit{0.0712} \\
    \(R_{90}\) vs \(R_0\) & \textbf{0.0001} & \textbf{0.0025} \\
    \(R_c\) vs \(H_0\) & 0.1229 & 0.4119 \\
    \(R_{90}\) vs \(H_0\) & \textit{0.0215} & \textit{0.1833} \\
    \(R_c\) vs \(A_K\) & 0.19 & 0.4617 \\
    \(R_{90}\) vs \(A_K\) & 0.63 & 1 \\
    \(R_c\) vs \(|b|\) & 0.19 & 0.4617 \\
    \(R_{90}\) vs \(|b|\) & 0.31 & 0.4967 \\
    \(R_c\) vs \(\cos(\ell)\) & 1.00 & 1 \\
    \(R_{90}\) vs \(\cos(\ell)\) & 0.36 & 0.4999 \\
    \midrule
    \multicolumn{3}{l}{\itshape Categorical} \dotfill\\
    \(R_c\) vs Facing & 0.71 & 1 \\
    \(R_c\) vs H II & 0.50 & 1 \\
    \(R_{90}\) vs Facing & 0.60 & 1 \\
    \(R_{90}\) vs H II & 0.52 & 1 \\
    \(R_c\) vs Multiple & 1.00 & 1 \\
    \(R_{90}\) vs Multiple & 0.34 & 0.4993 \\
    \(R_c\) vs \SI{8}{\um} & 0.82 & 1 \\
    \(R_{90}\) vs \SI{8}{\um} & 0.6 & 1 \\
    \midrule
    \multicolumn{3}{l}{\itshape Other datasets} \dotfill\\
    \(R_c\) vs Herschel & 0.074 & 0.3437 \\
    \(R_{90}\) vs Herschel & \textbf{0.00052} & \textbf{0.0106} \\
    \(R_c\) vs M42 & \textbf{0.00048} & \textbf{0.0099} \\
    \(R_{90}\) vs M42 & \textbf{0.0000105} & \textbf{0.0003} \\
    \bottomrule
  \end{tabular}
\end{table}


The \(p\)-values are the probability of finding a difference between
two populations as large as (or larger than) what is observed,
\emph{given} that there is no difference in the underlying
distribution from which the two populations are drawn (that is, given
that the null hypothesis is true).  However, what we really want to
know is something else: the probability that the null hypothesis is
true, \emph{given} the observations.  That is, the probability,
\(\alpha\), of a \textit{false positive}, also known as the \textit{Type I
  error rate}.  The common mistake of conflating these two definitions
is known as the ``\(p\)-value fallacy'' \citep{Goodman:1999a}, or``the
error of the transposed conditional'', as discussed in detail by
\citet{Colquhoun:2014a}.  It is possible to derive \(\alpha\) from
\(p\) using Bayes' theorem (e.g., \citealp{Goodman:1999b}), but that
requires an estimate of the prior probability of the null hypothesis,
independent of the observations.  Alternatively, it is also possible
to find a lower bound on \(\alpha\) from a frequentist approach
\citep{Sellke:2001a}:
\begin{equation}
  \label{eq:type-I}
  \alpha(p) \ge \bigg[ 1 - \big(e\, p \ln p\big)^{-1} \bigg]^{-1}
  \quad \text{valid for } p < 1/e.
\end{equation}
This is the approach we adopt here, which also numerically coincides
with the Bayesian approach for the case where the prior probability of
the null hypothesis is 0.5.  The reason that this is only a lower
limit for \(\alpha\) is that if we have overwhelming a priori evidence that
the null hypothesis is true (for instance, from previous empirical
studies, or because it follows from a well-supported theory), then a
Bayesian calculation would give a much higher value of \(\alpha\) than
\eqref{eq:type-I} does.  In our case, however, we have no strong
reasons for favoring any of the null hypotheses, so it is reasonable
to assume \(\alpha\) is close to the lower limit given in \eqref{eq:type-I}.

In order to choose a threshold \(p\)-value that counts as a
``significant'' result, one then needs to balance the risks of false
positives against the risks of \textit{false negatives}.  The false
negative probability, \(\beta\), also known as \textit{Type II error
  rate}, is the probability of failing to reject an untrue null
hypothesis.  That is, in the context of this paper, it is the
probability of failing to detect a real difference between two
sub-samples, or a real correlation between two variables.  The
complementary probability, \(1 - \beta\), is known as the
\textit{statistical power} or sensitivity of the test.  The value of
\(\beta\) depends on three factors:
\begin{enumerate}[1.]
\item The \textit{effect size}, which is a measure of the magnitude of
  the difference in a dependent variable between two sub-samples, or
  the degree of correlation between two continuous variables.  For the
  two sub-sample case, it is common to use a standardised mean
  difference, such as Cohen's \(d\) statistic \citep{Cohen:1988a}:
  \(d = (\bar{X}_A - \bar{X}_B) / s\), where \(\bar{X}_A\),
  \(\bar{X}_B\) are the means of the dependent variable \(X\) for
  samples A and B, while \(s\) is the pooled standard deviation of
  \(X\).  For the case of two continuous variables, the Pearson linear
  correlation coefficient, \(r\), can be used.  In both cases, rules
  of thumb have been developed \citep{Ruscio:2008a} for classifying an
  effect as ``large'' (\(d > 0.8\), \(r > 0.4\)) or ``small''
  (\(d < 0.2\), \(r < 0.1\)).  Alternatively, non-parametric
  statistics can be used, such as the \(A\) measure of stochastic
  superiority \citep{Delaney:2002a}.
\end{enumerate}

Obviously, this depends on the \textit{effect size}, which is the 


All astronomical data analysis is \emph{post hoc} analysis, since the universe was not set up to test a particular hypothesis (as far as we know).  It is therefore important to guard against the ``multiple comparisons problem'', whereby seemingly significant correlations are found where none really exist, simply by virtue of the large number of tests that were carried out.

Under the more conservative Holm--Bonferroni method, only comparisons with \(p < 0.001\) would be significant. 

The p-curve \citep{Head:2015a}

%%% Local Variables:
%%% mode: latex
%%% TeX-master: "quadrics-bowshock"
%%% End:


% Don't change these lines

\bsp	% typesetting comment
\label{lastpage}

\end{document}

%%% Local Variables:
%%% mode: latex
%%% TeX-master: t
%%% End:
