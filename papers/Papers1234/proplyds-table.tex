 \begin{table*}
\begin{tabular}{c|ccccccccc}\hline
Proplyd & LV2 & LV2b & LV3 & LV4  & LV5 & 177-341 & 168-328 & 169-338 & 180-331\\\hline
$D' (\arcsec)$ &7.83 & 7.01 &6.91 & 6.05 & 9.42 & 25.84 & 6.64 & 16.47 & 25.12 \\
$R'_c/D'$ (central)  & 0.23  & 0.19& 0.61  & 0.5  & 0.34  & 0.18  & 0.22 & 0.09 & 0.09\\
$R'_c/D'$ (max)
$R'_c/D'$ (min)
$R'_0/D'$ (central) & 0.32 & 0.1 & 0.34 & 0.19 & 0.21 & 0.14 & 0.15 & 0.06 & 0.07\\
$R'_0/D'$ (max)&&&&&&&&& \\
$R'_0/D'$ (min)&&&&&&&&& \\
$\beta$ &&&&&&&&& \\
$\xi$ &&&&&&&&& \\
$i$ &&&&&&&&& \\
$D$ (arcsec) &&&&&&&&& \\
$R_0/D$ &&&&&&&&& \\
$r_0$ &&&&&&&&& \\
$N_6$ &&&&&&&&& \\
$P$ (in) &&&&&&&&& \\
$P$ (wind) &&&&&&&&& \\
$F$ (photo) &&&&&&&&& \\
$F$ (star)  &&&&&&&&& \\
\end{tabular}
\caption{Characteristic Radii measurements for a sample of proplyds. The ``central label'' refers to the ``main'' 
measurements of the radii. The ``max'' and ``min'' labels refer to the maximum and minimum measurements of the radii
obtained in all variations. In the $\beta$, $\xi$ and $i$ rows we show the parameteres which more approaches to the 
expected stagnation pressure and flux to mantain pressure balance and the photoevaporation flow.
} 

\label{tab:proplyds}
\end{table*}

%%% Local Variables:
%%% mode: latex
%%% TeX-master: "proplyd-bowshocks.tex"
%%% End:
