\section{Discussion}
\label{sec:summary-discussion}

Are there any objects that might not be bow shocks?

Progression in density:
\begin{gather*}
  \text{Increasing density} \longrightarrow \\
  \text{WBS} \to \text{WBS} + \text{IDW} \to \text{WBS} + \text{DDW} \to \text{(RBW)} \to \text{RBS}
\end{gather*}

Chief diagnostic for radiation supported bows (RBW or RBS cases) is
infrared luminosity of bow.  Favored by high densities.

Dust waves favored by high velocities and intermediate densities.

\newcommand\IR{\ensuremath{_{\text{IR}}}}

In order to provide an empirical anchor to our theoretical
calculations, we now consider how the parameters of our models might
be determined from observations.  The parameter space diagrams, such
as Figures~\ref{fig:zones-v-n-plane} and
\ref{fig:existence-dust-wave}, are not particularly useful in this
regard, since in many cases the ambient density and relative stellar
velocity are not directly measured.  Instead, we aim to construct
diagnostics based on the most common observations, which are of the
infrared dust emission.

A fundamental parameter is the optical depth, \(\tau\), of the bow shell
to UV radiation, which determines what fraction of the stellar photon
momentum is available to support the shell (see
\S~\ref{sec:three-bow-regimes}).  But the same photons also heat the
dust grains in the bow, which re-radiate that energy predominantly at
mid-infrared wavelengths (roughly \SIrange{10}{100}{\um}) with
luminosity \(L\IR\).  Assuming that Ly\(\alpha\) and mechanical heating of
the dust shell is negligible and that the emitting shell subtends a
solid angle \(\Omega\), as seen from the star, then the optical depth can
be estimated as
\begin{equation}
  \label{eq:tau-empirical}
  \tau = -\ln \left( 1 - \frac{4\pi}{\Omega} \frac{L\IR}{L_*} \right)
  \approx \frac{2 L\IR}{L_*} \ ,
\end{equation}
where the last approximate equality holds if \(\tau \ll 1\) and the shell
emission covers one hemisphere.\footnote{%
  The \(\tau\) of \S~\ref{sec:three-bow-regimes} is not exactly the same
  as the \(\tau\) of equation~\eqref{eq:tau-empirical}, but is larger by
  a factor of \(1 + \varpi (1 - g)/(1 - \varpi)\), where \(\varpi\) is
  the grain albedo and \(g\) the scattering asymmetry (see
  App.~\ref{sec:gas-free-bow}).} %

A second important parameter is the thermal plus magnetic pressure in
the shocked shell, which is doubly useful since in a steady state it
is equal to \emph{both} the internal supporting pressure (wind ram
pressure plus absorbed stellar radiation) \emph{and} the external
confining pressure (ram pressure of ambient stream).  The shell pressure is
not given directly by the observations, but can be determined as
follows:
\begin{enumerate}[1.]
\item The shell mass column (\si{g.cm^{-2}}) can be estimated from the
  optical depth by assuming an effective UV opacity: \(\Sigma\shell = \tau / \kappa\)
\item The shell density (\si{g.cm^{-3}}) can be found from the mass
  column if the shell thickness is known:
  \(\rho\shell = \Sigma / h\shell\).  In the absence of other information, a
  fixed fraction of the shell radius can be used.  In particular, we
  normalize by a typical value of one~quarter\footnote{%
    This corresponds to a Mach number \(\M_0 = \surd 3\) if the stream
    shock is radiative, or \(\M_0 \gg 1\) if non-radiative (see
    \S~\ref{sec:radi-cool-lengths}).  Further discussion is given in
    Appendix~\ref{app:bow-shock-data}} %
  the star--apex distance:
  \(h_{1/4} = h\shell / (0.25 R_0)\).
\item Finally, the pressure (\si{dyne.cm^{-2}}) follows by assuming
  values for the sound speed and Alfvén speed:
  \(P\shell = \rho\shell (\sound^2 + \frac12 v\alfven^2) \).
\end{enumerate}
It is natural to normalize this pressure to the stellar radiation
pressure at the shell, so we define a shell momentum efficiency
\newcommand\pc{\ensuremath{_{\text{pc}}}}
\begin{equation}
  \label{eq:eta-shell}
  \eta\shell \equiv \frac{P\shell}{P\rad}
  = \frac{4\pi R_0^2\, (\sound^2 + \frac12 v\alfven^2)\, \tau}{L_*\, \kappa\, h\shell}
  \approx 245 \frac{R\pc \, T_4 \, \tau}{L_4 \, \kappa_{600} \, h_{1/4}} \ , 
\end{equation}
where in the last step we have assumed ionized gas with negligible
magnetic support (\(v\alfven \ll \sound\)) and written the stellar
luminosity and shell parameters in terms of typical values, as in
\S~\ref{sec:depend-stell-type}.  Note that the shell momentum
efficiency is simply the reciprocal of the radiation parameter of
equation~\eqref{eq:Xi-Prad-over-Pgas}:
\(\eta\shell = \Xi\shell^{-1}\), which provides yet a third use for
\(\eta\shell\), since \(\Xi\) is paramount in determining whether the
grains and gas remain well-coupled (see
\S~\ref{sec:exist-cond-separ}).

\begin{table}
  \centering
  \caption[Observational]{Key observational parameters for star/bow systems}
  \label{tab:observations}
  \begin{tabular}{l S S S}
    \toprule
    Star & {\(L_* / \si{L_\odot}\)} & {\(L_{\text{IR}} / \si{L_\odot}\)} & {\(R_0 / \si{pc}\)} \\
    \midrule
    \thD & 2.95e4 & 620 & 0.003 \\
    LP~Ori & 1600 & 240 & 0.01 \\
    \(\sigma\)~Ori & 6e4 & 15 & 0.12 \\[\smallskipamount]
    K18 Sources & \numrange{1.4e4}{8.7e5} & \numrange{8}{2800} & \numrange{0.02}{1.35} \\
    \bottomrule
  \end{tabular}
\end{table}


\begin{figure*}
  \centering
  \includegraphics[width=\linewidth]{figs/All-sources-eta-tau}
  \caption[Observational diagnostic diagram]{Observational diagnostic
    diagram for bow shocks.  The shell optical depth \(\tau\) (\(x\)
    axis) and momentum efficiency \(\eta\shell\) (\(y\) axis) can be
    estimated from observations of the bolometric stellar luminosity,
    infrared shell luminosity, and shell radius, as described in the
    text.  Results are shown for the 20 sources (circle symbols) from
    \citet{Kobulnicky:2018a} plus three further sources (square
    symbols), where we have obtained the measurements ourselves (see
    Tab.~\ref{tab:observations}).  The color of each symbol indicates
    the stellar luminosity (dark to light) as indicated by the scale
    bar. The shell pressure is determined assuming a gas temperature
    \(T = \SI{e4}{K}\), an absorption opacity
    \(\kappa = \SI{600}{cm^2.g^{-1}}\), and a thickness-to-radius ratio
    \(H/R = 0.25\).  The sensitivity of the results to a
    factor-of-three change in each parameter is shown in the upper
    inset box.  Exceptions are the two Orion Nebula sources, \thD{}
    and LP~Ori, where the small dim squares show the results of
    assuming the standard shell parameters, while the large squares
    show the results of modifications according to the peculiar
    circumstances of each object, as described in the text.  The lower
    inset box shows the sensitivity of the results to a factor-of-two
    uncertainty in each observed quantity: distance to source \(D\);
    stellar luminosity \(L_*\), shell infrared flux \(F\IR\); shell
    angular size \(\theta\).  Lines and shading indicate different
    theoretical bow regimes (see \S\S~\ref{sec:strong-gas-grain} and
    \ref{sec:imperf-coupl-betw}).  The dashed blue diagonal line
    corresponds to radiation-supported bows, while the upper left
    region corresponds to wind-supported bows.  The upper right corner
    (purple) corresponds to optically thick bow shocks, while the
    lower left corner (yellow) is the region where grain--gas
    separation \textit{may} occur, leading to a potential dust wave.
    However, the existence of a dust wave in this region is not
    automatic, since it only includes one of the four necessary
    conditions (\S~\ref{sec:exist-cond-separ} and
    \S~\ref{sec:grain-traj-with}). The lower-right region is strictly
    forbidden, except in case of violation of the assumption that dust
    heating be dominated by stellar radiation. }
  \label{fig:All-sources-eta-tau}
\end{figure*}



Mass loss rates - starting from \citet{Kobulnicky:2010a}

Different scenarios for producing velocities: dynamic ejection from
young clusters \citep{Hoogerwerf:2001a, Oh:2016c} produce high
velocities, dissolution of binary systems following core-collapse SN
\citep{Renzo:2018a} tend to produce lower velocities for the unbound
MS companion (walkaways, slower than 30 km/s).  Also, champagne flows
have low velocities.

How different regions of the \(\Pi\)--\(\Lambda\) plane are populated.
Bottom-right quadrant hard to get to (except for standing wave
oscillations), but may be due to finite shell thickness, which (for
low Mach number) will be more apparent in the wings, which might
decrease \(\Lambda\) more than \(\Pi\).  Fact that thin-shell solutions should
trace the contact discontinuity, but in some cases it may be only the
inner or the outer shell that is visible.

Justification for standing waves: Fig.~3 of \citet{Meyer:2016a} shows
a time sequence of thin-shell instability, which looks a bit like a
standing wave. But much larger amplitude than we are considering.

Deviations from axisymmetry as an alternative to oscillations. 


\subsection{The case of inside-out bows}
\label{sec:case-inside-out}

So far, we have considered the case where the inner source dominates
the radiation, while dust is present only in the outer stream, which
applies to hot stars interacting with the ISM.  However, in the case
of cool stars, the inner wind will also be dusty.  Examples are the
red supergiant (RSG) phase of high-mass evolution, or the asymptotic
giant branch (AGB) stage of low/intermediate-mass evolution.  In both
these cases, it is still the inner source that provides the radiation
field.  However, not all winds are radiatively driven and in those
cases it is conceivable that it is the outer source that dominates the
radiation field.  An example is the case of photoevaporating
protoplanetary disks (proplyds) in the Orion Nebula and other \hii{}
regions \citep{ODell:1994a}.  In the proplyds, the inner wind is a
thermally driven photoevaporation flow \citep{HA:1998, Henney:1999a},
while the outer stream is the stellar wind from an O~star
\citep{Garcia-Arredondo:2001a}.


\section{Summary and conclusions}
\label{sec:conclusions}



%%% Local Variables:
%%% mode: latex
%%% TeX-master: "dusty-bow-wave"
%%% End:
