
\section{Discussion}
\label{sec:discussion}

In this section we briefly discuss our predictions for the
appearance of dust waves and place our results in the context of
previous work on gas--grain coupling in the environs of high-mass
stars.  We concentrate on conceptual and theoretical aspects,
postponing empirical questions about particular sources for later
papers.

\subsection{Predicted appearance of dust waves}
\label{sec:pred-shape-struct}
By definition, dust waves do not correspond to any structure in the
gas and so are only detectable via the mid-infrared thermal emission
of the grains.  The emissivity is proportional to the grain density
but also depends on the grain temperature, which is a decreasing
function of radius, \(R\), from the star \citetext{see, for example,
  synthetic observations in \citealp{Mackey:2016a, Acreman:2016a,
    Meyer:2017a}}.  In radiative equilibrium, the bolometric
emissivity is proportional to the stellar flux \(\propto R^{-2}\), but the
radial dependence of the monochromatic emissivity will be much steeper
than this on the short-wavelength side of the average grain thermal
spectrum.  With this in mind, we can crudely estimate the appearance
of the example dust waves calculated in \S\S~\ref{sec:gas-free-bow}
and \ref{sec:grain-traj-with}.

In the inertia-confined case, the radius \(R\starstar\) is
proportional to the single-grain opacity \(\kappa\grain\) and will
therefore vary with grain size and composition.  For a spherical grain
of size \(\amu\,\si{\um}\) and solid density
\(\rho\grain \, \si{g.cm^{-3}}\), equation~\eqref{eq:kappa-grain} yields
\begin{equation}
  \label{eq:kappa-d-dependence}
  \kappa\grain = 7500 \frac{\Qpbar}{\rho\grain \amu} \ . 
\end{equation}
The radiation pressure efficiency is \(\Qpbar \sim 1\) for
\(\amu > 0.02\), but falls as \(\Qpbar \propto a\) for smaller grains
\citep[e.g., Fig.~7 of][]{Draine:2011a}.  Therefore, all the smallest
grains will form the dust wave at the same point, but grains larger
than \SI{0.02}{\um} will be spread out with
\(R\starstar \propto a^{-1}\).  It might be thought that this would produce
a very broad diffuse appearance to the dust wave, but this is
mitigated by the fact that it is the smaller grains that dominate the
UV opacity and mid-IR emissivity.  Also, the larger grains may be
destroyed by radiative torques (see \S~\ref{sec:grain-survival-dust}).
For the drag-confined case, the situation is clearer since the dust
wave will form just inside the rip point, \(R_\dag\), which is relatively
insensitive to the grain size.

The effect of a roughly parallel magnetic field is mainly seen in the
shape of the dust wave, which becomes hemispherical
(Fig.~\ref{fig:inertia-thB0}) instead of parabolic as was found in the
non-magnetic case (Fig.~\ref{fig:dust-trajectories}).  Interestingly,
this effect is the opposite of what is seen in simulations of MHD bow
shocks \citep{Meyer:2016a}, where a parallel B-field leads to
flatter-nosed bow shapes with a high planitude \citetext{see Fig.~25
  of \citealp{Tarango-Yong:2018a}}.  Figure~\ref{fig:frozen-stream}ab
shows that the field orientation has a similar effect on the dust wave
shape for drag-confined magnetic dust waves, but with the added
complication that a second dense shell forms on one side of the axis
at the stagnant drift radius, \(R_\ddag\), in cases where
\(v_\infty > \SI{60}{km.s^{-1}}\).  However, since \(R_\ddag\) is several times
larger than \(R_\dag\), the emission from this outer shell is likely to
be weak, given the steep radial dependence of the mid-IR emissivity
discussed above.

When the magnetic field is close to perpendicular to the axis, then a
dense dust shell does not form in the apex region for either the
inertia-confined (Fig.~\ref{fig:inertia-thB90}) or drag-confined
(Fig.~\ref{fig:frozen-stream}cd) cases.  Nonetheless, it is possible
that a dust wave might form in the wings in such cases.

\subsection{Gas--grain dynamics in \hii{} regions}
\label{sec:gas-grain-dynamics-hii}

The distinction that we draw in \S~\ref{sec:exist-cond-separ} between
inertia-confined dust waves and drag-confined dust waves is novel.
However, the concept of the \textit{rip point} can be found in
previous works, although not under that name and it has generally been
assumed to be of little interest.  For instance, Figure~8 of
\citet{Draine:2011a} clearly shows the discontinuity in drift velocity
that occurs at small radii within an \hii{} region (differences from
our own Figs.~\ref{fig:multi-dustprops} and~\ref{fig:drift-gn} are due
to \citeauthor{Draine:2011a}'s assumption of constant grain
potential).  However, the region of supersonic drift involves a very
small fraction of the total dust in the \hii{} region, so it was not
relevant to the concerns of that paper. Similarly, in \S~3.1 of
\citet{Hopkins:2018c} the possibility is raised of a ``decoupling
instability'' in the case that coupling strength decreases with
increasing slip velocity, but they go on to dismiss this as physically
irrelevant.  On the contrary, we show in this paper that it \emph{is}
relevant so long as the grain potential is high, so that a distinct
local maximum occurs in the drag-versus-velocity profile (see
Fig.~\ref{fig:gas-grain-drag-photoionized}).  In \S~9 of
\citet{Hopkins:2018a}, various issues relating to grain charging and
drag are discussed.  The quantity
\(e_{\text{rad}} / e_{\text{therm}}\) from \S~9.1.4 of that paper is
the same as our radiation parameter, \(\Xi\).  In \S~9.2.3 they give an
expression for the critical radius that divides subsonic from
supersonic drift, which is exactly equivalent to our
equation~\eqref{eq:Rdag-over-Rstar} for the rip point radius.  The
parameters assumed by \citeauthor{Hopkins:2018a} yield
\(\Xi_\dag = 3200\), which is within the range of values that we find for
O~stars in Table~\ref{tab:Xi-rip}.

\citet{Akimkin:2017a} studied the effect of radiation pressure on the
time-dependent evolution of an \hii{} region, taking full account of
dynamical coupling between grains and gas, thus extending a previous
study \citep{Akimkin:2015a} that had ignored the back-reaction on the
gas (see our \S~\ref{sec:back-reaction-gas}).  They found that the
dust completely decouples from the gas in the zone around the star
(\(< \SI{0.2}{pc}\) on timescales \(\sim \SI{0.5}{Myr}\)), which produces
an inner dust hole but partially eliminates the inner gas density hole
that is found in models that assume perfect coupling
\citep{Mathews:1967a, Draine:2011a, Kim:2016b}. The decoupling is more
pronounced for stars of lower effective temperature, which is probably
related to the fact that we find the critical radiation parameter
\(\Xi_\dag\) to be lower for B~stars than for O~stars
(Table~\ref{tab:Xi-rip}).  A similar dust hole for the central region
is found by \citet{Ishiki:2018a}.

The gas--grain decoupling that we have discussed so far occurs at high
radiation parameter, \(\Xi \sim 1000\), but there is another regime of
potential decoupling that occurs at low \(\Xi\).  It has long been
recognized \citep{Gail:1979a} that Coulomb gas--grain coupling should
weaken in the outer zones of an \hii{} region due to the fact that the
grain potential passes through (or close to) zero as a result of
electron collisions becoming competitive with photoelectric ejection
as the radiation field weakens.  This is the regime around
\(\Xi = \text{\numrange{0.1}{1.0}}\) in Figure~\ref{fig:drift-gn},
giving drift velocities of order \SI{1}{km.s^{-1}} for larger grains
(\SI{0.2}{\um}), which may be sufficient to produce significant
spatial variations in grain abundance on Myr timescales
\citep{Ishiki:2018a}. It remains to be seen whether this is still true
once effects that have been neglected in existing models are accounted
for, such as the \hii{} region's magnetic field \citep{Krumholz:2007a,
  Arthur:2011a, Gendelev:2012a} and internal turbulence
\citep{Arthur:2016a}.  On the other hand, grain abundance variations
may be enhanced by the resonant drag instability \citep{Squire:2018a,
  Hopkins:2018a}.  Yet another regime of potential decoupling occurs
in the neutral shell outside the \hii{} region
\citep{Gustafsson:2018a}, but since the drift velocity in both these
cases is much less than interesting values of the stream velocity,
\(v_\infty\), neither low-\(\Xi\) regime is relevant to dust waves.

\subsection{Gas--grain dynamics in stellar bows}
\label{sec:gas-grain-dynamics-bs}

The term \textit{dust wave} was first coined to described the
radiative decoupling of grains from the plasma flowing past a massive
star by \citet{Ochsendorf:2014b}, who attempted to explain the
infrared emission arc around \(\sigma\)~Ori.  The concept was subsequently
applied to additional dust arcs in RCW~82 and RCW~120
\citep{Ochsendorf:2014a} and a refined model was applied to further
observations of \(\sigma\)~Ori \citep{Ochsendorf:2015a}.  These works
provided the inspiration for the present paper, where we have
attempted to take a more a priori and systematic approach to the
problem, including factors such as the Lorentz force that were
neglected by \citeauthor{Ochsendorf:2014b}.

The most detailed simulations to date of the combined dynamics of
dust, gas, and magnetic field in an OB bow was carried out by
\citet{Katushkina:2017a}, who followed the trajectories of dust
particles after passing through an MHD bow shock, under the influence
of the stellar radiation force and Lorentz force.  The gas is
decelerated in the shock, but the dust initially carries on with its
pre-shock motion, which produces a relative velocity with respect to
the plasma. The component of this velocity perpendicular to the
magnetic field induces gyration about the field lines
(cf~\S~\ref{sec:magn-effects-grain}), which forms filaments of
enhanced dust density that are oriented perpendicular to the bow axis
and with characteristic separation equal to the bulk plasma velocity
times the gyration period.  In \citet{Katushkina:2018a}, similar such
simulations are applied to observations of the runaway B1 supergiant
\(\kappa\)~Cas, moving at \(\approx \SI{30}{km.s^{-1}}\), in order to explain the
filaments of infrared dust emission seen at \SI{24}{\um}.  It is found
that the simulations can only fit the observations with very large
dust grains (\(\approx \SI{1}{\um}\)) and a very strong perpendicular
magnetic field \(v\alfven \approx \SI{20}{km.s^{-1}}\).  The stellar
parameters of \(\kappa\)~Cas are roughly those given in the last row of
Table~\ref{tab:stars} and the observed radius of the dust arc (assumed
to correspond to the astropause) is \SI{0.75}{pc} for a distance of
\SI{1}{kpc}.  The \citeauthor{Katushkina:2018a} simulations do not
explicitly include the Coulomb drag force on the dust grains, although
our own results (\S~\ref{sec:cloudy-models-dust}) suggest that this
must be important around \(\kappa\)~Cas.  \citet{Katushkina:2018a} argue
that small grains are swept out by stellar radiation before reaching
the bow shock, and this would indeed be true if drag forces were
negligible.  However, we find \(\Xi \approx 20\) for the
\(\kappa\)~Cas shell, implying strong gas--grain coupling and that even
small grains cannot be repelled by the radiation force.  The root
cause of our difference with these authors is that they are assuming a
grain potential \(\phi \sim 1\), as seen in the ISM in the Solar
neighborhood, as opposed to \(\phi \sim 10\), which is more appropriate to
the environs of an OB star.  We also note that even when the Lorentz
force vastly exceeds all other forces, it does not necessarily follow
that the other forces are unimportant.  For example, in the
magnetized dust wave models that we present in
\S~\ref{sec:tight-magn-coupl} and \S~\ref{sec:grain-traj-with}, the
Lorentz force is infinitely stronger than other forces.  Nevertheless,
the radiation and drag forces are crucial in determining the structure
of the dust waves, which is possible because the component of the
Lorentz force projected along the field lines is zero.


\subsection{Grain survival}
\label{sec:grain-survival-dust}

The survival of dust grains of different sizes in close proximity to
OB stars is something that must be considered.  Potential destruction
mechanisms include thermal evaporation, particle sputtering, and
radiative torque destruction.  Thermal evaporation requires that the
grain radiative equilibrium temperature exceed the sublimation
temperature (\SIrange{1400}{1750}{K}, depending on composition), which
occurs for radiative fluxes about \SI{e9} times higher than the
interstellar radiation field.  For a drag-confined dust wave, we have
a fixed radiation parameter \(\Xi_\dag \sim 1000\), so this becomes a
threshold on the gas density, requiring \(n > \SI{3e5}{cm^{-3}}\).
Combining this with the maximum allowed density for a dust wave
(eq.~[\ref{eq:dust-wave-high-density-condition}]), we also obtain a
condition on the stream velocity:
\(v_\infty > 170 \kappa_{600}^{0.5} L_4^{0.15} \, \si{km.s^{-1}}\).  Therefore,
thermal evaporation is generally unimportant in dust waves, except for
fast-moving stars in very high density environments.

Ion sputtering is only effective for collider kinetic energies in
excess of \SI{100}{eV} \citep{Draine:1995a, Field:1997a}, which is
significantly larger than thermal energies in photoionized gas.  It
therefore requires supersonic gas--grain slip velocities
\(w > \SI{75}{km.s^{-1}}\), but this does not necessarily imply that
the stream velocity need be quite so high.  For inertia-confined dust
waves, \(w\) has a maximum value of \(2 v_\infty\) and for drag-confined
dust waves it can be even higher (for instance, reaching
\(w \approx 4 v_\infty\) in Fig.~\ref{fig:frozen-trajectories}a), so that
\(v_\infty > \SI{30}{km.s^{-1}}\) is probably sufficient.  However, in
order to be destroyed by sputtering it is also necessary that a dust
grain of radius \(\amu\)\,\si{\um} should traverse a gas column
density of \(\approx \num{2e21}\, \amu \,\si{cm^{-2}}\)
\citep{Draine:2011a}.  For magnetic field orientations close to
parallel (which is the case that most favors dust wave formation), the
grains linger in the dust wave for several dynamical times (see
\S~\ref{sec:grain-traj-with}), so we estimate a total column of
\(\approx 10 n R_\dag\).  Using equations~(\ref{eq:Rstar-typical},
\ref{eq:Rdag-over-Rstar}) then implies that grains smaller than
\(\amu \sim 0.001 (L_4 n)^{1/2}\) can be destroyed by sputtering in the
dust wave.

Radiative torque disruption \citep{Hoang:2018a} is the centrifugal
rupture of an irregular grain of non-zero helicity that has been
suprathermally spun up by the anisotropic absorption of radiation
\citep{Dolginov:1976a, Draine:1996b, Lazarian:2007a}.  The process has
a very steep size dependence, being most effective in destroying
larger grains.  Using equation~(27) of \citet{Hoang:2018a} and
assuming a grain tensile strength of
\(S_{\text{max}} = \SI{2e10}{dyne.cm^{-2}}\) \citetext{\S~2.4 of
  \citealp{Borkowski:1995a}}, we find that grains larger than
\(\amu \sim 0.04\, n^{-0.123}\) are efficiently destroyed by this
mechanism at a distance \(R_\dag\) from the star. The spin-up timescales
are less than \(100\, n^{-1}\,\si{yr}\), which is short compared with
the dust wave dynamic timescale.

In summary, refractory grains in the size range
\SIrange{0.001}{0.04}{\um} are predicted to survive in dust waves.
Larger grains than this are unlikely to be able to resist
disintegration by radiative torques and smaller grains may be
destroyed by sputtering.

% \subsection{Dust wave stability}
% \label{sec:dust-wave-stability}


% Instabilities: Resonant Drag Instability - application to stagnant drift case.




\section{Summary}
\label{sec:summary}

We have extended our previous study of bows around OB stars (Paper~I)
in order to consider the weak coupling case, in which a
radiation-supported dust wave decouples from the gas to form an
infrared emission arc outside of any hydrodynamic bow shock.  Our
principle findings are as follows:
\begin{enumerate}[1.]
\item Dust waves can only exist when the star's relative velocity with
  respect to its environment exceeds a critical value
  \(v_\infty > v_{\text{min}}\)
  (eq.~[\ref{eq:dust-wave-velocity-condition}]).  For O~stars with
  strong winds,
  \(v_{\text{min}} = \text{\SIrange{150}{300}{km.s^{-1}}}\), although
  for weak-wind stars and B~stars it can be as low as
  \SI{30}{km.s^{-1}}.
\item Additionally, the ambient density is constrained to lie within a
  certain range, \(n_{\text{min}} \to n_{\text{max}}\). For the lowest
  allowed relative velocities, \(v_\infty \approx v_{\text{min}}\), these are
  \(n_{\text{min}} = \SI{0.01}{cm^{-3}}\),
  \(n_{\text{max}} = \SI{100}{cm^{-3}}\) for B stars and
  \(n_{\text{min}} = \SI{1}{cm^{-3}}\),
  \(n_{\text{max}} = \SI{e5}{cm^{-3}}\) for strong-wind O~stars (see
  Fig.~\ref{fig:existence-dust-wave}).  Both these limits increase for
  higher velocities, as \(n_{\text{min}} \propto v_\infty^2\) and
  \(n_{\text{max}} \propto v_\infty^4\).
\item Dust waves may either be \textit{inertia-confined} or
  \textit{drag-confined}, where the inertia-confined regime (in which
  gas drag is always negligible) corresponds to a relatively narrow
  range of densities above \(n_{\text{min}}\).
\item In drag-confined dust waves, the gas--grain decoupling occurs
  suddenly at a \textit{rip point}, where the Coulomb drag
  catastrophically breaks down. The rip point occurs at a particular
  value of the radiation-to-gas pressure ratio:
  \(\Xi_\dag \sim 1000\), with little dependence on other parameters.
\item The post-rip grain trajectories are unstable for
  \(v_\infty = \text{\SIrange{10}{50}{km.s^{-1}}}\), exhibiting limit-cycle
  decoupling/recoupling behavior of repeated rip followed by
  snap-back.  For higher velocities a quasi-stationary stagnant drift
  shell can form on the axis.
\item Grain coupling to magnetic fields can modify these results, but
  this depends critically on the angle \(\theta\B\) between the field and
  the star's relative velocity vector.  For the quasi-parallel case
  (\(\theta\B < \ang{30}\)), the axial structure of the dust wave is
  little-changed, but the shape of the dust wave wings become more
  closed (hemispherical) than in the non-magnetic case.  For the
  quasi-perpendicular case (\(\theta\B > \ang{60}\)), a dust wave cannot form
  on the axis, although it is possible it may do so in the wings.
\end{enumerate}


%%% Local Variables:
%%% mode: latex
%%% TeX-master: "bs-bw-dw-02"
%%% End:
