
\section{Discussion}
\label{sec:discussion}

In this section we place our results in the context of previous work
on gas--grain coupling in the environs of high-mass stars.  We
concentrate on conceptual and theoretical aspects, postponing
empirical questions about particular sources for later papers.

The distinction that we draw in \S~\ref{sec:exist-cond-separ} between
inertia-confined dust waves and drag-confined dust waves is novel.
However, the concept of the \textit{rip point} can be found in
previous works, although not under that name and it has generally been
assumed to be of little interest.  For instance, Figure~8 of
\citet{Draine:2011a} clearly shows the discontinuity in drift velocity
that occurs at small radii within an \hii{} region (differences from
our own Figs.~\ref{fig:multi-dustprops} and~\ref{fig:drift-gn} are due
to \citeauthor{Draine:2011a}'s assumption of constant grain
potential).  However, the region of supersonic drift involves a very
small fraction of the total dust in the \hii{} region, so it was not
relevant to the concerns of that paper. Similarly, in \S~3.1 of
\citet{Hopkins:2018c} the possibility is raised of a ``decoupling
instability'' in the case that coupling strength decreases with
increasing slip velocity, but they go on to dismiss this as physically
irrelevant.  On the contrary, we show in this paper that it \emph{is}
relevant so long as the grain potential is high, so that a distinct
local maximum occurs in the drag-versus-velocity profile (see
Fig.~\ref{fig:gas-grain-drag-photoionized}).  In \S~9 of
\citet{Hopkins:2018a}, various issues relating to grain charging and
drag are discussed.  The quantity
\(e_{\text{rad}} / e_{\text{therm}}\) from \S~9.1.4 of that paper is
the same as our radiation parameter, \(\Xi\).  In \S~9.2.3 they give an
expression for the critical radius that divides subsonic from
supersonic drift, which is exactly equivalent to our
equation~\eqref{eq:Rdag-over-Rstar} for the rip point radius.  The
parameters assumed by \citeauthor{Hopkins:2018a} yield
\(\Xi_\dag = 3200\), which is within the range of values that we find for
O~stars in Table~\ref{tab:Xi-rip}.

\citet{Akimkin:2017a} studied the effect of radiation pressure on the
time-dependent evolution of an \hii{} region, taking full account of
dynamical coupling between grains and gas, thus extending a previous
study \citep{Akimkin:2015a} that had ignored the back-reaction on the
gas (see our \S~\ref{sec:back-reaction-gas}).  They found that the
dust completely decouples from the gas in the zone around the star
(\(< \SI{0.2}{pc}\) on timescales \(\sim \SI{0.5}{Myr}\)), which produces
an inner dust hole but partially eliminates the inner gas density hole
that is found in models that assume perfect coupling
\citep{Mathews:1967a, Draine:2011a, Kim:2016b}. The decoupling is more
pronounced for stars of lower effective temperature, which is probably
related to the fact that we find the critical radiation parameter
\(\Xi_\dag\) to be lower for B~stars than for O~stars
(Table~\ref{tab:Xi-rip}).  A similar dust hole for the central region
is found by \citet{Ishiki:2018a}.

The gas--grain decoupling that we have discussed so far occurs at high
radiation parameter, \(\Xi \sim 1000\), but there is another regime of
potential decoupling that occurs at low \(\Xi\).  It has long been
recognized \citep{Gail:1979a} that Coulomb gas--grain coupling should
weaken in the outer zones of an \hii{} region due to the fact that the
grain potential passes through (or close to) zero as a result of
electron collisions becoming competitive with photoelectric ejection
as the radiation field weakens.  This is the regime around
\(\Xi = \text{\numrange{0.1}{1.0}}\) in Figure~\ref{fig:drift-gn},
giving drift velocities of order \SI{1}{km.s^{-1}} for larger grains
(\SI{0.2}{\um}), which may be sufficient to produce significant
spatial variations in grain abundance on Myr timescales
\citep{Ishiki:2018a}. It remains to be seen whether this is still true
once effects that have been neglected in existing models are accounted
for, such as the \hii{} region's magnetic field \citep{Krumholz:2007a,
  Arthur:2011a, Gendelev:2012a} and internal turbulence
\citep{Arthur:2016a}.  Yet another regime of potential decoupling
occurs in the neutral shell outside the \hii{} region
\citep{Gustafsson:2018a}, but since the drift velocity in both these
cases is much less than interesting values of the stream velocity,
\(v_\infty\), neither low-\(\Xi\) regime is relevant to dust waves.

The most detailed simulations to date of the combined dynamics of
dust, gas, and magnetic field in an OB bow was carried out by
\citet{Katushkina:2017a}, who followed the trajectories of dust
particles after passing through an MHD bow shock, under the influence
of the stellar radiation force and Lorentz force.  The gas is
decelerated in the shock, but the dust initially carries on with its
pre-shock motion, which produces a relative velocity with respect to
the plasma. The component of this velocity perpendicular to the
magnetic field induces gyration about the field lines
(cf~\S~\ref{sec:magn-effects-grain}), which forms filaments of
enhanced dust density that are oriented perpendicular to the bow axis
and with characteristic separation equal to the bulk plasma velocity
times the gyration period.  In \citet{Katushkina:2018a}, similar such
simulations are applied to observations of the runaway B1 supergiant
\(\kappa\)~Cas, moving at \(\approx \SI{30}{km.s^{-1}}\), in order to explain the
filaments of infrared dust emission seen at \SI{24}{\um}.  It is found
that the simulations can only fit the observations with very large
dust grains (\(\approx \SI{1}{\um}\)) and a very strong perpendicular
magnetic field \(v\alfven \approx \SI{20}{km.s^{-1}}\).  The stellar
parameters of \(\kappa\)~Cas are roughly those given in the last row of
Table~\ref{tab:stars} and the observed radius of the dust arc (assumed
to correspond to the astropause) is \SI{0.75}{pc} for a distance of
\SI{1}{kpc}.  The \citeauthor{Katushkina:2018a} simulations do not
explicitly include the Coulomb drag force on the dust grains, although
our own results (\S~\ref{sec:cloudy-models-dust}) suggest that this
must be important around \(\kappa\)~Cas.  \citet{Katushkina:2018a} argue
that small grains are swept out by stellar radiation before reaching
the bow shock, and this would indeed be true if drag forces were
negligible.  However, we find \(\Xi \approx 20\) for the
\(\kappa\)~Cas shell, implying strong gas--grain coupling and that even
small grains cannot be repelled by the radiation force.  The root
cause of our difference with these authors is that they are assuming a
grain potential \(\phi \sim 1\), as seen in the ISM in the Solar
neighborhood, as opposed to \(\phi \sim 10\), which is more appropriate to
the environs of an OB star.  We also note that even when the Lorentz
force vastly exceeds all other forces, it does not necessarily follow
that the other forces are unimportant.  For example, in the
magnetized dust wave models that we present in
\S~\ref{sec:tight-magn-coupl} and \S~\ref{sec:grain-traj-with}, the
Lorentz force is infinitely stronger than other forces.  Nevertheless,
the radiation and drag forces are crucial in determining the structure
of the dust waves, which is possible because the component of the
Lorentz force projected along the field lines is zero.

The survival of dust grains of different sizes in close proximity to
OB stars is something that must be considered.  Potential destruction
mechanisms include thermal evaporation, photosputtering, particle
sputtering, and radiative torque destruction.  

Sputtering
\citep{Draine:2011a}.  Radiative Torque Disruption
\citep{Hoang:2018a}.


Instabilities: Resonant Drag Instability \citep{Squire:2018a, Hopkins:2018a} - application to stagnant drift case. 




\section{Summary}
\label{sec:summary}


%%% Local Variables:
%%% mode: latex
%%% TeX-master: "bs-bw-dw-02"
%%% End:
