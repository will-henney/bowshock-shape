
\section{Summary}
\label{sec:conclusion}

We have presented a statistical study of the shapes of three different
classes of stellar bow shocks, as characterized by their flatness of
apex (planitude, \(\Pi\)) and openness of wings (alatude,
\(\Lambda\)), following the terminology of
\citet[Paper~0]{Tarango-Yong:2018a}.  Our principal findings are as
follows:
\begin{enumerate}[1.]
\item Bow shocks driven by hot OB stars, from the mid-infrared catalog
  of \citet{Kobulnicky:2016a}, have an average shape
  \((\Pi, \Lambda) \approx (1.6, 1.7)\) that is consistent with predictions of the
  \citet{Wilkin:1996a} analytic model, but the dispersion in observed
  shapes \((\sigma_\Pi, \sigma_\Lambda) \approx (0.9, 0.3)\) is many times larger than
  predicted by that model.
\item The bow shock shapes show little correlation with other
  parameters of the source, such as stellar magnitude, Galactic
  latitude or longitude, extinction, or type of environment (cluster
  versus isolated).  The only exception is that the dispersion in
  alatude is higher for bow shocks of larger angular size.
\item A possible explanation for the previous results is that the
  variation in shapes is caused by time dependent oscillations in the
  bow shock surface, with relative amplitude of 10 to 20\% and
  wavelength of order the bow shock size.  The oscillations may either
  be due to dynamic instabilities in the bow shock shell or be driven
  by temporal variations in the stellar wind.  If the oscillations
  were more vigorous for stars with more powerful winds, it could
  explain the correlation with angular size.
\item Bow shocks driven by cool luminous stars (red supergiants and
  asymptotic giant branch stars), from the catalog of
  \citep{Cox:2012a}, have an average shape
  \((\Pi, \Lambda) \approx (1.5, 1.4)\), which has a significantly smaller alatude
  than the OB star sources, and which is not consistent with the
  \citet{Wilkin:1996a} model.  We suggest that this may be due to
  their dust emission being dominated by shocked stellar wind
  material, instead of shocked ambient material as is the case with
  the OB stars.
\item Bow shocks driven by proplyds and other young stars in the outer
  regions of the Orion Nebula, from the catalogs of
  \citet{Bally:2006a} and \citet{Gutierrez-Soto:2015a}, have an
  average shape of \((\Pi, \Lambda) \approx (2.7, 2.4)\), with a significant tail up
  to \((\Pi, \Lambda) \approx (7, 4)\).  A minority of these sources
  (\(\approx 20\%\)) have shapes similar to the OB star bow shocks, but the
  remainder have much flatter apexes and more open wings.  We suggest
  several possible mechanisms to explain this difference: divergent
  ambient flow; low Mach number; observational biases; influence of
  collimated jets, but the available evidence for and against each of
  these is mixed.
\end{enumerate}


% Difference in SEDs between FIR and MIR samples?  \citet{Meyer:2016a}
% say emission peaks at 3--50 micron for O star bow shocks.

% Al other things being equal, larger bows will be at higher
% inclinations (we should estimate how much variation in size we should
% get due to inclination -- it will be more for flatter bows).  Higher
% inclinations means less variation from the standing wave oscillations,
% which goes against our result that larger bows have greater dispersion
% in \(\Lambda'\).  Although the standing waves mainly give dispersion in
% \(\Pi'\) anyhow.

% \citet{Meyer:2014a} say that in cool stars the dust emission comes
% from the shocked inner wind, rather than from near the contact
% discontinuity for hot stars.  This could explain the difference in
% shape between the Herschel and Spitzer samples.


% How different regions of the \(\Pi\)--\(\Lambda\) plane are populated.
% Bottom-right quadrant hard to get to (except for standing wave
% oscillations), but may be due to finite shell thickness, which (for
% low Mach number) will be more apparent in the wings, which might
% decrease \(\Lambda\) more than \(\Pi\).  Fact that thin-shell solutions should
% trace the contact discontinuity, but in some cases it may be only the
% inner or the outer shell that is visible.

% Justification for standing waves: Fig.~3 of \citet{Meyer:2016a} shows
% a time sequence of thin-shell instability, which looks a bit like a
% standing wave. But much larger amplitude than we are considering.

% Deviations from axisymmetry as an alternative to oscillations. 

% \subsection{The case of inside-out bows}
% \label{sec:case-inside-out}

% So far, we have considered the case where the inner source dominates
% the radiation, while dust is present only in the outer stream, which
% applies to hot stars interacting with the ISM.  However, in the case
% of cool stars, the inner wind will also be dusty.  Examples are the
% red supergiant (RSG) phase of high-mass evolution, or the asymptotic
% giant branch (AGB) stage of low/intermediate-mass evolution.  In both
% these cases, it is still the inner source that provides the radiation
% field.  However, not all winds are radiatively driven and in those
% cases it is conceivable that it is the outer source that dominates the
% radiation field.  An example is the case of photoevaporating
% protoplanetary disks (proplyds) in the Orion Nebula and other \hii{}
% regions \citep{ODell:1994a}.  In the proplyds, the inner wind is a
% thermally driven photoevaporation flow \citep{Henney:1998b, Henney:1999a},
% while the outer stream is the stellar wind from an O~star
% \citep{Garcia-Arredondo:2001a}.


%%% Local Variables:
%%% mode: latex
%%% TeX-master: "obs-bowshocks"
%%% End:
