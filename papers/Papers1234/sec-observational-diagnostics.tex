
% Are there any objects that might not be bow shocks?

% Progression in density:
% \begin{gather*}
%   \text{Increasing density} \longrightarrow \\
%   \text{WBS} \to \text{WBS} + \text{IDW} \to \text{WBS} + \text{DDW} \to \text{(RBW)} \to \text{RBS}
% \end{gather*}

% Chief diagnostic for radiation supported bows (RBW or RBS cases) is
% infrared luminosity of bow.  Favored by high densities.

% Dust waves favored by high velocities and intermediate densities.

\section{Optical depth and pressure of the bow shell}
\label{sec:energy-trapp-vers}

\begin{figure*}
  \centering
  \includegraphics[width=0.8\linewidth]{figs/All-sources-eta-tau}
  \caption[Observational diagnostic diagram]{Observational diagnostic
    diagram for bow shocks.  The shell optical depth \(\tau\) (\(x\)
    axis) and momentum efficiency \(\eta\shell\) (\(y\) axis) can be
    estimated from observations of the bolometric stellar luminosity,
    infrared shell luminosity, and shell radius, as described in the
    text.  Results are shown for the 20 sources (circle symbols) from
    \citet{Kobulnicky:2018a} plus three further sources (square
    symbols), where we have obtained the measurements ourselves (see
    Tab.~\ref{tab:observations}).  The color of each symbol indicates
    the stellar luminosity (dark to light) as indicated by the scale
    bar. The shell pressure is determined assuming a gas temperature
    \(T = \SI{e4}{K}\), an absorption opacity
    \(\kappa = \SI{600}{cm^2.g^{-1}}\), and a thickness-to-radius ratio
    \(H/R = 0.25\).  The sensitivity of the results to a
    factor-of-three change in each parameter is shown in the upper
    inset box.  Exceptions are the two Orion Nebula sources, \thD{}
    and LP~Ori, where the small dim squares show the results of
    assuming the standard shell parameters, while the large squares
    show the results of modifications according to the peculiar
    circumstances of each object, as described in the text.  The lower
    inset box shows the sensitivity of the results to a factor-of-two
    uncertainty in each observed quantity: distance to source \(D\);
    stellar luminosity \(L_*\), shell infrared flux \(F\IR\); shell
    angular size \(\theta\).  Lines and shading indicate different
    theoretical bow regimes (see Paper~I and Paper~II).  The dashed
    blue diagonal line corresponds to radiation-supported bows, while
    the upper left region corresponds to wind-supported bows.  The
    upper right corner (purple) corresponds to optically thick bow
    shocks, while the lower left corner (yellow) is the region where
    grain--gas separation \textit{may} occur, leading to a potential
    dust wave.  However, the existence of a dust wave in this region
    is not automatic, since it only includes one of the four necessary
    conditions (\S\S~4.4 and 5.1 of Paper~II). The lower-right region
    is strictly forbidden, except in case of violation of the
    assumption that dust heating be dominated by stellar radiation. }
  \label{fig:All-sources-eta-tau}
\end{figure*}


A fundamental parameter is the optical depth, \(\tau\), of the bow
shell to UV radiation, which determines what fraction of the stellar
photon momentum is available to support the shell (see
\S~2.1 of Paper~I).  But the same photons also heat the
dust grains in the bow, which re-radiate that energy predominantly at
mid-infrared wavelengths (roughly \SIrange{10}{100}{\um}) with
luminosity \(L\IR\).  Assuming that Ly\(\alpha\) and mechanical
heating of the dust shell is negligible (see
\S~\ref{sec:unimp-other-heat}) and that the emitting shell subtends a
solid angle \(\Omega\), as seen from the star, then the optical depth
can be estimated as
\begin{equation}
  \label{eq:tau-empirical}
  \tau = -\ln \left( 1 - \frac{4\pi}{\Omega} \frac{L\IR}{L_*} \right)
  \approx \frac{2 L\IR}{L_*} \ ,
\end{equation}
where the last approximate equality holds if \(\tau \ll 1\) and the shell
emission covers one hemisphere.  Note that the \(\tau\) of Paper~I is not
exactly the same as the \(\tau\) of equation~\eqref{eq:tau-empirical},
but is larger by a factor of
\(Q_P / Q_{\text{abs}} = 1 + \varpi (1 - g)/(1 - \varpi)\), where
\(\varpi\) is the grain albedo and \(g\) the scattering asymmetry.
This is plotted in Figure~6a of Paper~II for Cloudy's default ISM
grain mixture (see \S~\ref{sec:grain-temp-emiss} below), where it can
be seen that \(Q_P / Q_{\text{abs}} = \text{\numrange{1.2}{1.3}}\) at
EUV/FUV wavelengths (\(\lambda = \text{\SIrange{0.05}{0.2}{\um}}\)), so the
correction is small.

A second important parameter is the thermal plus magnetic pressure in
the shocked shell, which is doubly useful since in a steady state it
is equal to \emph{both} the internal supporting pressure (wind ram
pressure plus absorbed stellar radiation) \emph{and} the external
confining pressure (ram pressure of ambient stream).  The shell
pressure is not given directly by the observations, but can be
determined by the following three steps:
\begin{enumerate}[P1.]
\item \label{P1} The shell mass column (\si{g.cm^{-2}}) can be
  estimated from the optical depth by assuming an effective UV
  opacity: \(\Sigma\shell = \tau / \kappa\)
\item \label{P2} The shell density (\si{g.cm^{-3}}) can be found from
  the mass column if the shell thickness is known:
  \(\rho\shell = \Sigma / h\shell\).  In the absence of other information, a
  fixed fraction of the shell radius can be used.  In particular, we
  normalize by a typical value of one~quarter the star--apex distance:
  \(h_{1/4} = h\shell / (0.25 R_0)\).  This corresponds to a Mach
  number \(\M_0 = \surd 3\) if the stream shock is radiative, or
  \(\M_0 \gg 1\) if non-radiative (see \S~3.2 of Paper~I). Further
  discussion is given in \S~\ref{app:bow-shock-data} below. 
\item \label{P3} Finally, the pressure (\si{dyne.cm^{-2}}) follows by
  assuming values for the sound speed and Alfvén speed:
  \(P\shell = \rho\shell (\sound^2 + \frac12 v\alfven^2) \).
\end{enumerate}
It is natural to normalize this pressure to the stellar radiation
pressure at the shell, so we define a shell momentum efficiency
\newcommand\pc{\ensuremath{_{\text{pc}}}}
\begin{equation}
  \label{eq:eta-shell}
  \eta\shell \equiv \frac{P\shell}{P\rad}
  = \frac{4\pi R_0^2\, (\sound^2 + \frac12 v\alfven^2)\, \tau}{L_*\, \kappa\, h\shell}
  \approx 245 \frac{R\pc \, T_4 \, \tau}{L_4 \, \kappa_{600} \, h_{1/4}} \ , 
\end{equation}
where in the last step we have assumed ionized gas with negligible
magnetic support (\(v\alfven \ll \sound\)) and written the stellar
luminosity and shell parameters in terms of typical values, which we
summarize below.  Note that the shell momentum efficiency is simply
the reciprocal of the radiation parameter from Paper~II's equation~(23):
\(\eta\shell = \Xi\shell^{-1}\), which provides yet a third use for
\(\eta\shell\), since \(\Xi\) is paramount in determining whether the
grains and gas remain well-coupled (see \S~4.4 of Paper~II).

In this section and the remainder of the paper, we employ
dimensionless versions of the stellar bolometric luminosity, \(L_*\),
wind mass-loss rate, \(\dot{M}\), and terminal velocity, \(V\wind\),
together with the ambient stream's mass density, \(\rho\), relative
velocity \(v_\infty\), and effective dust opacity, \(\kappa\).  These are
defined as follows:
\begin{align*}
  % \label{eq:stellar-parameters}
  \dot{M}_{-7} &= \dot{M} / \bigl(\SI{e-7}{M_\odot.yr^{-1}}\bigr) \\
  V_3 &= V\wind / \bigl(\SI{1000}{km.s^{-1}}\bigr) \\
  L_4 &= L_* / \bigl(\SI{e4}{L_\odot}\bigr) \\
  v_{10} &= v_\infty / \bigl( \SI{10}{km.s^{-1}} \bigr) \\
  n &= (\rho / \bar{m}) / \bigl( \SI{1}{cm^{-3}} \bigr) \\
  \kappa_{600} &= \kappa / \bigl( \SI{600}{cm^2.g^{-1}} \bigr) \ ,
\end{align*}
where \(\bar{m}\) is the mean mass per hydrogen nucleon
(\(\bar{m} \approx 1.3 m_{\text{p}} \approx \SI{2.17e-24}{g}\) for solar
abundances).

\section[The eta-tau diagnostic diagram]
{\boldmath The \(\eta\shell\)--\(\tau\) diagnostic diagram}
\label{sec:eta-tau-diagnostic}



\begin{table}
  \centering
  \caption[Observational]{Key observational parameters for star/bow systems}
  \label{tab:observations}
  \begin{tabular}{l S S S}
    \toprule
    Star & {\(L_* / \si{L_\odot}\)} & {\(L_{\text{IR}} / \si{L_\odot}\)} & {\(R_0 / \si{pc}\)} \\
    \midrule
    \thD & 2.95e4 & 620 & 0.003 \\
    LP~Ori & 1600 & 240 & 0.01 \\
    \(\sigma\)~Ori & 6e4 & 15 & 0.12 \\[\smallskipamount]
    K18 Sources & \numrange{1.4e4}{8.7e5} & \numrange{8}{2800} & \numrange{0.02}{1.35} \\
    \bottomrule
  \end{tabular}
\end{table}




In Figure~\ref{fig:All-sources-eta-tau} we show the resultant
diagnostic diagram: \(\eta\shell\) versus \(\tau\).  The horizontal axis
shows the fraction of the stellar radiative \emph{energy} that is
reprocessed by the bow shell, while the vertical axis shows the
fraction of stellar radiative \emph{momentum} that is imparted to the
shell, either directly by absorption, or indirectly by the stellar
wind (which is itself radiatively driven).  Radiatively supported bows
(DW, RBW, or RBS, or cases) should lie on the diagonal line
\(\eta\shell = ( Q_P / Q_{\text{abs}}) \tau \approx 1.25 \tau\), where we have used
the ratio of grain radiation pressure efficiency to absorption
efficiency found in the FUV band for the dust mixture shown in
Paper~II's Figure~6.  Wind-supported bows should lie above this line
and no bows should lie below the \(\eta\shell = \tau\) line, since
\(Q_P\) cannot be smaller than \(Q_{\text{abs}}\).

We have calculated \(\eta\shell\) and \(\tau\) using the above-described
methods for the 20 mid-infrared sources studied by
\citet{Kobulnicky:2018a} (K18) and plotted them on our diagnostic
diagram.  Details of our treatment of this observational material are
provided in the following subsection.  In order to expand the range of
physical conditions, we have included three additional sources (data
in Table~\ref{tab:observations}): bows around \thD{}
\citep{Smith:2005a} and LP~Ori \citep{ODell:2001c} in the Orion
Nebula, which show larger optical depths, plus the inner bow around
\(\sigma\)~Ori, which illuminates the Horsehead Nebula and has previously
been claimed to be a dust wave \citep{Ochsendorf:2014b,
  Ochsendorf:2015a}.  Details of the observations of these additional
sources will be published elsewhere.


\subsection{Treatment of sources from \citeauthor{Kobulnicky:2018a}}
\label{sec:kobulnicky}


In a series of papers \citeauthor{Kobulnicky:2018a} provide an
extensive mid-infrared-selected sample of over 700 candidate stellar
bow shock nebulae (\citealp{Kobulnicky:2016a, Kobulnicky:2017a,
  Kobulnicky:2018a}, hereafter K16, K17, and K18).  For 20 of these
sources, reliable distances and spectral classifications are provided
in Table~5 of K17 and Tables~1 and~2 of K18. In this section, we
outline how we obtain \(\tau\) and \(\eta\shell\) from the data in these
catalogs, while further aspects of the \citeauthor{Kobulnicky:2018a}
material are discussed in \S~\ref{app:bow-shock-data}.

The UV optical depth of the bow shell is obtained
(eq.~[\ref{eq:tau-empirical}]) from the ratio of infrared shell
luminosity to stellar luminosity.  The inverse of this ratio is given
in Table~5 of K17, but we choose to re-derive the values since the
spectral classification of some of the sources was revised between K17
and K18.  Although K17 found the total shell fluxes from fitting dust
emission models to the observed SEDs, we adopt the simpler approach of
taking a weighted sum of the flux densities \(F_\nu\) (in \si{Jy}) in
three mid-infrared bands:
\begin{multline}
  \label{eq:total-ir-flux}
  F_{\text{IR}}  \approx \bigl[  2.4\,(F_8 \text{ or } F_{12})
    + 1.6\,(F_{22} \text{ or } F_{24})  \\
  + 0.51\,F_{70}\bigr]
  \,\times \SI{e-10}{erg.s^{-1}.cm^{-2}} \ ,
\end{multline}
where \(F_8\) is Spitzer IRAC \SI{8.0}{\um}, \(F_{24}\) is Spitzer
MIPS \SI{23.7}{\um}, \(F_{12}\) and \(F_{22}\) are WISE bands 3 and 4,
and \(F_{70}\) is Herschel PACS \SI{70}{\um}.  The weights are chosen
so that the integral \(\int_0^\infty F_\nu \,d\nu\) is approximated by the sum
\(\Sigma_k F_k\, \Delta\nu_k\), under the assumption that fluxes in shorter (e.g.,
IRAC \SI{5.8}{\um}) and longer (e.g., PACS \SI{150}{\um}) wavebands
are negligible.  Shell fluxes are converted to luminosities using the
assumed distance to each source, and stellar luminosities are taken
directly from K18 Table~2, based on spectroscopic classification and
the calibrations of gravity and effective temperature from
\citet{Martins:2005a}.  

\begin{figure}
  \centering
  \includegraphics[width=\linewidth]{figs/K17-tau-comparison}
  \caption{Comparison between shell-to-star luminosity ratios
    calculated as described in the text (\(y\) axis) with those given
    in K17 (\(x\) axis).  The blue dashed line signifies equality and
    the gray band shows ratios between 1/2 and 2.}
  \label{fig:k17-k18-comparison}
\end{figure}

In Figure~\ref{fig:k17-k18-comparison} we compare the \(\tau\) obtained
using the shell luminosity as described above with that obtained using
the luminosity ratios directly from K17 Table~5.  It can be seen that
for the majority of sources the two measurements are consistent within
a factor of two (gray band).  The four furthest-flung outliers can be
understood as follows:
\begin{description}
\item[\textit{Source 67}] This has a very poor-quality spectral fit in
  K17 (see lower left panel of their Fig.~12) and so 
  \(F_{\text{IR}}\) is overestimated by them by a factor of 10.
\item[\textit{Sources 341 and 342}] The spectral classes changed from
  B2V in K17 to O9V and B1V, respectively, in K18, increasing the
  derived \(L_*\), which lowers \(\tau\).
\item[\textit{Source 411}] The luminosity class changed from Ib (K17)
  to V (K18), so \(L_*\) has been greatly reduced, which increases
  \(\tau\).
\end{description}


\subsection{Random uncertainties due to observational errors}
\label{sec:rand-syst-uncert}

The fundamental observational quantities that go into determining
\(\tau\) and \(\eta\shell\) for each source are distance, \(D\); stellar
luminosity, \(L_*\); total infrared flux, \(F_{\text{IR}}\); and bow
angular apex distance, \(\theta\).  From these, the shell radius and
infrared luminosity are found as \(R_0 = \theta D\) and
\(L_{\text{IR}} = 4\pi D^2 F_{\text{IR}}\).  Rather than clutter the
diagram with error bars, we instead show the sensitivity to
observational errors in the lower-right box, where each arrow
corresponds to a factor of two increase (0.3~dex) in each quantity:
\(D\), \(L_*\), \(F_{\text{IR}}\), and \(\theta\).  We now calculate
uncertainty estimates for individual observational quantities that are
used in deriving not only \(\tau\) and \(\eta\shell\), but also mass-loss
rates, as discussed later in \S~\ref{sec:stellar-wind-mass}.

\subsubsection{Distance}
\label{sec:distance}

Most sources are members of known high-mass clusters with distance
uncertainties less than 20\% (0.08~dex). The only exception is
Source~329 in Cygnus, for which the distance uncertainty is roughly a
factor of~2 \citep{Kobulnicky:2018a}.

\subsubsection{Stellar luminosity}
\label{sec:stellar-luminosity}

The stellar luminosity is determined from spectral classification,
which makes it independent of distance.  Taking a \SI{2000}{K}
dispersion in the effective temperature scale \citep{Martins:2005a}
gives an uncertainty of 25\% in the luminosity, and adding in possible
errors in gravity and the effect of binaries, we estimate a total
uncertainty in \(L_*\) of 50\% (0.45~mag or 0.18~dex).

\subsubsection{Shell flux and surface brightness}
\label{sec:shell-flux-surface}

We estimate the uncertainty in shell bolometric flux,
\(F_{\text{IR}}\), by comparing two different methods: model fitting
\citep{Kobulnicky:2017a} and a weighted sum of the 8, 24, and
\SI{70}{\um} bands (eq.~[\ref{eq:total-ir-flux}]), giving a standard
deviation of 17\% (0.07~dex).  To this, we add the estimate of 25\%
for the effects of background subtraction uncertainties on individual
photometric measurements \citep{Kobulnicky:2017a}. The absolute flux
calibration uncertainty for both Herschel PACS \citep{Balog:2014a} and
Spitzer MIPS \citep{Engelbracht:2007a} is less than 5\%, which is
small in comparison. Combining the 3 contributions in quadrature gives
a total uncertainty of 0.12~dex.  We adopt the same uncertainty for
the \SI{70}{\um} surface brightness.

\subsubsection{Angular sizes}
\label{sec:angular-sizes}

For the angular apex distance, \(\theta\), the largest uncertainty for
well-resolved sources is due to the unknown inclination.
\citet{Tarango-Yong:2018a} show that the dispersion in true to
projected distances can introduce an uncertainty of \(30\%\)
(0.11~dex) in unfavorable cases (e.g., their Fig.~26).  For 5 of the
20 sources from \citet{Kobulnicky:2018a}, \(\theta\) is of order the
Spitzer PSF width at \SI{24}{\um}, so the errors may be larger.

\subsubsection{Stellar wind velocity}
\label{sec:stell-wind-veloc}

Although this is not strictly an observed quantity for the K18 sample,
we will treat it as such since it is estimated per star, based on the
spectral type.  K18 estimate 50\% uncertainty, and we adopt the same
here (0.18~dex).


\subsubsection{Combined effect of uncertainties on the
  \(\tau\)--\(\eta\shell\) diagram}
\label{sec:comb-effect-uncert}

Assuming that the uncertainty in each observational quantity is
independent, we can now combine them using the techniques described in
Appendix~\ref{sec:comb-uncert-covar} to find the \(\pm 1~\sigma\) error
ellipse, shown in blue in the figure.  It can be seen that
observational uncertainties in \(\tau\) and \(\eta\shell\) are highly
correlated: the dispersion is \SI{0.7}{dex} in the product
\(\eta\shell \tau\) but only \SI{0.16}{dex} in the ratio
\(\eta\shell/\tau\), with stellar luminosity errors dominating in both
cases.  Observational uncertainties are therefore relatively
unimportant in determining whether a given source is wind-driven or
radiation-driven, which depends only on \(\eta\shell/\tau\).  On the other
hand, they do significantly effect the question of whether a source
has a sufficiently high radiation parameter \(\Xi\) to possibly be a
dust wave.


\subsection{Systematic uncertainties due to assumed shell parameters}
\label{sec:syst-uncert-due}

A further source of uncertainty arises from the parameters of the
shocked shell that are assumed in steps P\ref{P1}--P\ref{P3}. Namely,
the relative shell thickness, \(h\shell/R_0\), the ultraviolet grain
opacity per mass of gas, \(\kappa\), and the shell temperature, \(T\).
These parameters effect only \(\eta\shell\), not \(\tau\), with a
sensitivity shown by arrows in the upper left box of
Figure~\ref{fig:All-sources-eta-tau}.

\subsubsection{Shell thickness}
\label{sec:shell-thickness}
We do not expect a great deal of variation in the shell thickness,
except for in the case of fast runaway stars
(\(v > \SI{100}{km.s^{-1}}\)), for which the shell may be dramatically
thinner if the post-shock cooling is sufficiently rapid
(\(h / R_0 \sim M_0^{-2}\)).  For ambient densities less than about
\SI{10}{cm^{-3}}, the minimum thickness is about ten times smaller
than we are assuming.  This occurs at
\(v \approx \SI{60}{km.s^{-1}}\), corresponding to the peak in the cooling
curve at \SI{e5}{K} (see \S~3.2 of Paper~I),
% XREF: Paper I \S~\ref{sec:radi-cool-lengths}
since the thickness is set by the cooling length at higher speeds.  In
principle, the shell thickness can be measured observationally if the
source is sufficiently well resolved \citep{Kobulnicky:2017a},
although this is complicated by projection effects.

\subsubsection{Dust opacity}
\label{sec:dust-opacity}
The dust opacity will depend on the total dust-gas ratio and on the
composition and size distribution of the grains.  Our adopted value of
\SI{600}{cm^2.g^{-1}}, or \SI{1.3e-21}{cm^2.H^{-1}}, is appropriate
for average Galactic interstellar grains in the EUV and FUV (e.g.,
\citealp{Weingartner:2001a}), but there is ample evidence for
substantial spatial variations in grain extinction properties
\citep{Fitzpatrick:2007a}, both on Galactic scales
\citep{Schlafly:2016a} and within a single star forming region
\citep{Beitia-Antero:2017a}.  The properties of grains within
photoionized regions are very poorly constrained observationally
because the optical depth is generally much lower than in overlying
neutral material.  In the Orion Nebula, there is some evidence
\citep{Salgado:2016a} that the FUV dust opacity in the ionized gas may
be as low as \SI{90}{cm^2.g^{-1}}, although the uncertainties in this
estimate are large and different results are obtained in other
regions, such as W3(A) \citep{Salgado:2012a}.  It is even possible
that the FUV dust opacity may be larger than the ISM value if the
abundance of very small grains is enhanced through radiative torque
disruption of larger grains \citep{Hoang:2018a}.

\subsubsection{Shell gas temperature}
\label{sec:shell-gas-temp}
For bows around O~stars, the shell temperature should be close to the
photoionization equilibrium value of \(\approx \SI{e4}{K}\), since the
post-shock cooling length is short in ambient densities above
\SI{0.1}{cm^{-3}} (\S~3.2 of Paper~I)
% XREF: Paper I \ref{sec:radi-cool-lengths}
and the shell does not trap the ionization front for ambient densities
below \SI{e4}{cm^{-3}} (\S~3.1 of Paper~I).
% XREF: Paper I \\ref{sec:trapp-ioniz-front}
For B~stars, on the other hand, these two density limits move closer
together (cf.~the smaller gap between the blue and red lines in
% XREF: Paper I \ref{fig:zones-v-n-plane}a
% XREF: Paper I \ref{fig:B-supergiant}
% XREF: Paper I \ref{fig:zones-v-n-plane}bc
Paper~I's Figs.~2a and 4, as compared with 2b and~c), making it more
likely that a bow will lie in a different temperature regime.  The
only source for which we have evidence that this occurs is LP~Ori, as
discussed in the next section.

\subsection{Special treatment of particular sources}
\label{sec:spec-treatm-part}

For two of the additional bows listed in Table~\ref{tab:observations},
we are forced to deviate from the default values for the shell
parameters.  For LP~Ori, the bow shell appears to be formed from
neutral gas \citep{ODell:2001c} and its relatively high \(\tau\) value
is more than sufficient to trap the weak ionizing photon output of a
B3 star.  We therefore move its point in
Figure~\ref{fig:All-sources-eta-tau} downward to reflect a temperature
of \SI{1000}{K} (or, equivalently, a magnetically supported shell with
\(v\alfven \approx \SI{3}{km.s^{-1}}\)).  For the case of the Orion
Trapezium star \thD{}, we find that using the default parameters
results in a placement well inside the forbidden zone of
Figure~\ref{fig:All-sources-eta-tau} (indicated by fainter symbol).
For this object there is no reason to suspect anything but the usual
photoionized temperature of \SI{e4}{K}, but its placement could be
resolved either by decreasing the shell thickness, or decreasing the
UV dust opacity, or both. Given the moderate limb brightening seen in
the highest resolution images of the Ney--Allen nebula
\citep{Robberto:2005a, Smith:2005a}, the shell thickness is unlikely
to be less than half our default value.  But, if this were combined
with a factor 5 decrease in \(\kappa\), as suggested by
\citet{Salgado:2016a}, then this would be sufficient to move the
source up to the RBW line, or slightly above.


\section{Candidate radiation-supported bows}
\label{sec:cand-radi-supp}

Four sources are sufficiently close to the diagonal line
\(\eta\shell = 1.25 \tau\) in Figure~\ref{fig:All-sources-eta-tau} that they
should be treated as potential candidates for radiation-supported
bows. These are K18 sources~380 (HD~53367, V750~Mon) and 407 (HD~93249
in Carina) plus \thD{} and LP~Ori.  Of the four, source 380 is the
only one that is also a candidate for grain-gas decoupling.  Further
details of the two K18 sources are presented in following subsections,
where for source~380 we show that reducing both luminosity and
distance by a factor of roughly 2 with respect to the values used by
K18 would provide a better fit to the totality of observational data.
However, the ratio \(\eta\shell /\tau\) is proportional to \(D / L_*\) so it
would not be affected by such an adjustment and the bow remains
radiation-supported.  On the other hand \(\eta\shell\) (proportional to
\(D / L_*^2\)) would increase by 2, making the classification as dust
wave candidate more marginal. 

% \subsection{Notes on particular sources}
% \label{sec:notes-part-sourc}

\subsection{Source~380, HD~53367, V750~Mon}
\label{sec:hd-53367-v750}
  
This is a Herbig Be star with spectral type B0V--B0III and mass 12 to
\SI{15}{M_\odot}, which shows long-scale irregular photometric
variability \citep{Tjin-A-Djie:2001a, Pogodin:2006a} together with
cyclic radial velocity variations, which are interpreted as an
eccentric binary with a \SI{5}{M_\odot} pre-main-sequence companion.
It is located outside the solar circle and, although it was originally
classified as part of the CMa OB association at about \SI{1}{kpc}
\citep{Tjin-A-Djie:2001a}, more recent estimates put it much closer.
K18 assume a distance of \SI{260}{pc}, based on Hipparcos measurements
\citep{van-Leeuwen:2007a}, but its Gaia DR2 parallax
\citep{Gaia-Collaboration:2016a, Gaia-Collaboration:2018a, Luri:2018a}
puts it closer still at \SI{140 \pm 30}{pc}.\footnote{%
  Median and 90\% confidence interval, estimated from Bayesian
  inference using an exponential density distribution with scale
  length of \SI{1350}{pc} as a prior (see
  \url{https://github.com/agabrown/astrometry-inference-tutorials/}). }

\citet{Quireza:2006b} report a kinematic distance to the associated
\hii{} region IC~2177 (G\num{223.70}\num{-1.90}) of \SI{1.6}{kpc},
based an LSR radio recombination line velocity of \SI{+16}{km.s^{-1}}
\citep{Quireza:2006a} and the outer Galaxy rotation curve of
\citet{Brand:1993a}.  However, given the dispersion in peculiar
velocities of star-forming clouds (\SIrange{7}{9}{km.s^{-1}},
\citealp{Stark:1984a}) and likely streaming motions of the ionized gas
(\(\approx \SI{10}{km.s^{-1}}\), \citealp{Matzner:2002a, Lee:2012a}),
this is still consistent with a much smaller distance, which would
also help in bringing the radio continuum-derived nebular electron
density into agreement with the value (\(\approx \SI{100}{cm^{-3}}\))
derived from the optical [\ion{S}{ii}] line ratio
\citep{Hawley:1978a}.

\citet{Fairlamb:2015a} derive a distance of \SI{340 \pm 60}{pc} from
combining a spectroscopically determined effective temperature,
gravity, and reddening with photometry and pre-main sequence
evolutionary tracks.  They also determine a luminosity of \SI{13000
  \pm 1000}{L_\odot}, which is only half that assumed by K18, and
would have to be even lower to bring the photometry into concordance
with the Gaia distance.  Alternatively, the luminosity could remain
the same if the total-to-selective extinction ratio were higher than
the \(R_V = 3.1\) assumed by \citet{Fairlamb:2015a}.  Taking
\(R_V = 5.5\) instead, as in Orion, would give \(A_V = 3.34\) and a
predicted \(V\) magnitude of 6.7 if we assume \(L_4 = 1.3\),
\(D = \SI{170}{pc}\) (furthest distance in Gaia 90\% confidence
interval) and \(T_{\text{eff}} = \SI{29.5}{kK}\)
\citep{Fairlamb:2015a} for a bolometric correction of \(-2.86\)
\citep{Nieva:2013a}.  The observed brightness varies between
\(V = 6.9\) and 7.2, which is still at least 20\% fainter than
predicted, but this is the best that can be achieved without rejecting
the Gaia parallax distance entirely.

A final sanity check can be performed by considering the free-free
radio continuum flux of V750~Mon's surrounding \hii{} region, which,
after converting to a luminosity, should be proportional to the total
recombination rate in the nebula and therefore, assuming
photoionization equilibrium and negligible dust absorption in the
ionized gas, also proportional to the ionizing photon luminosity of
the star.  \citet{Quireza:2006b} report a flux density of \SI{6}{Jy}
at \SI{8.6}{Ghz} for G\num{223.70}\num{-1.90}, as compared with a flux
density of \SI{260}{Jy} for the Orion Nebula using the same
instrumental setup.  Assuming an ionizing photon luminosity of
\(S_{49} = 1\) and distance \SI{410}{pc} for the ionizing Trapezium
stars in Orion, therefore implies
\(S_{49} = 0.0027 (D/\SI{140}{pc})^2\) for V750~Mon.  Using an
ionizing flux of \SI{3.2e22}{cm^{-2}.s^{-1}} from the curves in Fig.~4
of \citet{Sternberg:2003a}, this translates to a bolometric luminosity
of \(L_4 = 0.96 (D/\SI{140}{pc})^2\), which is consistent with the
\citet{Fairlamb:2015a} value if we take a distance towards the high
end of the Gaia range.

\subsection{Source 407, HD~93249}
\label{sec:source-407}

TODO

\section{Mid-infrared grain emissivity}
\label{sec:grain-temp-emiss}

In preparation for our discussion of different wind mass-loss
diagnostic methods below, in this section we calculate the grain
emissivity predicted by models of dust heated by a nearby OB star.  We
use the same simulations that we employed in \S~4.2 of Paper~II,
% XREF: Paper II \ref{sec:cloudy-models-dust}
which employ the plasma physics code Cloudy \citep{Ferland:2013a,
  Ferland:2017a}.  In summary, simulations of spherically symmetric,
steady-state, constant density \hii{} regions were carried out for
four different stellar types from B1.5 to O5 (Table~2 of Paper~II), a
range of gas densities from \SIrange{1}{e4}{cm^{-3}}, and using
Cloudy's default ``ISM'' graphite/silicate dust mixture with 10 size
bins from \SIrange{0.005}{0.25}{\um}.  

\begin{figure}
  \centering
  \includegraphics[width=\linewidth]{figs/grain-T-vs-U}
  \caption{Grain temperature versus radiation field mean intensity,
    \(U\), in units of the interstellar radiation field in the solar
    neighborhood.  Line types and colors correspond to a variety of
    stellar spectral shapes, gas densities, and grain species.  Dashed
    lines show carbon grains, solid lines show silicate grains, with
    line thickness and transparency increasing with grain size.
    Stellar spectral types are O5\,V (blue), O9\,V (orange), B1.5\,V
    (purple), and B0.7\,Ia (green), with lighter shades denoting
    higher gas densities. }
  \label{fig:grain-T-vs-U}
\end{figure}

Figure~\ref{fig:grain-T-vs-U} shows equilibrium grain temperatures for
these Cloudy models as a function of the nominal energy density of the
radiation field, \(U = u / u\mmp \), where \(u = L / 4 \pi R^2 c\) and
\(u\mmp\) is the energy density of the interstellar radiation field
for \(\lambda < \SI{8}{\um}\) in the solar neighborhood
\citep{Mathis:1983a}:
\begin{equation}
  \label{eq:u-mmp83}
  u\mmp\,c = \SI{0.0217}{erg.s^{-1}.cm^{-2}} \ .
\end{equation}
The tight relationship seen in Figure~\ref{fig:grain-T-vs-U} between
\(T\) and \(U\) is evidence for the dominance of stellar radiative
heating, which we justify on theoretical grounds in
\S~\ref{sec:unimp-other-heat} below.  The variation about the mean
relation is mainly due to differences in grain size and composition,
with smaller grains and graphite grains being relatively hotter.  The
downward hooks seen on the left end of each simulation's individual
curve are due to the fact that our calculation of \(U\) does not
account for internal absorption, which starts to become important near
the ionization front.

\begin{figure}
  \centering
  \includegraphics[width=\linewidth]{figs/grain-j70-vs-U-edited}
  \caption{Grain emissivity at \SI{70}{\um} for all Cloudy models,
    with lines colored as in Fig.~\ref{fig:grain-T-vs-U}, but without
    the variation in line type and thickness since emissivity is
    integrated over all grain types and sizes.  For comparison, the
    emissivity from the grain models of \citet{Draine:2007a} are shown
    as dark gray symbols, which assume illumination by a scaled
    interstellar radiation field with a SED with a very different
    shape from that of an OB star, see Fig.~\ref{fig:sed-comparison}.
    The light gray symbols show the effect of using an 8 times higher
    \(U\) with the \citeauthor{Draine:2007a} models, which
    approximately compensates for this difference in SED.  }
  \label{fig:grain-j70}
\end{figure}

The grain emissivity at \SI{70}{\um} (Herschel PACS blue band) for the
Cloudy simulations (colored lines) is shown in
Figure~\ref{fig:grain-j70}, where it is compared with the same
quantity from the grain models (dark gray symbols) of
\citet{Draine:2007a}.  A clear difference is seen between the two sets
of models, but this is due almost entirely to a difference in the
assumed spectrum of the illuminating radiation, as illustrated in
Figure~\ref{fig:sed-comparison}.  \citet{Draine:2007a} use a SED that
is typical of the interstellar radiation field in the Galaxy, which is
dominated by an old stellar population, which peaks in the near
infrared, with only a small FUV contribution from younger stars (about
8\% of the total energy density).  This is very different from the OB
star SEDs, which are dominated by the FUV and EUV bands.  Since the
grain absorption opacity is substantially higher at UV wavelengths
than in the visible/IR (see Fig.~6 of Paper~II),
% XREF: Paper II \ref{fig:cloudy-ism-dust-opacity}
the effective grain heating efficiency of the OB star SED is
correspondingly higher.  The light gray symbols show the effect on the
\citet{Draine:2007a} models of multiplying the radiation field by a
factor of \num{8} in order to offset this difference in efficiency,
which can be seen to bring them into close agreement with the Cloudy
models.  A further difference is that the \citet{Draine:2007a} model
includes small PAH particles, which we do not include in our Cloudy
models, since they are believed to be largely absent in photoionized
regions \citep{Giard:1994a, Lebouteiller:2011a}.  However, this only
effects the emissivity at shorter mid-infrared wavelengths
\(< \SI{20}{\um}\).

\begin{figure}
  \centering
  \includegraphics[width=\linewidth]{figs/sed-comparison}
  \caption{Comparison between the spectral energy distribution (SED)
    of a typical OB star (blue line) and the interstellar radiation
    field in the solar neighborhood (orange line).  The OB star is the
    \SI{20}{M_\odot} model from Table~1 of Paper~I and is plotted for
    a distance from the star of \SI{1}{pc}.  The interstellar SED is
    from \citet{Mathis:1983a} and is multiplied by \num{80} so that
    the total FUV-to-NIR flux is equal for the two SEDs.}
  \label{fig:sed-comparison}
\end{figure}

In terms of the characteristic parameters introduced in
\S~\ref{sec:energy-trapp-vers} the dimensionless radiation field
becomes
\begin{equation}
  \label{eq:U-from-L4-and-Rpc}
  U = 14.7\, L_4\, R_{\text{pc}}^{-2} \ ,
\end{equation}
or, alternatively, it can be expressed in terms of the ambient stream as
\begin{equation}
  \label{eq:U-from-ambient}
  U = 3.01 \, n \, v_{10}^2 / x^2 \ , 
\end{equation}
where \(x = R_0/R_*\) is given by Paper~I's equation~(12).
% XREF: Paper I \eqref{eq:x-cases}
It can also be related to the radiation parameter \(\Xi\), defined in
Paper~II's equation~(23), as
% XREF: Paper II \eqref{eq:Xi-Prad-over-Pgas}
\begin{equation}
  \label{eq:U-vs-Xi}
  U = 3.82 \, n T_4 \, \Xi \ .
\end{equation}
A common alternative approach to scaling the radiation field (see
\citealp{Tielens:1985a} and citations thereof) is to normalize in the
FUV band (\SIrange{0.0912}{0.24}{\um}), where the local interstellar
value is known as the Habing flux \citep{Habing:1968a}:
\begin{equation}
  \label{eq:Habing-flux}
  F\Hab = \SI{0.0016}{erg.s^{-1}.cm^{-2}} \ .
\end{equation}
The resultant dimensionless flux is often denoted by \(G_0\), and the
relationship between \(G_0\) and \(U\) depends on the fraction
\(f_{\text{fuv}}\) of the stellar luminosity that is emitted in the
FUV band:
\begin{equation}
  \label{eq:G-vs-U}
  G_0 = f_{\text{fuv}} \frac{u\mmp\, c}{F\Hab} \,U = (\text{\numrange{6}{10}}) \,U \ ,
\end{equation}
where we give the range corresponding to early O
(\(f_{\text{fuv}} \approx 0.4\)) to early B
(\(f_{\text{fuv}} \approx 0.7\)) stars, calculated from TLUSTY stellar
atmosphere models \citep{Lanz:2003a, Lanz:2007a}.

%% Single-photon heating of small grains

\subsection{Unimportance of other heating mechanisms}
\label{sec:unimp-other-heat}
The grain temperature in bows around OB stars is determined
principally by the steady-state equilibrium between the absorption of
stellar UV radiation (heating) and the thermal emission of infrared
radiation (cooling).  Other processes such as single-photon stochastic
heating, Lyman~\(\alpha\) line radiation, and post-shock collisional
heating can dominate in other contexts, but these are generally
unimportant for circumstellar bows, as we now demonstrate.

\subsubsection{Stochastic single-photon heating}
\label{sec:stoch-single-phot}
When the radiation field is sufficiently dilute, then a grain that
absorbs a photon has sufficient time to radiate all that energy away
before it absorbs another photon \citep{Duley:1973a}.  In this case,
the emitted infrared spectrum for \(\lambda < \SI{50}{\um}\) becomes
relatively insensitive of the energy density of the incident radiation
\citep{Draine:2001a}.  However, this is most important for the very
smallest grains.  From equation~(47) of \citet{Draine:2001a}, one
finds that grains with sizes larger than
\(a = \SI{0.005}{\um} = \SI{5}{nm}\) (the smallest size included in
our Cloudy models) should be close to thermal equilibrium for
\(U > 30\), which is small compared with typical bow shock values
(\(U = \text{\numrange{e3}{e6}}\)).  As mentioned above, PAHs are not
expected to be present in the interior of \hii{} regions.
\citealp{Desert:1990a} found them to be strongly depleted for
\(U > 100\) around O stars.  However, other types of ultra-small
grains, down to sub-nm sizes \citep{Xie:2018a} may be present in bows,
and stochastic heating \emph{would} be important for grains with
\(a = \SI{1}{nm}\) if \(U < \num{e5}\).  Note, however that grains
smaller than \SI{0.6}{nm} would be destroyed by sublimation after
absorbing a single He-ionizing photon.


\subsubsection{Lyman \(\alpha\) heating}

On the scale of an entire \hii{} region, the dust heating is typically
dominated by Lyman \(\alpha\) hydrogen recombination line photons, which
are trapped by resonant scattering (e.g., \citealp{Spitzer:1978a}
\S~9.1b).  However, this is no longer true on the much smaller scale
of typical bow shocks.  An upper limit on the Lyman \(\alpha\) energy
density can be found by assuming all line photons are ultimately
destroyed by dust absorption rather than escaping in the line wings
(e.g., \citealp{Henney:1998b}), which yields
\begin{equation}
  \label{eq:U-Lya}
  U\Lya \approx 0.1 n / \kappa_{600} \ .
\end{equation}
This can be combined with equation~\eqref{eq:U-from-ambient} to give
the ratio of Lyman \(\alpha\) to direct stellar radiation as
\begin{equation}
  \label{eq:Lya-over-stellar}
  \frac{U\Lya}{U} \approx 0.03 \frac{x^2}{v_{10}^2 \kappa_{600}} \ .
\end{equation}
Taking the most favorable parameters imaginable of a slow stream
(\(v_{10} = 2\)), very strong wind (\(x \approx 1\)), and reduced dust
opacity (\(\kappa_{600} = 0.1\)) gives a Lyman \(\alpha\) contribution of only
10\% of the stellar radiative energy density.  In any other
circumstances, the fraction would be even lower.

\subsubsection{Shock heating}
\newcommand\kin{\ensuremath{_{\text{kin}}}}
The outer shock thermalizes the kinetic energy of the ambient stream,
which may in principle contribute to the infrared emission of the bow.
In order for this process to be competitive, the following three
conditions must all hold:
\begin{enumerate}[1.]
\item The post-shock gas must radiate efficiently with a cooling
  length less than the bow size, see \S~3.2 of Paper~I.\@
  % XREF Paper I \ref{sec:radi-cool-lengths}
  This is satisfied for all but the lowest densities (Paper~I's
  Fig.~2).
  % XREF Paper I \ref{fig:zones-v-n-plane}
\item A significant fraction of the shock energy must be radiated by
  dust.  This requires that the post-shock temperature be greater than
  \SI{e6}{K}, which requires a stream velocity
  \(v_\infty > \SI{200}{km.s^{-1}}\) \citep{Draine:1981a}.  This also
  coincides with the range of shock velocities where the smaller
  grains will start to be destroyed by sputtering in the post-shock
  gas.
\item The kinetic energy flux through the shock must be significant,
  compared with the fraction of the stellar radiation flux that is
  absorbed and reprocessed by the bow shell.
\end{enumerate}
It turns out that the third condition is the most stringent, so we
will consider it in detail.  The kinetic energy flux through the outer shock for an ambient stream of density \(\rho_\infty\) and velocity \(v_\infty\) is
\begin{equation}
  \label{eq:Fkin}
  F\ke = \tfrac12 \rho_\infty v_\infty^3 = \tfrac12 P\shell v_\infty \ , 
\end{equation}
while the stellar radiative energy flux absorbed by the shell is
\begin{equation}
  \label{eq:Ftrap}
  F\trap \approx \tau L / 4 \pi R_0^2 \ ,
\end{equation}
assuming an absorption optical depth \(\tau \ll 1\). The shell
pressure in the WBS case can be equated to the ram pressure of the
internal stellar wind (see
% XREF Paper I \ref{sec:three-bow-regimes}
\S~2.1 of Paper~I), so that the ratio of the two energy fluxes is
\begin{equation}
  \label{eq:F-ratio-shock}
  \frac{F\ke}{F\trap} = \frac12 \frac{\eta\wind}{\tau} \frac{v_\infty}{c} \ .
\end{equation}
An upper limit to the stellar wind momentum efficiency \(\eta\wind\) is
the shell momentum efficiency \(\eta\shell\) that is derived
observationally in \S~\ref{sec:energy-trapp-vers}, where it is found
that \(\eta\shell / \tau < 30\) for all sources considered.  Therefore, for
a stream velocity \(v_\infty = \SI{200}{km.s^{-1}}\), we have
\(F\ke/F\trap < 0.01\) and the shock-excited dust emission is still
negligible.  Only in stars with \(v_\infty > \SI{1000}{km.s^{-1}}\) would
the shock emission start to be significant, and such hyper-velocity
stars \citep{Brown:2015a} do not show detectable bow shocks.

So far, we have only considered the outer shock, but the inner shock
that decelerates the stellar wind will have a velocity of
\SIrange{1000}{3000}{km.s^{-1}} and therefore might have a significant
kinetic energy flux by eq.~\eqref{eq:F-ratio-shock}.  However, the
stellar wind from hot stars will be free of dust,\footnote{%
  With the exception of Wolf-Rayet colliding wind binary systems
  \citep{Tuthill:1999a, Callingham:2019a}.} %
so that it would be necessary for the stellar wind protons to cross
the contact/tangential discontinuity and deposit their energy in the
dusty plasma of the shocked ambient stream in order for this source of
energy to contribute to the grain emission.  This is not possible
because the Larmor radius (see \S~5 of Paper~II)
% XREF Paper II \ref{sec:magn-effects-grain}
of a \SI{3000}{km.s^{-1}} proton in a \SI{1}{\micro G} field is only
\SI{3e10}{cm}, which is millions of times smaller than typical bow
sizes.  The magnetic field in the outer shell is unlikely to be
smaller than \(\approx n^{1/2} \si{\micro G}\), given that Alfvén speeds of
\SI{2}{km.s^{-1}} are typical of photoionized regions \citep{Arthur:2011a, Planck-Collaboration:2016c}
% XREF Paper II \\ref{eq:alfven}
and if the density were much lower than
\SI{1}{cm^{-3}}, then the scale of the bow would be commensurately
larger anyway.  Three-dimensional MHD simulations of bow shocks
\citep{Katushkina:2017a, Gvaramadze:2018a} show that the magnetic
field lines are always oriented parallel to the shell, so that high
energy particles from the stellar wind would be efficiently reflected
in a very thin layer and cannot contribute to grain heating.  For the
same reason, heat conduction by electrons across the contact
discontinuity is also greatly suppressed \citep{Meyer:2017a}.



%%% Local Variables:
%%% mode: latex
%%% TeX-master: "bs-bw-dw-03"
%%% End:


\section{Stellar wind mass-loss rates}
\label{sec:stellar-wind-mass}

Various methods have been proposed to derive stellar wind mass-loss
rates from observations of stellar bow shocks \citetext{for example,
  \citealp{Kobulnicky:2010a, Gvaramadze:2012a, Kobulnicky:2018a}}.  In
this section, we will show how the \(\tau\)--\(\eta\shell\) diagram can be
used to derive the mass-loss rate for bows in the wind-supported
regime.  We then compare our method with the method used by
\citet{Kobulnicky:2018a}, which is a refinement of that originally
proposed in \citet{Kobulnicky:2010a}.  The method used by
\citet{Gvaramadze:2012a} uses combined measurements of the bow shock
and the surrounding \hii{} region.  It has the advantage of depending
on fewer free parameters than the other methods, but can only be used
in the case of isolated stars, whereas the majority of the sources
considered here are in cluster environments.

\subsection{Mass loss determination from the \boldmath \(\tau\)--\(\eta\shell\) diagram}
\label{sec:mass-loss-determ}
\begin{figure*}
  \centering
  \includegraphics[width=\linewidth]{figs/Mdot-vs-lum-combo-edited}
  \caption{Wind mass-loss rates as a function of stellar luminosity,
    derived from (a)~our trapped energy/momentum method and (b)~the
    grain emissivity method of \citet{Kobulnicky:2018a}, with
    corrections as described in our Appendix~\ref{app:bow-shock-data}.
    Circle symbols show the sources from K18, colored according to the
    stellar effective temperature (see key at far right). In panel a,
    squares show two of our additional sources
    (Tab.~\ref{tab:observations}). Upper limits to the mass loss are
    given for sources that lie close to the radiation-supported line
    in Fig.~\ref{fig:All-sources-eta-tau}, represented by faint
    symbols and downward-pointing arrows.  Sources that lie less than
    a factor of three above the line have enhanced downward
    uncertainties and are also shown by slightly fainter symbols.  For
    \(\sigma\)~Ori, the large symbol corresponds to the sum of the
    luminosities of the triple OB system Aa, Ab, and B
    \citep{Simon-Diaz:2015a}, while the small symbol corresponds to
    the luminosity of only the most massive component Aa.  Lines show
    the predictions of stellar wind models: red lines are the commonly
    used recipes from \citet{Vink:2000a} for dwarfs (solid), giants
    (dashed), and supergiants (dotted), while black lines show
    eq.~(11) of \citet{Krticka:2017a} for O~stars (solid) and models
    of \citet{Krticka:2014a} for B stars.  Orange plus symbols show
    mass-loss measurements from NUV lines for weak-wind O~dwarfs
    \citep{Marcolino:2009a}, while the green plus symbol shows the
    measurement from infrared H recombination lines for \(\sigma\)~Ori
    \citep{Najarro:2011a}.  Boxes show the sensitivity of the results
    to observational uncertainties (lower right) and assumed shell
    parameters (upper left).}
  \label{fig:mass-loss-vs-luminosity}
\end{figure*}




\subsection{Mass loss determination method of \protect\citet{Kobulnicky:2018a}}
\label{app:bow-shock-data}


% \begin{figure*}
%   \centering
%   \includegraphics[width=\linewidth]{figs/K18-pairplot-edited}
%   \caption[K18 pair plot]{Pair plots of correlations between observed
%     and derived parameters of bows from the \citep{Kobulnicky:2018a}
%     sample.}
%   \label{fig:K18-pairplot-edited}
% \end{figure*}


\begin{figure}
  \centering
  \includegraphics[width=\linewidth]{figs/K18-emissivity-vs-U}
  \caption{Discrepancies in \SI{70}{\um} emissivities that we have
    identified in K18.  Red crosses show the emissivities given in
    K18's Table~2 as a function of the radiation field \(U\), while
    the solid gray line shows the emissivities that they claim to be
    using from \citet{Draine:2007a}.  The dashed line shows the
    emissivities that we believe they should have been using, which
    correct for the marked difference in spectral energy distribution
    between OB stars and the Galactic interstellar radiation field
    (see App.~\ref{sec:grain-temp-emiss}).}
  \label{fig:k18-emissivity}
\end{figure}

\newcommand\LOS{\ensuremath{_{\text{los}}}}
K18 derive mass loss rates for their sources using a method that is
different from the one that we employ in
\S~\ref{sec:summary-discussion}.  Both methods are based on
determining the stellar wind ram pressure that supports the bow shell,
but K18 do so via the following steps:
\begin{enumerate}[K1.]
\item \label{K1} The line-of-sight mass column through the shell is
  calculated by combining the peak surface brightness at \SI{70}{\um},
  \(S_{70}\), with a theoretical emissivity per nucleon,
  \(j_{70}(U)\), from \citet{Draine:2007a}:
  \(\Sigma\LOS = S_{70} / \bar{m} j_{70}(U)\).  This depends on
  knowledge of the stellar radiation field at the shell:
  \(U \propto L_* / R_0^2\).
\item \label{K2} The shell density is found from the line-of-sight
  mass column using an observationally determined ``chord diameter'',
  \(\ell\), which is assumed to be equal to the depth along the line
  of sight: \(\rho\shell = \Sigma\LOS / \ell\).
\item \label{K3} The internal ram pressure is equated to the external
  ram pressure, which is found by assuming a stream velocity of
  \SI{30}{km.s^{-1}} and a compression factor of 4 across the outer
  shock:
  \(P_{\text{stream}} = 0.25 \rho\shell \times
  (\SI{30}{km.s^{-1}})^2\).
\end{enumerate}
There are clear parallels but also differences between steps
K\ref{K1}--K\ref{K3} and our own steps P\ref{P1}--P\ref{P3}.  Our
step~P\ref{P1} depends on the total observed infrared flux of the bow
combined with an assumption about the grain opacity at ultraviolet
wavelengths, while step~K\ref{K1} depends on the peak brightness at a
single wavelength combined with an assumption about the grain
emissivity at infrared wavelengths.  Our step~P\ref{P2} requires an
assumption about the relative thickness of the shell, while
step~K\ref{K2} is more directly tied to observations.\footnote{%
  Note that there is a relation between the chord length, \(\ell\),
  and the shell thickness, \(h\shell\), but this depends on the
  planitude, \(\Pi\) of the bow shape \citep{Tarango-Yong:2018a}:
  \( h\shell/R_0 = \Pi \bigl(1 - \{1 - [\ell/(2 \Pi R_0)]^2\}^{1/2}\bigr)\).  } %
On the other hand, step~K\ref{K3} makes a roughly equivalent
assumption about the shock compression factor,\footnote{%
  In reality, the compression factor may be larger or smaller than 4,
  depending on the efficiency of the post-shock cooling (see
  \S~\ref{sec:radi-cool-lengths}).  For instance, for
  \(v = \SI{30}{km.s^{-1}}\) as assumed by K18 and \(T = \SI{e4}{K}\),
  one has a Mach number of \(\M_0 = 2.63\) and a compression factor of
  2.8 for a non-radiative shock (by eq.~[\ref{eq:shock-n-jump}]) or a
  factor of \(\M_0^2 =6.9\) for a strongly radiative one.  } %
and a further assumption about the stream velocity. These assumptions
are not necessary for our step~P\ref{P3}, but we do need to assume a
value for the shell gas temperature.

In principle, both methods are valid and their different assumptions
and dependencies on observed quantities and auxiliary parameters
provide an important cross check on one another.  However, as
explained in detail in Appendix~\ref{sec:grain-temp-emiss}, the
\(j_\nu(U)\) relation depends on the shape of the illuminating SED,
which means that the \citet{Draine:2007a} models require modification
when applied to grains around OB stars.  In addition, when attempting
to replicate the K18 values of \(j_{70}\) we find that they only
follow the \citet{Draine:2007a} values for \(U > \num{e4}\), tending
to a constant value for weaker radiation fields.  The situation is
summarized in Figure~\ref{fig:k18-emissivity}, where the values taken
directly from Table~2 of K18 are shown by red crosses, the values they
claim to be using are shown by the gray solid line (this curve is
consistent with that shown in K18 Fig.~2), and the values they
\textit{should} have been using are shown by the gray dashed line.

% If we replaced K3 with M3, then mass loss rates would be reduced by 1.86 across the board

\begin{figure}
  \centering
  \includegraphics[width=\linewidth]{figs/K18-mdot-Ux8-comparison}
  \caption{Effects on mass-loss determination of correcting the K18
    emissivities.  The mass-loss rates from Table~2 of K18 are shown
    on the \(x\) axis, while the corrected values are shown on the
    \(y\) axis.  Symbols are color coded by the strength of the
    radiation field, \(U\). The corrected mass-loss rates are
    predominantly lower by a factor of roughly 2.}
  \label{fig:k18-mdot-corrected-emissivity}
\end{figure}

After correcting the \SI{70}{\um} emissivities in this way, we
re-derive the mass loss rates, following the same steps as in K18,
which are then used in Figure~\ref{fig:mass-loss-vs-luminosity}b of
\S~\ref{sec:summary-discussion}.  The difference between these
corrected mass-loss rates and those published in K18 is shown in
Figure~\ref{fig:k18-mdot-corrected-emissivity}.  It can be seen that
sources with \(U \approx \num{e3}\) (darker shading) are relatively
unaffected but that sources with stronger radiation fields (lighter
shading) have their mass-loss increasingly reduced, as could be
surmised from Figure~\ref{fig:k18-emissivity}.  The average reduction
is by a factor of about two.

\begin{figure}
  \centering
  \includegraphics[width=\linewidth]{figs/K18-mdot-corrected-comparison-R0-edited}
  \caption{Comparison of the two mass-loss methods: K18 corrected
    method (\(x\) axis) versus our method (\(y\) axis).  Error bars on
    the \(y\) axis correspond to a factor-three uncertainty in
    \(\eta\shell\).  Sources for which these error bars overlap with
    the RBW zone are only upper limits for the wind mass-loss rate,
    and are indicated by faint symbols.}
  \label{fig:mass-loss-comparison}
\end{figure}

Finally, in Figure~\ref{fig:mass-loss-comparison} we compare our own
mass-loss determinations with the corrected K18 values.  The plot
symbols are shaded according to the physical size of the bow, \(R_0\).
It is apparent that the two methods are broadly in agreement on
average, but there is considerable dispersion for individual objects,
with only a weak correlation between the results of the two techniques
(Pearson correlation coefficient \(r = 0.67\)).  Interestingly, the
smaller bows (darker shading) show much better agreement than the
larger bows (lighter shading).  The five bows with
\(R_0 > \SI{0.4}{pc}\) (339, 344, 369, 381, 382) show a difference of
nearly an order of magnitude, in the sense that our method
consistently predicts lower mass-loss rates than K18.  For a further
five of the sources (380, 407, 409, 410, 411), our method gives only
an upper limit to the wind mass-loss rate if one assumes a
factor-three uncertainty in \(\eta\shell\), and these bows are among
the smallest in the sample, all with \(R_0 < \SI{0.1}{pc}\).







Source~342 is anomalous in both methods.

Evidence for small grains around Arches cluster near Galactic center \citep{Hankins:2017a}



Different scenarios for producing velocities: dynamic ejection from
young clusters \citep{Hoogerwerf:2001a, Oh:2016c} produce high
velocities, dissolution of binary systems following core-collapse SN
\citep{Renzo:2018a} tend to produce lower velocities for the unbound
MS companion (walkaways, slower than 30 km/s).  Also, champagne flows
have low velocities.

How different regions of the \(\Pi\)--\(\Lambda\) plane are populated.
Bottom-right quadrant hard to get to (except for standing wave
oscillations), but may be due to finite shell thickness, which (for
low Mach number) will be more apparent in the wings, which might
decrease \(\Lambda\) more than \(\Pi\).  Fact that thin-shell solutions should
trace the contact discontinuity, but in some cases it may be only the
inner or the outer shell that is visible.

Justification for standing waves: Fig.~3 of \citet{Meyer:2016a} shows
a time sequence of thin-shell instability, which looks a bit like a
standing wave. But much larger amplitude than we are considering.

Deviations from axisymmetry as an alternative to oscillations. 

\subsection{Diagnosing bow type from shell emission profile}
\label{sec:diagnosing-bow-type}

Given that the predicted shell density profile varies between
wind-supported and radiation-supported bows (Paper~I) it is worth
considering whether it may be possible to use observed emission
profiles to distinguish between the two.



\subsection{The case of inside-out bows}
\label{sec:case-inside-out}

So far, we have considered the case where the inner source dominates
the radiation, while dust is present only in the outer stream, which
applies to hot stars interacting with the ISM.  However, in the case
of cool stars, the inner wind will also be dusty.  Examples are the
red supergiant (RSG) phase of high-mass evolution, or the asymptotic
giant branch (AGB) stage of low/intermediate-mass evolution.  In both
these cases, it is still the inner source that provides the radiation
field.  However, not all winds are radiatively driven and in those
cases it is conceivable that it is the outer source that dominates the
radiation field.  An example is the case of photoevaporating
protoplanetary disks (proplyds) in the Orion Nebula and other \hii{}
regions \citep{ODell:1994a}.  In the proplyds, the inner wind is a
thermally driven photoevaporation flow \citep{Henney:1998b, Henney:1999a},
while the outer stream is the stellar wind from an O~star
\citep{Garcia-Arredondo:2001a}.


\section{Summary and conclusions}
\label{sec:conclusions}



%%% Local Variables:
%%% mode: latex
%%% TeX-master: "bs-bw-dw-03"
%%% End:
