

\section{Gas--grain coupling and decoupling}
\label{sec:imperf-coupl-betw}

\begin{figure}
  \includegraphics[width=\linewidth]{figs/test-Fdrag-components}
  \caption{Contributions of different collider species to the
    dimensionless drag force, \(f\drag / f_*\), as a function of
    gas--grain slip velocity, \(w\).  Solid lines show the Coulomb
    (electrostatic) drag, while dashed lines show the Epstein
    (solid-body) drag.  Results are shown for dimensionless grain
    potential \(\phi = 10\).  All Coulomb forces scale with
    \(\phi^2\), while the Epstein forces are independent of \(\phi\).  The
    species labelled ``CNO++'' represents the combined effect of all
    metals (see App.~\ref{sec:equat-moti-grains}).}
  \label{fig:drag-components}
\end{figure}


\begin{figure}
  \includegraphics[width=\linewidth]{figs/test-Fdrag-param-space}
  \caption{Regimes of gas--grain drag as a function of slip velocity
    and grain potential.  The different regimes are indicated by bold
    roman numerals, as explained in
    Table~\ref{tab:fdrag-regimes}. Blue shading indicates regions
    dominated by Epstein (solid-body) drag, whereas red and green
    shading indicate regions dominated by Coulomb drag due to protons
    and electrons, respectively.  In each case, the saturated color
    represents a contribution \(> 70\%\) of the relevant component to
    the total drag force, while progressively lighter shading
    represents the \(> 60\%\) and \(> 50\%\) levels.  The thick white
    dotted line indicates the transition between the subthermal and
    superthermal regimes for protons, while the thin white dotted line
    indicates the corresponding transition for electrons.  Contours
    show the total drag force in units of \(f_*\) (see
    eq.~[\ref{eq:fstar}]) in decade intervals from \(0.1\) to
    \(10^4\), as labelled.  Results are shown for
    \(T = \SI{8000}{K}\) and \(n = \SI{100}{cm^{-3}}\), but the
    differences are very slight throughout the ranges
    \(T = \text{\SIrange{5000}{15000}{K}}\) and
    \(n = \text{\SIrange{e-3}{e6}{cm^{-3}}}\).}
  \label{fig:drag-v-phi-plane}
\end{figure}


% \begin{figure}
%   \includegraphics[width=\linewidth]{figs/decouple-v-n-plane}
%   \caption{As Fig.~\ref{fig:zones-v-n-plane}(a), but accounting for
%     gas-grain decoupling with constant efficiency \(\xi = 0.07\). }
%   \label{fig:decouple-v-n-plane}
% \end{figure}



If the radiation field is sufficiently strong, then the collisional
coupling between grains and gas will break down.  In this section, we
calculate the regions of star/stream parameter space where this might
occur, leading to a separation of the bow into an outer dust wave and
an inner, dust-free bow shock.

\subsection{Drag force on grains}
\label{sec:drag-force-grains}

\begin{table}
  \centering
  \caption{Regimes of drag force as function of grain potential and slip speed}
  \label{tab:fdrag-regimes}
  \renewcommand\arraystretch{1.3}
  \resizebox{\linewidth}{!}{%
    \begin{tabular}{@{}r l l l@{}}
    \toprule
      & Regime & Approximate criteria & \(f\drag / f_*\) \\ \midrule
      I & Epstein subsonic & \(\phi^2 \ll 1\)
                             and \(w_{10} < 1\) & \(1.5\, w_{10}\) \\
      II & Epstein supersonic & \(w_{10} > 1\)
                                and \(w_{10} > 5\,\abs{\phi}\)& \( w_{10}^2\) \\
      III & Coulomb p\(^+\) subthermal & \(\phi^2 > 1\)
                                         and \(w_{10} < 1\) & \((1 + 20\, \phi^2)\,
                                                              w_{10}\) \\
    % & Coulomb p\(^+\) peak & \(\phi^2 > 1\)
    %                          and \(w_{10} \approx 1\) & \(1 + 10\, \phi^2\) \\
      IV & Coulomb p\(^+\) superthermal & \(\phi^2 > 1\)
                                          and \(1 < w_{10} < 5\) & \(w_{10}^2
                                                                   + 10\, \phi^2/w_{10}^2 \) \\
      V & Coulomb e\(^-\) subthermal & \(\phi^2 > 20\)
                                 and \(5 < w_{10} < 42\) & \(0.48\, \phi^2 \,
                                                           w_{10}\) \\
    \bottomrule
  \end{tabular}
  }
\end{table}

The drag force \(f\drag\) on a charged dust grain moving at a relative
speed \(w\) through a plasma has contributions from both direct
collisions and from electrostatic Coulomb interactions with ions and
electrons \citep{Draine:1979a}.  Full details of the equations and
collider species used are given in
Appendix~\ref{sec:equat-moti-grains}. Results are shown in
Figure~\ref{fig:drag-components}, where dashed lines correspond to
direct solid body collisions and solid lines to electrostatic
interactions.  The latter depend on the grain potential, which is
described in dimensionless terms by \(\phi\), the electrostatic potential
energy of a unit charge at the surface of a grain of charge
\(z\grain\) and radius \(a\), in units of the characteristic thermal
energy of a gas particle:
\begin{equation}
  \label{eq:phi-potential}
  \phi = \frac{e^2 z\grain}{a kT} \ .
\end{equation}
The electrostatic contributions to \(f\drag\) are proportional to
\(\phi^2\) (results are shown for \(\abs{\phi} = 10\)), whereas the
solid-body contributions are independent of \(\phi\).  The drag force is
put in dimensionless units by dividing by a characteristic force:
\begin{equation}
  \label{eq:fstar}
  f_* = 2 n k T \cdot \pi a^2 \ , 
\end{equation}
which is approximately the ionized gas pressure multiplied by the
grain geometric cross section.

For grains with low electric charge, \(\phi^2 \ll 1\), the drag force is
dominated by direct collisions of protons with the grain (dashed red
line in Fig.~\ref{fig:drag-components}).  The gas collisional mean
free path is much larger than the grain size, so the drag is in the
Epstein regime \citep{Weidenschilling:1977b}.
% This is illustrated by
% the \(\phi = 0.25\) case (blue line) in
% Figure~\ref{fig:gas-grain-drag-photoionized}.
As the relative gas--grain slip speed, \(w\), increases, \(f\drag\)
first increases linearly with \(w\) reaching \(f\drag \approx f_*\) at
\(w = \sound \approx \SI{10}{km.s^{-1}}\), then transitions to a quadratic
increase in the supersonic regime.

As \(\abs{\phi}\) increases, long-range electrostatic interactions with
protons within the Debye radius (Coulomb drag) become increasingly
important at subsonic relative velocities, as shown by the solid lines
in Figure~\ref{fig:drag-components}).
% in
% Figure~\ref{fig:gas-grain-drag-photoionized} by the orange
% (\(\phi = 1\)), green (\(\phi = 4\)), and red (\(\phi = 16\)) lines.
However, the Coulomb drag has a peak when \(w\) is equal to the
thermal speed of the colliders, which is
\(\approx \SI{10}{km.s^{-1}}\) for protons, giving a maximum strength of
\begin{equation}
  \label{eq:fdrag-maximum}
  f_{\mathrm{max}} = 0.5\, (\ln\Lambda)\, \phi^2 f_* \approx 10\, \phi^2 f_* \ , 
\end{equation}
where \(\Lambda\) is the plasma parameter (number of particles within a
Debye volume), such that
\(\ln\Lambda = 23.267 + 1.5 \ln T_4 - 0.5 \ln n\).  At highly super-thermal
speeds, the Coulomb drag falls asymptotically as
\(f\drag \propto 1 / w^{2}\).  The thermal speed of electrons is higher than
that of the protons by a factor of \((m_p / m_e)^{1/2}\), so that the
electron Coulomb drag (solid light blue line) gives a second peak of
similar strength, but at \(w \approx \SI{430}{km.s^{-1}}\).  The behavior of
\(f\drag\) in all these different regimes is summarised in
Table~\ref{tab:fdrag-regimes}, in terms of \(\phi\) and
\(w_{10} = w / \SI{10}{km.s^{-1}}\).  This is further illustrated in
Figure~\ref{fig:drag-v-phi-plane}, where each of the drag regimes is
located on the \((w, \abs{\phi})\) plane.


% \begin{equation}
%   \label{eq:fdrag-regimes}
%   f\drag \approx
%   \begin{cases}
%     \text{Epstein subsonic (\(\phi^2 \ll 1\) and \(w_{10} < 1\)):}
%     & w_{10}\, f_* \\
%     \text{Epstein supersonic (\(\phi^2 \ll 1\) and \(w_{10} > 1\)):}
%     & w_{10}^2\, f_* \\
%     \text{Coulomb p\(^+\) subthermal (\(\phi^2 > 1\) and \(w_{10} < 1\)):}
%     & (1 + 20 \phi^2)\, w_{10}\, f_* \\
%     \text{Coulomb p\(^+\) peak (\(\phi^2 > 1\) and \(w_{10} \approx 1\)):}
%     & (1 + 10 \phi^2)\, f_* \\
%     \text{Coulomb p\(^+\) superthermal (\(\phi^2 > 1\) and \(1 < w_{10} < 5\)):}
%     & (w_{10}^2 + 10 \phi^2/w_{10}^2) \, f_* \\
%     \text{Coulomb e\(^-\) subthermal (\(\phi^2 > 20\) and \(5 < w_{10} < 42\)):}
%     & 0.48 \phi^2 \, w_{10}\, f_* \\
%   \end{cases}
% \end{equation}

% Or \Lambda = (4 pi / 3) n r_D^3?
% Where Debye length is r_D^2 = k T / 4 pi n e^2
% => \Lambda = n r_D (4 pi / 3) k T / 4 pi n e^2
% = k T r_D / 3 e^2
% This is 9 times less than the other expression
% Anyway, from kappa notes I have
% \ln\Lambda = 9.452 + 1.5 ln(T) - 0.5 ln(n)

% If I am going to use log10, then the coefficients get divided by
% ln(10) = 2.30258509299, and if we use T4, then we add 1.5 ln(1e4) =
% 13.815.  So we get 23.267 + 0.651 log10(T4) - 0.217 log10(n).  Nope,
% best with natural log



\begin{figure}
  \centering
  \includegraphics[width=\linewidth]{figs/cloudy-ism-dust-opacity}
  \caption{Extinction properties of Cloudy's standard ``ISM'' dust
    mixture. %
    (a)~Wavelength dependence of mean values over the entire mixture
    of three dimensionless quantities related to scattering: albedo,
    \(\varpi\) (solid line); scattering asymmetry,
    \(g = \langle \cos\theta \rangle\) (dashed line); ratio of radiation pressure
    efficiency to absorption efficiency, \(Q_P / Q_{\text{abs}}\)
    (dotted line).
    %
    (b)~Wavelength dependence of mass opacity (cross section per unit
    mass of gas) for the whole mixture (heavy black line) and broken
    down by size bin and grain composition (colored lines, see key). }
  \label{fig:cloudy-ism-dust-opacity}
\end{figure}

\begin{table*}
  \centering
  \caption{Stellar parameters for example stars}
  \label{tab:stars}
  \begin{tabular}{l S S S S S S S S S}
    \toprule
    & {\(M / \si{M_\odot}\)} & {\(L_4\)}
    & {\(\dot{M}_{-7}\)} & {\(V_3\)} & {\( \eta\wind \)}
    & {Sp.~Type} 
    & {\(T_{\text{eff}} / \si{kK}\)} & {\(\lambda_{\text{eff}}\) / \si{\um}}
    & {\(S_{49}\)} 
    \\
    \midrule
    & 10 & 0.63 & 0.0034 & 2.47 & 0.0066 & {B1.5\,V} & 25.2 & 0.115 & 0.00013
                   \\
    Main-sequence OB stars
    & 20 & 5.45 & 0.492 & 2.66 & 0.1199 & {O9\,V} & 33.9 & 0.086 & 0.16
                    \\
    & 40 & 22.2 & 5.1 & 3.31 & 0.4468 & {O5\,V} & 42.5 & 0.068 & 1.41
                   \\[\smallskipamount]
    Blue supergiant star
    & 33 & 30.2 & 20.2 & 0.93 & 0.3079 & {B0.7\,Ia} & 23.5 & 0.123 & 0.016
                   \\
    \bottomrule
  \end{tabular}
\end{table*}

\subsection{Grain charging and gas--grain coupling}
\label{sec:cloudy-models-dust}

We calculate models of the physical properties of dust grains using
the plasma physics code Cloudy \citep{Ferland:2013a, Ferland:2017a},
which self-consistently solves the multi-frequency radiative transfer
together with thermal, ionization, and excitation balance of all
plasma constituents.  Cloudy incorporates grain charging as described
in \citet{Baldwin:1991a} and \citet{van-Hoof:2004a} with photoelectric
emission theory from \citet{Weingartner:2001b, Weingartner:2006a}.  We
use the default ``ISM'' dust mixture included in Cloudy, which
comprises ten size bins each for spherical silicate and graphite
grains in the range \num{0.005} to \SI{0.25}{\um}, and which is
designed to reproduce the average Galactic extinction curve
\citep{Weingartner:2001a, Abel:2008a}.  The optical properties of each
grain species are calculated using Mie theory \citep{Bohren:1983a},
assuming solid spheres.  The resultant wavelength-dependent extinction
properties of the mixture are summarised in
Figure~\ref{fig:cloudy-ism-dust-opacity}.


To ascertain the expected variation in grain properties in the
circumstellar environs of luminous stars, we calculate a series of
spherically symmetric, steady-state, constant density Cloudy
simulations, illuminated by the stars listed in Table~\ref{tab:stars},
which are the same ones as used in Paper~I.\@ Stellar spectra are
taken from the OSTAR2002 and BSTAR2006 grids, calculated with the
TLUSTY model atmosphere code \citep{Lanz:2003a, Lanz:2007a}.
Simulations are run for hydrogen densities of \numlist{1;10;100;e3;e4}
\si{cm^{-3}} and assuming standard \hii{} region gas phase abundances.
The calculation is stopped when the ionization front is reached and
the inner radius is chosen to be roughly 1\% of this.



\begin{figure*}
  \includegraphics[width=\linewidth]{figs/multi-dustprops}
  \caption{Dust properties as a function of radius from star for three
    selected Cloudy simulations. (a)~\SI{40}{M_\odot} main-sequence star in
    medium of density \SI{e4}{cm^{-3}}. (b)~\SI{10}{M_\odot} main-sequence
    star in medium of density \SI{1}{cm^{-3}}. (c)~Blue supergiant
    star in medium of density \SI{1}{cm^{-3}}}.
  \label{fig:multi-dustprops}
\end{figure*}

Figure~\ref{fig:multi-dustprops} shows resultant radial profiles of
dust properties for representative simulations: grain temperature, grain
abundance, grain potential, and grain drift velocity.  Line types
correspond to the different size bins of graphite and silicate grains,
as indicated in the key from smallest to largest. The left hand panels
show results for a high-density (\(n = \SI{e4}{cm^{-3}}\)), compact
(\(R \approx \SI{0.1}{pc}\)) region around an early O~star, where the grain
temperature is very high, especially for the smaller silicate grains,
and sublimation significantly reduces the grain abundance in the inner
regions.  The remaining columns show low-density
(\(n = \SI{1}{cm^{-3}}\)), extended (\(R \sim \SI{10}{pc}\)) regions
around main-sequence and supergiant B-type stars, in which the grain
temperatures are much lower, ranging from \SIrange{20}{50}{K} in the
outer parts up to \SIrange{100}{200}{K} in the inner parts.

Unlike the strong differences in thermal properties, the radial
dependence of grain electrostatic potential (third row in
Fig.~\ref{fig:multi-dustprops}) is qualitatively similar for all the
simulations.  The grains are predominantly positively charged, with high
potentials (\(> 10\) times the thermal energy of gas particles) close
to the star due to the strong EUV and FUV photo-ejection.  The
potential falls to much lower values in the outer ionized region, as
the EUV flux falls off, and then climbs again at the ionization front
due to the fall in electron density, while the FUV photo-ejection
persists well into the neutral region.  There are small differences
between the simulations due to the increasing relative importance of the
EUV radiation for hotter stars, which leads to a deeper dip in the
potential just inside the ionization front for the \SI{40}{M_\odot} case,
even reaching negative values for some grain species.

Equilibrium drift velocity for each grain species is calculated in the
Cloudy simulations using the same \citet{Draine:1979a} theory as
described in \S~\ref{sec:drag-force-grains} and
Appendix~\ref{sec:equat-moti-grains}.  The way that this is
implemented by default in Cloudy means that if the only solution at
the inner radius is a superthermal one, then the superthermal solution
branch (upper right corner of
Fig.~\ref{fig:gas-grain-drag-photoionized}) is followed as far as
possible through the outer spatial zones.  We have modified the code
so as to instead always prefer the slower subthermal branch whenever
multiple solutions are available.  This makes more sense than the
default behavior for our context, where the grains are moving towards
the star and so the radiative force is gradually increasing from an
initial low value.

Example results are shown in the bottom row of
Figure~\ref{fig:multi-dustprops} and again they are qualitatively
similar for all the simulations.  Close to the star, the radiation
force is higher than the upper limit on the Coulomb drag force
(eq.~\eqref{eq:fdrag-maximum}), so that the equilibrium drift velocity
is exceedingly high.  We discuss this situation in greater detail
below in \S~\ref{sec:drag-force-grains}.  Note that such high drift
velocities are much higher than any realistic true relative velocity
between grains and gas, since they are based on the assumption that
the radiation force remains constant while the grain is accelerated,
which is not the case under these conditions.  Instead, they are
simply an indication that the gas and grains have completely
decoupled.

As the radial distance from the star increases, the radiation field is
increasingly diluted but the grain potential falls only slowly, so
eventually one reaches a point where an equilibrium between Coulomb
drag and radiation force can be established, which corresponds to a
discontinuity in the drift velocity.  The drift velocity carries
on falling towards the outside of the \hii{} region, but then
increases again just inside the ionization front due to the drop in
grain potential there.

\subsection{Gas--grain separation: drift and rip}
\label{sec:gas-grain-separ}

In \S~\ref{sec:gas-free-bow} we calculate the behaviour of an
incoming stream of dust grains, subject only to the repulsive
radiation force from a star.  For an initial inward radial trajectory,
the dust grain motion is decelerated and turned around, reaching a
minimum radius \(R\starstar\), given by equation~\eqref{eq:dust-r0}.
This drag-free radiative turnaround radius, \(R\starstar\), is smaller
for higher initial inward velocities, but is independent of the
density of the incoming stream.  We are now in a position to see how
gas--grain drag will modify this picture.

We introduce the local radiation parameter, \(\Xi\), defined as the
ratio of direct stellar radiation pressure to gas pressure:
\begin{equation}
  \label{eq:Xi-Prad-over-Pgas}
  \Xi \equiv \frac{P\rad}{P\gas} \approx \frac{L}{4 \pi R^2 c\, (2 n k T)} \ ,
\end{equation}
where the last expression corresponds to the optically thin limit.  If
we define the grain's frequency-averaged radiation pressure efficiency
as
\begin{equation}
  \label{eq:Qpbar}
  \Qpbar = \frac{1}{L} \int_0^\infty \!\Qp \, L_\nu \, d\nu \ ,
\end{equation}
then equations~\eqref{eq:dust-rad-force} and~\eqref{eq:fstar} give the
radiation force acting on a grain as
\begin{equation}
  \label{eq:frad-Xi}
  f\rad = \Qpbar\, \Xi\, f_* \ .
\end{equation}

\begin{figure}
  \centering
  \includegraphics[width=\linewidth]{figs/phi-versus-xi-annotate}
  \caption{Grain potential in thermal units (linear scale) versus
    radiation parameter (logarithmic scale). All densities and stellar
    types are shown, with line colors as in Fig.~\ref{fig:drift-gn}.
    Solid lines show silicate grains and dashed lines show graphite
    grains.  Line width increases with grain size (to reduce clutter,
    only every second size bin is shown).  Straight black lines show
    the logarithmic fits discussed in the text:
    eq.~\eqref{eq:phi-vs-Xi}, most appropriate for carbon grains
    around cooler stars, is shown by the dashed line, while the solid
    lines show the modifications for silicate grains and for hotter
    stars.}
  \label{fig:phi-vs-Xi}
\end{figure}

\begin{figure*}
  \includegraphics[width=\linewidth]{figs/gas-grain-drag-photoionized}
  \caption{Dimensionless drag force, \(f\drag / f_*\), as a function
    of gas--grain slip velocity, \(w\), for different values of the
    grain potential in thermal units, \(\phi\).  Contributions from
    proton and electron Coulomb (electrostatic) drag, as well as
    Epstein (solid-body) drag are indicated.  Examples of subsonic and
    highly supersonic stable drift velocities are shown (thin dark
    blue arrows), where the drag force is in equilibrium with the
    radiation force (thick dark blue dashed lines), while blue shading
    indicates the unstable, mildly supersonic velocity regime, where
    no stable drift equilibrium exists. }
  \label{fig:gas-grain-drag-photoionized}
\end{figure*}

The grain potential \(\phi\), which is crucial for determining
\(f\drag\) (Tab.~\ref{tab:fdrag-regimes}), is due to competition
between the photons and the charged particles that interact with the
grain.  It is therefore reasonable to suppose that \(\phi\) should also
be primarily determined by \(\Xi\).  This is confirmed in
Figure~\ref{fig:phi-vs-Xi} using the Cloudy simulations of
\S~\ref{sec:cloudy-models-dust}, for which we find a slow dependence
that can be approximated as
\begin{equation}
  \label{eq:phi-vs-Xi}
  \phi(\Xi) \approx 1.5 \bigl( 2.3 +  \ln \Xi \bigr) \ .
\end{equation}
There are also slight secondary dependencies on the grain composition and
stellar spectrum.  The relationship given in eq.~\eqref{eq:phi-vs-Xi}
is appropriate for graphite grains and for stellar effective
temperatures in the range \SIrange{20}{30}{kK}.  For hotter stars than
this, \(\phi\) should be multiplied by a further factor of \(1.5\), while
for silicate grains it should be divided by \(1.5\).

In the outer regions of the photoionized volume around an OB star,
close to the ionization front, the radiation parameter is low, with
typical value \(\Xi \sim 0.1\).  In this regime, the negative charge
current at the grain surface due to electron collisions is roughly in
balance with the positive current due to the ultraviolet photoelectric
effect \citep{Weingartner:2001b}, leading to a low grain potential,
\(\abs{\phi} < 1\), which may be positive or negative.  The low
\(\Xi\) means that the radiative force is also weak:
\(f\rad \sim 0.1 f_*\) from equation~\eqref{eq:frad-Xi} if
\(\Qpbar \approx 1\) at UV wavelengths, which is true for all but the smallest
grains.  Thus, from the equations for \(f\drag\) given in
Table~\ref{tab:fdrag-regimes}, the radiative force can be balanced by
Epstein drag if \(w_{10} \sim 0.1\), leading to a small equilibrium drift
velocity, \(w\drift < \SI{1}{km.s^{-1}}\), of the grains with respect
to the gas.  This drift is much smaller than the inward stream
velocities that we are considering
(\(v_\infty > \SI{10}{km.s^{-1}}\)), so the dust follows the gas stream at
a slightly reduced velocity (\(< 10\%\)), and (by mass conservation) a
slightly increased density.  Each grain exerts an exactly opposite
force to \(f\drag\) upon the gas, but since the dust-gas mass ratio,
\(Z\grain\), is small, this produces a negligible acceleration of the
gas.

\begin{table}
  \caption{Critical values of radiation parameter at the rip point: \(\Xi_\dag\)}
  \centering
  \begin{tabular*}{0.75\columnwidth}{l @{\quad\quad\quad\quad} S S} \toprule
    & \multicolumn{2}{c}{Grain composition} \\
    Spectrum & {Graphite} & {Silicate}
    \\ \midrule
    B star & 1000 +- 400 & 350 +- 150 \\
    O star & 3000 +- 500 & 2500 +- 500 \\
    \bottomrule
    \addlinespace
    \multicolumn{3}{@{}p{0.75\columnwidth}@{}}{
    Calculated from the Cloudy models shown in Figure~\ref{fig:drift-gn}. 
    Uncertainties represent variations with grain size and gas density. 
    See Appendix~\ref{sec:cloudy-models-dust} for further details.}
  \end{tabular*}
  \label{tab:Xi-rip}
\end{table}

As the dusty stream approaches the star, the radiation parameter
\(\Xi\) will increase, with a dependence of \(R^{-2}\) once the stream
is well inside the ionization front, assuming roughly constant
pressure in the \hii{} region.  This increases \(f\rad\)
(eq.~[\ref{eq:frad-Xi}]), but also increases the grain potential
\(\phi\) (eq.~[\ref{eq:phi-vs-Xi}]) due to the increasing dominance of
grain charging by photoelectric ejection.  Initially, this results in
a lowering of the equilibrium drift velocity to \(w_{10} \sim 0.01\) as
the Coulomb drag kicks in (see lower panels of
Fig.~\ref{fig:multi-dustprops}).  However, at smaller radii the slow
logarithmic increase in \(\phi(\Xi)\) means that the drift velocity must
start increasing again to accommodate the linear increase of
\(f\rad(\Xi)\).  Eventually, \(f\rad\) exceeds \(f_{\mathrm{max}}\), the
maximum drag force that proton Coulomb interactions can provide
(eq.~[\ref{eq:fdrag-maximum}]).  This occurs at a critical value of
the radiation parameter, which we denote the \textit{rip point}:
\(\Xi_\dag \sim 1000\).

\begin{figure*}
  \includegraphics[width=\linewidth]{figs/drift-pratio-4panel}
  \caption{Drift velocity \(w\drift\) versus radiation parameter
    \(\Xi\). Each line represents a simulation with ambient density and
    stellar type as indicated in the key.  Results are shown for
    graphite and silicate grains of two different sizes.  The rip
    point, which corresponds to gas--grain decoupling, is the
    discontinuity in the curves at
    \(w\drift \approx \SI{10}{km.s^{-1}}\), indicated by the upper
    horizontal dashed line.  The vertical dashed lines show the narrow
    range of radiation parameter, \(\Xi = 1000 \pm \SI{0.5}{dex}\), that
    encompasses the rip point for all simulations. }
  \label{fig:drift-gn}
\end{figure*}

To test these ideas, we plot in Figure~\ref{fig:drift-gn} the grain
drift velocity from the Cloudy simulations as a function of \(\Xi\),
showing results for four different grain types and for all
combinations of stellar parameters and ambient densities for which we
have run simulations.  It can be seen that the radiation parameter at
the rip point \(\Xi_\dagger\) is indeed confined to a narrow range.  The
fundamental explanation for this is that both the charge balance and
the force balance are essentially due to competition between the
photons and the charged particles that interact with the grain.  The
small variations in \(\Xi_\dag\) with stellar type and grain composition,
which are of order \SI{+- 0.5}{dex}, are listed in
Table~\ref{tab:Xi-rip}.  The gas density and grain size have very
little influence on this critical value \(\Xi_\dag\), with the only
exception being the very smallest grains (\(a < \SI{0.006}{\um}\), not
illustrated), which show \(\Xi_\dag \approx \num{e4}\), but such grains are only
minor contributors to the UV opacity for our adopted dust mixture
(\(< 10\%\) in EUV and \(< 1\%\) in FUV, see
Fig.~\ref{fig:cloudy-ism-dust-opacity}).


The radius of the rip point, \(R_\dag\), can be expressed in terms of
\(R_*\), the fiducial optically thick bow shock radius introduced in
Paper~I's \S~2.1:
%% XREF \ref{sec:three-bow-regimes}
\begin{equation}
  \label{eq:Rdag-over-Rstar}
  R_\dag = \frac{v_\infty}{\sound}\, \Xi_\dag^{-1/2} R_* \approx v_{10}\, \Xi_\dag^{-1/2} R_* \ ,
\end{equation}
where we have made use of equation~\eqref{eq:Xi-Prad-over-Pgas} and
Paper~I's equation~(3).
%% XREF \eqref{eq:Rstar}



\begin{figure*}
  \centering
  \includegraphics[width=\linewidth]{figs/existence-dust-wave}
  \caption{Regions of stream parameter space \((v, n)\) where dust
    waves may form around main-sequence OB stars of
    \SIlist{10;20;40}{M_\odot} (see Paper~I's Tab.~1).  This figure is
    similar to Paper~I's Fig.~2,
    %% XREF\ref{fig:zones-v-n-plane},
    except that the velocity axis is logarithmic and extends out to
    \SI{1000}{km.s^{-1}}.  Overlapping colored shapes show parameters
    where dust waves may be allowed in the cases of large
    (\(a = \SI{0.2}{\um}\)) and small (\(a = \SI{0.02}{\um}\))
    graphite and silicate grains, as labeled in the left panel.  For
    \((v, n)\) outside of these shapes, dust waves cannot occur for
    the reasons indicated by labeled orange arrows in the center
    panel.  Labeled dashed lines in the right panel show the
    correspondence between the region boundaries and each dust wave
    existence condition given in
    equations~(\ref{eq:dust-wave-velocity-condition},
    \ref{eq:dust-wave-low-density-condition},
    \ref{eq:dust-wave-high-density-condition}). Heavy dashed lines in
    the left panel show where the rip point and the drag-free
    turnaround radius coincide.  Dust waves above these lines are drag
    confined, while dust waves below the lines are inertia confined.
  }
  \label{fig:existence-dust-wave}
\end{figure*}

\subsection{Existence conditions for dust waves}
\label{sec:exist-cond-separ}

In order for a separate outer dust wave to exist, it is necessary for
the grains to decouple from the incoming gas stream before the stream
hits the hydrodynamic bow shock caused by the stellar wind.  The wind
bow shock radius is \(R_0 = \eta\wind^{1/2} R_*\) (Paper~I's eq.~[12]),
%% XREF \ref{eq:x-cases}
where \(\eta\wind\) is the wind momentum efficiency
(Paper~I's eq.~[13]).
%% XREF \ref{eq:wind-eta-typical}
Therefore, the condition
\(R_\dag > R_0\) becomes from equation~\eqref{eq:Rdag-over-Rstar}:
\begin{equation}
  \label{eq:dust-wave-velocity-condition}
  v_{10} > v_{10,\text{min}} = \bigl( \Xi_\dag \, \eta\wind \bigr)^{1/2} \ . 
\end{equation}
For early O main-sequence stars and OB supergiants, the wind
efficiency is generally high (\(\eta\wind > 0.1\)) and
\(\Xi_\dag > 2000\) (Tab.~\ref{tab:Xi-rip}), so that dust waves can only
exist when the stream velocity is very high
(\(v_\infty > \SI{150}{km.s^{-1}}\)).  For main-sequence B~stars, in
contrast, the wind can be much weaker (\(\eta\wind < 0.01\)) and
\(\Xi_\dag\) is also smaller, so that dust waves are permitted by this
criterion for much lower stream velocities:
(\(v_\infty > \SI{30}{km.s^{-1}}\)).  The same will be true of the
``weak-wind'' class of late O main-sequence stars, which also show
\(\eta\wind < 0.01\) (see Paper~I, \S~4).

However, there are other conditions that need to be satisfied in order
for the dust wave to exist.  For instance, the drag-free turnaround
radius must also be outside the bow shock: \(R\starstar > R_0\),
otherwise the radiation is incapable of repelling the grain
opportunely, even once it has decoupled from the gas.  From
equation~\eqref{eq:dust-r0}, together with Paper~I's equations~(3, 9),
%% XREF \eqref{eq:Rstar}
%% XREF \eqref{eq:tau-star},
we find
\begin{equation}
  \label{eq:Rstarstar-over-Rstar}
  \frac{R\starstar}{R_*} = \frac{2 \kappa\grain \tau_*}{\kappa} \ , 
\end{equation}
so the condition becomes
\begin{equation}
  \label{eq:dust-wave-low-density-condition}
  \tau_* >  \tau_{*,\text{min}} = 0.5\, \frac{\kappa}{\kappa\grain}\, \eta\wind^{1/2} 
  % \frac{\kappa \eta\wind^{1/2}}{2 \kappa\grain}
  \ . 
\end{equation}
The average value of the factor \(\kappa / \kappa\grain\) over the entire grain
population must be equal to the dust--gas mass ratio,
\(Z\grain \approx 0.01\), but the factor will vary between grains, according
to their size and composition.\footnote{%
  Remember that \(\kappa\) is the opacity per unit mass of gas, while
  \(\kappa\grain\) is the opacity per unit mass of a particular grain. In
  both cases, averaged over the stellar spectrum.} %
In particular, it will be relatively larger for the largest grains
(\(a \approx \SI{0.2}{\um}\)), which dominate the total dust mass, and
smaller for the smaller grains (\(a \approx \SI{0.02}{\um}\)), which
dominate the UV opacity.  Given the dependence of \(\tau_*\) on the
stream parameters (Paper~I's eq.~[15]),
%% XREF \ref{eq:taustar-typical}
for a given
stellar luminosity this condition corresponds to a minimum value for
\(n / v_\infty^2\).

A third condition comes from requiring \(R_\dag > R_0\) in the radiation
bow wave regime (see Paper~I's \S~2.1),
%% XREF \ref{sec:three-bow-regimes}
where
\(R_0 \approx 2 \tau_* R_*\).  This yields
\begin{equation}
  \label{eq:dust-wave-high-density-condition}
  \tau_* < \tau_{*,\text{max}} = 0.5\, v_{10}\, \Xi_\dag^{-1/2} \ , 
\end{equation}
which, for a given stellar luminosity, corresponds to a maximum value
for \(n / v_\infty^4\).  Thus, for a given stream velocity that satisfies
equation~\eqref{eq:dust-wave-velocity-condition},
equations~(\ref{eq:dust-wave-low-density-condition},
\ref{eq:dust-wave-high-density-condition}) determine respectively the
minimum and maximum stream density for which a dust wave can exist.

The combined effects of the three conditions are illustrated in
Figure~\ref{fig:existence-dust-wave} for each of the three example
main sequence stars from Table~\ref{tab:stars}.  Further restrictions
on the existence of dust waves arise when the effects of magnetic
fields are considered, as will be discussed in
\S~\ref{sec:magn-effects-grain} below.  Note that the three conditions
are restrictions solely on the formation of an \textit{outer} dust
wave, that is, outside of the wind-supported hydrodynamic bow shock.
In the case of the equation~\eqref{eq:dust-wave-low-density-condition}
condition, there is a further possibility: if the gas--grain coupling
(and magnetic coupling) is so weak that it is still unimportant at the
higher densities found in the bow shock shell, then an
inertia-confined \textit{inner} dust wave may form inside the bow
shock, even when \(\tau_* < \tau_{*,\text{min}}\).  Although the same might
be thought to apply to the condition of
equation~\eqref{eq:dust-wave-velocity-condition}, this is not the
case, since the density compression in the bow shock will reduce the
radiation parameter, \(\Xi\), which moves the rip point, \(R_\dag\), to an
even smaller radius.  Therefore, if radiation has not managed to
decouple a grain before it passes through the shock, it is unlikely to
be able to do it afterwards.

\begin{figure*}
  \centering
  \includegraphics[width=\linewidth]{figs/dust-wave-phase-trajectories-annotate}
  \caption{Trajectories of small graphite grains
    (\(a = \SI{0.02}{\um}\)) at impact parameter \(b = 0\) for two
    example cases (see yellow ``+'' symbols in left panel of
    Fig.~\ref{fig:existence-dust-wave}), which differ only in the
    stream velocity: \(v = \SI{40}{km.s^{-1}}\) (left panels) and
    \SI{80}{km.s^{-1}} (right panels).  In both cases, the stream
    density is \(n = \SI{1}{cm^{-3}}\) and the central star is a
    \SI{10}{M_\odot} main-sequence B star (see Tab.~\ref{tab:stars}).
    Upper panels show the evolution of grain radius, \(R\) (blue
    curve, normalized by the rip point radius, \(R_\dag\)), and grain
    velocity, \(v\) (orange curve, normalized by the gas stream
    velocity).  The origin of the time axis is set to the moment of
    closest approach of the grain to the star: \(R = \Rmin\).  Lower
    panels show the trajectories in phase space: position versus
    gas--grain relative slip velocity (\(w = \abs{v - v_\infty}\)).  Filled
    contours show the net force on the grain: \(f\rad - f\drag\), with
    positive values in blue and negative values in red.  The heavy
    dotted line shows where there is no net force: \(f\rad = f\drag\).
    The grain trajectory (thick, solid black line with arrows)
    initially follows this line, but departs from it after the rip
    point. In the left panel, the grain enters a limit cycle between
    decoupling (rip) and re-coupling (snap back).  In the right panel,
    the grain spirals in on the stagnant drift point.  See text for
    further details.}
    \label{fig:phase-space-trajectories}
\end{figure*}

\subsection{Post-rip grain dynamics}
\label{sec:two-regimes-post}

We now investigate the trajectory of the dust grain following the
catastrophic breakdown of gas--grain coupling at the rip point.  Two
regimes are possible, depending on the relation between the rip point
radius, \(R_\dag\), and the drag-free radiative turnaround radius,
\(R\starstar\).  If \(R_\dag > R\starstar\), then the grain's inertia
will still carry it in as far as \(R\starstar\) and the initial
trajectory will be almost identical to that described in
\S~\ref{sec:gas-free-bow} for the drag-free case. But once the grain
has been turned around by the radiation field and pushed out past
\(R_\dag\) again, it will \emph{recouple} to the gas.  
We will refer to this as an \textit{inertia-confined dust wave} (IDW).
From equations~(\ref{eq:Rdag-over-Rstar},
\ref{eq:Rstarstar-over-Rstar},
\ref{eq:dust-wave-high-density-condition}), the condition
\(R_\dag > R\starstar\) corresponds to
\begin{equation}
  \label{eq:IDW}
  \tau_* < \frac{\kappa}{\kappa\grain} \tau_{*,\text{max}} \ , 
\end{equation}
which is indicated by dashed lines in the left panel of
Figure~\ref{fig:existence-dust-wave}.  If, on the other hand,
\(R_\dag < R\starstar\), then the tail wind provided by the gas carries
the grain closer to the star than its inertia would naturally take it.
When the grain finally decouples at \(R_\dag\) it experiences a much
higher unbalanced \(f\rad\), which can initially accelerate it to
outward velocities significantly higher than the inflow velocity if
\(R_\dag \ll R\starstar\).  We will refer to this case as a
\textit{drag-confined dust wave} (DDW).  As in the IDW case, the
expelled grain will eventually recouple to the gas once it moves away
from the star.

\label{sec:grain-traj-along}

What happens to the grain after recoupling depends on the sign of
\(d f\drag / d w\) when \(w = \abs{v_\infty}\).  If this derivative is
positive, as is the case in drag regimes~II and V (see
Tab.~\ref{tab:fdrag-regimes} and Fig.~\ref{fig:drag-v-phi-plane}),
then the grain can reach a stable equilibrium drift at rest with
respect to the star at a point \(R_\ddag\), which we call the
\textit{stagnant drift radius}. If the stream velocity is not
excessively high (\(v_\infty < \SI{150}{km.s^{-1}}\) when
\(\phi = 4\), or \(< \SI{300}{km.s^{-1}}\) when \(\phi = 16\)), then the
equilibrium \(f\rad\) is less than the value at the rip point,
requiring a lower value of the radiation parameter:
\(\Xi_\ddag < \Xi_\dag\).  The resultant stagnant drift radius is therefore
outside the rip point: \(R_\ddag > R_\dag\).  Of course, a static equilibrium
is only possible when the impact parameter is exactly zero.
Otherwise, there will be an unbalanced lateral component of the
radiation force, which will cause a sideways drift.  However, as we
show below, strong coupling to the magnetic field means that the
strictly on-axis calculation is a reasonable approximation over a
range of impact parameters in the case where the angle between the
magnetic field direction and the stream velocity is not too large.

On the other hand, if \(d f\drag / d w < 0\) when
\(w = \abs{v_\infty}\), then the equilibrium is unstable and no stagnant
drift is possible.  This occurs for drag regime~IV, which applies when
\(\phi > 1\) and
\(\SI{10}{km.s^{-1}} < v_\infty < \SI{50}{km.s^{-1}}\), as illustrated in
Figure~\ref{fig:gas-grain-drag-photoionized}.  There is also a second
unstable regime (partially visible in the upper-right corner of
Fig.~\ref{fig:drag-v-phi-plane}), which is related to the thermal peak
in the electron Coulomb drag when \(\phi > 30\) and
\(\SI{400}{km.s^{-1}} < v_\infty < \SI{2000}{km.s^{-1}}\).  This is not
relevant to bow shocks around OB~stars since \(\phi\) does not reach such
high values, but it may apply in other contexts, such as outflows from
AGN, where grain potentials as high as \(\phi \sim 100\) can be achieved
\citep{Weingartner:2006a}.

An example of each of these two behaviors is illustrated in
Figure~\ref{fig:phase-space-trajectories}.  The left panels show the
case where \(v_\infty = \SI{40}{km.s^{-1}}\), which is in the unstable
regime, resulting in periodic ``limit-cycle'' behavior (the parameters
of this model correspond to the yellow ``plus'' symbol labeled ``40''
in the left panel of Fig.~\ref{fig:existence-dust-wave}).  During the
grain's first approach, it starts to follow a phase trajectory (lower
left panel) along the \(f\rad - f\drag = 0\) contour, corresponding to
equilibrium drift, in which the grain begins to move a few
\si{km.s^{-1}} slower than the gas stream.  Then, when it reaches the
rip point (\(R = R_\dag\), \(w \approx \SI{10}{km.s^{-1}}\)) it suddenly
experiences a large unbalanced outward radiation force (blue region of
phase space in Fig.~\ref{fig:phase-space-trajectories}). The grain's
inward momentum carries it to the point \(\Rmin \approx 0.85 R_\dag\), before
it is expelled at roughly twice the inflow speed.  However, after
moving outward, it finds itself in a drag-dominated region of phase
space (red in the figure), and so recouples to the inflowing gas
stream.  The recoupling initiates gradually, as the grain's outward
motion is slowed and it begins to move inward again, but is completed
suddenly once \(w\) again falls below \SI{10}{km.s^{-1}}, in what we
term \textit{snap back}. The net result is that the grain has returned
to exactly the same phase track that it started in on, and so repeats
the cycle indefinitely.

The right panels of Figure~\ref{fig:phase-space-trajectories} show the
case where the stream velocity is doubled to
\(v_\infty = \SI{80}{km.s^{-1}}\), but all other parameters remain the
same.  At this velocity, the equilibrium drift is stable and so the
grain can achieve a stagnant drift solution, where it is stationary
with respect to the star.  The trajectory during the first approach is
similar to the previous case, except that the overshoot of the rip
point is greater, so that \(\Rmin \approx 0.65 R_\dag\) in this case.  This is
a consequence of the fact that the rip point is closer to the
drag-free turnaround radius (\(R_\dag / R\starstar\) is larger than in
the lower velocity case), so that the grain inertia is relatively more
important.  A second consequence of this is that the speed of the
initial expulsion is not so large, being only a little higher than the
inflow velocity.  The qualitative difference between the two cases
emerges after the first recoupling: instead of the snap back and
endless limit cycle, the grain oscillates about the stagnant drift
radius with ever decreasing amplitude, so that after a few oscillation
periods it has come to almost a complete rest.

\begin{figure}
  \includegraphics[width=\linewidth]{figs/onaxis-stats-plot-MS10-v080-gra002}
  \caption{Bow radius as a function of stream density for a stream of
    initial velocity \SI{80}{km.s^{-1}}, which interacts with a
    \SI{10}{M_\odot} main-sequence B~star.  This corresponds to a vertical
    slice through the left panel of
    Fig.~\ref{fig:existence-dust-wave}.  At low densities, the
    hydrodynamic bow shock (blue line) is larger than the drag-free
    turnaround radius for small carbon grains, meaning that a grain's
    inertia carries it into the bow shock along with the gas, even
    though the gas--grain coupling is not particularly strong.  At
    densities above about \SI{0.05}{cm^{-3}}, however, this is no
    longer true and a separate dust wave forms outside of the
    hydrodynamic bow shock, which is now dust-free (green dashed
    line).  The grains in the dust wave will occupy a range of radii
    (pale orange shading) between \(\Rmin\) (solid orange line) and
    \(R_\ddag\), the stagnant drift radius.  At densities above about
    \SI{1000}{cm^{-3}}, the gas stream starts to feel the effect of
    passing through the dust wave, and above \SI{3e4}{cm^{-3}}, the
    dust wave and bow shock merge to form a radiative bow wave (red
    line), which becomes an optically thick radiative bow shock
    (purple line) above \SI{e6}{cm^{-3}}.}
  \label{fig:decouple-vertical-cut}
\end{figure}

\subsection{Back reaction on the gas flow}
\label{sec:back-reaction-gas}

So far we have ignored the effect of the drag force on the gas stream
itself, but it is clear that this must become important as \(\tau_*\)
approaches \(\tau_{*,\text{max}}\), since that is the point where the
dust wave transitions to a bow wave, in which the dust and gas are
perfectly coupled.  A full treatment of this problem would require
solving the hydrodynamic equations simultaneously with the equations
of motion of the dust grains, which is beyond the scope of this paper.
Instead, we outline a heuristic approach that qualitatively captures
the physics involved.

The maximum drag force experienced by a grain is at the rip point.
Since the grain follows a zero-net-force phase track up until that
point, this can be written with the help of
equations~(\ref{eq:dust-rad-force}, \ref{eq:dust-r0}) as
\begin{equation}
  \label{eq:fdrag-max}
  f\drag (R_\dag) = f\rad(R_\dag) =   \frac{m\grain v_\infty^2 R\starstar}{ 2 R_\dag^2} 
\end{equation}
The timescale of the flow can be characterized by the crossing time
\(R_\dag / v_\infty\), but the residence time of the grain at the bow apex
will be several times larger than this (see previous section).  On the
other hand, the average drag force during this residence will be
several times smaller than \(f\drag (R_\dag)\) if
\(R_\ddag > R_\dag\), which is typically the case.  We therefore parameterize
our ignorance via a dimensionless factor, \(\alpha\), which we expect to be
of order unity, and write the total impulse imparted to the grain by
drag as
\begin{equation}
  \label{eq:grain-impulse}
  J\drag \equiv \int \!f\drag \, dt \approx \alpha f\drag (R_\dag) \frac{R_\dag}{v_\infty}
  = \tfrac12 \alpha \, m\grain v_\infty \, \frac{R\starstar}{R_\dag} \ .
\end{equation}

By Newton's Third Law, an equal and opposite impulse is imparted to
the gas, which will act to decelerate the gas stream as it decouples
from the grains.  Realistically, \(J\drag\) should be summed over the
grain size distribution, but for simplicity we assume that all grains
are identical, so that the mass of gas that accompanies each grain is
given by
\begin{equation}
  \label{eq:gas-mass}
  m_{\text{gas}} = \frac{m\grain}{Z\grain} =  m\grain \, \frac{\kappa\grain}{\kappa} \ . 
\end{equation}
If the gas remains supersonic after decoupling, then thermal pressure
can be ignored and the gas will suffer a change in momentum equal to
\(J\drag\), so that its velocity is reduced by
\(\Delta v = J\drag / m_{\text{gas}}\), which by
equations~(\ref{eq:Rdag-over-Rstar}, \ref{eq:Rstarstar-over-Rstar},
\ref{eq:dust-wave-high-density-condition}, \ref{eq:grain-impulse},
\ref{eq:gas-mass}) is
\begin{equation}
  \label{eq:gas-dv}
  \Delta v = \tfrac12 \alpha \frac{\tau_*}{\tau_{*, \text{max}}} v_\infty\ .
\end{equation}
This deceleration reduces the gas stream's ram pressure before it
interacts with the central star's stellar wind.  The radius of the
dust-free bow shock formed by this interaction is therefore increased
by a factor \((1 - \Delta v / v_\infty)^{-1}\) with respect to the case
calculated in Paper~I's \S~2.1,
% XREF \ref{sec:three-bow-regimes}
yielding
\begin{equation}
  \label{eq:gas-free-bow-shock}
  R_{\text{dfbs}} \approx \frac{\eta\wind^{1/2} R_*}{1 - \tfrac12 \alpha \tau_* / \tau_{*, \text{max}}} \ .
\end{equation}

An example is illustrated in Figure~\ref{fig:decouple-vertical-cut},
where the dust-free bow shock radius is shown by the green dashed line
as a function of stream density, \(n\).  This is calculated for fixed
stream velocity and grain and star properties, so that
\(\tau_* \propto n^{1/2}\) (Paper~I's eq.~[15]).
% XREF \ref{eq:taustar-typical}
In order for
\(R_{\text{dfbs}}\) to match the dust-wave and bow-wave radii at the
point where they cross at \(\tau_* = \tau_{*, \text{max}}\), we find
\(\alpha \approx 1.5\) is required.  It can be seen that the gas deceleration is
negligible over most of the density range for which a separate dust
wave arises.  Only for \(n > \SI{e3}{cm^{-3}}\) does
\(R_{\text{dfbs}}\) begin to curve up from the general \(n^{-1/2}\)
trend, becoming essentially flat at a value
\(R_{\text{dfbs}} \approx (\kappa/\kappa\grain) R\starstar\) until full-coupling is
established at \(n > \SI{3e4}{cm^{-3}}\).  Note, however, that the
treatment described here is very approximate: it does not take into
account the shock that must form once \(J\drag\) reaches an
appreciable fraction of \(m_{\text{gas}} v_\infty\) and, additionally, it
includes a factor, \(\alpha\), whose value has not been rigorously
justified.  More detailed modeling is required to fully understand the
bow behavior in this transition regime.



\section{Magnetic coupling of grains}
\label{sec:magn-effects-grain}


It remains to calculate in detail the effects on grain dynamics of the
plasma's magnetic field, in order to justify the approach taken in
\S~\ref{sec:tight-magn-coupl} and extend those results to include the
effects of drag forces from the gas.  The Lorentz force on charged
grains due to a magnetic field is
\begin{equation}
  \label{eq:f-lorentz}
  \bm{f}\!\B = \frac{z\grain e}{c} \, \bm{w} \times \bm{B} \ . 
\end{equation}
The direction of the force is perpendicular both to the magnetic
field, \(\bm{B}\), and to the relative velocity, \(\bm{w}\), of the
grain with respect to the plasma.  If \(\bm{w}\) and \(\bm{B}\) (as
seen by the grain) are changing slowly, compared with the
gyrofrequency, \(\omega\B = z\grain e B / m\grain c\), then the grain
motion perpendicular to \(\bm{B}\) is constrained to be a circle of
radius equal to the Larmor radius:
\begin{equation}
  \label{eq:Larmor}
  r\B = \frac{m\grain c w_\perp} {\abs{z\grain} e B} \ ,
\end{equation}
where \(B = \abs{\bm{B}}\) and \(w_\perp\) is the perpendicular component
of \(\bm{w}\).  The component of \(\bm{w}\) parallel to \(\bm{B}\) is
unaffected by \(\bm{f}\!\B\), so the resultant trajectory is helical.

The relative importance of the magnetic field can be characterized by
the ratio of the Larmor radius to the minimum radius, \(\Rmin\),
reached by the grain in the dust wave (see
\S~\ref{sec:grain-traj-along}), where \(\Rmin \approx R_\dag\) for
drag-confined dust waves (DDW), or \(\Rmin \approx R\starstar\) for
inertia-confined dust waves (IDW).  We write the field strength in
terms of the Alfvén speed,
\begin{equation}
  \label{eq:alfven}
  v\alfven = \frac{B}{(4\pi\rho\gas)^{1/2}}
  = 1.9\, \frac{B}{\si{\micro G}} n^{-1/2} \, \si{km.s^{-1}} \ ,
\end{equation}
and the grain charge \(z\grain e\) in terms of the potential \(\phi\) (eq.~[\ref{eq:phi-potential}]) to obtain
\begin{equation}
  \label{eq:larmor-over-Rdag}
  \text{DDW:}\quad \frac{r\B}{\Rmin} =  
  \frac{r\B}{R_\dag} = 0.0140 \,
  a_{\si{\um}}^2 \,
  \frac{w_\perp}{v\alfven} \,
  \left( \frac{\Xi_\dag}{L_4 T_4} \right)^{1/2}
  \frac{\rho\grain}{\phi_\dag}
\end{equation}
and
\begin{equation}
  \label{eq:larmor-over-Rstarstar}
  \text{IDW:}\quad \frac{r\B}{\Rmin} =  
  \frac{r\B}{R\starstar} = 0.0544 \,
  a_{\si{\um}}^3 \,
  \frac{w_\perp}{v\alfven} \,
  \frac{v_{10}^2 }{n^{1/2}} \,
  \frac{1}{L_4 T_4} \,
  \frac{\rho\grain^2}{\Qp \phi\starstar} \ ,
\end{equation}
where \(a_{\si{\um}} = a / \SI{1}{\um}\), \(\rho\grain\) is the grain
material density in \si{g.cm^{-3}}, and we have made use of
equations~(\ref{eq:dust-r0}, \ref{eq:Rdag-over-Rstar}), together with Paper~I's equation~(14).
%% XREF \ref{eq:Rstar-typical}

If \(r\B / \Rmin \ll 1\), then the grains are so strongly coupled to the
field that they can be treated in the guiding-center approximation, in
which the trajectory is decomposed into a tight circular gyromotion
around the field lines, plus a sliding of the guiding center along the
field lines, which is governed by the radiation and drag forces.  The
radiation force will also produce an out-of-plane drift, given by
\begin{equation}
  \label{eq:perpendicular-drift}
  \bm{v}_{\text{drift}} = \frac{c}{e z\grain} \, \frac{\bm{f}\!\rad \times \bm{B}}{B^2}
  \ ,
\end{equation}
but from equations~(\ref{eq:Larmor}, \ref{eq:dust-rad-force},
\ref{eq:dust-r0}) it follows that
\begin{equation}
  \label{eq:vdrift-over-vinfinity}
  \frac{v_{\text{drift}}(R\starstar)}{v_\infty} = \frac{r\B}{2 R\starstar} \ ,
\end{equation}
so it is valid to ignore this drift in the limit of small \(r\B\).
This is the limit that was applied in \S~\ref{sec:tight-magn-coupl}
for the case of zero drag.  In the opposite limit,
\(r\B / \Rmin \gg 1\), magnetic coupling is so weak that the
non-magnetic results of \S~\ref{sec:imperf-coupl-betw} are scarcely
modified.  Assuming \(w_\perp \sim v_\infty\) and adopting a threshold of
\(r\B / \Rmin < 0.1\), equations~(\ref{eq:larmor-over-Rdag},
\ref{eq:larmor-over-Rstarstar}) can be transformed into conditions on
the stream velocity (in \si{km.s^{-1}}) where tight magnetic coupling
will apply: \newcommand\freeze{\ensuremath{_{\text{tight}}}}
\begin{equation}
  \label{eq:velocity-strong-B-coupling}
  v_{\infty} < v\freeze \approx
  \begin{cases}
    \text{drag-confined:}
    & 0.8 \, a_{\si{\um}}^{-2} \, v\alfven \, L_4^{1/2}\\
    \text{inertia-confined:}
    & 6 \, a_{\si{\um}}^{-1} \, v\alfven^{1/3} \, n^{1/6} \, L_4^{1/3} \\
  \end{cases} \ ,
\end{equation}
where we have substituted typical values of the minor parameters
\(\Xi_\dag\), \(\phi_{\dag}\), \(\phi_{**}\), \(\rho\grain\),
\(T_4\).\footnote{%
  The most significant systematic variation in \(v\freeze\) from these
  suppressed parameters is due to grain composition, yielding slightly
  higher values for graphite than for silicate (\SI{+-0.15}{dex}).} %
It is apparent that \(v\freeze\) is very sensitive to the grain size.
For instance, taking a typical \hii{} region value of
\(v\alfven = \SI{2}{km.s^{-1}}\) \citep{Arthur:2011a} and
\(L_4 = 0.63\) (Tab.~\ref{tab:stars}, B1.5~V star), then for the
drag-confined case \(v\freeze \approx \SI{30}{km.s^{-1}}\) for \SI{0.2}{\um}
grains but \(v\freeze \approx \SI{3000}{km.s^{-1}}\) for \SI{0.02}{\um}
grains. Thus, for typical stream velocities of
\SIrange{20}{100}{km.s^{-1}}, the small grains are always tightly
coupled to the magnetic field, but the large grains are only loosely
coupled for the faster streams.

\subsection{Grain trajectories with tight magnetic coupling}
\label{sec:grain-traj-with}

\begin{figure*}
  \centering
  \includegraphics[width=\linewidth]{figs/frozen-stream-map-multi}
  \caption{Drag-confined dust waves with tight magnetic coupling.
    Upper panels show a quasi-parallel field, \(\theta\B = \ang{10}\),
    while lower panels show a quasi-perpendicular field,
    \(\theta\B = \ang{75}\).  Left panels show an incident stream velocity of
    \(v_\infty = \SI{40}{km.s^{-1}}\), while right panels show
    \(v_\infty = \SI{80}{km.s^{-1}}\).  In all cases, the stream density is
    \(n = \SI{10}{cm^{-3}}\) and the calculations are performed for
    small graphite grains, \(a = \SI{0.02}{\um}\), and the
    \SI{10}{M_\odot} main-sequence B~star.  Continuous black lines show
    grain trajectories, with triplets of colored symbols indicating
    the progress of individual cohorts, which entered from the right
    edge at a particular time.  Continuous blue lines show the
    magnetic field, which flows from right to left along with the
    incident stream.  The radius of the rip point, \(R_\dag\), and the
    stagnant drift point, \(R_\ddag\), are shown respectively by red and
    blue dashed lines.  The approximate shape of the wind-supported
    bow shock is shown by the dotted gray line.  The calculations are
    no longer valid after trajectories cross this surface.}
  \label{fig:frozen-stream}
\end{figure*}

\begin{figure}
  \centering
  \includegraphics[width=\linewidth]{figs/frozen-trajectories-multi}
  \caption{Sample grain trajectories for drag-confined dust waves with
    tight magnetic coupling and a quasi-parallel field
    orientation. These are the same models as in the upper row of
    Fig.~\ref{fig:frozen-stream}. (a)~Incident stream velocity of
    \(v_\infty = \SI{40}{km.s^{-1}}\), showing quasi-limit-cycle behavior
    (\numrange{0}{4000} years).  (b)~Incident stream velocity of
    \(v_\infty = \SI{80}{km.s^{-1}}\), showing quasi-stagnation behavior
    (\numrange{500}{2000} years).  In both cases, the streamline with
    initial impact parameter \(y = -0.4 R_\dag\) is shown.}
  \label{fig:frozen-trajectories}
\end{figure}

We can now investigate how the results of the
section~\ref{sec:imperf-coupl-betw} are modified by magnetic fields in
the tight coupling limit.  For simplicity, we assume a uniform field
in the incoming stream, with field lines oriented at an angle
\(\theta\B\) to the velocity vector that defines the bow axis.  We also
assume a super-alfvénic stream, \(v_\infty > v\alfven\), so that the
radius, \(R_0\), of the wind bow shock is unaffected by the magnetic
field, and additionally assume \(\tau_* \ll \tau_{*, \text{max}}\), so that
the back-reaction of the grain drag on the plasma is negligible
(\S~\ref{sec:back-reaction-gas}) and \(B\) remains uniform in
magnitude and direction in the dust wave region, outside of the bow
shock.

In \S~\ref{sec:tight-magn-coupl}, we derived analytic and
semi-analytic results in the limit of zero gas--grain drag, which is
appropriate for inertia-confined dust waves.  The resultant dust wave
structure is highly dependent on the field orientation.  For a
parallel field (Fig.~\ref{fig:inertia-thB0}), the apex of the dust
wave occurs at the same point, \(R\starstar\), as in the non-magnetic
case, but the shape of the dust wave wings is more closed, being
hemispherical rather than parabolic in shape.  For a perpendicular
field (Fig.~\ref{fig:inertia-thB90}), on the other hand, grains in the
apex region are dragged very close to the star and no dust wave forms
there.  A dust wave can form in the wings, with impact parameter
\(> R\starstar\), which is roughly parabolic in shape, but more
swept-back than in the non-magnetic case.  Whether such a dust wave
will exist in practice depends critically on the size and shape of the
interior MHD wind-supported bow shock.

In Figures~\ref{fig:frozen-stream} and~\ref{fig:frozen-trajectories}
we show example results for drag-confined dust waves, which are
calculated by numerically integrating the grain's equation of motion,
as described in Appendix~\ref{sec:equat-moti-grains}. Apart from the
inclusion of the magnetic field, the model parameters are the same as
used in Figure~\ref{fig:phase-space-trajectories}, with the exception
that the stream density is increased to \SI{10}{cm^{-3}}.\footnote{The
  reason for using a higher density is to decrease the amplitude of
  the radial oscillations of the trajectories, which allows the dust
  wave structure to be more clearly perceived in the figures. } This
time, we use quasi-parallel (\(\theta\B = \ang{10}\)) and
quasi-perpendicular (\(\theta\B = \ang{75}\)) field orientations.  The
quasi-parallel field is most similar to the non-magnetic case, and the
models shown in the two upper panels of Figure~\ref{fig:frozen-stream}
closely mirror the cases shown in
Figure~\ref{fig:phase-space-trajectories}, with a limit cycle behavior
when \(v_\infty = \SI{40}{km.s^{-1}}\) and stagnant drift when
\(v_\infty = \SI{80}{km.s^{-1}}\).

The principle difference from the 1-D axial trajectories discussed in
\S~\ref{sec:grain-traj-along} is that the small \ang{10} misalignment
of \(\bm{B}\) from the incident stream direction causes a slow
sideways migration, which puts a finite limit on the time a grain can
reside in front of the star.  This can be appreciated more clearly in
Figure~\ref{fig:frozen-trajectories}, which shows the grain position
and velocity along a sample streamline with initial impact parameter
\(b = -0.4 R_\dag\) for the two quasi-parallel models.  In panel~a,
corresponding to \(v_\infty = \SI{40}{km.s^{-1}}\), we see the same
rip-and-snap-back cycle of de-coupling and re-coupling that was
discussed previously, but only two periods of the cycle are completed
before the grain's lateral migration takes it as far as
\(y \approx +R_\dag\), at which point nothing can stop the gas stream from
dragging it past the star and away.  All told, the grain remains in
the apex region for about \(1/\sin\theta\B\) times longer than the crossing
time, \(R_\dag / v_\infty\).

In panel~b, corresponding to \(v_\infty = \SI{80}{km.s^{-1}}\), the grain
settles for a while around the stagnant drift radius, \(R_\ddag\), after
the initial rip and turn around.  Again, it slowly migrates sideways,
and eventually recouples to the incident stream, but this time after
reaching \(y \approx +R_\ddag\).  The grain residence time in the apex region,
measured in crossing times, is slightly longer than in panel~a, but is
of the same order.  In both these cases, a second, exterior dust
shell is formed in addition to the hemispherical one produced by the
initial turn around inside \(R_\dag\). In the limit-cycle case, it forms
at the snap-back point, while in the stagnant-drift case, it forms at
\(R_\ddag\) and is significantly denser than the interior shell (upper
right panel of Fig.~\ref{fig:frozen-stream}).

The assumptions behind these models break down when the grain
trajectory intersects the outer shock of the dust-free wind-supported
bow shock.  The increased gas density in the bow shock shell will
reduce \(\Xi\), which is likely to cause gas--grain recoupling in the
case of drag-confined dust waves (see also discussion in the final
paragraph of \S~\ref{sec:exist-cond-separ}).  However, detailed
modeling of this requires magnetohydrodynamical simulations, which are
beyond the scope of this paper.  In Figure~\ref{fig:frozen-stream} we
show in gray dotted lines the Wilkinoid surface \citep{Wilkin:1996a},
which is an approximation to the shape of the inner bow shock, see
\citet{Tarango-Yong:2018a}.  It can be seen that the majority of dust
streamlines do not cross this surface until well into the wings of the
bow, so that a separate exterior dust wave can exist for this
quasi-parallel magnetic field orientation.

This is no longer the case for the quasi-perpendicular field
orientation, as illustrated in the lower panels of
Figure~\ref{fig:frozen-stream}, where the behavior is similar to the
perpendicular inertia-confined case studied in
\S~\ref{sec:perp-magn-field}.  Although the grains decouple from the
gas inside the rip point, for small impact parameters the magnetic
geometry does not allow the radiation field to expel them until they
are much nearer to the star.  Since the inner wind-supported bow shock
radius is only a few times smaller than \(R_\dag\), it is possible that
they will pass through the shock surface and re-couple before
expulsion can occur.  For the \SI{40}{km.s^{-1}} model (lower left
panel), all of the streamlines with impact parameters
\(\abs{b} < R_\dag\) intersect the Wilkinoid surface before they are
deflected, and so no separate dust wave would form in this case.  For
the \SI{80}{km.s^{-1}} model (lower left panel), some of the
streamlines (mainly those with \(b > 0\)) do manage to avoid crossing
the bow shock surface, so it is possible that a dust wave may still
form, although it would be on only one side of the axis.


%%% Local Variables:
%%% mode: latex
%%% TeX-master: "bs-bw-dw-02"
%%% End:
